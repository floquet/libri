\section{Solution using the SVD}
Let's return to using an SVD to solve the linear systems like $\ls$ as shown in \S\eqref{sec:formal:simple}.

Once you have the SVD life is good.
 
%%
\subsection{Extended example}
We don't want 3 x 3, go to 3 x 5
\begin{equation}
  \begin{array}{cccc}
    \A{}&x &=& b\\
    \mat{rrrrr}
    {
     1 & -1 &  1 & -1 &  1\\
    -1 &  1 & -1 &  1 & -1\\
     1 & -1 &  1 & -1 &  1\\
    }
    &
    \mat{c}{x_{1}\\x_{2}\\x_{3}\\x_{4}\\x_{5}}
    &=& \phivector.
  \end{array}
\end{equation}

\textbf{Recycling:} The good news is that we can solve this system using the simple \svdl \ method of the first chapter allowing us to bypass the eigensystem problem. Even more good] news is that we can recycle the codomain matrix $\Y{}$. The active column vector in the domain matrix is this:
\begin{equation}
  \X{}_{*,1} = \sfive\mat{r}{1 \\ -1 \\  1 \\ -1 \\  1}.
\end{equation}
Using the fact that
\begin{equation}
  \begin{split}
    \A{}\X{}_{*,1}=\sigma_{1}\Y{}_{*,1}
  \end{split}
\end{equation}
we can compute the lone singular value of
\begin{equation}
  \sigma_{1} = 15^{-1/2}.
\end{equation}

Without doing noticeable calculation, we already have the partial decomposition
\begin{equation}
  \begin{split}
    \svda{T}\\
    &=
    \Yshade
    \mat{c|cccc}
    {
     15^{-1/2} & 0 & 0 & 0 & 0\\\hline
      0 & 0 & 0 & 0 & 0\\
      0 & 0 & 0 & 0 & 0\\
    }
    \mat{ccccc}
    {\sfive & -\sfive & \sfive & -\sfive & \sfive\\
     \rowcolor{ltgray}
     \cdot  &  \cdot  & \cdot  &  \cdot  & \cdot \\
     \rowcolor{ltgray}
     \cdot  &  \cdot  & \cdot  &  \cdot  & \cdot \\
     \rowcolor{ltgray}
     \cdot  &  \cdot  & \cdot  &  \cdot  & \cdot \\
     \rowcolor{ltgray}
     \cdot  &  \cdot  & \cdot  &  \cdot  & \cdot}\\
  \end{split}
\end{equation}

The Gram-Schmidt process in the appendix \eqref{sec:gs} will complete the $\X{}$ matrix. Using the seed vectors
\begin{equation}
  U = \lst{
  \mat{r}{1 \\ -1 \\  1 \\ -1 \\  1},
  \mat{c}{1\\0\\0\\0\\0},
  \mat{c}{0\\1\\0\\0\\0},
  \mat{c}{0\\0\\1\\0\\0},
  \mat{c}{0\\0\\0\\1\\0}
  }
\end{equation}
the domain matrix becomes
\begin{equation}
  \X{} = 
  \left[
\begin{array}{ r >{\columncolor{ltgray}}r >{\columncolor{ltgray}}r >{\columncolor{ltgray}}r >{\columncolor{ltgray}}c }
  \sfive &  \frac{4}{2\sqrt{5}} &  0 &  0 &  0 \\
 -\sfive &  \frac{1}{2\sqrt{5}} &  \frac{3}{2\sqrt{3}} &  0 &  0\\
  \sfive & -\frac{1}{2\sqrt{5}} &  \frac{1}{2\sqrt{3}} & -\frac{2}{6} &  0\\
 -\sfive &  \frac{1}{2\sqrt{5}} & -\frac{1}{2\sqrt{3}} &  \ssix & \stwo\\
  \sfive & -\frac{1}{2\sqrt{5}} &  \frac{1}{2\sqrt{3}} & -\ssix & \stwo\\
\end{array}
\right]
\end{equation}

Now do the change of coordinates stuff in the section on a more formal introduction.



\endinput