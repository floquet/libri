\chapter{Modal Example Continued}

Other solution methods

\section{Normal Equations - Again}  %    S    S    S    S    S    S    S    S    S
  \begin{equation*}   %  =   =   =   =   =
  %\begin{split}
    {\bf{1}} = \mat{c}{1 \\ 1 \\ 1 \\ 1 \\ 1 \\ 1 \\ 1 \\ 1 \\ 1}, \quad
    x        = \mat{c}{1 \\ 2 \\ 3 \\ 4 \\ 5 \\ 6 \\ 7 \\ 8 \\ 9}, \quad
    T        = \frac{1}{10}\mat{c}{156 \\ 175 \\ 366 \\ 438 \\ 582 \\ 616 \\ 642 \\ 704 \\ 988}
    %\label{eq:}
  %\end{split}
  \end{equation*}

  \begin{equation*}   %  =   =   =   =   =
  %\begin{split}
    \A{} = 
      \mat{c|c}{{\bf{1}} & x}
    %\label{eq:}
  %\end{split}
  \end{equation*}
The linear system $\A{}a = T$ looks like this
  \begin{equation}   %  =   =   =   =   =
  %\begin{split}
      \mat{cc}{
         1 & 1 \\
         1 & 2 \\
         1 & 3 \\
         1 & 4 \\
         1 & 5 \\
         1 & 6 \\
         1 & 7 \\
         1 & 8 \\
         1 & 9 } 
      \mat{c}{ a_{0} \\ a_{1} } =
         \frac{1}{10}\mat{c}{156 \\ 175 \\ 366 \\ 438 \\ 582 \\ 616 \\ 642 \\ 704 \\ 988} .
    \label{eq:bevington axeb}
  %\end{split}
  \end{equation}

  \begin{equation*}   %  =   =   =   =   =
  \begin{split}
    \oto &= m = 9 \\
    \otx &= \xto = 45 \\
    \xtx &= 285 \\
    {\bf{1}}^{\mathrm{T}}T &= \frac{4667}{10} \\
    x^{\mathrm{T}}T        &= 2898
    %\label{eq:}
  \end{split}
  \end{equation*}
  
  \begin{equation*}   %  =   =   =   =   =
  %\begin{split}
    \wx{*} = \mat{cc}{\oto & \otx \\ \xto & \xtx} = \mat{cc}{9 & 45 \\ 45 & 285}
    %\label{eq:}
  %\end{split}
  \end{equation*}

  \begin{equation*}   %  =   =   =   =   =
  %\begin{split}
    \A{*}T = 
      \mat{c}{{ \bf{1}}^{\mathrm{T}}T \\ x^{\mathrm{T}}T } =
      \frac{1}{10} \mat{c}{4667 \\28\,980}
    %\label{eq:}
  %\end{split}
  \end{equation*}

\eqref{eq:bevington axeb} becomes
  \begin{equation}   %  =   =   =   =   =
  %\begin{split}
    \mat{cc}{\oto & \otx \\ \xto & \xtx}
    \mat{c}{a_{0} \\ a_{1}} = 
    \mat{c}{{ \bf{1}}^{\mathrm{T}}T \\ x^{\mathrm{T}}T }
    \label{eq:bevington vectors}
  %\end{split}
  \end{equation}

\section{Singular Value Decomposition}  %    S    S    S    S    S    S    S    S    S
Solution steps
\begin{enumerate}
  \item Compute $\lambda\paren{\wx{*}}$.
  \item Educated guess at domain matrix $\V{}$.
  \item Compute codomain matrix $\U{}$.
\end{enumerate}



\section{$\Q{}\R{}$ Decomposition}  %    S    S    S    S    S    S    S    S    S

\subsection{Problem Statement}  %   SS   SS   SS   SS   SS   SS   SS   SS   SS   SS   SS   SS

\endinput