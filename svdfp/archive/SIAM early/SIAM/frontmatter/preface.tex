\begin{thepreface}\index{Preface}

\section*{Purpose}
For so many people, the \svdp \ is an objet d'art: elegant, powerful and intuitive. A foundation concept in linear algebra - the four fundamental subspaces - is brilliantly illuminated by the SVD. A fascinating topic worthy of study in its own right, the decomposition is also of central primacy in linear algebra. It is intimately connected to the generalized matrix inverse. It provides an obvious way to quantify the information content in a matrix. It is a work horse in the prolific field of least squares: the singularity of system matrices can be quantified, and it becomes a general tool in the attack of ill-posed and ill-conditioned problems.

However, this broad and far-reaching tool is used by many who feel that SVD is a Byzantine method whose machinations are murky or even opaque. This is a needless shame and the motivation for preparing this pedestrian manual.

The purpose of this work is to bring the joy and utility of the singular value decomposition to the masses.

\section*{Audience}
\begin{chapterquote}[20pt]
Life has obliged him to remember so much useful knowledge that he has lost not only his history, but his whole original cargo of useless knowledge; history, languages, literatures, the higher mathematics, or what you will - are all gone. \\
---{\upshape Albert J. Nock}
\end{chapterquote}
\ \\
Albert J. Nock 
The target audience for this book is working professionals. As people build a career they acquire and reinforce a great deal of knowledge. Somewhat by unpleasant necessity we lose contact with some of our formal education. But whether we encounter new fields of study or need to reestablish contact with older skills, we sit down and teach ourselves what we need to learn.

And so here we are with the SVD and this book is an tool to help the reader to understand and to master the technique. The foundation presumption is that the reader has a need to master the SVD, and the background material is not immediate. This may be because
\begin{itemize}
\item You studied mathematics, but the linear algebra did not cover the SVD, or,
\item the presentation of the SVD was cursorial, or,
\item your contact with formal mathematics was many years in the past.
\end{itemize}
However, effort has been taken to incorporate classroom experience and to enrich graduate students in the physical sciences and engineering.

In order to be a useful tool for study, we have made the text a bit verbose and the examples quite detailed. This is because there is no instructor available to help extricate one from the innumerable little mistakes that can derail a computation.

Also, we yield to the reality that books are not read \emph{in toto} from front cover to back. Therefore we have endeavored to make the summaries and examples as complete as possible to preclude a mad scanning of the index to decipher the clues.

\section*{The beauty of the SVD}
\begin{chapterquote}[20pt]
Beauty is the first test: there is no permanent place in the world for ugly mathematics. \\
---{\upshape Godfrey Harold Hardy }
\end{chapterquote}
\ \\

The SVD is one of the most elegant results of linear algebra. Too often mathematicians in the classroom present the subject for mathematicians-in-training. Certainly, presenting a constructive proof of the SVD can be expected to disenfranchise some the practical students outside of the math department.

Clearly, there is great utility to the method of theorem-proof-examples. But the focus here is to address the practical matters  and this leads to an examples-theorem-proof paradigm. Instead of presenting a theorem and discussing important results, we present the results and ask the reader to decide upon their importance. This is a blatant attempt to hook the reader.

A more typical linear algebra class will build up to the SVD and expose it as a wonderful embodiment of \ftola and reveal it as an ultimate form of diagonalization and orthogonalization. 


\begin{quote}
The \svdl \ decomposes a target matrix into orthonormal basis matrices for the domain and the codomain as well as a diagonal matrix of scale factors.
\end{quote}

One subtle goal of this work is show the beauty and power of linear algebra from the perspective of the SVD.

Books on mathematics and books on exercise share a common failing: just reading them won't provide the desired improvement. Your reward is tied directly to your effort. The examples have been written with sufficient detail to allow readers to allow readers to spot their mistakes as they follow along with pencil and paper. Those fortunate few in excellent mathematical shape may find the exercises trifling. However, the goal is to include a broad audience of people who may not have covered this material or may have covered it in the distant past.

\end{thepreface}

\endinput