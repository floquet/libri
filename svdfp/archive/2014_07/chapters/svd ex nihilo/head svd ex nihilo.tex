\chapter{SVD {\it{Ex Nihilo}}}
%
The \ftola \ is a thunderous statement about the vector spaces associated with a matrix. It proclaims the existence of the row and column spaces and their complementary structure. Given our prototype matrix $\aicmnr$
\begin{equation}
  \begin{split}
    \cmplxm &= \brnga{}  \oplus \rnlla{*} \\
    \cmplxn &= \brnga{*} \oplus \rnlla{} \\
  \end{split}
\end{equation}
We know that these range and \ns s exist, we know their dimensions, and we know the are orthogonal complements, the boon of the \ft . 

But the \ft \ is mute about the mapping action of a matrix. For example, in the maps for matrix (c), unit circles are mapped into ellipses.  

The previous chapter starts with the \ftola \ and it's subspace decomposition. Then it explores how the target matrix connects the domain and codomain. This brings up the alignment of the spaces and scale factors, issues not addressed by the \ft. 

The goal of this chapter is to look at the \asvd s of the example matrices and show that they provide the missing alignment and scaling information. The issue of how to compute a \asvd \ is deferred until the next chapter. Here, the SVDs appear {\it{ex nihilo}} and the focus is on how this takes us beyond the \ftola.

\input{chapters/"svd ex nihilo"/"sec shape and configuration of the decomposition matrices"}
\input{chapters/"svd ex nihilo"/"sec comparing decompositions"}
\input{chapters/"svd ex nihilo"/"sec essence of the SVD"}

\endinput