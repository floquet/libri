\section{Jargon}

%%%%%%
\subsection{Overview}
The goal here is to prepare the reader for the terminology which follows.

%%%%%%
\subsection{AKA}
Here is a table of ``also known as''
\begin{table}[htdp]
\caption{default}
\begin{center}
\begin{tabular}{ll}
null space $\nll{\A{}}$ & kernel\\
 & right null space\\
null space $\nll{\A{}}$ & cokernel\\
 & left null space\\
\end{tabular}
\end{center}
\label{default}
\end{table}%


%%%%%%
\subsection{Range and image}
The column space refers to the vector space induced by the columns. The range, or image, is the collection of all possible combinations of the column vectors. For a matrix with $m$ rows, the induced vector space is $\cmplxm$. If there are fewer than $m-$columns, or in general if there are fewer than $m$ linearly independent column vectors, than the host space is incomplete. When the host space is incomplete there will be a complementary space.


%%%%%%
\subsection{Domains}
Here is the tricky part. Both the domain and the codomain are domains. When we speak of the domain matrices they are, in general, the matrices $\X{}$ and $\Y{}$. Individually, $\X{}$ is a set of basis vectors for the domain. Or the domain matrix for short. The matrix $\Y{}$ is the codomain matrix. The old saying ``the usage is clear from context'' ignores the fact that many readers are trying to develop the intuition of context.

%%%%%%
\subsection{The singular values}
We will see that the singular values derive from the eigenvalues of the product matrices $\prdm{*}$ and $\prdmm{*}$. The eigenvalues are the roots of the characteristic polynomials and may include zero and may 
in no particular order. When we speak of the eigenvalues of the product matrix, we mean a list ordered by decreasing size without zero values. While it is perfectly valid to write an eigenvalue spectrum as this
\begin{equation}
  \lambda\paren{\prdm{*}} = \lst{9,2,0,2},
\end{equation}
in this work we consider this spectrum to be
\begin{equation}
  \lambda\paren{\prdm{*}} = \lst{9,2,2}.
\end{equation}

%%%%%%
\subsection{The $\sig{}$ matrix}
\begin{enumerate}
\item $\ess{}$, the full rank diagonal matrix of singular values
\item $\Xi$, the sabot matrix, a matrix of zeros of the same dimension as the target matrix.
\item $\sig{}$, the sabot matrix with $\ess{}$ embedded starting at the $\lst{\text{r},\text{c}}=\lst{1,1}$
\end{enumerate}

\subsection{The SVD of a matrix}
The gymnastics of articles. The domain matrices are not unique. Therefore the \svdl\ is not unique. The singular values are always unique.
be ordered in any fashion

\endinput