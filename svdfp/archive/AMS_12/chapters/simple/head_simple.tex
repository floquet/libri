\chapter[SVD without eigenvalues]{Special case: SVD without eigenvalues}
\label{chap:simple}
Consider the quadratic equation
\begin{equation}
  ax^{2}+bx+c = 0.
\end{equation}
We know that the two roots are given by
\begin{equation}
  x_{\pm} = \frac{1}{2a}\paren{-b\pm\sqrt{b^{2}-4ac}}.
\end{equation}
However, when we see an equation like
\begin{equation}
  x^{2}-x-6 = 0
\end{equation}
we don't compute the general solution; we factor the equation into
\begin{equation}
  \paren{x-3}\paren{x+2} = 0.
\end{equation}
We recognize that there are special cases when the roots can be found by a process that is simpler than the general method.

Similarly, while there is a general method for computing the \svdp \ there is also an elementary method for special cases; such a method will be shown here.

The general SVD requires solving for the eigenvalues of one of the product matrices, either $\prdm{*}$ or $\prdmm{*}$. However, in special cases we can completely bypass the eigenvalue problem and construct the SVD using only Gaussian elimination or even inspection.

%%
\section{Simple example}
What are the solutions to the linear system
\begin{equation}
  \begin{split}
    \A{}x &= y \\
    \mat{cc}{1 & 0 \\ 0 & 1 \\ 0 & 0}
    \mat{c}{x_{1} \\ x_{2}} &=
    \mat{c}{y_{1} \\ y_{2} \\ y_{3}}?
  \end{split}
\end{equation}
We can rewrite the left hand side in a more telling form:
\begin{equation}
  \A{}x = x_{1} \mat{c}{1\\0\\0} + x_{2} \mat{c}{0\\1\\0}.
\end{equation}
There is no solution for this system when $y_{3}$ is different from 0. For example, if
\begin{equation}
  y = \mat{c}{0\\0\\1}
\end{equation}
there is no solution. If
\begin{equation}
  y = \mat{c}{a\\b\\0}
\end{equation}
the solution is 
\begin{equation}
  x = \mat{c}{a\\b}.
\end{equation}


\begin{equation}
  \mat{c}{x_{1} \\ x_{2}} = \mat{c}{y_{1} \\ y_{2}} + \alpha \mat{r}{1 \\ -1}
\end{equation}
where $\alpha$ is an arbitrary complex constant.



\section{An example from the method of least squares}

Here we will use the \svdl \ to solve a basic problem in linear algebra using the workhorse method of least squares. This example is constructed to be simple enough to solve with pencil and paper to allow the reader to follow along and duplicate building the solution. This will reinforce the presentation and demonstrate the simplicity of the process.

The sample problem is this: find all of the least squares solutions to the linear system
%%
\begin{equation}
\begin{array}{cccc}
    \A{} & \xi & = & \phi\\
    \Aexample &
    \mat{c}{\xi\\ \eta}
    & = &
    \phivector.
\end{array}
\label{eq:simple:problem}
\end{equation}

If you are comfortable with least squares problems you may want to skip ahead to \S \eqref{sec:svd}. Otherwise the following material which builds up the problem should be helpful.

%%
\subsection{The generic problem in least squares}
This class of problems is characterized in the literature as $\ls$. In the context of least squares the terms have the following meanings. The desired solution is the vector $x$ which contains the fit parameters. The inputs are the matrix $\A{}$ which encodes information about the measurement device and the vector $b$ which contains the data, the recurring measurement. Schematically we have the following:
%%
\begin{equation}
  \begin{array}{cccc}
  \text{device} & \text{solution} && \text{measurement}\\
  \A{} & x & = & b\\
  \text{input} & \text{output} && \text{input}\\
  \text{matrix} & \text{vector} && \text{vector}\\
  \end{array}
\end{equation}
The matrix $\A{}$ is called the device matrix\index{device matrix} or the system matrix\index{system matrix}. The vector $b$ is called the data vector\index{data vector} or the measurement vector\index{measurement vector}. The output vector $x$ is called the solution vector\index{solution vector} or the parameter vector\index{parameter vector}.

If the device matrix is not singular it can be inverted and the solution vector for the system is
\begin{equation}
  x=\A{-1}b.
\end{equation}
This solution is unique. However, the more interesting cases, and the case studied here, do not have a single solution and the device matrix cannot be inverted. But these cases can be solved using the method of least squares.

%%
\subsection{The example}
Consider a probe which measures over a plane a scalar value $\phi$, such as the electrostatic field. At every position the scalar field has a unique value. However when measurements are repeated at a location the measurement usually varies from earlier values due to instrumental uncertainties. The measurement sequence\index{measurement sequence} consists of recording the probe position and the value of the measurement. 

%%
\subsubsection{Problem specification}
Table \eqref{tab:2:taxon} details the specification of this problem. Such a table is a good way to start least squares problems. Failure to specify the problem often leads to failure to find the solution.

\begin{table}[h]
\begin{center}
\begin{tabular}{llc}
  type & quantity & variables \\ \hline
  model & $f(x,y) = \phi$ & $\xi x + \eta y = \phi$ \\[3pt]
  merit function & $M(\xi)$ & $\paren{\A{}\xi - \phi}^{\mathrm{T}}\cdot\paren{\A{}\xi - \phi}$ \\[3pt]
  fixed input & measurement locations &
  $\mat{c}
  {
  x_{k}\\[3pt]
  y_{k}
  }$ \\[3pt]
  recurring input & measurements & $\phi_{k}$ \\[2pt]
  output & fit parameters &
  $\mat{c}
  {
  \xi\\
  \eta
  }$ \\
\end{tabular}
\end{center}
\caption[Bilinear fit parameters]{Bilinear fit parameters. Start the solution by specifying the problem. Identify the model and the merit function. Distinguish between inputs and outputs.}
\label{tab:2:taxon}
\end{table}%

%%
\subsubsection{Solution strategies}
The matrix form for this model is
\begin{equation}
  \A{}\xi = \phi.
  \label{eq:2:ls}
\end{equation}
The $\by{n}{d}$ system  matrix $\A{}$ encodes information about the device because it relates where measurements were taken over a domain of $d$ dimensions. The $n-$vector $\phi$ contains the $n$ measurement values. The $d-$vector $\xi$ is the parameter vector\index{parameter vector} or the solution vector\index{solution vector}. Finding the solution vector is an inverse problem because we must invert equation \eqref{eq:2:ls} to solve for the parameter vector. This solution represents a point in an abstract parameter space.

To solve the problem one could minimize the summation form of the merit function 
\begin{equation}
M(\xi)=\sum_{k=1}^{n}{\paren{\xi x_{k} + \eta y_{k} - \phi_{k}}}^{2}.
\end{equation}
This will lead to the \textit{normal equations}\index{normal equations}:
\begin{equation}
  \A{T}\A{}\xi = \A{T} \phi.
  \label{eq:2:normal}
\end{equation}
The problem with these normal equations is that they degrade the quality of solution. If the condition number of equation \eqref{eq:2:ls} is $\kappa$, the condition number of the normal equations of equation \eqref{eq:2:normal} is $\kappa^{2}$. In ill-posed problems this can be a fatal flaw. So why use the normal equations? Because often times the data vector is not in the range of $\A{}$, but the vector $\A{T}\phi$ will be in the range of $\A{T}\A{}$. (See problem 7.)

Note however that if the system is large, the product matrix will be much smaller. The dimensions are
\begin{equation}
  \A{} \in \real{\by{n}{d}}, \quad \prdm{*} \in \real{\by{d}{d}}.
\end{equation}
If there were 1000 measurements the system matrix would be $\by{1000}{2}$ while the product matrix would be just $\by{2}{2}$ and much easier to work with. Solving the normal equations is a useful technique, but not of interest here.

The difference between the measurement and the prediction is called the residual error\index{residual error} and is expressed in this form
\begin{equation}
  \begin{split}
    \epsilon_{k} &= (\xi x_{1} + \eta y_{1}) - \phi_{k},\\
    \text{error} &= \text{prediction} - \text{measurement}.
  \end{split}
\end{equation}
The goal is to minimize the quadratic sum of these errors
\begin{equation}
  \epsilon_{k}^{\mathrm{T}}\,\epsilon_{k} = \paren{\A{}\xi - \phi}^{\mathrm{T}}\cdot\paren{\A{}\xi - \phi} = \normt{\A{}\xi - \phi}^{2}.
\end{equation}
The ``least'' in least squares implies minimization; ``squares'' implies the $2-$norm $L_{2}$.

%%
\subsubsection{Data}
The measurement sequence is shown in table \eqref{tab:2:data}.
\begin{table}[h]
\begin{center}
\begin{tabular}{ccc}
    & location $\paren{\A{}}$ & measurement $\paren{\phi}$ \\ \hline
  1 & $\mat{c}{x_{1}\\y_{1}}=\mat{r}{1\\-1}$ & $\phi_{1}=2$ \\[13pt]
  2 & $\mat{c}{x_{2}\\y_{2}}=\mat{r}{-1\\1}$ & $\phi_{2}=1$ \\[13pt]
  3 & $\mat{c}{x_{3}\\y_{3}}=\mat{r}{1\\-1}$ & $\phi_{3}=0$ \\[13pt]
\end{tabular}
\end{center}
\caption[The measurement sequence]{The measurement sequence. These values form the system matrix $\A{}$ and the data vector $\phi$. The duplicate measurement locations will completely change the nature of the problem and the solution.}
\label{tab:2:data}
\end{table}%
The linear system\index{linear system} is composed of three equations, one for each measurement:
\begin{equation}
  \begin{split}
    \xi x_{1} + \eta y_{1} &= \phi_{1}, \\
    \xi x_{2} + \eta y_{2} &= \phi_{2}, \\
    \xi x_{3} + \eta y_{3} &= \phi_{3}.
  \end{split}
\end{equation}
Usually, and in this case, there is no value for the parameters $\xi$ and $\eta$ which solve all three equations. 

The system matrix will be the target matrix and it describes the device because it contains measurement locations:
\begin{equation}
  \A{} = \Aexample.
\label{eq:simple:IamA}
\end{equation}

The data vector then contains the measurements:
\begin{equation}
  \phi = \mat{c}{\phi_{1} \\\phi_{2} \\ \phi_{3}} = \mat{c}{2 \\ 1 \\ 0}.
  \label{eq:phi}
\end{equation}

These measurements complete the linear system
\begin{equation}
\begin{array}{cccc}
    \mat{cc}
    {
    x_{1} & y_{1} \\
    x_{2} & y_{2} \\
    x_{3} & y_{3} \\
    }
    &\mat{c}{\xi\\ \eta}
    &=
    &\mat{c}
    {
    \phi_{1} \\
    \phi_{2} \\
    \phi_{3}
    }, \\[16pt]
    \Aexample
    &\mat{c}{\xi\\ \eta}
    & =
    &\phivector.
\end{array}
\end{equation}

Going back to the concept of a matrix as a set of vectors, we can write the matrix equation in terms of column vectors:
\begin{equation}
  \A{} \xi = \xi\, \textbf{x} + \eta\, \textbf{y} = \xi \mat{r}{1\\-1\\1} + \eta \mat{r}{1\\-1\\1} = \mat{c}{2 \\ 1 \\ 0}.
\end{equation}
It should be apparent that the data vector is not in the range of the target matrix because there is no way that we can combine these two column vectors to produce the data vector.
%%
\subsection{The solution}
We have covered how we generated the matrix equation
\begin{equation*}
\begin{array}{cccc}
    \A{} & \xi &=& \phi\\
    \Aexample
    & \mat{c}{\xi\\ \eta}
    & =
    &\phivector
    \label{eq:2:problem}
\end{array}
\end{equation*}
and motivated the strategy of solving this form instead of the normal equations. Now we explore the solution using the SVD.

%%
\subsubsection{Diagnosis}
Step back and inspect problem \eqref{eq:2:problem} from a linear algebra perspective. Is the system matrix $\A{}$ invertible? Is it well conditioned? How many solutions do we expect, one or an infinite number\footnote{Because $\A{}$ is not square it cannot be inverted. Because it cannot be inverted the condition number is not bounded; this is a poorly conditioned matrix. Because of the rank deficiency we expect an infinite number of solutions: a particular solution plus null space vectors scaled by an arbitrary scalars.}?

Examine the \textit{image} of the $\A{}$ matrix:
\begin{equation}
  \Aexample \ximatrix = \paren{\xi-\eta}\mat{r}{1\\-1\\1}.
\end{equation}
Clearly there is no number $\xi-\eta$ that solves the problem:
\begin{equation}
 \paren{\xi-\eta}\mat{r}{1\\-1\\1} = \phivector.
\end{equation}

This conundrum reminds us that there is only one fundamental column in $\A{}$. The second column is the negative of the first column:
\begin{equation}
  \A{} =
\left(
\begin{array}{r|r}
  c_{1} & -c_{1}    
\end{array}
\right).
\end{equation}
Since there is only one independent column, the matrix rank $\rho = 1$. Since the rank is less than both the number of rows and the number of columns
\begin{equation}
  \rho < m, \quad \rho < n,
\end{equation}
the problem is singular. Therefore the condition number $\kappa = \infty$ and there are an infinite number of solutions.

%%
\subsubsection{Solving singular systems}
Solutions of singular systems like \eqref{eq:2:ls} have the following form:
\begin{equation}
  \xi = \xi_{p} + \alpha\, \xi_{h}.
  \label{eq:fact}
\end{equation}
Here $\xi_{p}$ represents a \index{particular solution}\textit{particular solution}, $\xi_{h}$ a \index{homogeneous solution}\textit{homogeneous solution}, and $\alpha$ is an arbitrary scalar. 

The particular solution is a point solution to the least squares problem. Since the data vector is not in the image of the system matrix we know that the solution vector $\A{}\xi_{p}$ does not reach the data vector $\phi$, but it comes as close as possible using the $2-$norm to measure proximity.

The classic diagram that illustrates this point is shown below in figure \eqref{fig:simple:classic}. Simplify the range of the system matrix as a plane. The data vector $\phi$ projects from the origin to a point outside of the plane. This shows that $b$ is not in the range of $\A{}$, that is
$$
\phi\notin \rng{\A{}}.
$$
The solution that we compute is $\A{+}\xi=\xi_{p}$ and this is the point in the range closest to the data. Expressed another way, the solution vector is the orthogonal projection of the data vector onto the range of $\A{}$.
\begin{figure}[htbp] %  figure placement: here, top, bottom, or page
   \centering
   \includegraphics[]{pdf/simple/projection} 
   \caption[The geometry of the least squares solution]{The geometry of the least squares solution. The data vector $\phi$ is not in the image of the target matrix. The plane represents $\rnga{}$, the range of $\A{}$. Orthogonal to the range is $\nll{A}$, the null space of $\A{}$. The least squares solution is the orthogonal projection of the data vector into the range. This point $\xi_{p}$ is the particular solution.}
   \label{fig:simple:classic}
\end{figure}

The homogeneous solution satisfies the equation
\begin{equation}
  \A{}\xi_{h} = \zero.
\end{equation}
In this problem, the homogeneous solution is obvious:
\begin{equation}
  \Aexample \mat{r}{1\\1} = \mat{c}{0\\0\\0}.
\end{equation}
As such,
\begin{equation}
  \xi_{h} = \mat{r}{1\\1}.
  \label{eq:2:xih}
\end{equation}

%%
\subsubsection{A special solution via pseudoinverse}
However, the object of desire is the particular solution. This solution is computed according to the prescription
\begin{equation}
  \A{+}\xi = \xi_{p}.
\end{equation}
The matrix $\A{+}$ is called the pseudoinverse\index{pseudoinverse} or the generalized matrix inverse\index{generalized matrix inverse}. While we don't know how to construct a pseudoinverse, we will see how the \svdl \ can be exploited to find this special solution.

\endinput
\section[The column vectors]{The column vectors of the domain matrices}
In this chapter we see the singular values\index{singular values!scale factors, as} are called scale factors. Consider the notion of distance. Every point in the domain maps to a point in the codomain. How does the distance between two points in the domain compare to the distance between these two points in the codomain?
\begin{equation}
  \A{}\X{}_{*,c} = \sigma_{c} \Y{}_{*,c}, \quad c = 1, \rho.
\end{equation}

\endinput
\section{The pseudoinverse}

When can a matrix be inverted? When it is square, that is when the number of rows equals the number of columns. Also, one of these equivalent statements must be satisfied:
\begin{enumerate}
\item the determinant is nonzero;
\item the column vectors are linearly independent and form a complete spanning set;
\item none of the eigenvalues are zero.
\end{enumerate}
In this case, linear systems like $\ls$ have a single solution
\begin{equation}
  x = \A{-1}b.
\end{equation}

The concept of a matrix inverse can be generalized to all matrices. This means rectangular matrices, matrices with zero eigenvalues, matrices with linearly dependent columns, and matrices where the column vectors are an incomplete set.

This topic deserves a fuller treatment after some theoretical formalities. But for now an excellent perspective to use is the method of least squares. Clearly a linear system may not have a point solution, but it will always have a  least squares solution $x_{ls}$. What matrix $\A{+}$ produces the least squares solution
\begin{equation}
  x_{ls}=\A{+}b?
\end{equation}

Again we take an intuitive approach to this topic, leaving formalities for later. Using the SVD as our starting point we nominate the inverse of the decomposition 
\begin{equation}
  \begin{split}
    \A{+}&=\paren{\svd{T}}^{-1}=\paren{\X{*}}^{-1}\paren{\sig{}}^{-1}\paren{\Y{}}^{-1} \\
     &= \mpgi{*}
  \end{split}
\end{equation}
One of the great joys of unitary matrices is that they are trivial to invert.

The fly in the ointment is the $\sig{}$ matrix. For rank deficient target matrices such as this case there is no standard matrix inverse. To invert this matrix form the transpose. This insures that we can pre- or post multiply $\sig{}$ with $\sig{(+)}$. Next replace each singular value $\sigma_{k}$ with its reciprocal $\sigma_{k}^{-1}$.

In the example problem these matrices are
\begin{equation}
  \sig{}= \Sigmaexampleb, \qquad 
  \sig{(+)}= \mat{c|cc}
  {
  \ssix & 0 & 0\\\hline
  0 & 0 & 0
  }.
\end{equation}

The pseudoinverse for this example is
\begin{equation}
  \begin{split}
  \mpgiax{T} &= \Xshade \mat{c|cc}
  {
  \ssix & 0 & 0\\\hline
  0 & 0 & 0
  } \Ytshade \\
  & =
  \frac{1}{6}
  \mat{rrr}
  {
   1 & -1 &  1 \\
  -1 &  1 & -1
  }.
  \end{split}
  \label{eq:simple:mppdecomp}
\end{equation}

%%
\subsection{Particular solution via pseudoinverse}
\label{sec:pi}
The pseudoinverse is not applied in the same fashion as the standard inverse. For the standard inverse, the application looks like this:
\begin{equation}
  \begin{split}
    \lsa\\
    \A{-1}\paren{\A{}x} & = \A{-1}b\\
    x&= \A{-1}b.
  \end{split}
\end{equation}

When the usual matrix inverse exists we have
\begin{equation}
  \A{-1} = \A{+}.
\end{equation}
However the game is different when the usual inverse does not exists. The issue with the pseudoinverse is this instance is that in general
\begin{equation}
  \A{+}\A{} \ne \I{n}.
\end{equation}
This restricts the solution process
\begin{equation}
  \begin{split}
    \lsa,\\
    \A{+}\paren{\A{}x} & = \A{+}b.\\
  \end{split}
  \label{eq:solution:general}
\end{equation}
%%
Later on in \S\eqref{sec:orthproj} we will see that the operator $\A{+}\A{}$ is a projection operator and discuss its geometric interpretation. For now rest assured that the solution we are seeking is given by this relation
\begin{equation}
  x = \A{+}b.
\end{equation}

This is a particular solution. For the example matrix  we find that
\begin{equation}
  \xi_{p} = \A{+}\phi = 
  \frac{1}{6}
  \mat{rrr}
  {
   1 & -1 &  1 \\
  -1 &  1 & -1
  } 
  \mat{c}
  {
  2\\1\\0
  }
  =
  \frac{1}{6}
  \mat{r}
  {1\\-1},
  \label{eq:soln:particular}
\end{equation}
the particular solution of equation \eqref{eq:fact}.

%%
\subsection{Complete solution}
\label{sec:solution:complete}
The homogeneous solution represents the null space vectors. There will be a free variable for each null space vector. In this case there is only a single null vector
\begin{equation}
  \X{}_{*,2} \propto \mat{r}{1\\1}.
\end{equation}
The homogeneous solution\footnote{The difference between $\alpha'$ and $\alpha$ is the normalization factor on the column vector.} is then
\begin{equation}
  \xi_{h} = \alpha' \X{}_{*,2} = \alpha \mat{c}{1\\1},\quad \alpha \in \cmplx{}.
\end{equation}
This is a null solution
\begin{equation}
  \A{} \xi_{h} = 0.
\end{equation}

Using the prescription for the general solution established in equation \eqref{eq:fact} we can state the least squares solutions for equation \eqref{eq:2:problem} are given by
\begin{equation}
\boxed{
  \begin{array}{rccccc}
    \xi &=& \xi_{p} &\oplus& \xi_{h},\\[7pt]
      &=& \A{+}\phi &\oplus& \alpha \X{}_{*,2}, & \alpha\in\cmplx{}\\[7pt]
      &=& \frac{1}{6}\mat{r}{1\\-1} &\oplus& \alpha \mat{c}{1\\1}\\[17pt]
      &=& \textit{point} &\oplus& \textit{line}
  \end{array}
}
\label{eq:simple:fullsoln}
\end{equation}

The solution is not a point, it is a direct sum of a point and line; the line implies a continuum solution which implies a null space. This implies that the merit function does not have a minimum point. This implies that the solutions are indistinguishable in the domain. Any point on the line through $\xi_{p}$ and parallel to $\xi=\eta$ 
\begin{equation}
  \eta = \xi - \frac{1}{3}
\end{equation}
produces the same value for the merit function. This is the red line in figure \eqref{fig:simple:solution}. The invariance is more apparent in the matrix formulation.

Since the vector $(1,1)^{\mathrm{T}}$ defines the null space we have for any vector $\xi$
\begin{equation}
  \begin{split}
      \A{}\paren{\xi_{p}+\xi_{h}} &= \A{}\paren{\xi_{p}+\alpha\mat{c}{1\\1}}\\ 
      &= \A{}\xi_{p}+\A{}\paren{\alpha\mat{c}{1\\1}}\\
      &= \A{}\xi_{p} + \mat{c}{0\\0\\0} \\
      &= \A{}\xi_{p}.
  \end{split}
\end{equation}

All of these solutions map to the same point $p$ in the range\index{range} of the target matrix:
\begin{equation}
  \A{}\xi_{p} = \frac{1}{3}\mat{r}{1\\-1\\1} = p.
  \label{eq:simple:map}
\end{equation}

What did the least squares solution minimize? It minimized the distance between the data point $\phi$ and the solution point $p$. Another way to look at the problem is the find the point in the range of the column vectors closest to the data. Since the data point is not in the range of the column vectors there is no direct solution and we need to find the least squares solution. As we will see later, another way to find the solution is to ask
\begin{equation}
  \min_{\alpha\in\cmplx{}} \normt{\phi-\alpha\mat{r}{1\\-1\\1}}^{2}=\min_{\alpha\in\cmplx{}} \paren{(2-\alpha)^{2}+(1+\alpha)^{2}+(-\alpha)^{2}}.
\end{equation}
The minimization is in the codomain while the solution is in the domain. This means that after the minimum vector is found in the codomain it must be mapped to the solution vector in the domain.

%%
\subsection{The solution error}
We have a solution. In fact, we have the best solution as measured with the $2-$norm. But is the best solution a good solution? The first step in addressing this question is computing the residual error vector
\begin{equation}
  r = \A{}x-b
\end{equation}
which for this example is the following
\begin{equation}
  r = \rthree \mat{r}{-5,-4,1}
\end{equation}
\begin{equation}
  \begin{split}
     r &= \A{}x-b, \\
       &= \rthree \mat{r}{1\\-1\\1} - \phivector,\\
       &= \rthree \mat{r}{-5,-4,1}
  \end{split}
  \label{eq:soln:error}
\end{equation}
which has a magnitude given by this
\begin{equation}
  r^{T}r = r\cdot r = \frac{14}{3}.
\end{equation}

We computing the error vector explicitly to demonstrate that the error is measured in the codmain, that it, it is in the data space, not the fit parameter space. It is difficult to say much more about the error without knowing the context of the measurement. For example, we might want to look at the magnitude of the error or the relative error or the absolute error. But one point is clear: don't assume that the best fit is a good fit.

%%
\section{The geometry of the solution}
The geometry of the solution has a certain grace about it. The solution resides in the vector space induced by the row vectors, here $\real{2}$. This is the space of the solution parameters, in this case $\xi$ and $\eta$. For another type of regression these parameters could represent the slope and the intercept of a solution curve.

The minimization problem is solved in the vector space induced by the column vectors, here $\real{3}$. The closest point to the data is the point $p$ and this point is connected to the solution point $\xi_{p}$ by the target matrix $\A{}$. The solution to the least squares problem is not the $3-$vector $p$, it is the $2-$vector $\xi_{p}$. The correspondence is given by equation \eqref{eq:simple:map}.

The full solution to the least squares problem is given in equation \eqref{eq:simple:fullsoln} and contains the particular solution given by the pseudoinverse and the homogeneous solution given by the null space. The solution can also be seen in terms of the the decomposition of $\real{2}$ below showing the range of the row vectors - the image space - and the null space.
\begin{equation}
    \xi = \frac{1}{6}\underbrace{\mat{r}{1\\-1}}_{\text{image space}} + \alpha \underbrace{\mat{c}{1\\1}}_{\text{null space}}.
\end{equation}

The \svdl \ has resolved the both the domain and the codomain into basis vectors for the range and the null space. These basis matrices are these:
\begin{equation}
  \begin{split}
  \B{}_{\real{2}} &=\X{}= \Xshade,\\[5pt]
  \B{}_{\real{3}} &=\Y{}=\Yshade.
  \end{split}
\end{equation}

Then the decomposition is used to build the pseudoinverse and the homogeneous solution. The seminal point from this example is that the \textit{pseudoinverse identifies the particular solution to the least squares problem.} 

\begin{figure}[htbp] %  figure placement: here, top, bottom, or page
   \centering
   \includegraphics[ ]{pdf/simple/solution02} 
   \caption[The domain resolved into image space and null space vectors]{The domain resolved into image space and null space vectors. These vectors are the an orthogonal basis for $\real{2}$. The particular solution is the point $\xi_{p}$. The homogeneous solution is shown by the red line; this is the line of equivalent solutions. The thick line represents the range of the row vectors, the red line represents the orthogonal null space. This vector space is also the parameter space where the solution resides. It is not a physical space like the measurement space of the codomain.}
   \label{fig:simple:solution}
\end{figure}

%%
\section{Uniqueness of the SVD}
We would not expect the \svdl \ to be unique. After all, the null space vectors can be reordered. The key fact is that when we resolve the domain and codomain there are sign ambiguities. For example, these two spans are equivalent:
\begin{equation}
  \spn \lst{\mat{r}{1\\-1\\1}} \equiv  \spn \lst{\mat{r}{-1\\1\\-1}}.
\end{equation}
So we should not expect the decomposition to be unique. For example in this case we could also write that
\begin{equation}
  \begin{split}
    \svda{T}\\
     & = 
    \mat{r>{\columncolor{ltgray}}r >{\columncolor{ltgray}}r}
    {-\sthree & -\stwo & \ssix\\
      \sthree &    0   & \frac{2}{\sqrt{6}}\\
     -\sthree &  \stwo & \ssix}
    \Sigmaexampleb
    \stwo \mat{rr}
    {-1 & 1\\
    \rowcolor{ltgray}
      1 & 1}.
  \end{split}
\end{equation}
Notice the change of sign of the image space vectors and the reordering of the null space vectors.

However while the basis vectors are malleable, the singular values are not. The $\sig{}$ matrix is \textit{invariant}\index{invariance!$\sig{}$ matrix}. 

\endinput
\section{Summary}
When the pencil shavings and eraser trailings are cleared away a simple process remains. The $\X{}$ matrix is an orthonormal basis for the the vector space induced by the row vectors of $\A{}$. The $\Y{}$ matrix is an orthonormal basis for the the vector space induced by the column vectors of $\A{}$. The $\sig{}$ matrix in between them has these critical properties:
\begin{itemize}
\item this matrix is unique;
\item this matrix has the same shape as the target matrix;
\item the diagonal singular values matrix $\ess{}$ is embedded and 
\subitem the singular values are ordered;
\subitem the singular values are positive;
\subitem the singular values populate the diagonal.
\end{itemize}
 

To conjure a mental image of the elementary decomposition process table \eqref{tab:simple:summary} shows a schematic interpretation of the the process we have just completed. The column vectors of the target matrix $\A{T}$ and their perpendicular complements are used to build the domain matrix $X{}$. The column vectors of matrix $\A{}$ and their perpendicular complements are used to build the codomain matrix $\Y{}$. In this example, each image space had dimension one which allowed us to harvest the first column for the span of the induced vector spaces.

The top line of the figure represents the calculations used to resolve the domain matrix as shown in \S\eqref{domain}. 
The bottom line represents the calculation of the codomain matrix as detailed in \S\eqref{codomain}. The middle section refers to the construction of the $\sig{}$ matrix in \S\eqref{scale}.

\begin{landscape}
\thispagestyle{empty}

\begin{table}[htdp]
\begin{center}
\begin{tabular}{ccccccccc}
 &&image space&&null space\\
 $\A{T}$ & $\to$ & row space & $\to$ & row space$^{\perp}$ & $\to$ & $\X{}$ \\\hline
 $\Atexample$�&&�$\stwo\mat{r}{1\\-1}$�&&�$\stwo\mat{c}{1\\1}$�&&�$\Xshade$ \\
  & \\
  &&&&&&$\downarrow$ \\
  &&&&&&$\A{}x_{1}=\sigma_{1}y_{1}$ & $\to$ & $\sig{} = \Sigmaexampleb $\\
  &&&&&&$\uparrow$ \\
  & \\
 &&image space&&null space\\
 $\A{}$ & $\to$ & column space & $\to$ & column space$^{\perp}$ & $\to$ & $\Y{}$ \\\hline
 $\Aexample$�&&�$\sthree\mat{r}{1\\-1\\1}$�&&�$\lst{\stwo\mat{c}{0\\1\\1},�\ssix\mat{r}{2\\1\\-1}�}$�&&�$\Yshade$ \\
\end{tabular}
\end{center}
\label{tab:simple:summary}
\caption{This special case for the \svdl \ involves using $\A{}$ to construct the codomain matrix $\Y{}$, using $\A{T}$ to construct the domain matrix $\X{}$ and then using these domain matrices to construct $\sig{}$. To construct the codomain and domain spaces extract the first column vector from $\A{}$ and $\A{T}$ respectively, then construct perpendicular vectors for the null space.}
\end{table}%

\end{landscape}

\endinput
\section{Exercises}
\begin{enumerate}
\item Consider the SVD given for $\Arrr{2}{2}{2}$:
\begin{equation*}
  \svdax{T} = 
  \mat{c|c}{y_{11} & y_{12} \\ y_{21} & y_{22}}
  \mat{cc}{\sigma_{1} & 0 \\ 0 & \sigma_{1}}
  \mat{cc}{x_{11} & x_{12} \\\hline x_{21} & x_{22}}.
\end{equation*}
Show by direct computation of the product that
\begin{equation*}
\begin{split}
  \A{} 
  &= \mat{cc}{
  \sigma_{1} x_{11} y_{11} + \sigma_{2} x_{12} y_{12} & \sigma_{1} x_{21} y_{11} + \sigma_{2} x_{22} y_{12} \\
  \sigma_{1} x_{11} y_{12} + \sigma_{2} x_{12} y_{22} & \sigma_{1} x_{21} y_{12} + \sigma_{2} x_{22} y_{22} } \\
  &= \sigma_{1} \mat{cc}{
  y_{11} \mat{cc}{x_{11} & x_{21}} \\
  y_{12} \mat{cc}{x_{11} & x_{21}}}
  + \sigma_{2} \mat{cc}{
  y_{21} \mat{cc}{x_{12} & x_{22}} \\
  y_{22} \mat{cc}{x_{12} & x_{22}}} \\
  &= \sigma_{1} y_{1}x_{1}^{T} + \sigma_{2} y_{2}x_{2}^{T}.
\end{split}
\end{equation*}
\item
\item
\end{enumerate}


\endinput

\endinput