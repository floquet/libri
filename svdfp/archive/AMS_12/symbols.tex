\chapter*{Literature survey}

By design, this book relies on prior coursework and therefore other texts. Readers who need to work on the foundations or who wish to learn more will want to refer to other works.

%%%
\subsection*{Introductory linear algebra texts}
New and highly priced texts appear year after year. They are thicker and they are heavier and the graphics are more impressive. They may include DVDs and have elaborate web sites associated with them. Survey these by using a bookseller web site and search by bestselling order.

Here we include less audacious and more economical books
\begin{quote}
\textit{Linear Algebra}\\
Marcus and Minc\\
McGraw-Hill, 1985\\
ISBN 0-971-12345-1
\end{quote}

Here we list a few inexpensive books of considerable benefit. 

%%%%
\subsection*{Intermediate texts}
There is one text which stands above the crowd thanks to the masterful lectures which complement it.
\begin{quote}
textit{Linear Algebra}\\
Gilbert Strang\\
McGraw-Hill, 1985\\
ISBN 0-971-12345-1
\end{quote}
The lectures can be found at the MIT OCW web site:
\begin{verbatim} 
MIT/OCW
\end{verbatim}

%%%
\subsection*{Advanced texts}
The subject of the \svdl \ is covered in detail with considerable skill in
\begin{quote}
textit{Matrix Analysis}\\
Roger Horn\\
McGraw-Hill, 1985\\
ISBN 0-971-12345-1
\end{quote}

%%%
\subsection*{Journal articles}
Certainly a pedestrian text should not expect readers to be conversant with material in journals covering theory and applications. We restrict ourselves to a few noteworthy articles which would be a natural progression.

history

\endinput