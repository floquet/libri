%-----------------------------------------------------------------------
% Beginning of chapter-template.tex
%-----------------------------------------------------------------------
%
%    This is a template file for monographs prepared with AMS author
%    packages, for use with AMS-LaTeX.  Separate chapters should be
%    included at the appropriate position.
%
%    Templates for various common text, math and figure elements are
%    given following the \end{document} line.
%
%    Start by copying this file to <filename>.tex, using a distinctive
%    name suitable for your book in place of <filename>.  This will
%    be the driver file for your book.
%
%%%%%%%%%%%%%%%%%%%%%%%%%%%%%%%%%%%%%%%%%%%%%%%%%%%%%%%%%%%%%%%%%%%%%%%%

%    Replace amsbook by the documentclass code for the monograph series.
\documentclass{amsbook}

%    If you need symbols beyond the basic set, uncomment this command.
%\usepackage{amsmath}
\usepackage{amsmath,amssymb}

%    If your book includes graphics, or such features as rotation or
%    scaling, uncomment this command.
\usepackage{graphicx}
\usepackage{colortbl}

%    If the book includes commutative diagrams:
%\usepackage[cmtip,all]{xy}

%    If you are using the author-year citation style:
%\usepackage{natbib}

%    Include other referenced packages here.
\usepackage{color}

\newcommand{\pathname}     {../common/}
\newcommand{\fullpath}     {\pathname}
\newcommand{\pathchapter}  {../chapters/}
\newcommand{\pathappendix} {../appendices/}
\newcommand{\pathgraphics} {../graphics/}
\newcommand{\pathequations}{../equations/}
\newcommand{\pathfigures}  {../figures/}
\newcommand{\pathtables}   {../tables/}
\newcommand{\paththeorems} {../theorems/}

\input{\pathname declarations.tex}

%    For use when working on individual chapters
%\includeonly{}

%    For a single index; for multiple indexes, see the manual
%    "Instructions for preparation of papers and monographs:
%    AMS-LaTeX" (instr-l.pdf in the AMS-LaTeX distribution).
%    Do not \usepackage{makeidx}; all facilities are contained
%    within the AMS document classes.
\makeindex

\begin{document}

\frontmatter

\title{Excursions in linear least squares}

%    Remove any unused author tags.

%    author one information
\author{Daniel Topa}
\address{}
\curraddr{}
\email{dantopa@gmail.com}
\thanks{}

%    If any version of the Mathematics Subject Classification other
%    than the 2010 edition appears, then you have an old version
%    of the AMS-LaTeX collection and need to upgrade.  Download from
%    http://www.ams.org/tex/amslatex.html .
\subjclass[2010]{Primary }

\keywords{least squares, \Ltwo, \ltwo}

\date{}

\begin{abstract}
\end{abstract}

\maketitle

%    Dedication.  If the dedication is longer than a line or two,
%    remove the centering instructions and the line break.
%\cleardoublepage
%\thispagestyle{empty}
%    If this book uses the documentclass stml-l or mmono-s, change
%    13.5pc to 10.5pc.
%\vspace*{13.5pc}
%\begin{center}
%  Dedication text (use \\[2pt] for line break if necessary)
%\end{center}
%\cleardoublepage

%    Change page number to 7 if a dedication is present.
\setcounter{page}{5}

\tableofcontents

%    Include unnumbered chapters (preface, acknowledgments, etc.) here.
\include{}

\mainmatter

%    Include main chapters here.
\part{\label{part:first}Rudiments}  %  *  *  *  *  *  *  *  *  *  *  *  *
%  \include{../chapters/formulation/"chap formulation"}
\input{\pathchapter formulation/"chap formulation"}
\input{\pathchapter archetype/"chap archetype"}

\part{\label{part:zonal}Zonal fits}  %  *  *  *  *  *  *  *  *  *  *  *  *
\input{\pathchapter matrices/"chap matrices"}
%\input{\pathchapter matrices/"chap gradients"}

\part{\label{part:zonal}Modal fits}  %  *  *  *  *  *  *  *  *  *  *  *  *
\input{\pathchapter polynomials/"chap polynomials"}
\input{\pathchapter orthogonals/"chap orthogonals"}

%    Include main chapters here.
\part{\label{part:nonlinear}Nonlinear problems}  %  *  *  *  *  *  *  *  *  *  *  *  *
\chapter{Strays}

Blindly applying linear tools to nonlinear problems presents many paths to perdition. Hope, no matter how fervent, cannot remedy mathematical maladies.

We stress the definition of the least squares problem in \eqref{eqn:definition} as the first indication that something is amiss. We stress visualization methods to help reveal the status of a calculation.
\begin{enumerate}
\item finding reasonable approximations which nudge the problem into linearity;
\item iterating the solution to a linear problem to improve a nonlinear problem; 
\item separating a problem into linear and nonlinear components.
\end{enumerate}

%\part{\label{part:first}Scalar fields}  %  *  *  *  *  *  *  *  *  *  *  *  *
%
%\part{\label{part:first}Vector fields}  %  *  *  *  *  *  *  *  *  *  *  *  *
%
%\part{\label{part:first}Tensor fields}  %  *  *  *  *  *  *  *  *  *  *  *  *
%
%
%\part{\label{part:first}Zonal fits}  %  *  *  *  *  *  *  *  *  *  *  *  *
%  
%\part{\label{part:}Stitching}  %  *  *  *  *  *  *  *  *  *  *  *  *

%\part{\label{part:}This is a Part Title Sample}  %  *  *  *  *  *  *  *  *  *  *  *  *

\appendix
%    Include appendix "chapters" here.
\input{\pathappendix exemplars/"app precis"}

\backmatter
%    Bibliographies can be prepared with BibTeX using amsplain,
%    amsalpha, or (for "historical" overviews) natbib style.
\bibliographystyle{amsplain}
\bibliography{}

%    See note above about multiple indexes.
\printindex

\end{document}