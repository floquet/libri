\chapter{Elementary examples}

\section*{Basic matrices}  %  S  S  S  S  S
We analyze the least squares solutions for three examples of the form
$$\axeb .$$
\input{\pathtables elementary/"tab matrices"}  %  <  <  <  <  <  <  <  <  <  <  <  <

%\input{chapters/XXX/"sec YYY"}  %  <  <  <  <  <  <  <  <  <  <  <  <  <  <  <

%% SECTION
\section{Example (a): row and column rank deficient}  %  S  S  S  S  S
A \asvd\ is \ 
% = =  e q u a t i o n
  \begin{equation}
    \begin{split}
      \aaesvd{*} \\
      %
      \matrixa & = 
      \mat{rrc}{
      \bl{ \recip{\sqrt{3}}} & \rd{-\frac{1}{\sqrt{2}}} &  \rd{\frac{1}{\sqrt{6}}} \\[5pt]
      \bl{-\recip{\sqrt{3}}} & \rd{0} \phantom{i}       &  \rd{\frac{2}{\sqrt{6}}} \\[5pt]
      \bl{ \recip{\sqrt{3}}} & \rd{ \frac{1}{\sqrt{2}}} &  \rd{\frac{1}{\sqrt{6}}} \\
      }
      %
      \sigmaa
      %
      \mat{rrc}{
      \bl{\recip{\sqrt{2}}} &  \bl{-\recip{\sqrt{2}}} \\[5pt]
      \rd{\recip{\sqrt{2}}} &  \rd{ \recip{\sqrt{2}}}
      }
    \end{split}
    %\label{eqn:}
  \end{equation}
% = =
The pseudoinverse matrix is
% = =  e q u a t i o n
  \begin{equation}
    %\begin{split}
      \Ap = \mpp{*} = \recip{6} \mat{rrr}{1 & -1 &  1\\-1 & 1 & -1}
    %\end{split}
    %\label{eqn:}
  \end{equation}
% = =
The projection operator
% = =  e q u a t i o n
  \begin{equation}
    %\begin{split}
      \ApA = \recip{2} \mat{rr}{1 & -1 \\ -1 & 1}
    %\end{split}
    %\label{eqn:}
  \end{equation}
% = =
Given an arbitrary vector in the domain $\paren{y\in\cmplx{n=2}}$, the general least squares solutions is
% = =  e q u a t i o n
  \begin{equation}
    \begin{split}
      \xls &= \Ap b + \paren{\I{n} - \ApA} y \\
       &=  \recip{6} \mat{r}{1\\-1} + \half \mat{rr}{3&-1\\-1&3} y
    \end{split}
    %\label{eqn:}
  \end{equation}
% = =
Another way to write this is
% = =  e q u a t i o n
  \begin{equation}
    %\begin{split}
      \xls = \recip{6} \mat{r}{1\\-1} + \alpha \mat{c}{1\\1}, \qquad \alpha\in\cmplx{}
    %\end{split}
    %\label{eqn:}
  \end{equation}
% = =


%% SECTION
\section{Example (b): full column rank}  %  S  S  S  S  S
A \asvd\ is \ 
% = =  e q u a t i o n
  \begin{equation}
    \begin{split}
      \aaesvd{*} \\
      %
      \matrixb & = 
      \mat{crr}{
      \bl{  \recip{\sqrt{30}}} & \bl{  \recip{\sqrt{6}}} &  \rd{\frac{2}{\sqrt{5}}} \\[5pt]
      \bl{\frac{5}{\sqrt{30}}} & \bl{ -\recip{\sqrt{6}}} &  \rd{0} \phantom{i} \\[5pt]
      \bl{\frac{2}{\sqrt{30}}} & \bl{\frac{2}{\sqrt{6}}} &  \rd{ -\recip{\sqrt{5}}} \\
      }
      %
      \sigmab
      %
      \recip{\sqrt{2}}\mat{rc}{
      \bl{1}  & \bl{1} \\
      \bl{-1} & \bl{1}
      }
    \end{split}
    %\label{eqn:}
  \end{equation}
% = =
The pseudoinverse matrix is
% = =  e q u a t i o n
  \begin{equation}
    %\begin{split}
      \Ap = \mpp{*} = \recip{15} \mat{rrr}{-8 & 20 & -16\\12 & 0 & 24}
    %\end{split}
    %\label{eqn:}
  \end{equation}
% = =
The projection operator
% = =  e q u a t i o n
  \begin{equation}
    %\begin{split}
      \ApA = \I{2} = \idtwo
    %\end{split}
    %\label{eqn:}
  \end{equation}
% = =
The general least squares solutions is
% = =  e q u a t i o n
  \begin{equation}
      \xls = \Ap b = \recip{15} \mat{c}{4\\24}
    %\label{eqn:}
  \end{equation}
% = =

%% SECTION
\section{Example (c): full row and column rank}  %  S  S  S  S  S
A \asvd\ is \ 
% = =  e q u a t i o n
  \begin{equation}
    \begin{split}
      \aaesvd{*} \\
      %
      \matrixc & = 
      \mat{cr}{
      \bl{0} & \bl{-1} \\
      \bl{1} & \bl{0}
      }
      %
      \sigmac
      %
      \recip{\sqrt{2}}\mat{rc}{
      \bl{1}  & \bl{1} \\
      \bl{-1} & \bl{1}
      }
    \end{split}
    %\label{eqn:}
  \end{equation}
% = =
We can build the inverse using the adjoint and cofactor:
% = =  e q u a t i o n
  \begin{equation}
    %\begin{split}
      \A{-1} = \frac{\text{adj} \A{}}{\det \A{}} = \Ap = \recip{4} \mat{rc}{2&1\\-2&1}
    %\end{split}
    %\label{eqn:}
  \end{equation}
% = =
Again the projection operator is the identity matrix
% = =  e q u a t i o n
  \begin{equation}
    %\begin{split}
      \pras =\A{-1}\A{} = \ApA = \I{2} = \idtwo
    %\end{split}
    %\label{eqn:}
  \end{equation}
% = =
The general least squares solutions is
% = =  e q u a t i o n
  \begin{equation}
      \xls = \A{-1}b = \Ap b = \mat{c}{1\\6}
    %\label{eqn:}
  \end{equation}
% = =
%\input{\pathfigures matrices/"fig c domains"}  %  <  <  <  <  <  <  <  <  <  <  <  <
\input{\pathtables matrices/"tab c domains"}  %  <  <  <  <  <  <  <  <  <  <  <  <

%% SUBSECTION
\subsection{Domain coordinates}
  % = =  e q u a t i o n
  \begin{equation}
    \brac{\xls}_{\V{}} = \V{*}\xls = \recip{\sqrt{2}} \mat{r}{7\\-5}
    %\label{eqn:}
  \end{equation}
  % = =
  % = =  e q u a t i o n
  \begin{equation}
    \brac{b}_{\U{}} = \U{*}b = \mat{r}{1\\-2}
    %\label{eqn:}
  \end{equation}
  % = =
\endinput  %  -  -  -  -  -  -  -  -  -  -  -  -  -  -  -  -  -  -  -  -

%\input{\pathchapter "chapter name"/"chap XXX"}  %  <  <  <  <  <  <  <  <  <  <  <  <