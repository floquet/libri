\subsection{Nilpotent matrices}
%
\begin{equation}
  \A{k} = \zero, \quad k=2,3,4,\dots
\end{equation}

Typically, textbooks use an example where the matrix is upper (or lower) triangular and has a zero diagonal. In this manner, each successive power has fewer and fewer entries; the zeros overwhelm the matrix. This makes multiplication by hand much easier.

By way of example, let us turn our attention to a type of nilpotent matrices expounded by Mercer \cite{Mercer}.
%
\begin{equation*}
 \A{} = \mat{cccccr}{
  2 & 2 & 2 & 2 & 2 & -5 \\
  8 & 1 & 1 & 1 & 1 & -6 \\
  1 & 8 & 1 & 1 & 1 & -6 \\
  1 & 1 & 8 & 1 & 1 & -6 \\
  1 & 1 & 1 & 8 & 1 & -6 \\
  1 & 1 & 1 & 1 & 8 & -6 }
\end{equation*}
%
\begin{equation*}
 \A{2} = \recip{7}
 \mat{cccccr}{
  3 & 3 & 3 & 3 & -4 & -4 \\
  3 & 3 & 3 & 3 & -4 & -4 \\
  9 & 2 & 2 & 2 & -5 & -5 \\
  2 & 9 & 2 & 2 & -5 & -5 \\
  2 & 2 & 9 & 2 & -5 & -5 \\
  2 & 2 & 2 & 9 & -5 & -5 }
\end{equation*}
%
\begin{equation*}
 \A{3} = \recip{49}
 \mat{cccccr}{
  4 & 4 & 4 & -3 & -3 & -3 \\
  4 & 4 & 4 & -3 & -3 & -3 \\
  4 & 4 & 4 & -3 & -3 & -3 \\
  10 & 3 & 3 & -4 & -4 & -4 \\
  3 & 10 & 3 & -4 & -4 & -4 \\
  3 & 3 & 10 & -4 & -4 & -4 }
\end{equation*}
%
\begin{equation*}
 \A{4} = \recip{343}
 \mat{ccrrrr}{
 5 & 5 & -2 & -2 & -2 & -2 \\
 5 & 5 & -2 & -2 & -2 & -2 \\
 5 & 5 & -2 & -2 & -2 & -2 \\
 5 & 5 & -2 & -2 & -2 & -2 \\
 11 & 4 & -3 & -3 & -3 & -3 \\
 4 & 11 & -3 & -3 & -3 & -3 } \\
\end{equation*}
%
\begin{equation*}
 \A{5} = \recip{2401}
 \mat{crrrrr}{
 6 & -1 & -1 & -1 & -1 & -1 \\
 6 & -1 & -1 & -1 & -1 & -1 \\
 6 & -1 & -1 & -1 & -1 & -1 \\
 6 & -1 & -1 & -1 & -1 & -1 \\
 6 & -1 & -1 & -1 & -1 & -1 \\
 12 & -2 & -2 & -2 & -2 & -2 } \\
\end{equation*}
%
\begin{equation*}
 \A{6} = \recip{49}
 \mat{cccccr}{
  4 & 4 & 4 & -3 & -3 & -3 \\
  4 & 4 & 4 & -3 & -3 & -3 \\
  4 & 4 & 4 & -3 & -3 & -3 \\
  10 & 3 & 3 & -4 & -4 & -4 \\
  3 & 10 & 3 & -4 & -4 & -4 \\
  3 & 3 & 10 & -4 & -4 & -4}
\end{equation*}
%
\begin{equation*}
 \A{6} = \zero
\end{equation*}

%%%%%%
\input{chapters/examples/"tab nilpotent"}        % table

\endinput