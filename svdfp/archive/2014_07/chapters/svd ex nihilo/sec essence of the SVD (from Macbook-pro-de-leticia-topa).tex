\section{Essence of the SVD}

The essence of the \asvd \ is resolving the eigensystem for the product matrix $\wxe{*}$. This means finding both the eigenvalues and the eigenvectors.

The eigensystem problem varies in complexity from trivial to tedious to overwhelming. A good part of this work shows how to avoid the eigenproblem altogether. But first, we can see the problem geometrically by looking at the image of the product matrix as shown in tables \eqref{tab:computing I:star (a)}-{tab:computing I:star (c)}. The extremal vectors\index{singular vectors!right!extremal vectors} are form an orthogonal set, and when normalized become the right singular vectors. Their lengths determine the singular values\index{singular vectors!length of extremal vectors}. 

The following figures are discrete versions of the maps shown in figures \eqref{tab:ftola:maps:(a)}-\eqref{tab:ftola:maps:(c)}. The first figure corresponds to matrix (a) and shows the angles $\theta$ and $\phi$ sampled in intervals of $\pi / 4$.	All of the image vectors fall on the same line given in equation \eqref{ftola:image:(a)}. Therefore the eigenvector is
\begin{equation}
  \bvo = \obavo \in \brnga{*}.
\end{equation}
The largest vector has length $\sqrt{6}$. Therefore the singular value is
\begin{equation}
  \sigma_{1} = \sqrt{6}.
\end{equation}

The figures from matrices (b) and (c) follow. Here the lone parameter $\theta$ is sampled at intervals of $\pi / 12$, and the coloring scheme is the same as in table \eqref{tab:ftola:maps:(c)}. From inspection of figure \eqref{tab:ftola:maps:(b)}, the eigenvectors are
\begin{equation}
  \bvo = 1
\end{equation}

Let's form an image in our minds.

%%%%%%
\input{chapters/"svd ex nihilo"/"tab image (a)"}  % table
\input{chapters/"svd ex nihilo"/"tab image (b)"}  % table
\input{chapters/"svd ex nihilo"/"tab image (c)"}  % table

\endinput