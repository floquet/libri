\chapter{Polynomial fits}

  % = =  e q u a t i o n
  \begin{equation}
    p_{d}\paren{x} = \sum_{j=0}^{d} \alpha_{j} x^{j} = \alpha_{0} + \alpha_{1}x + \alpha_{2}x^{2} + \cdots + \alpha_{d}x^{d}
    %\label{eqn:}
  \end{equation}
  % = =
As an aside, we point out a crucial distinction between the summation forms and the polynomial forms. The summation series $\sum_{j=0}^{d} \alpha_{j} x^{j}$ is undefined when $x=0$ and $j=0$ because it generates the indeterminate $0^{0}$. Again we define the iterated limit $p_{d}\paren{0} = \alpha_{0}$.

Caution is demanded here.  Summation form
  % = =  e q u a t i o n
  \begin{equation}
    p_{d}\paren{x} = \sum_{j=0}^{d} \alpha_{j} x^{j}
    \label{eqn:summation form}
  \end{equation}
  % = =
generates the indeterminate	form $0^{0}$ when $x=0$ and $j=0$. That is $p_{d}\paren{0}$ is not defined. The root cause is the iterated integral
  % = =  e q u a t i o n
  \begin{equation}
    \lim_{x\rightarrow 0} \lim_{j\rightarrow 0} x^{j} \ne \lim_{j\rightarrow 0} \lim_{x\rightarrow 0} x^{j}
    %\label{eqn:}
  \end{equation}
  % = =
The former tends to unity, the former to zero.
Longform
  % = =  e q u a t i o n
  \begin{equation}
    p_{d}\paren{x} = \alpha_{0} + \alpha_{1}x + \alpha_{2}x^{2} + \cdots + \alpha_{d}x^{d}
    %\label{eqn:}
  \end{equation}
  % = =
In this form $p_{d}(0) = \alpha_{0}$.
This is more than mathematical hairsplitting. In a computer code
  % = =  e q u a t i o n
  \begin{equation}
    q_{d}(x) = \alpha_{0} + \sum_{j=1}^{d} \alpha_{j} x^{j}
    %\label{eqn:}
  \end{equation}
  % = =

% SECTION
\section{Constant value}
Given a set of measurements which constant best typifies the data?
\input{\pathtables polynomials/"tab fit types"}  %  <  <  <  <  <  <  <  <  <  <  <  <

% SECTION
\section{Linear fit}

% SECTION
\section{The Weierstrass approximation theorem}
\input{\paththeorems "thm weierstrass"}

\input{\pathchapter polynomials/"sec optimal dof"}


\endinput  %  -  -  -  -  -  -  -  -  -  -  -  -  -  -  -  -  -  -  -  -