%%%%%%%%%%%%%%
%%%%%%%%%%%%%%
\subsubsection{n = 2}
\textbf{Try an identity matrix}
This is the simplest nontrivial Jordan block.
\begin{equation}
  \J{2} = \mat{cc}{1&1 \\ 0&1}
\end{equation}
Does it have a \asvd \ of the form
\begin{equation}
  \J{2} = \U{} \, \sig{} \, \I{2}?
\end{equation}
This implies that
\begin{equation}
  \sig{-1} \J{2} = \U{} 
\end{equation}
is a unitary matrix. Are there numbers $a$ and $b$ such that
\begin{equation}
  \mat{cc}{a & 0 \\ 0 & b} \mat{cc}{1&1 \\ 0&1} = \mat{cc}{a&a \\ 0&b}
\end{equation}
is a unitary matrix? Clearly not. For example, consider the first column vector; the norm is $\abs{a}$ which must be 1. The norm of the first row vector is $\sqrt{2}\abs{a}$ which also must be 1. $\bang$

\textbf{Try a single rotation matrix}
Check to see if an arbitrary rotation with one angle can describe the matrix. Let the domain matrices be
\begin{equation}
  \begin{split}
    \bur{}  &= \R{}\paren{\theta}  = {\bl{ \rot{\theta} }}, \\
    \bvr{*} &= \R{}\paren{-\theta} = {\bl{ \mat{rc}{\cos \theta & \sin \theta \\ -\sin \theta & \cos \theta} }}.
  \end{split}
\end{equation}
The decomposition is then
\begin{equation}
  \begin{split}
    \J{2} 
      &= \svd{*} \\
      &= \R{} \paren{\theta}\,
         \sig{} \,
         \R{} \paren{-\theta}
  \end{split}
\end{equation}
A quick test is to check if the $\sig{}$ is diagonal:
%
\begin{equation}
  \begin{split}
    \R{} \paren{-\theta}\, \J{2}\, \R{} \paren{\theta} = \sig{} 
      &= 
    \mat{cc}{ 1 + \cos \theta \sin \theta & \cos^{2} \theta \\
             -\sin^{2} \theta & 1 - \cos \theta \sin \theta } \\
      &\iseq \mat{cc}{ \num & 0 \\ 0 & \num }
  \end{split}
\end{equation}
There is no angle $\theta$ which allows both the counterdiagonal terms to both be 0.

\textbf{Try two rotation matrices}
Let the domain and codomain matrices be characterized by two distinct rotation angles:
%
\begin{equation}
  \begin{split}
    \bur{}  &= \R{} \paren{\theta} = {\bl{ \rot{\theta} }}, \\
    \bvr{*} &= \R{} \paren{-\phi}  = {\bl{ \mat{rc}{\cos \phi & \sin \phi \\ -\sin \phi & \cos \phi} }}.
  \end{split}
\end{equation}
%
Of course this method must work as the unitary matrices $\bur{}$ and $\bvr{}$ are equivalent to rotation matrices. The goal now is to bypass solving for the left eigenvectors.
%
\begin{equation}
  \begin{split}
    \J{2} 
      &= \svd{*} \\
      &= {\bl{ \rot{\theta} }}
         \mat{cc}{ \sigma_{1} & 0 \\ 0 & \sigma_{2} }
         {\bl{ \rott{\phi} }}
  \end{split}
  \label{eq:jcf:three matrices}
\end{equation}
%
The goal now is to find the angles by a shortcut. The singular values are quick to find and they are
%
\begin{equation}
  \sigma_{\pm} = \sqrt{ \half \paren{3 \pm \sqrt{5}} }.
\end{equation}
%
After multiplying the three matrices in equation {eq:jcf:three matrices}, a new angle characterizes the result
%
\begin{equation}
  \psi_{\pm} = \theta \pm \phi.
\end{equation}
%
The Jordan normal form is then
\begin{equation}
  \J{2} = \half
          \mat{rr}{
          \cos \psi_{+} + \sqrt{5} \cos \psi_{-} &  \sin \psi_{+} - \sqrt{5} \sin \psi_{-} \\
          \sin \psi_{+} + \sqrt{5} \sin \psi_{-} & -\cos \psi_{+} + \sqrt{5} \cos \psi_{-} }
        = \mat{cc}{1&1 \\ 0&1} 
\end{equation}
%
There are a few ways to solve this system for the angles $\psi_{\pm}$: Solve for a row, solve for a column, solve for the diagonal. We choose to concentrate on the counterdiagonal terms. The linear system is
%
\begin{equation}
  \mat{cr}{1 & \sqrt{5} \\ 1 & -\sqrt{5}} 
  \mat{c}{ \sin \psi_{+} \\ \sin \psi_{-}} =
  \mat{c}{ 0 \\ 2 } .
\end{equation}
%
The solution in terms of $\psi_{\pm}$ is then
%
\begin{equation}
  \mat{c}{ \psi_{+} \\ \psi_{-} } = 
  \mat{r}{ \frac{\pi}{2} + 2j\pi \\ \arcsin\paren{-\recip{\sqrt{5}}} + 2k\pi }, \qquad j,k \iints.
\end{equation}
%
In terms of the rotation angles the solution is
%
\begin{equation}
  \mat{c}{ \theta_{k} \\ \phi_{k} } = \half \arcsin\paren{-\frac{1} {\sqrt{5}}} + \pi \paren{k+\recip{4}} \mat{r}{1\\-1} , \quad k \iints
\label{eq:examples:jcf2}
\end{equation}
%
The angles are $2\pi-$periodic since
\begin{equation}
  \begin{split}
    e^{i\theta_{k}} &= e^{i\theta_{k+2}}, \\
    e^{i\phi_{k}} &= e^{i\phi_{k+2}}.
  \end{split}
\end{equation}
%
With this formalism, there are two distinct states corresponding to $k=\lst{0,1}$.

%%%%%%%%
\input{chapters/examples/"tab the two different SVDs for the Jordan form J2"}  % table

\endinput