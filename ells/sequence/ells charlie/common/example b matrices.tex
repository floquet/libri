%% Example matrices
\newcommand{ \dimsb }[0]    { \cmplxall{3}{2}{2} }

\newcommand{ \matrixb }
{
    \recip{4}
    \mat{ cc }
      {
       0 & 1 \\
       3 & 2 \\
       0 & 2
      }
}

%% transpose
\newcommand{ \matrixbt }
{
    \recip{4}
    \mat{ ccc }
      {
       0 & 3 & 0 \\
       1 & 2 & 2
      }
}

%% product matrices
\newcommand{ \wxb }[0]
{
    \frac{3}{16}
    \mat{ rrr }
      {
       3 &  2 \\
       2 &  3
      }
}

\newcommand{ \wyb }[0]
{
    \recip{16}
    \mat{ rrr }
      {
       1 &  2 & 3 \\
       2 & 13 & 4 \\
       2 &  4 & 4
      }
}

% x-ray view
\newcommand{ \xaycolb }[0]
{
  \recip{4}
  \mat{c|c|c}
  { 
   \bzero & \bthree & \gzero \\ 
   \bone  & \btwo   & \gtwo 
  }
}

\newcommand{ \xaycolbt }[0]
{
  \recip{4}
  \mat{c|c}
  { 
   \bzero  & \bone \\ 
   \bthree & \btwo  \\
   \bzero  & \btwo  
  }
}
\newcommand{ \xayrowb }[0]
{
  \recip{4}
  \mat{ccc} 
  { 
   \bzero & \bthree & \bzero \\\hline 
   \bone  & \btwo   & \btwo 
  }
}

%% pseudoinverse
\newcommand{ \Aplusb }[0]
{
  \recip{15}
  \mat{rrr}
  {
  -8 &  \phantom{-}20 & -16\\
  12 & 0 & 24
  }
}

%% decomposition
\newcommand{ \essb }[0] 
{
    \recip{4}
    \mat{ cc }
      {
        \sqrt{15} & 0 \\
         0 & \sqrt{3}
       }
}

\newcommand{ \sigmab }[0] 
{
    \recip{4}
    \mat{ cc }
      {
        \sqrt{15} & 0 \\
         0 & \sqrt{3} \\\hline
         0 & 0
      }
}

\newcommand{ \sigmabt }[0] 
{
    \recip{4}
    \matt{ cc|c }
      {
        \sqrt{15} & 0 & 0 \\
         0 & \sqrt{3} & 0
      }
}

\newcommand{ \sigmabnude }[0] 
{
    \recip{4}
    \mat{ cc }
      {
        \frac{1}{\sqrt{15}} & 0 \\
        0 & \frac{1}{\sqrt{3}} \\\hline
        0 & 0
      }
}

\newcommand{ \sigmabinv }[0] 
{
    \matt{ cc|c }
      {
        \frac{4}{\sqrt{15}} & 0 & 0 \\
         0 & \frac{4}{\sqrt{3}} & 0
      }
}

\newcommand{ \Sb }[0] 
{
    \recip{4}
    \mat{ cc }
      {
        \sqrt{15} & 0 \\
         0 & \sqrt{3}
      }
}

\newcommand{ \Sbinv }[0] 
{
    \mat{ cc }
      {
        \frac{4}{\sqrt{15}} & 0 \\
         0 & \frac{4}{\sqrt{3}}
      }
}

\newcommand{ \matrixbY}[0] 
{
    \mat{ rrr }
      {
        {\bl{ \rsthirty }}             & {\bl{ \rssix }}  & {\rd{-\rstfive }} \\
        {\bl{ \rsfthirty }}            & {\bl{-\rssix }}  & \rzero\phantom{,} \\
        \phantom{-}{\bl{ \rstthirty }} & {\bl{ \rstsix }} & {\rd{ \rsfive }} \\
      }
}

\newcommand{ \matrixbYt }[0]
{
    \mat{ rrr }
      {
        {\bl{ \rsthirty }} & {\bl{ \rsfthirty }} & {\bl{ \rsfthirty }} \\
        {\bl{ \rssix }}    & {\bl{ -\rssix }}    & {\bl{ \rssix }} \\
        {\rd{-\rstfive }}  & \rzero              & \phantom{-}{\rd{ \rsfive }} \\
      }
}


\newcommand{ \matrixbYg}[0] 
{
    \mat{ rrr }
      {
        {\bl{ \frac{e^{i\phi}} {\sqrt{30}}}} &
        {\bl{ \frac{e^{i\theta}}   {\sqrt{6}}}}  & 
        {\rd{ \frac{2} {5}}} \\
        %
        {\bl{ \frac{5 e^{i\phi}} {\sqrt{30}}}} &
        {\bl{ \frac{-e^{i\theta}}   {\sqrt{6}}}}  & 
        {\rd{ 0 }} \\
        %
        {\bl{ \frac{2 e^{i\phi}} {\sqrt{30}}}} &
        {\bl{ \frac{2e^{i\theta}}   {\sqrt{6}}}}  & 
        {\rd{-\frac{1} {5}}} 
      }
}

\newcommand{ \matrixbYgi}[0] 
{
    \mat{ rrr }
      {
        {\bl{  \frac{1} {\sqrt{30}} i}} &
        {\bl{ -\frac{1}{\sqrt{6}} i}}  & 
        {\rd{  \frac{2} {5}}} \\
        %
        {\bl{ \frac{5} {\sqrt{30}} i}} &
        {\bl{ \frac{1} {\sqrt{6}}  i}}  & 
        {\rd{ 0 }} \\
        %
        {\bl{ \frac{2} {\sqrt{30}} i}} &
        {\bl{-\frac{2} {\sqrt{6}}  i}}  & 
        {\rd{-\frac{1} {5}}} 
      }
}

\newcommand{ \matrixbX}[0]
{
    \rstwo
    {\bl{ \mat{ rr }
      {
       \phantom{-}1 &  -1\\
       1 &  1
      }
     }}
}

\newcommand{ \matrixbXgt}[0]
{
    \rstwo
    {\bl{ \mat{ rr }
      {
        e^{-i\phi} &  \phantom{-}e^{-i\phi}\\
       -e^{-i\theta} &  e^{-i\theta}
      }
     }}
}

\newcommand{ \matrixbXt}[0]
{   
    {\bl{ \rstwo \mat{ rr }
      {
        1 &  \phantom{-}1\\
       -1 &  1
      }
     }}
}
%% subspace decomposition
%% codomain
\newcommand{ \rngbo }[0]
{
   {\bl{  
    \mat{ c }
      {
       \recip{\sqrt{30}} \\
       \frac{5}{\sqrt{30}} \\
       \frac{5}{\sqrt{30}}
      }
    }}
}

\newcommand{ \nllbso }[0]
{
   {\rd{  
    \mat{ c }
      {
       \recip{\sqrt{6}} \\
       \frac{-1}{\sqrt{6}} \\
       \frac{2}{\sqrt{6}}
      }
    }}
}

\newcommand{ \nllbst }[0]
{
   {\rd{  
    \mat{ c }
      {
       \frac{-2}{\sqrt{5}} \\
       0 \\
       \recip{\sqrt{5}}
      }
    }}
}

%% domain
\newcommand{ \rngbso }[0]
{
   {\bl{  
    \mat{ c }
      {
       \recip{\sqrt{2}} \\
       \frac{-1}{\sqrt{2}}
      }
    }}
}

\newcommand{ \rngbst }[0]
{
   {\bl{  
    \mat{ c }
      {
       \recip{\sqrt{2}} \\
       \recip{\sqrt{2}} 
      }
    }}
}

%% decomposition
\newcommand{ \ssinv }[0] 
{
    \mat{ cc|c }
      {
         1 & 0 & 0 \\
         0 & 1 & 0 \\\hline
         0 & 0 & 0
      }
}

\newcommand{ \sinvs }[0] 
{
    \idtwo
}

% qr decomposition
\newcommand{ \matrixbQ}[0] 
{
    \rsfive
    \mat{ cc|r }
      {
        \bzero & \bone  & \rtwo \\
        \bone  & \bzero & \rzero \\
        \bzero & \btwo  & \rminus\rone \\
      }
}

\newcommand{ \matrixbR}[0] 
{
    \recip{4}
    \mat{ cc }
      {
        3 & 2 \\
        0 & \sqrt{5} \\ \hline
        0 & 0 \\
      }
}

\newcommand{ \matrixbQt}[0] 
{
    \rsfive
    \mat{ ccc }
      {
        \bzero & \bone  & \bzero \\
        \bone  & \bzero & \btwo
      }
}
\newcommand{ \matrixbRinv}[0] 
{
    \mat{ cc }
      {
        \frac{4}{3} & 0 \\
        \half & \frac{4} {\sqrt{5}}
      }
}

%% projectors
\newcommand{ \prab }[0]
{
  \recip{5}
  \mat{rrr}
  {
   1 & 0 & 2 \\
   0 & 5 & 0 \\
   \ps2 & \ps0 & \ps4 \\
  }
}

\newcommand{ \pnasb }[0]
{
  \recip{5}
  \mat{rrr}
  {
   4 & 0 & -2 \\
   0 & 0 &  0 \\
  -2 & \ps0 &  1 \\
  }
}

\newcommand{ \prasb }[0]
{
  \idtwo
}

\newcommand{ \pnab }[0]
{
  \gztwo
}

\newcommand{ \prabs }[0]  {\prasb}
\newcommand{ \pnasbs }[0] {\pnab}
\newcommand{ \prasbs }[0] {\prab}
\newcommand{ \pnabs }[0]  {\pnasb}


% vectors
\newcommand{ \prabeta }[0]
{\bl{ \recip{5}
   \mat{c}{   \eta_{1} + 2\eta_{3} \\ 
             5\eta_{2}             \\
             2\eta_{1} + 4\eta_{3} } }
}
             
\newcommand{ \prasbxi }[0]
{\bl{ \mat{c}{ \xi_{1} \\ 
               \xi_{2} } }
}

\newcommand{ \pnabxi }[0]
{
  \gztwo
}

\newcommand{ \pnasbeta }[0]
{\rd{ \recip{5}
   \mat{c}{ 4\eta_{1} - 2\eta_{3} \\ 
            0                     \\
           -2\eta_{1} +  \eta_{3} } }
}

\newcommand{ \pnasbetat }[0]
{\rd{ \recip{5}
   \mat{ccc}{ 4\eta_{1} - 2\eta_{3} & 
               0                    &
             -2\eta_{1} +  \eta_{3} } }
}


% colorized

\newcommand{ \matrixbXtclr}[0]
{
    \mat{ rrr }
      {
        {\bl{\frac{1}{\sqrt{30}}}} &  {\bl{\frac{5}{\sqrt{30}}}} & {\bl{\frac{2}{\sqrt{30}}}} \\
        {\bl{\frac{1}{\sqrt{6}}}}  & {\bl{-\frac{1}{\sqrt{6}}}}  & {\bl{\frac{2}{\sqrt{6}}}} \\
        {\rd{\frac{2}{\sqrt{5}}}}  & {\rd{0}}                    & {\rd{-\frac{1}{\sqrt{5}}}}
      }
}

\newcommand{ \matrixbYclr}[0] 
{
    \rstwo
    \mat{ rr }
      {
       \bone &  \bminus \bone \\
      \phantom{-}\bone &  \bone
      }
}

\newcommand{ \matrixbUclri}[0]
{
    \mat{ rrr }
      {
        {\bl{\frac{1}{\sqrt{30}}}} &  {\bl{\frac{5}{\sqrt{30}}}} & {\bl{\frac{2}{\sqrt{30}}}} \\
        {\bl{\frac{1}{\sqrt{6}}}}  & {\bl{-\frac{1}{\sqrt{6}}}}  & {\bl{\frac{2}{\sqrt{6}}}} \\
        {\rd{\frac{2}{\sqrt{5}}}}  & {\rd{0}}                    & {\rd{-\frac{1}{\sqrt{5}}}}
      }
}

%% SVD
\newcommand{ \svdecompb }[0]     { \matrixbY \, \sigmab \, \matrixbXt} 
\newcommand{ \aesvdecompb }[0]   { \A{} = \svdecompb } 
\newcommand{ \svdecompbg }[0]    { \matrixbYg \, \sigmab \, \matrixbXgt} 
%\newcommand{ \svdecompbg }[0]    { \matrixbY \, \sigmab \, \matrixbXgt} 

\newcommand{ \svdecompbclr }[0]  { \matrixbYclr \, \sigmab \, \matrixbXtclr} 
\newcommand{ \aesvdecompbclr }[0]{ \A{} = \svdecompb } 

%% spaces
\newcommand{ \examplebrngdo }[0] { \lst{ {\color{blue}{\rsthirty \threevecj}}, {\color{blue}{\rsfive \threeveck }}} }
\newcommand{ \examplebnlldo }[0] { \lst{ {\color{red}{\rsfive \threevecm}} } }

\newcommand{ \examplebrngco }[0] { \lst{ {\color{blue}{\rstwo \twovecb}},  {\color{blue}{\rstwo \twoveci}} } }
\newcommand{ \examplebnllco }[0] { \lst{ {\color{red}{\zerotwo}} } }

% partitions
\newcommand{ \matrixbp }
{
    \recip{4}
    \mat{ c|c }
      {
       0 & 1 \\
       3 & 2 \\
       0 & 2
      }
}

\newcommand{ \matrixbtp }
{
    \recip{4}
    \mat{ c|c|c }
      {
       0 & 3 & 0 \\
       1 & 2 & 2
      }
}

%% data vector b
\newcommand{ \datab }[0]
{
    \mat{ c }
      {
       2 \\
       1 \\
       0
      }
}

\newcommand{ \databt }[0]
{
    \mat{ ccc }
      {
       2 & 1 & 0
      }
}

%% x_{LS}
\newcommand{ \xlsb }[0]
{
   {\bl{  
    \recip{15}
    \mat{ c }
      {
        4 \\
       24
      }
    }}
}

%% Ax_{LS}
\newcommand{ \databAxls }[0]
{
   {\bl{ 
    \recip{3}
    \mat{ r }
      {
       1 \\
      -1 \\
       1
      }
    }}
}

%% r
\newcommand{ \databr }[0]
{
    {\rd{ \recip{3}
    \mat{ r }
      {
       -5 \\
       -4 \\
        1
      }
     }}
}

\newcommand{ \nllb }[0]
{
   {\rd{  
    \mat{ r }
      {
        1 \\
        1
      }
    }}
}

%% ||r||
\newcommand{ \normbr }[0]
{
    \sqrt{\frac{14} {3}}
}

%% ||r||^{2}
\newcommand{ \normbrs }[0]
{
    \frac{14} {3}
}

%% vectors
\newcommand{ \vecbxa }  { {\bl{ \rstwo \mat{c}{  1\\ 1 } }} }
\newcommand{ \vecbxb }  { {\bl{ \rstwo \mat{r}{ -1\\ 1 } }} }

\newcommand{ \vecbyra }  { {\bl{ \mat{c}{ \frac{1}{} \\ 1 } }} }
\newcommand{ \vecbyrb }  { \mat{r}{ \bminus\bone\\\bone\\\bminus\bone } }
\newcommand{ \vecbyna }  { {\rd{ \rsfive \mat{r}{ -2 \\ 0 \\ 1 } }} }
\newcommand{ \vecbynat } { {\rd{ \rsfive \mat{rcc}{ -2 & 0 & 1 } }} }

%% 3 vectors
\newcommand{ \vecba }      { \mat{r}{ 0\\ 3\\ 0 } }
\newcommand{ \vecbb }      { \mat{r}{ 1\\ 2\\ 2 } }
\newcommand{ \vecbc }      { \mat{r}{-2\\ 0\\ 1 } }
\newcommand{ \vecbm }      { \mat{r}{ 1\\ 5\\ 2 } }
\newcommand{ \vecbn }      { \mat{r}{ 1\\-1\\ 2 } }
\newcommand{ \vecbo }      { \mat{r}{-2\\ 0\\ 1 } }

\newcommand{ \bvecba }     { \frac{1}{4} {\bl{ \vecba }} }
\newcommand{ \bvecbb }     { \frac{1}{4} {\bl{ \vecbb }} }
\newcommand{ \bvecbm }     { {\bl{ \vecbm }} }
\newcommand{ \bvecbn }     { {\bl{ \vecbn }} }
\newcommand{ \rvecbo }     { {\rd{ \vecbo }} }
\newcommand{ \rvecbc }     { {\rd{ \vecbc }} }

\newcommand{ \obvecba }    { {\bl{ \rsthree  \vecba } } }
\newcommand{ \obvecbb }    { {\bl{ \rsfive   \vecbb } } }
\newcommand{ \obvecbm }    { {\bl{ \rsthirty \vecbm } } }
\newcommand{ \obvecbn }    { {\bl{ \rssix    \vecbn } } }
\newcommand{ \orvecbo }    { {\rd{ \rsfive   \vecbo } } }
\newcommand{ \orvecbc }    { {\rd{ \rsfive   \vecbc } } }

%% 2 vectors
\newcommand{ \vecbd }      { \mat{c}{ 1 \\ 1} }
\newcommand{ \vecbe }      { \mat{r}{-1 \\ 1} }

\newcommand{ \bvecbd }     { \frac{1}{4} {\bl{ \mat{c}{ 0 \\ 1} }} }
\newcommand{ \bvecbe }     { \frac{1}{4} {\bl{ \mat{c}{ 3 \\ 2} }} }

\newcommand{ \obvecbd }    { \bl{ \rstwo \vecbd } }
\newcommand{ \obvecbe }    { \bl{ \rstwo \vecbe } }

\newcommand{ \obvecbdt }   { \bl{ \rstwo \mat{cc}{ 1 & 1} } }
\newcommand{ \obvecbet }   { \bl{ \rstwo \mat{cc}{-1 & 1} } }

%%%%%%%%%%%
%%%%%%%%%%%
%%%%%%%%%%%
\newcommand{ \bveca }      { \mat{r}{ 1\\ 5\\ 2 } }
\newcommand{ \bvecb }      { \mat{r}{ 1\\-1\\ 2 } }
\newcommand{ \bvecc }      { \mat{r}{ 2\\ 0\\-1 } }

\newcommand{ \bvecat }     { \mat{ccc}{ 1 &  5 &  2 } }
\newcommand{ \bvecbt }     { \mat{rrr}{ 1 & -1 &  2 } }
\newcommand{ \bvecct }     { \mat{rrr}{ 2 &  0 & -1 } }

\newcommand{ \obbuo }      { \bl{ \rsthirty \bveca } }
\newcommand{ \obbut }      { \bl{ \rssix    \bvecb } }
\newcommand{ \orbuo }      { \rd{ \rsfive   \bvecc } }

\newcommand{ \obbuot }     { \bl{ \rsthirty \bvecat } }
\newcommand{ \obbutt }     { \bl{ \rssix    \bvecbt } }
\newcommand{ \orbuot }     { \rd{ \rsfive   \bvecct } }

%%%%%%%%%%%
%%%%%%%%%%%
\newcommand{ \bvecd }      { \mat{c} { 1\\ 1 } }
\newcommand{ \bvece }      { \mat{r} {-1\\ 1 } }

\newcommand{ \bvecdt }     { \mat{cc}{ 1 & 1 } }
\newcommand{ \bvecet }     { \mat{cc}{-1 & 1 } }

\newcommand{ \obbvo }      { \bl{ \rstwo \bvecd } }
\newcommand{ \obbvt }      { \bl{ \rstwo \bvece } }
\newcommand{ \obbvot }     { \bl{ \rstwo \bvecdt } }
\newcommand{ \obbvtt }     { \bl{ \rstwo \bvecet } }



\endinput