%-----------------------------------------------------------------------
% Beginning of chapter-template.tex
%-----------------------------------------------------------------------
%
%    This is a template file for monographs prepared with AMS author
%    packages, for use with AMS-LaTeX.  Separate chapters should be
%    included at the appropriate position.
%
%    Templates for various common text, math and figure elements are
%    given following the \end{document} line.
%
%    Start by copying this file to <filename>.tex, using a distinctive
%    name suitable for your book in place of <filename>.  This will
%    be the driver file for your book.
%
%%%%%%%%%%%%%%%%%%%%%%%%%%%%%%%%%%%%%%%%%%%%%%%%%%%%%%%%%%%%%%%%%%%%%%%%

%    Replace amsbook by the documentclass code for the monograph series.
\documentclass{amsbook}

%    If you need symbols beyond the basic set, uncomment this command.
%\usepackage{amssymb}

%    If your book includes graphics, or such features as rotation or
%    scaling, uncomment this command.
%\usepackage{graphicx}

%    If the book includes commutative diagrams:
%\usepackage[cmtip,all]{xy}

%    If you are using the author-year citation style:
%\usepackage{natbib}

%    Include other referenced packages here.
\usepackage{}

%    For use when working on individual chapters
%\includeonly{}

%    As set up here, all theorem-class objects will be numbered with
%    the same counter, starting with 1 at every new chapter; numbers
%    will have the form <chapter>.<theorem>.  This may be changed if
%    the author prefers.
\newtheorem{theorem}{Theorem}[chapter]
\newtheorem{lemma}[theorem]{Lemma}

\theoremstyle{definition}
\newtheorem{definition}[theorem]{Definition}
\newtheorem{example}[theorem]{Example}
\newtheorem{xca}[theorem]{Exercise}

\theoremstyle{remark}
\newtheorem{remark}[theorem]{Remark}

\numberwithin{section}{chapter}
\numberwithin{equation}{chapter}

%    For a single index; for multiple indexes, see the manual
%    "Instructions for preparation of papers and monographs:
%    AMS-LaTeX" (instr-l.pdf in the AMS-LaTeX distribution).
%    Do not \usepackage{makeidx}; all facilities are contained
%    within the AMS document classes.
\makeindex

\begin{document}

\frontmatter

\title{}

%    Remove any unused author tags.

%    author one information
\author{}
\address{}
\curraddr{}
\email{}
\thanks{}

%    author two information
\author{}
\address{}
\curraddr{}
\email{}
\thanks{}

%    If any version of the Mathematics Subject Classification other
%    than the 2010 edition appears, then you have an old version
%    of the AMS-LaTeX collection and need to upgrade.  Download from
%    http://www.ams.org/tex/amslatex.html .
\subjclass[2010]{Primary }

\keywords{}

\date{}

\begin{abstract}
\end{abstract}

\maketitle

%    Dedication.  If the dedication is longer than a line or two,
%    remove the centering instructions and the line break.
%\cleardoublepage
%\thispagestyle{empty}
%    If this book uses the documentclass stml-l or mmono-s, change
%    13.5pc to 10.5pc.
%\vspace*{13.5pc}
%\begin{center}
%  Dedication text (use \\[2pt] for line break if necessary)
%\end{center}
%\cleardoublepage

%    Change page number to 7 if a dedication is present.
\setcounter{page}{5}

\tableofcontents

%    Include unnumbered chapters (preface, acknowledgments, etc.) here.
\include{}

\mainmatter
%    Include main chapters here.
\include{}

\appendix
%    Include appendix "chapters" here.
\include{}

\backmatter
%    Bibliographies can be prepared with BibTeX using amsplain,
%    amsalpha, or (for "historical" overviews) natbib style.
\bibliographystyle{amsplain}
\bibliography{}

%    See note above about multiple indexes.
\printindex

\end{document}

%%%%%%%%%%%%%%%%%%%%%%%%%%%%%%%%%%%%%%%%%%%%%%%%%%%%%%%%%%%%%%%%%%%%%%%%

%    Templates for common elements of a monograph; for additional
%    information, see the AMS-LaTeX instructions manual, instr-l.pdf,
%    included in every AMS author package, and the amsthm user's guide,
%    linked from http://www.ams.org/tex/amslatex.html .

%    Note that [optional] short forms may be needed for running heads.

%    Chapter titles
\chapter[short form]{full title}

%    Section headings
\section[short form]{full heading}
\subsection{}

%    Ordinary theorem and proof
\begin{theorem}[Optional addition to theorem head]
% text of theorem
\end{theorem}

\begin{proof}[Optional replacement proof heading]
% text of proof
\end{proof}

%    Figure insertion; default placement is top; if the figure occupies
%    more than 75% of a page, the [p] option should be specified.
\begin{figure}
\includegraphics{filename}
\caption{text of caption}
\label{}
\end{figure}

%    Mathematical displays; for additional information, see the amsmath
%    user's guide, linked from http://www.ams.org/tex/amslatex.html .

% Numbered equation
\begin{equation}
\end{equation}

% Unnumbered equation
\begin{equation*}
\end{equation*}

% Aligned equations
\begin{align}
  &  \\
  &
\end{align}

%-----------------------------------------------------------------------
% End of chapter-template.tex
%-----------------------------------------------------------------------
