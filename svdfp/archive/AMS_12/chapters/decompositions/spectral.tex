\section[$\A{} = \ess{}\Lambda\,\ess{-1}$]{Spectral decomposition}

%%
\subsection{Theory}
A foundation theorem is the \index{spectral theorem}spectral theorem.

\begin{equation}
\A{} = \ess{}\Lambda\,\ess{-1}.
\end{equation}

In conversational mathematics: 
\begin{quote}
Every Hermitian matrix has a spectral decomposition. The component matrices are
\begin{enumerate}
\item a diagonal matrix of eigenvalues, and
\item a unitary matrix of eigenvectors.
\end{enumerate}
The spectrum is the list of matrix eigenvalues.
 \end{quote}

%%
\subsection{Connection to the \svdl}
All matrices have a \svdl. Normal matrices also have spectral a decomposition. For a normal matrix $\A{}$ these decompositions are given by
\begin{equation}
  \begin{split}
    \A{}&=\ess{-1}\Lambda\,\ess{} \\
    \svda{*}
  \end{split}
\end{equation}
The target matrix is normal if and only if
\begin{equation}
  \brac{\A{},\A{*}} = 0
\end{equation}
or equivalently
\begin{equation}
  \prdmm{*} = \prdm{*}.
  \label{eq:decomp:normal}
\end{equation}
In terms of the spectral decomposition equality \eqref{eq:decomp:normal} is expressed in this manner:
\begin{equation}
  \begin{split}
     \prdmm{*} &= \prdm{*},\\
     \paren{\ess{-1}\Lambda\,\ess{}}\paren{\ess{-1}\Lambda\,\ess{}} &= \paren{\ess{-1}\Lambda\,\ess{}}\paren{\ess{-1}\Lambda\,\ess{}},\\     
     \ess{-1}\Lambda^{2}\,\ess{} &= \ess{-1}\Lambda^{2}\,\ess{}.     
  \end{split}
\end{equation}
The diagonal matrix $\Lambda^{2}$ contains the squares of the eigenvalues of the full rank matrix $\A{}$ and the $\ess{}$ matrix contains the eigenvectors.

In terms of the SVD equality \eqref{eq:decomp:normal} is expressed this way:
\begin{equation}
  \begin{split}
     \prdmm{*} &= \prdm{*},\\
     \X{}\, \sig{2}\, \X{*} &= \Y{}\, \sig{2}\, \Y{*}.
  \end{split}
\end{equation}
We have used the facts that the matrix $\sig{}$ is square and full rank to write
\begin{equation}
  \sig{}\,\sig{T} = \sig{T}\,\sig{} = \sig{2}.
\end{equation}

We see
\begin{equation}
  \begin{split}
    \ess{} &= \X{*}\\
    \ess{-1} &= \Y{}
  \end{split}
\end{equation}


%%
\subsection{Examples}


%%
\subsection{Application}
Use the $\L\,\U{}$ decomposition to solve the linear system

\endinput