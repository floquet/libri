\chapter{Post Mortem II}

The previous chapter introduced the general method for resolving a matrix into its \svdl. This method involved finding the singular values by resolving the eigensystem of a product matrix. The next section explores the geometry of the eigenvalues and motivates the perception that the singular values are a set of scale factors.

However, because of the weight of the details in the last chapter, we will exhibit another example. This will provide more insight into the geometry of the singular values.

The trouble with the target matrix in the preceding example is that while it does have enough linearly independent column vectors to span the domain it lacks enough linearly independent row vectors to span the codomain. Therefore there is a left inverse but no right inverse. Therefore the linear system can't be solved using obvious methods again we need the SVD.
 
%%
\section{Solution using the SVD}
Let's return to using an SVD to solve the linear systems like $\ls$ as shown in \S\eqref{sec:formal:simple}.

Once you have the SVD life is good.
 
%%
\subsection{Extended example SVD}
We don't want 3 x 3, go to 3 x 5
\begin{equation}
  \begin{array}{cccc}
    \A{}&x &=& b\\
    \mat{rrrrr}
    {
     1 & -1 &  1 & -1 &  1\\
    -1 &  1 & -1 &  1 & -1\\
     1 & -1 &  1 & -1 &  1\\
    }
    &
    \mat{c}{x_{1}\\x_{2}\\x_{3}\\x_{4}\\x_{5}}
    &=& \phivector.
  \end{array}
\end{equation}

\textbf{Recycling:} The good news is that we can solve this system using the simple \svdl \ method of the first chapter allowing us to bypass the eigensystem problem. Even more good] news is that we can recycle the codomain matrix $\Y{}$. The active column vector in the domain matrix is this:
\begin{equation}
  \X{}_{*,1} = \sfive\mat{r}{1 \\ -1 \\  1 \\ -1 \\  1}.
\end{equation}
Using the fact that
\begin{equation}
  \begin{split}
    \A{}\X{}_{*,1}=\sigma_{1}\Y{}_{*,1}
  \end{split}
\end{equation}
we can compute the lone singular value of
\begin{equation}
  \sigma_{1} = 15^{-1/2}.
\end{equation}

Without doing noticeable calculation, we already have the partial decomposition
\begin{equation}
  \begin{split}
    \svda{T}\\
    &=
    \Yshade
    \mat{c|cccc}
    {
     15^{-1/2} & 0 & 0 & 0 & 0\\\hline
      0 & 0 & 0 & 0 & 0\\
      0 & 0 & 0 & 0 & 0\\
    }
    \mat{ccccc}
    {\sfive & -\sfive & \phantom{-}\sfive & -\sfive & \phantom{-}\sfive\\
     \rowcolor{ltgray}
     \cdot  &  \phantom{-}\cdot  & \phantom{-}\cdot  &  \phantom{-}\cdot  & \phantom{-}\cdot \\
     \rowcolor{ltgray}
     \cdot  &  \phantom{-}\cdot  & \phantom{-}\cdot  &  \phantom{-}\cdot  & \phantom{-}\cdot \\
     \rowcolor{ltgray}
     \cdot  &  \phantom{-}\cdot  & \phantom{-}\cdot  &  \phantom{-}\cdot  & \phantom{-}\cdot \\
     \rowcolor{ltgray}
     \cdot  &  \phantom{-}\cdot  & \phantom{-}\cdot  &  \phantom{-}\cdot  & \phantom{-}\cdot}\\
  \end{split}
\end{equation}

The Gram-Schmidt process in the appendix \eqref{sec:gs} will complete the $\X{}$ matrix. Using the seed vectors
\begin{equation}
  U = \lst{
  \mat{r}{1 \\ -1 \\  1 \\ -1 \\  1},
  \mat{c}{1\\0\\0\\0\\0},
  \mat{c}{0\\1\\0\\0\\0},
  \mat{c}{0\\0\\1\\0\\0},
  \mat{c}{0\\0\\0\\1\\0}
  }
\end{equation}
the domain matrix becomes
\begin{equation}
  \X{} = 
  \left[
\begin{array}{ r >{\columncolor{ltgray}}r >{\columncolor{ltgray}}r >{\columncolor{ltgray}}r >{\columncolor{ltgray}}c }
  \sfive &  \frac{4}{2\sqrt{5}} &  0 &  0 &  0 \\
 -\sfive &  \frac{1}{2\sqrt{5}} &  \frac{3}{2\sqrt{3}} &  0 &  0\\
  \sfive & -\frac{1}{2\sqrt{5}} &  \frac{1}{2\sqrt{3}} & -\frac{2}{6} &  0\\
 -\sfive &  \frac{1}{2\sqrt{5}} & -\frac{1}{2\sqrt{3}} &  \ssix & \stwo\\
  \sfive & -\frac{1}{2\sqrt{5}} &  \frac{1}{2\sqrt{3}} & -\ssix & \stwo\\
\end{array}
\right]
\end{equation}

Now do the change of coordinates stuff in the section on a more formal introduction.
 
%%
\subsection{Extended example solution}
The particular solution for the problem is given by this
\begin{equation}
  x_{p} = \A{+}b.
  \label{eq:lsq:a}
\end{equation}
Since the domain matrices are mainly composed of null vectors, the pseudoinverse is a quick construction. We need only one outer product
\begin{equation}
  \begin{split}
    \A{+} &= \sigma_{1}\X{}_{*,1}\otimes \Y{T}_{1,*}\\
     &= \paren{15^{-1/2}}
     \paren{\sfive \mat{r}{1\\-1\\1\\-1\\1}}
     \paren{\sthree \mat{rrr}{1&-1&1}} =
     \frac{1}{15}\mat{rrr}{1 & -1 & 1\\-1 & 1 & -1\\1 & -1 & 1\\-1 & 1 & -1\\1 & -1 & 1}.
  \end{split}
\end{equation}
The point solution, the particular solution, of equation \eqref{eq:lsq:a} is then
\begin{equation}
  x_{p} = \frac{1}{15}\mat{rrrrr}{1 & -1 & 1 & -1 & 1}^{\mathrm{T}}.
\end{equation}
The homogenous solutions add considerable flavor to the full solution. The null space is spanned by four vectors.
\begin{equation}
  \begin{split}
    x &= x_{p} + x_{h}\\
      &= \underbrace{\frac{1}{15}\mat{r}{1\\-1\\1\\-1\\1}}_{\text{particular}}
       + \underbrace{
         \alpha_{1}  \mat{r}{4\\1\\-1\\1\\-1} 
       + \alpha_{2}  \mat{r}{0\\3\\1\\-1\\1} 
       + \alpha_{3}  \mat{r}{0\\0\\-2\\1\\-1} 
       + \alpha_{4}  \mat{r}{0\\0\\0\\1\\1}
         }_{\text{homogeneous}} 
  \end{split}
\end{equation}
where the constants $\alpha$ are arbitrary complex numbers.
 
%%
\subsection{Extended example solution}
Using the coordinate transformations of equation \eqref{eq:moreformal:a}
\begin{equation*}
  \begin{split}
    \mathbb{X} & = \X{*} x  \\
    \mathbb{B} & = \Y{*} b
    \label{eq:moreformal:a}
  \end{split}
\end{equation*}
The $\mathbb{B}$ vector is unchanged from equation \eqref{eq:morefomal:B}
\begin{equation*}
    \mathbb{B} = \mat{r}{\sthree \\ \stwo \\ \frac{-2}{\sqrt{6}}}.
\end{equation*}
The new $\mathbb{X}$ vector is now given by this
\begin{equation}
  \mathbb{X} = \mat{c}{
  \frac{1}{\sqrt{5}}\paren{x_{1}+2x_{2}} \\
 -\frac{1}{\sqrt{5}}x_{1}+\frac{1}{2\sqrt{5}}x_{2}+\frac{\sqrt{3}}{2}x_{3} \\
  \frac{1}{\sqrt{5}}x_{1}-\frac{1}{2 \sqrt{5}}x_{2}+\frac{1}{2 \sqrt{3}}x_{3}+\sqrt{\frac{2}{3}} x_{4}\\[5pt]
 -\frac{1}{\sqrt{5}}x_{1}+\frac{1}{2 \sqrt{5}}x_{2}-\frac{1}{2 \sqrt{3}}x_{3}+\frac{1}{\sqrt{6}}x_{4}+\frac{1}{\sqrt{2}}x_{5}\\[5pt]
  \frac{1}{\sqrt{5}}x_{1}-\frac{1}{2 \sqrt{5}}x_{2}+\frac{1}{2 \sqrt{3}}x_{3}-\frac{1}{\sqrt{6}}x_{4}+\frac{1}{\sqrt{2}}x_{5}
  }.
\end{equation}

The simplified solution of equation \eqref{eq:formal:svdsoln} is given by
\begin{equation}
  \begin{split}
    \mathbb{X} &= \sig{(+)}\, \mathbb{B}\\
    \mat{r}{
    \sfive            \paren{x_{1} - x_{2} + x_{3} - x_{4} + x_{5}} \\
    \frac{2}{\sqrt{5}}\paren{4x_{1}+ x_{2} - x_{3} + x_{4} - x_{5}} \\
    \frac{2}{\sqrt{3}}\paren{       3x_{2} + x_{3} - x_{4} + x_{5}} \\
    \ssix             \paren{              -2x_{3} + x_{4} - x_{5}} \\
    \stwo             \paren{                        x_{4} + x_{5}} \\
  }
  &= \mat{c}{\sqrt{5}\\[5pt]0\\[5pt]0\\[5pt]0\\[5pt]0}
  \end{split}
\end{equation}

\endinput
\section[Left and right inverses]{Left and right inverses: a first look}
\label{lrfirst}

This section whets the appetite for a topic which will be developed later in the the section on the Moore-Penrose pseudoinverse, \S\eqref{sec:chiral}. For now, we present a basic observation. One may suspect that when we generalize the matrix inverse, we will also generalize the properties of the matrix inverse. The careworn requirement is that a square nonsingular matrix $\A{}$ must satisfy
\begin{equation}
  \A{-1}\A{} = \A{}\,\A{-1} = \I{m}.
\end{equation}
However for the pseudoinverse, the sizes of the resultant matrices don't even match. For the prototypical $\Acc{m}{n}$:
\begin{equation}
  \begin{array}{rcl}
    \leftinv &\in&\cmplx{\by{n}{n}},\\
    \rightinv &\in&\cmplx{\by{m}{m}}.
  \end{array}
\end{equation}

Is it possible only one of the matrix products $\leftinv $ or $\rightinv$ might be an identity matrix?


The first step is to assemble the pseudoinverse matrix for $\A{}\in\cmplx{\by{3}{2}}_{2}$:
\begin{equation}
  \begin{split}
    \mpgia{T} \\
      &=
      \left[
\begin{array}{ cr >{\columncolor{ltgray}}r }
  \frac{1}{\sqrt{30}} & \frac{ 1}{\sqrt{6}} & \frac{-2}{\sqrt{5}}\\
  \frac{5}{\sqrt{30}} & \frac{-1}{\sqrt{6}} & 0 \\
  \frac{2}{\sqrt{30}} & \frac{ 2}{\sqrt{6}} & \frac{ 1}{\sqrt{5}}\\
\end{array}
\right]  
  \mat{cc}
  {
  \frac{1}{\sqrt{15}} & 0\\
  0 & \frac{1}{\sqrt{3}}\\\hline
  0 & 0
  }
  \frac{1}{\sqrt{2}}
  \mat{rr}{1 & 1\\-1 & 1}\\
  &=\frac{1}{15}
  \mat{rr}
  {
 -2 & 3 \\
  5 & 0 \\
 -4 & 6
  }.
  \end{split}
\end{equation}

What is the action of the pseudoinverse matrix when it pre- and post-multiplies the target matrix?
%%
\begin{equation}
  \begin{array}{rcccc}
    \leftinv&=&
    \frac{1}{15}
  \mat{rr}
  {
 -2 & 3 \\
  5 & 0 \\
 -4 & 6
  }
  \mat{ccc}
  {
  0 & 3 & 0 \\
  1 & 2 & 2
  } &=&
  \frac{1}{5}
  \mat{ccc}
  {
 1 & 0 & 2 \\
 0 & 5 & 0 \\
 2 & 0 & 4
  },\\
    \rightinv&=&
  \mat{ccc}
  {
  0 & 3 & 0 \\
  1 & 2 & 2
  } 
    \frac{1}{15}
    \mat{rr}
  {
 -2 & 3 \\
  5 & 0 \\
 -4 & 6
  }
&=& \itwo.
  \end{array}
  \label{eq:gen:lr}
\end{equation}

In this case with full row rank the pseudoinverse is also a \index{right inverse}right inverse. In the chapter on the pseudoinverse we will uncover a geometric interpretation for the product of a matrix and its pseudoinverse. For now we simply note the following behaviors:
\begin{table}[htdp]
\begin{center}
\begin{tabular}{lll}
rank condition   & \ parameters \ & \ inverse condition\\\hline
full row rank    & \ $\rho = m $  & \ $\A{+} = \AinvL$ $\phantom{A^{-1^{-1^{-1}}}}$ \\[3pt]
full column rank & \ $\rho = n $  & \ $\A{+} = \AinvR$ \\[3pt]
full row and column rank \ & \ $\rho = m = n $ \ & \ $\A{+} = \AinvL = \AinvR = \A{-1}$ \\[13pt]
\end{tabular}
\end{center}
\label{tab:pmii:rank}
\caption[When the pseudoinverse will behave like a standard inverse]{Full rank is the criterion which indicates when the pseudoinverse will behave like a standard inverse. For matrices with full \textit{row} rank, the pseudoinverse is a \textit{right} inverse. For matrices with full \textit{column} rank, the pseudoinverse is a \textit{left} inverse. Of course if the matrix is square and of full rank then the pseudoinverse is the standard inverse.}
\end{table}%
\\
%%%
The formulaically minded may prefer this more mathematical presentation:
\begin{equation}
  \begin{array}{rclcrclcrcl}
     \A{}&\in&\cmplx{\by{m}{\textbf{n}}}_{\textbf{n}} & \Longrightarrow & \A{+} &=& \AinvL & \Longrightarrow & \leftinv &=& \I{n},\\
     \A{}&\in&\cmplx{\by{\textbf{m}}{n}}_{\textbf{m}} & \Longrightarrow & \A{+} &=& \AinvR & \Longrightarrow & \rightinv &=& \I{m},\\
     \A{}&\in&\cmplx{\by{m}{\textbf{m}}}_{\textbf{m}} & \Longrightarrow & \A{+} &=& \A{-1} & \Longrightarrow & \leftinv = \rightinv &=& \I{m},.\\
  \end{array}
\end{equation}

\section{Proximity to the identity}
\label{piproximity}

A quick note before closing. What about the first result in equation \eqref{eq:gen:lr}? Clearly $\leftinv$ is not a left inverse because the product is not an identity matrix. But how ``close'' is this matrix to the identity matrix? We can use the concept of the matrix norm to measure the distance between two matrices. 
\begin{equation}
\normt{\leftinv-\I{m}} = 
\normt{\frac{1}{5}
  \mat{ccc}
  {
 1 & 0 & 2 \\
 0 & 5 & 0 \\
 2 & 0 & 4
  }
  -
  \ithree} = 1.
\end{equation}
We will see this last result a few more times.

The point is that while we did not reach the target matrix 
\begin{equation}
  \I{3} = \ithree
\end{equation}
we can measure how close we came. In fact this line of reasoning opens up a vital property of the SVD: it enables us to quantify how close a target matrix is to the nearest matrix of lower rank.


\endinput


\endinput