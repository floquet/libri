%%%%%%%%%%%
\subsection{Properties of the pseudoinverse}
The bulk of the paper lies in lemma 1 and its proof.

%%%%%%%%%%%%%%
%%%%%%%%%%%%%%
\subsubsection{Lemma 1.1: pseudoinverse of the pseudoinverse}
The pseudoinverse of the pseudoinverse is the original matrix:
\begin{equation}
  \paren{\Ap}^{\sym} = \A{}.
  \label{eqn:mpp:app}
\end{equation}
The inverse of a unitary matrix is the Hermitian conjugate matrix; this is one property which makes unitary matrices so desireable. The critical observation here the repeated application of the pseudoinverse to the $\sig{}$ matrix. In analog to equation \eqref{eqn:mpp:app} we have
%\begin{equation*}
%  \sig{} = \sbf{} 
%    \Rightarrow \quad \sig{\sym} = \sbg{-1}
%    \Rightarrow \quad \paren{\sig{\sym}}^{\sym}  = \sbf{}
%\end{equation*}
\begin{equation}
  \paren{\sig{\ssym}}^{\sym} = \sig{}.
  \label{eqn:mpp:spp}
\end{equation}
This is a straightforward exercise in $\sig{}$ matrix gymnastics
%%
\begin{table}[htdp]
%\caption[Inverting the $\sig{}$ matrix]{default}
\begin{center}
\begin{tabular}{lllllll}
%
    $\sig{} = \sbf{}$ & $\Rightarrow$ \\
%
  & $\sig{\ssym} = \sbg{-1}$ & $\Rightarrow$ \\
%
 && $\paren{\sig{\ssym}}^{\sym}  = \sbf{}$
\end{tabular}
\end{center}
\label{tab:sigma gymnastics}
\end{table}\\
The process applied to the $\sig{}$ matrix to form the pseudoinverse is to transpose the sabot and invert the diagonal matrix $\ess{}$.
%%

%%%%%%%%%%%%%%
%%%%%%%%%%%%%%
\subsubsection{Lemma 1.2: Hermitian conjugate of the pseudoinverse}
The pseudoinverse of the Hermitian conjugate matrix is the Hermitian conjugate of the pseudoinverse matrix:
\begin{equation}
  \Bigl(\A{*}\Bigr)^{\sym} = \paren{\A{\sym}}^{*}.
\end{equation}
The first half of the identity is this
\begin{equation}
  \paren{\A{*}}^{\sym} = \paren{\svdt{*}}^{\sym} = \svds{*}{-T}
\end{equation}
where
\begin{equation}
  \sig{-T} = \sbg{-1}
\end{equation}
represents the inverse matrix $\ess{-1}$ in a transposed sabot matrix.

%%%%%%%%%%%%%%
%%%%%%%%%%%%%%
\subsubsection{Lemma 1.3: pseudoinverse and standard inverse}
If the matrix $\A{}$ is nonsingular then the pseudoinverse matrix is the standard matrix inverse:
\begin{equation}
  \A{-1} = \Ap.
\end{equation}
If $\A{-1}$ exists, then $\rho = m = n$. Therefore the \asvd \ is given as
\begin{equation}
  \A{} = \svdblockbf.
\end{equation}
The pseudoinverse for this matrix is
\begin{equation}
  \Ap = \mppa.
\end{equation}
It should be apparent that this matrix acts as the standard inverse:
\begin{equation}
  \ApA = \AAp = \I{m}.
\end{equation}


%%%%%%%%%%%%%%
%%%%%%%%%%%%%%
\subsubsection{Lemma 1.4: pseudoinverse of matrix products}
Given conformable matrices $\A{}$ and $\B{}$,
\begin{equation}
  \paren{\A{}\,\B{}}^{\sym} = \B{\sym}\Ap.
\end{equation}
This proof relies upon the fact that the product of two unitary matrices is also unitary. To see this, let the matrices $\C{}$ and $\D{}$ be unitary. The Hermitian conjugate of the product matrix is this
\begin{equation}
  \paren{ \C{}\D{} }^{*} = \D{*}\C{*}.
\end{equation}
Therefore
\begin{equation}
  \paren{ \C{}\D{} }^{*} \paren{\C{}\D{}}  = \I{}
\end{equation}
which establishes that the product matrix is also unitary.

Going back to the premise, we state the \asvd s for the target matrices
\begin{equation}
  \begin{split}
    \A{} &= \U{}_{A}\, \sig{}_{A}\, \V{*}_{A}, \\
    \B{} &= \U{}_{B}\, \sig{}_{B}\, \V{*}_{B}.
  \end{split}
\end{equation}


%%%%%%%%%%%%%%
%%%%%%%%%%%%%%
\subsubsection{Lemma 1.5: pseudoinverse and Hermitian conjugate}
\begin{equation}
  \Bigl(\wx{*}\Bigr)^{\sym} = \Ap \paren{\Ap}^{*} 
\end{equation}



%%%%%%%%%%%%%%
%%%%%%%%%%%%%%
\subsubsection{Lemma 1.6: unitary matrices}
Given unitary matrices $\C{}$ and $\D{}$ the triple product has the following property
\begin{equation}
  \paren{\C{} \A{}\, \D{}}^{\sym} = \D{*} \Ap \C{*}
\end{equation}


%%%%%%%%%%%%%%
%%%%%%%%%%%%%%
\subsubsection{Lemma 1.7: pseudoinverse of an unitary decomposition}
Given a target matrix $\A{}$ with the following resolution
\begin{equation}
  \A{} = \sum_{i} \A{}_{i} 
\end{equation}
with the constraint
\begin{equation}
  \A{}_{j}\A{*}_{k} = \A{*}_{j}\A{}_{k} = 0, \quad \text{ for } j\ne k.
\end{equation}
Then the pseudoinverse can be written as the sum of pseudoinverses
\begin{equation}
  \Ap = \sum_{i} \Ap_{i}
\end{equation}


%%%%%%%%%%%%%%
%%%%%%%%%%%%%%
\subsubsection{Lemma 1.8: pseudoinverse of normal matrices}
Given a normal matrix $\A{}$, that is a matrix where
\begin{equation}
  \wx{*} = \wy{*},
\end{equation}
then we know that
\begin{equation}
  \ApA = \AAp,
\end{equation}
and that for $n\in\mathbb{Q}$
\begin{equation}
  \Bigl(\textbf{A}^{n}\Bigr)^{\sym}  = \paren{\Ap}^{n}.
\end{equation}



%%%%%%%%%%%%%%
%%%%%%%%%%%%%%
\subsubsection{Lemma 1.9: pseudoinverse and rank}
The matrices $\A{}$, $\A{*}$, $\wx{*}$, and $\ApA$ all have a rank given by
\begin{equation}
  \rho = \text{tr} \paren{\ApA}. 
\end{equation}




\endinput