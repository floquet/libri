\chapter{Post Mortem II}

The previous chapter introduced the general method for resolving a matrix into its \svdl. This method involved finding the singular values by resolving the eigensystem of a product matrix. The next section explores the geometry of the eigenvalues and motivates the perception that the singular values are a set of scale factors.

However, because of the weight of the details in the last chapter, we will exhibit another example. This will provide more insight into the geometry of the singular values.

The trouble with the target matrix in the preceding example is that while it does have enough linearly independent column vectors to span the domain it lacks enough linearly independent row vectors to span the codomain. Therefore there is a left inverse but no right inverse. Therefore the linear system can't be solved using obvious methods again we need the SVD.
 
%%
\section{Review}
But first we will review the general method for \svdl \ this time using a much simpler example. This example will illuminate the decomposition scheme and leave us with a helpful geometric example.

\begin{landscape}
\thispagestyle{empty}
%%
\subsection{A quick review with a simpler matrix}
\begin{equation}
\boxed{
  \begin{array}{ccccccccccccc}
  \A{}&\longrightarrow&\A{*}&\longrightarrow&\W{x}&\longrightarrow&\lambda\paren{\W{x}}&\longrightarrow&\lst{\sigma_{k}}&\longrightarrow&\sig{}\\
  &&&&\downarrow\\
  &&&&\lst{x_{k}}&\longrightarrow&\lst{x_{k}}^{\perp}&\longrightarrow&\X{}\\
  &&&&&&&&\downarrow\\
  &&&&&&&&\lst{y_{k}}&\longrightarrow&\lst{y_{k}}^{\perp}&\longrightarrow&\Y{}\\
  \end{array}
 }
\end{equation}
Flow chart for a typical \svdl. This layout shows the simplicity of the underlying decomposition process. In general, it is the eigenvalue problem thats gives the SVD a reputation as being difficult. Readers more comfortable with a cookbook format may prefer table \eqref{tab:3:input}.

Essentially we the process involves finding the eigenvectors of the product matrix $\W{x}$ and completing this with a set of null space vectors to form the basis matrix $\X{}$. The singular values are the square root of the eigenvalues of $\W{x}$ and they are used to construct $\sig{}$. Given the matrices $\W{x}$ and $\sig{}$ one can construct the final matrix $\Y{}$.

A sample process with a full rank matrix is shown below.
\begin{equation*}
  \begin{array}{ccccccccccccc}
  \mat{rr}{1&2\\-1&2}&\rightarrow&\mat{rr}{1&1\\2&-2}&\rightarrow&\mat{cc}{2&0\\0&8}&\rightarrow&\lst{8,2}&\rightarrow&\lst{2\sqrt{2},\sqrt{2}}&\rightarrow&\sig{}=\sqrt{2}\mat{cc}{2&0\\0&1}\\
  &&&&\downarrow\\
  &&&&\lst{\mat{c}{0\\1},\mat{c}{1\\0}}&\rightarrow&\lst{\emptyset}&\rightarrow&\X{}=\mat{cc}{0&1\\1&0}\\
  &&&&&&&&\ \; \downarrow\\
  &&&&&&&&\frac{1}{\sqrt{2}}\lst{\mat{c}{1\\1},\mat{r}{-1\\1}}&\rightarrow&\lst{\emptyset}&\rightarrow&\Y{}=\frac{1}{\sqrt{2}}\mat{cr}{1&-1\\1&1}\\
  \end{array}
  \label{eq:pmII}
\end{equation*}
\end{landscape}

The new target matrix $\A{}\in\real{\by{2}{2}}_{2}$ is decomposed as
\begin{equation}
  \begin{array}{ccccc}
    \A{} &=& \Y{} & \sig{} & \X{T}, \\
    \mat{rr}{1&2\\-1&2} &=& \stwo\mat{cr}{1&-1\\1&1}
    & \sqrt{2}\mat{cc}{2&0\\0&1}
    & \mat{cc}{0&1\\1&0}.
  \end{array}
  \label{eq:pmII:A}
\end{equation}
This example \eqref{eq:pmII} is shown against the flow chart in \eqref{eq:gen:flow}. Since this matrix has full rank there are no null spaces this is a very basic case. The matrices involved in the decomposition are all squares as depicted in figure \eqref{fig:full_rank}.

\begin{figure}[htbp] %  figure placement: here, top, bottom, or page
   \centering
%   \includegraphics[width=2in]{pdf/post_mortemII/svd_02_02_02} 
   \includegraphics[ ]{pdf/post_mortemII/C222} 
   \caption[The matrices in nonsingular decompositions are all square]{The matrices in nonsingular decompositions are all square. There are no null space vectors in either domain matrix. THe matrix of singular values is a full diagonal matrix.}
   \label{fig:full_rank}
\end{figure}

\section{SVD by inspection}
Notice another way to decompose the matrix in example.
\begin{enumerate}
\item Observe that 
\begin{equation}
  \A{}=\mat{rr}{1&2\\-1&2}
\end{equation}
has full rank. Therefore there will be no sabot matrix.
\item Compute the product matrix
\begin{equation}
  \A{T}\A{} = \W{x} = \mat{cc}{2&0\\0&8}.
\end{equation}
\item The eigenvalue spectrum is then
\begin{equation}
  \lambda\paren{\W{x}} = \lst{8,2}.
\end{equation}
\item The matrix of singular values is then
\begin{equation}
  \ess{} = \sqrt{2}\mat{cc}{2&0\\0&1}
\end{equation}
Therefore
\begin{equation}
  \sig{} = \ess{} = \sqrt{2}\mat{cc}{2&0\\0&1}.
\end{equation}
\item Because the diagonal elements are not arranged in descending order, we need a permutation matrix. Postulate the simplest possible permutation matrix for the domain:
\begin{equation}
  \X{} = \ktwo.
\end{equation}
\item Solve the two equations 
\begin{equation*}
  \A{}\X{}_{*,k} = \sigma_{k} \Y{}_{*,k}, \quad k=\lst{1,2}
\end{equation*}
to discover that
\begin{equation}
  \Y{} = \stwo \mat{rr}{1&-1\\1&1}.
\end{equation}
\end{enumerate}

\endinput
\section[Left and right inverses]{Left and right inverses: a first look}
\label{lrfirst}

This section whets the appetite for a topic which will be developed later in the the section on the Moore-Penrose pseudoinverse, \S\eqref{sec:chiral}. For now, we present a basic observation. One may suspect that when we generalize the matrix inverse, we will also generalize the properties of the matrix inverse. The careworn requirement is that a square nonsingular matrix $\A{}$ must satisfy
\begin{equation}
  \A{-1}\A{} = \A{}\,\A{-1} = \I{m}.
\end{equation}
However for the pseudoinverse, the sizes of the resultant matrices don't even match. For the prototypical $\Acc{m}{n}$:
\begin{equation}
  \begin{array}{rcl}
    \leftinv &\in&\cmplx{\by{n}{n}},\\
    \rightinv &\in&\cmplx{\by{m}{m}}.
  \end{array}
\end{equation}

Is it possible only one of the matrix products $\leftinv $ or $\rightinv$ might be an identity matrix?


The first step is to assemble the pseudoinverse matrix:
\begin{equation}
  \begin{split}
    \mpgia{T} \\
      &=
      \left[
\begin{array}{ cr >{\columncolor{ltgray}}r }
  \frac{1}{\sqrt{30}} & \frac{ 1}{\sqrt{6}} & \frac{-2}{\sqrt{5}}\\
  \frac{5}{\sqrt{30}} & \frac{-1}{\sqrt{6}} & 0 \\
  \frac{2}{\sqrt{30}} & \frac{ 2}{\sqrt{6}} & \frac{ 1}{\sqrt{5}}\\
\end{array}
\right]  
  \mat{cc}
  {
  \frac{1}{\sqrt{15}} & 0\\
  0 & \frac{1}{\sqrt{3}}\\\hline
  0 & 0
  }
  \frac{1}{\sqrt{2}}
  \mat{rr}{1 & 1\\-1 & 1}\\
  &=\frac{1}{15}
  \mat{rr}
  {
 -2 & 3 \\
  5 & 0 \\
 -4 & 6
  }.
  \end{split}
\end{equation}

What is the action of the pseudoinverse matrix when it pre- and post-multiplies the target matrix?
%%
\begin{equation}
  \begin{array}{rcccc}
    \leftinv&=&
    \frac{1}{15}
  \mat{rr}
  {
 -2 & 3 \\
  5 & 0 \\
 -4 & 6
  }
  \mat{ccc}
  {
  0 & 3 & 0 \\
  1 & 2 & 2
  } &=&
  \frac{1}{5}
  \mat{ccc}
  {
 1 & 0 & 2 \\
 0 & 5 & 0 \\
 2 & 0 & 4
  },\\
    \rightinv&=&
  \mat{ccc}
  {
  0 & 3 & 0 \\
  1 & 2 & 2
  } 
    \frac{1}{15}
    \mat{rr}
  {
 -2 & 3 \\
  5 & 0 \\
 -4 & 6
  }
&=& \itwo.
  \end{array}
  \label{eq:gen:lr}
\end{equation}

In this case with full row rank the pseudoinverse is also a \index{right inverse}right inverse. In the chapter on the pseudoinverse we will uncover a geometric interpretation for the product of a matrix and its pseudoinverse. For now we simply note the following behaviors:
\begin{table}[htdp]
\begin{center}
\begin{tabular}{lll}
rank condition   & \ parameters \ & \ inverse condition\\\hline
full row rank    & \ $\rho = m $  & \ $\A{+} = \AinvR$ $\phantom{A^{-1^{-1^{-1}}}}$ \\[3pt]
full column rank & \ $\rho = n $  & \ $\A{+} = \AinvL$ \\[3pt]
full row and column rank \ & \ $\rho = m = n $ \ & \ $\A{+} = \AinvL = \AinvR = \A{-1}$ \\[13pt]
\end{tabular}
\end{center}
\label{tab:pmii:rank}
\caption{Full rank is the criterion which indicates when the pseudoinverse will behave like a standard inverse. For matrices with full \textit{row} rank, the pseudoinverse is a \textit{right} inverse. For matrices with full \textit{column} rank, the pseudoinverse is a \textit{left} inverse. Of course if the matrix is square and of full rank then the pseudoinverse is the standard inverse.}
\end{table}%
\\
%%%
The formulaically minded may prefer this more mathematical presentation:
\begin{equation}
  \begin{array}{rclrcr}
    \leftinv &=& \I{n}, & \quad \A{}&\in&\cmplx{\by{m}{\textbf{n}}}_{\textbf{n}},\\
    \rightinv &=& \I{m}, & \quad \A{}&\in&\cmplx{\by{\textbf{m}}{n}}_{\textbf{m}},\\
    \leftinv = \rightinv &=& \I{m}, & \quad \A{}&\in&\cmplx{\by{\textbf{m}}{\textbf{m}}}_{\textbf{m}}.\\
  \end{array}
\end{equation}

\section[]{Proximity to the identity}
\label{piproximity}

A quick note before closing. What about the first result in equation \eqref{eq:gen:lr}? Clearly $\leftinv$ is not a left inverse because the product is not an identity matrix. But how ``close'' is this matrix to the identity matrix? We can use the concept of the matrix norm to measure the distance between two matrices. 
\begin{equation}
\normt{\leftinv-\I{m}} = 
\normt{\frac{1}{5}
  \mat{ccc}
  {
 1 & 0 & 2 \\
 0 & 5 & 0 \\
 2 & 0 & 4
  }
  -
  \ithree} = 1.
\end{equation}
We will see this last result a few more times.

The point is that while we did not reach the target matrix 
\begin{equation}
  \I{3} = \ithree
\end{equation}
we can measure how close we came. In fact this line of reasoning opens up a vital property of the SVD: it enables us to quantify how close a target matrix is to the nearest matrix of lower rank.


\endinput

\section{Anatomy of the SVD}
To close out the chapter we will summarize the important findings on the anatomy of the \svdl \ as seen through central examples. We have seen that the SVD can be viewed in the general case as flowing fromthe solution to the eigensystem for the product matrix $\W{x}$. This is not an absolute rule of course; we opened up in chapter 1 with a class of examples where there was no need to find the eigensystem. We have discussed cases where $m<n$ and it is more expedient to find the eigenvalue to $\W{y}$. This kind of variability makes the decomposition a fascinating subject. But the framework best used to order the different variants is the framework of the eigensystem solution. 

For the thin SVD, we must find the nonzero eigenvalues and their corresponding eigenvectors. For a full SVD we will need to resolve an orthogonal basis for the null vectors, the vectors corresponding to the zero eigenvalues.

The cases on the following pages consider the standard three examples. The target matrix $\A{}$ has either:
\begin{enumerate}
\item rank deficiency in both rows and columns,
\item full row rank with column rank deficiency,
\item full row and column rank.
\end{enumerate}

Notice how the eigenvectors of the product matrices appear in the domain matrices. We show that the general method for finding the column vectors of $\Y{}$ involves solve equation \eqref{ax=sy}. This will guarantee that the signs of the terms are correct.  Notice, though, the we can construct eigenvectors to match the column vectors of $\Y{}$.

%%%
\break
\subsection[Anatomy of the SVD: row, column rank deficiency]{Anatomy of the SVD: row and column rank deficiency}

\begin{equation*}
  \begin{array}{ccccc}
    \A{} &=& \Y{} & \sig{} & \X{T} \\
    \Aexample & = & \Yshade & \Sigmaexampleb & \Xtshade
  \end{array}
\end{equation*}

Here the target matrix $\Arrr{3}{2}{1}$ has $m = 3$ rows, $n = 2$ columns and matrix rank $\rho = 1$. This matrix has rank deficiency in both the rows and the columns. All components needed for the SVD come from the product matrices $\prdm{T}$ and $\prdmm{T}$. The square root of the nonzero eigenvalues comprise the diagonal entries of $\sig{}$.

\begin{table}[htdp]
\begin{center}
\begin{tabular}{c|c}
product matrix: & product matrix: \\
$\W{x} = \A{T}\A{} = 3 \mat{rr}{1&-1\\-1&1}$ &
$\W{y} = \A{}\A{T} = 2 \mat{rrr}{1&-1&1\\-1&1&-1\\1&-1&1}$ \\[30pt]
%%%
eigenvalues: & eigenvalues: \\
$\lambda\paren{\W{x}} = \lst{6,0}$ &
$\lambda\paren{\W{y}} = \lst{6,0,0}$ \\[20pt]
%%%
eigenvectors: & eigenvectors: \\
$\lst{\mat{r}{-1\\1},\mat{>{\columncolor{ltgray}}c}{1\\1}}$ &
$\lst{\mat{r}{-1\\1\\-1},\mat{>{\columncolor{ltgray}}r}{0\\1\\1},\mat{>{\columncolor{ltgray}}r}{1\\0\\-1}}$ \\[30pt]
%%%
domain matrix: & codomain matrix: \\
$\X{} = \mat{c|>{\columncolor{ltgray}}c}{\stwo \mat{r}{-1\\1} & \stwo \mat{>{\columncolor{ltgray}}r}{1\\1}}$ &
$\Y{} = \mat{c|>{\columncolor{ltgray}}c>{\columncolor{ltgray}}c}{\sthree \mat{r}{-1\\1\\-1} & \stwo \mat{r}{0\\1\\1} & \stwo \mat{r}{2\\1\\-1}}$\\[25pt]
%%%
\end{tabular}
\end{center}
\label{default}
%\caption{Origin of the components in a \svdl. The singular values are the square root of the nonzero eigenvalues. The domain matrices are shown in terms of column vectors to emphasize the connection to the eigenvectors of the product matrices.}
\end{table}%

\begin{equation*}
  \ess{} = \mat{c}{6}, \qquad \sig{}= \mat{c|c}{\ess{} & \zero \\\hline \zero & \zero} = \Sigmaexampleb
\end{equation*}

%    Include unnumbered chapters (preface, acknowledgments, etc.) here.
\subsection{Anatomy of the SVD: full row rank}

\begin{equation*}
  \begin{array}{ccccc}
    \A{} &=& \Y{} & \sig{} & \X{T} \\
    \matrixbravo & = & \matrixbravoY & \matrixbravosigma & \matrixbravoXt
  \end{array}
\end{equation*}

Here the target matrix $\Arrr{2}{3}{2}$ has $m=2$ rows, $n=3$ columns and matrix rank $\rho =2$ (full column rank).

\begin{table}[htdp]
\begin{center}
\begin{tabular}{c|c}
product matrix: & product matrix: \\
$\W{x} = \A{T}\A{} = \mat{ccc}{1&2&2\\2&13&4\\2&4&4}$ &
$\W{y} = \A{}\A{T} = \mat{cc}{9&6\\6&9}$ \\[30pt]
%%%
eigenvalues: & eigenvalues: \\
$\lambda\paren{\W{x}} = \lst{15,3,0}$ &
$\lambda\paren{\W{y}} = \lst{15,3}$ \\[20pt]
%%%
eigenvectors: & eigenvectors: \\
$\lst{\mat{c}{1\\5\\2},\mat{r}{1\\-1\\2},\mat{>{\columncolor{ltgray}}r}{-2\\0\\1}}$ &
$\lst{\mat{c}{1\\1},\mat{r}{-1\\1}}$ \\[30pt]
%%%
domain matrix: & codomain matrix: \\
$\X{} = \mat{cc|>{\columncolor{ltgray}}c}{\frac{1}{30}\mat{c}{1\\5\\2} & \ssix \mat{r}{1\\-1\\2} & \sfive \mat{r}{-2\\0\\1}}$ &
$\Y{} = \stwo \mat{cc}{\mat{c}{1\\1} & \mat{r}{-1\\1}}$\\[25pt]
%%%
\end{tabular}
\end{center}
\label{default}
%\caption{Origin of the components in a \svdl. The singular values are the square root of the nonzero eigenvalues. The domain matrices are shown in terms of column vectors to emphasize the connection to the eigenvectors of the product matrices.}
\end{table}%

\begin{equation*}
  \ess{} = \mat{cc}{\sqrt{15}&0\\0&\sqrt{3}}, \qquad \sig{}= \mat{c|c}{\ess{} & \zero} = \matrixbravosigma
\end{equation*}

%%%%
\subsection[Anatomy of the SVD: full row, column rank]{Anatomy of the SVD: full row and column rank}

\begin{equation*}
  \begin{array}{ccccc}
    \A{} &=& \Y{} & \sig{} & \X{T} \\
    \matrixalpha & = & \matrixalphaY & \matrixalphasigma & \matrixalphaXt
  \end{array}
\end{equation*}

The target matrix $\Arrr{2}{2}{2}$ has $m=2$ rows, $n=2$ columns and matrix rank $\rho = 2$ (full column rank, full row rank). Notice how the eigenvalues are presented in the target matrix $\W{x}$. This is a diagonal matrix and it is customary (but not obligatory) to read the eigenvalues from the diagonal as $\lst{\lambda_{1}, \lambda_{2}}$. When we order the singular values we must reorder the eigenvectors as well.

\begin{table}[htdp]
\begin{center}
\begin{tabular}{c|c}
product matrix: & product matrix: \\
$\W{x} = \A{T}\A{} = \mat{cc}{2 & 0\\0 & 8}$ &
$\W{y} = \A{}\A{T} = \mat{cc}{5 & 3 \\ 3 & 5}$ \\[30pt]
%%%
eigenvalues: & eigenvalues: \\
$\lambda\paren{\W{x}} = \lst{2,8}$ &
$\lambda\paren{\W{y}} = \lst{2,8}$ \\[20pt]
%%%
eigenvectors: & eigenvectors: \\
$\lst{\mat{c}{0\\1}, \mat{c}{1\\0}}$ &
$\lst{\mat{r}{1\\-1},\mat{c}{1\\1}}$ \\[30pt]
%%%
domain matrix: & codomain matrix: \\
$\X{} = \mat{cc}{\xhatt & \yhatt}$ &
$\Y{} = \mat{cc}{\stwo\mat{c}{1\\1} & \stwo\mat{r}{1\\-1}}$\\[25pt]
%%%
\end{tabular}
\end{center}
\label{default}
%\caption{Origin of the components in a \svdl. The singular values are the square root of the nonzero eigenvalues. The domain matrices are shown in terms of column vectors to emphasize the connection to the eigenvectors of the product matrices.}
\end{table}%

\begin{equation*}
  \ess{} = \sqrt{2} \mat{cc}{2 & 0 \\ 0 & 1}, \qquad \sig{}= \mat{c}{\ess{}} = \matrixalphasigma
\end{equation*}\endinput

\endinput