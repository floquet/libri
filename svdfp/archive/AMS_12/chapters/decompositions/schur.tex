\section{Schur decomposition}

%%
\subsection{Theory}
The Schur decomposition deals only with square matrices. So given 
\begin{equation}
  \A{}\in\cmplx{\by{m}{m}}_{\rho}
\end{equation}
the Schur theorem states that there is a unitary matrix $\U{}$ that satisfies the similarity transformation
\begin{equation}
  \U{*}\A{}\,\U{} = \T{}
\end{equation}
with the matrix $\T{}$ being upper triangular.

In conversational mathematics
\begin{quote}
  Every square matrix is unitarily similar to an upper-triangular matrix.
\end{quote}

%%
\subsection{Example}
You can check that
\begin{equation}
  \begin{array}{ccccc}
    \A{}&=&\U{}&\T{}&\U{*}\\
    \frac{1}{2}\mat{cc}{2-i & -1 \\ i & 2 + i}&=&
    \frac{1}{\sqrt{2}}\mat{rr}{-i&1\\i&1}&
                      \mat{cc}{1&1\\0&1}&
    \frac{1}{\sqrt{2}}\mat{rr}{i&-i\\1&1}.
  \end{array}
  \label{eq:decomp:schur}
\end{equation}

%%
\subsection{Application}
Solve a simple linear system using the Schur decomposition above in equation \eqref{eq:decomp:schur}. The linear system is the following
\begin{equation}
  \begin{array}{cccc}
    \A{}&x&=&b\\
    \frac{1}{2}\mat{cc}{2-i & -1 \\ i & 2 + i}&\mat{r}{x_{1}\\x_{2}}&
    =&\mat{r}{1\\1}
  \end{array}
  \label{eq:schur:app}
\end{equation}

The inverse of the target matrix is quickly assembled from the Schur decomposition:
\begin{equation}
  \A{-1}=\paren{\U{}\,\T{}\,\U{*}}^{-1}=\paren{\U{*}}^{-1}\paren{\T{}}^{-1}\paren{\U{}}^{-1}=\U{}\,\T{-1}\U{*}.
\end{equation}

The solution to equation \eqref{eq:schur:app} follows after the inversion of the matrix $\T{}$. Specifically we see that
\begin{equation}
  \begin{split}
    x &= \A{-1}b\\
      &= \U{}\,\T{-1}\U{*} b\\
      &= \frac{1}{\sqrt{2}}\mat{rr}{-i&1\\i&1}
                      \mat{rr}{1&-1\\0&1}
    \frac{1}{\sqrt{2}}\mat{rr}{i&-i\\1&1}\mat{c}{1\\1}\\
    & = \mat{c}{1+i\\1-i}.
  \end{split}
\end{equation}

%%
\subsection{Connection to the SVD}
Go back to equation \eqref{eq:A}. If we have the \svdl \ then we can assemble the Schur decomposition of the product matrices. These matrices are
\begin{equation*}
  \begin{split}
    \W{x} &= \A{T}\A{} = \Atexample\Aexample = \mat{rrr}
    {
  3 & -3  \\
 -3 &  3},\\
    \W{y} &= \A{}\A{T} = \Aexample\Atexample = \mat{rrr}
    {
  2 & -2 &  2 \\
 -2 &  2 & -2 \\
  2 & -2 &  2}.
  \end{split}
\end{equation*}

To refresh, the \svdl \ for the target matrix is
\begin{equation}
  \begin{split}
    \svda{T} \\
    \Aexample &= \Yshade\Sigmaexampleb\Xshade.
  \end{split}
\end{equation}

In terms of the \svdl \ the first product matrix takes the form
\begin{equation}
  \W{x} = \A{T}\A{} = \paren{\svdt{T}}\paren{\svd{T}} = \Y{}\,\sig{T}\sig{}\,\Y{T}.
\end{equation}
This is the Schur decomposition with the unitary matrix played by the codomain matrix and the upper triangular matrix given as
\begin{equation}
  \T{} = \sig{T}\,\sig{}.
\end{equation}
In the most general case with complex domain matrices the relationship between the SVD and Schur decomposition is
\begin{equation}
  \begin{split}
    \W{x} &= \U{}\,\T{}\,\U{*}\\
      &= \Y{}\,\sig{}\,\sig{T}\,\Y{T}.
  \end{split}
\end{equation}
As you might imagine the other product matrix is breaks down as follows
\begin{equation}
  \begin{split}
    \W{y} &= \U{}\,\T{}\,\U{*}\\
      &= \X{}\,\sig{T}\sig{}\,\X{*}.
  \end{split}
\end{equation}
The triangular matrices of interest are these
\begin{equation}
  \begin{split}
    \sig{T}\sig{} &= \mat{c|c}{6&0\\\hline0&0}, \\
    \sig{}\sig{T} &= \mat{c|cc}{6&0&0\\\hline0&0&0\\0&0&0}. \\
  \end{split}
\end{equation}

Given the SVD you can assemble the Schur decomposition for either product matrix. However, both Schur decompositions are needed to gather an SVD.

The Schur decomposition is not unique. You can verify that this version also provides the correct product:
\begin{equation}
  \W{x} = \U{}\,\T{}\,\U{*} = 
  \mat{>{\columncolor{ltgray}}rr>{\columncolor{ltgray}}r}
  {
  -\frac{2}{\sqrt{6}} & -\frac{1}{\sqrt{3}} & 0 \\
  -\frac{1}{\sqrt{6}} &  \frac{1}{\sqrt{3}} & \frac{1}{\sqrt{2}} \\
   \frac{1}{\sqrt{6}} & -\frac{1}{\sqrt{3}} & \frac{1}{\sqrt{2}}
  }
  \mat{ccc}
  {
  0 & 0 & 0 \\
  0 & 6 & 0 \\
  0 & 0 & 0
  }
  \mat{rrr}
  {
  \rowcolor{ltgray}
  -\frac{2}{\sqrt{6}} & -\frac{1}{\sqrt{6}} &  \frac{1}{\sqrt{6}} \\      
  -\frac{1}{\sqrt{3}} &  \frac{1}{\sqrt{3}} & -\frac{1}{\sqrt{3}} \\
  \rowcolor{ltgray}
   0 &                   \frac{1}{\sqrt{2}} &  \frac{1}{\sqrt{2}}
  }.
\end{equation}

\endinput