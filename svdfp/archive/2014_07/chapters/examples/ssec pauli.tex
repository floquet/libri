%%%%%%%%%%%%%%%%%%%%%%%%%%%%%%
\subsection{The Pauli matrices}
The Pauli spin matrices $\sigma_{k}$ are a set of three $2 \times 2$ matrices which arise in quantum mechanics in the treatment of spin.

The singular value decompositions are shown in table \eqref{tab:pauli:decomposition}.
%%
\begin{table}[htdp]
\caption[Simple decompositions of the Pauli matrices]{Simple decompositions of the Pauli matrices.}
\begin{center}
\begin{tabular}{ccccccc}
%
 && $\A{}$ & = & $\U{}$ & $\sig{}$& $\V{*}$ \\\hline
%
 $\sigma_{1}$ & = & $\mat{cc}{ 0 & 1 \\ 1 & 0 }$  & = & $\I{2}$ &  $\I{2}$ &  $\K{2}$ \\
%
 $\sigma_{2}$ & = & $\mat{cr}{ 0 & -i \\ i & 0 }$ & = & $\mat{cr}{ \aut & \auo }$ & $\I{2}$& $\K{2}$ \\
%
 $\sigma_{3}$ & = & $\mat{cr}{ 1 & 0 \\ 0 & -1 }$ & = & $\mat{rr}{ \aut & \auo }$ & $\I{2}$ & $\K{2}$
%
\end{tabular}
\end{center}
\label{tab:pauli:decomposition}
\end{table}
%%
%
The permutation matrices $\PP{}$ are versions of the identity matrix with column vectors reordered. For examples,
\begin{equation}
  \PP{1,2} = \I{2}, \quad \PP{2,1} = \mat{cc}{ 0 & 1 \\ 1 & 0 }.
\end{equation}
Notice the incidental equality
\begin{equation}
   \PP{2,1} = \sigma_{1}.
\end{equation}


{\it{Observations:}} The singular values are always one, therefore these matrices are unitary.
The matrices are involutary
\begin{equation}
  \sigma_{k}^{2} = \I{2}, \quad k = 1,2,3.
\end{equation}
They satisfy the anticommutation relationship
\begin{equation}
  \lst{\sigma_{j}, \sigma_{k}} = \sigma_{j} \sigma_{k} + \sigma_{k} \sigma_{j} = 2 \delta_{j,k} \I{2}, \quad j,k = 1,2,3.
\end{equation}
Mystery equation
\begin{equation}
   \sigma_{j} \sigma_{k} = \delta_{j,k} \I{2} + \eps_{jkl} \sigma_{l} .
\end{equation}
These matrices, along with the identity, form a basis for $\cmplx{\byy{2}}$
\begin{equation}
  A = \alpha_{0} \I{2} + \sum_{k=1}^{3} \alpha_{k} \sigma_{k}.
\end{equation}
For example,
\begin{equation}
  \mat{rr}{ 1+i & -1+i \\ -1-i & -1+i } = i \I{2} - \sigma_{1} - \sigma_{2} + \sigma_{3}
\end{equation}

\endinput