\section{\label{sec:optimal dof}Optimal degree of fit}

Nature selects a functional form and Man guesses a functional form.

To what end do we strive? To interpolate or extrapolate? To determine the gradient or the Laplacian? To determine the first or second integral?

Let us turn our attention to having a data set and finding the optimal degree of fit.

% SUBSECTION
\subsection{Bevington data}  %  SS  SS  SS  SS  SS
What is going on with the font sizes?

\input{\pathfigures polynomials/"fig bevington r2"}  %  <  <  <  <  <  <  <  <  <  <  <  <
\input{\pathfigures polynomials/"fig bevington k"}   %  <  <  <  <  <  <  <  <  <  <  <  <
\input{\pathtables polynomials/"tab bevington amplitudes"}    %  <  <  <  <  <  <  <  <  <  <  <  <
\input{\pathtables polynomials/"tab bevington fit and residuals"}   %  <  <  <  <  <  <  <  <  <  <  <  <
\input{\pathtables polynomials/"tab bevington amplitudes graphs"}   %  <  <  <  <  <  <  <  <  <  <  <  <
\input{\pathtables polynomials/"tab bevington spectra"}    %  <  <  <  <  <  <  <  <  <  <  <  <
\input{\pathfigures polynomials/"fig bevington spectra"}   %  <  <  <  <  <  <  <  <  <  <  <  <

\endinput  %  -  -  -  -  -  -  -  -  -  -  -  -  -  -  -  -  -  -  -  -

%\input{\pathchapter "chapter name"/"sec XXX"}  %  <  <  <  <  <  <  <  <  <  <  <  <