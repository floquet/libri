%    Figure insertion; default placement is top; if the figure occupies
%    more than 75% of a page, the [p] option should be specified.
\begin{figure}[t]
	\includegraphics[ width = 5in ]{\pathgraphics "bevington codomain"}
	\caption[Measurement space for Bevington example]{Measurement space $\brnga{}$ for Bevington example. The black vector is $y$, the measured data. The range space $\bur{}$ is spanned by the vectors $[\bur{}]_{1}$ and $[\bur{}]_{2}$; the least squares solution $\bl{\A{}\,a}$ lies within this plane and is represented by the blue line. The difference between the measurement $y$ and the approximation $\bl{\A{}\,a}$ is the residual error vector, shown in red. The black and red lines are actually hyperplanes of dimension 7.}
	\label{fig:bevington codomain}
\end{figure}

\endinput  %  -  -  -  -  -  -  -  -  -  -  -  -  -  -  -  -  -  -  -  -

%\input{\pathfigures "fig XXX"}  %  <  <  <  <  <  <  <  <  <  <  <  <