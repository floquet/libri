\numberwithin{equation}{section}
\numberwithin{figure}{section}
\numberwithin{table}{section}

\setlength\minrowclearance{4pt}

\newtheorem{theorem}{Theorem}[chapter]
\newtheorem{lemma}[theorem]{Lemma}

\theoremstyle{definition}
\newtheorem{definition}[theorem]{Definition}
\newtheorem{example}[theorem]{Example}
\newtheorem{xca}[theorem]{Exercise}

\theoremstyle{remark}
\newtheorem{remark}[theorem]{Remark}

\numberwithin{section}{chapter}
\numberwithin{equation}{section}

%%
\providecommand{\abs}[1]    {\left|#1\right|}
\providecommand{\norm}[1]   {\left\lVert#1\right\rVert}
\providecommand{\normo}[1]  {\left\lVert#1\right\rVert_{1}}
\providecommand{\normt}[1]  {\left\lVert#1\right\rVert_{2}}
\providecommand{\norminf}[1]{\left\lVert#1\right\rVert_\infty}
\providecommand{\normp}[1]  {\left\lVert#1\right\rVert_{p}}
\providecommand{\normf}[1]  {\left\lVert#1\right\rVert_F}

\providecommand{\real}[1]{\mathbb{R}^{#1}}
\providecommand{\cmplx}[1]{\mathbb{C}^{#1}}
\providecommand{\complex}[1] { \mathbb{C}^{\by{m}{n}}_{#1} }

\providecommand{\nll}[1]{\mathcal{N}\paren{#1}}
\providecommand{\rng}[1]{\mathcal{R}\paren{#1}}

\providecommand{\Ain}[1] {\A{\by{m}{n}}_{#1}}
\providecommand{\Ar}[1]  {\A{}\in\real{\by{#1}{#1}}}
\providecommand{\Ac}[1]  {\A{}\in\cmplx{\by{#1}{#1}}}
\providecommand{\Acc}[2] {\A{}\in\cmplx{\by{#1}{#2}}}
\providecommand{\Accmn}  {\Acc{m}{n}}
\providecommand{\Amnr}   {\Accmn_{\rho}}

\providecommand{\svd}[1]{\Y{}\, \Sigma \, \X{ \mathrm{#1} }}
\providecommand{\svdthin}[1]{\widetilde{\Y{}}\, \widetilde{\sig{}}\, \widetilde{\mathbf{X}}^{\mathrm{#1}}}
\providecommand{\svda}[1]{\A{} & = \svd{#1}}
\providecommand{\svdax}[1]{\A{}  = \svd{#1}}
\providecommand{\svdt}[1]{\X{} \, \sig{T} \, \Y{ \mathrm{#1} }}
\providecommand{\mpgi}[1]{\X{} \, \sig{(+)} \, \Y{ \mathrm{#1} }}
\providecommand{\mpgia}[1]{\Ap & = \mpgi{#1}}
\providecommand{\mpgiax}[1]{\Ap = \mpgi{#1}}

\providecommand{\svdtag}[1]{\Y{}_{ #1 }\, \Sigma_{ #1 }\, \X{*}_{ #1 }}
\providecommand{\svdtagA}[1]{\A{}_{ #1 } = \svdtag{#1}}

\providecommand{\A}[1] { \textbf{A}^{\mathrm{#1}} }
\providecommand{\B}[1] { \textbf{B}^{\mathrm{#1}} }
\providecommand{\C}[1] { \textbf{C}^{\mathrm{#1}} }
\providecommand{\D}[1] { \textbf{D}^{\mathrm{#1}} }
\providecommand{\LL}[1]{ \textbf{L}^{\mathrm{#1}} }
\providecommand{\M}[1] { \textbf{M}^{\mathrm{#1}} }
\providecommand{\N}[1] { \textbf{N}^{\mathrm{#1}} }
\providecommand{\pee}[1]{\textbf{P}^{\mathrm{#1}} }
\providecommand{\Q}[1] { \textbf{Q}^{\mathrm{#1}} }
\providecommand{\R}[1] { \textbf{R}^{\mathrm{#1}} }
\providecommand{\T}[1] { \textbf{T}^{\mathrm{#1}} }
\providecommand{\V}[1] { \textbf{V}^{\mathrm{#1}} }
\providecommand{\U}[1] { \textbf{U}^{\mathrm{#1}} }
\providecommand{\X}[1] { \textbf{X}^{\mathrm{#1}} }
\providecommand{\Y}[1] { \textbf{Y}^{\mathrm{#1}} }
\providecommand{\Z}[1] { \textbf{Z}^{\mathrm{#1}} }
\providecommand{\ess}[1]{\textbf{S}^{\mathrm{#1}} }
\providecommand{\J}[2] { \textbf{J}^{#2}_{#1} }
\providecommand{\E}[1] { \textbf{E}_{#1} }
\providecommand{\G}[1] { \textbf{G}_{\mathrm{#1}} }
\providecommand{\I}[1] { \textbf{I}_{#1} }
\providecommand{\K}[1] { \textbf{K}_{#1} }
\providecommand{\W}[1] { \textbf{W}_{\mathrm{#1}} }
\providecommand{\sig}[1] { \Sigma^{\mathrm{#1}} }

\providecommand{\prdm}[1]  {\A{ \mathrm{#1} }\A{}}
\providecommand{\prdmm}[1] {\A{}\,\A{ \mathrm{#1} }}
\providecommand{\wx}[1]    {\X{}\,\sig{T}\,\sig{}\,\X{#1}}
\providecommand{\wy}[1]    {\Y{}\,\sig{}\,\sig{T}\,\Y{#1}}
\providecommand{\pra}[1]   {\Y{}\,\J{m}{\rho}\Y{#1}}
\providecommand{\pnap}[1]  {\X{}\,\J{n}{\rho}\,\X{#1}}
\providecommand{\prodx}[2] {\X{}_{#2}\,\X{#1}_{#2}}
\providecommand{\prody}[2] {\Y{}_{#2}\,\Y{#1}_{#2}}

\providecommand{\paren}[1] { \left( #1 \right) }
\providecommand{\inner}[1] { \langle #1 \rangle }
\providecommand{\brac}[1]  { \left[ #1 \right] }
%\providecommand{\list}[1]  { \lbrace #1 \rbrace }
\providecommand{\lst}[1]   { \left\{ #1 \right\} }
\providecommand{\dim}[1]   { \text{dim}\paren{ #1 } }
\providecommand{\rank}[1]  {\text{rank}\paren{#1}}

\providecommand{\mat}[2] {\left[\begin{array}{#1}#2\end{array}\right]}
\providecommand{\rat}[2] {\mathbf{R}^{\mathrm{#2}}\paren{#1}}

\providecommand{\by}[2]  {#1 \times #2}
\providecommand{\bys}[1] {#1 \times #1}
\providecommand{\byt}[2] {\mathit{#1} \times \mathit{#2}}
\providecommand{\bymn}   {\by{m}{n}}
\providecommand{\bymm}   {\bys{m}}

\providecommand{\xrng}{\X{}_{\mathcal{R}}}
\providecommand{\xnll}{\X{}_{\mathcal{N}}}
\providecommand{\yrng}{\Y{}_{\mathcal{R}}}
\providecommand{\ynll}{\Y{}_{\mathcal{N}}}


%% order
\providecommand{\order}[1] { \mathcal{o}\paren{#1} }
\providecommand{\Order}[1] { \mathcal{O}\paren{#1} }

%%
\DeclareMathOperator{\tr}{tr}
\DeclareMathOperator{\mx}{max}
\DeclareMathOperator{\dm}{dim}
\DeclareMathOperator{\adj}{adj}
\DeclareMathOperator{\Arg}{Arg}
\DeclareMathOperator{\sgn}{sign}
\DeclareMathOperator{\spn}{sp}
\DeclareMathOperator{\rref}{rref}

\DeclareMathOperator{\rot}{\textbf{R}\paren{\theta}}

\DeclareMathOperator{\Ap}{\A{+}}
\DeclareMathOperator{\leftinv}{\A{+}\A{}}
\DeclareMathOperator{\rightinv}{\A{}\A{+}}
\DeclareMathOperator{\AinvL}{\A{-1}_{\mathrm{L}}}
\DeclareMathOperator{\AinvR}{\A{-1}_{\mathrm{R}}}
\DeclareMathOperator{\AinvB}{\A{-1}_{\mathrm{L,R}}}

\DeclareMathOperator{\projra} {\mathbf{P}_{\rng{\A{}}}}
\DeclareMathOperator{\projrap}{\mathbf{P}_{\rng{\A{}}^{\perp}}}
\DeclareMathOperator{\projna} {\mathbf{P}_{\nll{\A{}}}}
\DeclareMathOperator{\projnap}{\mathbf{P}_{\nll{\A{}}^{\perp}}}

\DeclareMathOperator{\projrat} {\mathbf{P}_{\rng{\A{T}}}}
\DeclareMathOperator{\projratp}{\mathbf{P}_{\rng{\A{T}}^{\perp}}}
\DeclareMathOperator{\projnat} {\mathbf{P}_{\nll{\A{T}}}}
\DeclareMathOperator{\projnatp}{\mathbf{P}_{\nll{\A{T}}^{\perp}}}

\newcommand{\archetype}{\Aexample = \Yshade \Sigmaexampleb \Xtshade}
\newcommand{\archetypez}{\Aexample &= \Yshade \Sigmaexampleb \Xtshade}

\newcommand{ \svdl }{singular value decomposition}
\newcommand{ \svdp }{singular value decomposition (SVD)}
\newcommand{ \ft }{Fundamental Theorem}
\newcommand{ \ftola }{\ft \ of Linear Algebra}
\newcommand{ \qr }{$\Q{}\,\R{}$ decomposition}

\newcommand{ \ls }  {\A{} x = b}
\newcommand{ \lsa } {\A{} x & = b }
\newcommand{ \zero }{\textbf{0}}
\newcommand{ \one } {\textbf{1}}

\newcommand{ \mv }  {$m-$vector}
\newcommand{ \nv }  {$n-$vector}
\newcommand{ \vv }  {$2-$vector}
\newcommand{ \vvv } {$3-$vector}

%\newcommand{ \pivot } {1 \negthickspace\negthickspace\negmedspace\bigcirc}
\newcommand{ \pivot } {\boxed{1}}
\definecolor{ltgray}{gray}{0.9}
\definecolor{veryltgray}{gray}{0.95}


%% common fractions
\newcommand{\rtwo}  { \frac{1}{2} }
\newcommand{\rthree}{ \frac{1}{3} }
\newcommand{\rfour} { \frac{1}{4} }
\newcommand{\rfive} { \frac{1}{5} }
\newcommand{\rsix}  { \frac{1}{6} }

\newcommand{\stwo}  { \frac{1}{\sqrt{2}} }
\newcommand{\sthree}{ \frac{1}{\sqrt{3}} }
\newcommand{\sfive} { \frac{1}{\sqrt{5}} }
\newcommand{\ssix}  { \frac{1}{\sqrt{6}} }

\newcommand{\nstwo}  { \frac{-1}{\sqrt{2}} }
\newcommand{\nsthree}{ \frac{-1}{\sqrt{3}} }
\newcommand{\nsfive} { \frac{-1}{\sqrt{5}} }
\newcommand{\nssix}  { \frac{-1}{\sqrt{6}} }

%%other definitions
%%
%% vectors
%%
\newcommand{ \xhatt } {\left[
\begin{array}{c}
 1 \\
 0
\end{array}
\right]
}

\newcommand{ \yhatt } {\left[
\begin{array}{c}
  0 \\
  1
\end{array}
\right]
}

\newcommand{ \xhattt } {\left[
\begin{array}{c}
 1 \\
 0 \\
 0
\end{array}
\right]
}

\newcommand{ \yhattt } {\left[
\begin{array}{c}
  0 \\
  1 \\
  0
\end{array}
\right]
}

\newcommand{ \zhattt } {\left[
\begin{array}{c}
  0 \\
  0 \\
  1
\end{array}
\right]
}

\newcommand{ \xtwo } {\left[
\begin{array}{c}
 x \\
 y
\end{array}
\right]
}

\newcommand{ \xthree } {\left[
\begin{array}{c}
 x \\
 y \\
 z
\end{array}
\right]
}

%%
%% matrices
%%
\newcommand{ \itwo } {
\mat{cc}{
 1 & 0 \\
 0 & 1 
}}

\newcommand{ \ithree } {
\mat{ccc}{
 1 & 0 & 0 \\
 0 & 1 & 0 \\
 0 & 0 & 1
}}

\newcommand{ \ifour } {
\mat{cccc}{
 1 & 0 & 0 & 0 \\
 0 & 1 & 0 & 0 \\
 0 & 0 & 1 & 0 \\
 0 & 0 & 0 & 1 \\
}}

\newcommand{ \ktwo } {
\mat{cc}{
 0 & 1 \\
 1 & 0 
}}

\newcommand{ \kthree } {
\mat{ccc}{
 0 & 0 & 1 \\
 0 & 1 & 0 \\
 1 & 0 & 0
}}

\newcommand{ \kfour } {
\mat{cccc}{
 0 & 0 & 0 & 1 \\
 0 & 0 & 1 & 0 \\
 0 & 1 & 0 & 0 \\
 1 & 0 & 0 & 0 \\
}}

\input{common/pauli.tex}
% The 8 Gell-Mann matrices for SU(3)
\newcommand{ \gma } {
\mat{ccc}{
 0 & 1 & 0 \\
 1 & 0 & 0 \\
 0 & 0 & 0
}}
\newcommand{ \gmb } {
\mat{crc}{
 0 & -i & 0 \\
 i & 0 & 0 \\
 0 & 0 & 0
}}
\newcommand{ \gmc } {
\mat{crc}{
 1 & 0 & 0 \\
 0 & -1 & 0 \\
 0 & 0 & 0
}}
\newcommand{ \gmd } {
\mat{ccc}{
 0 & 0 & 1 \\
 0 & 0 & 0 \\
 1 & 0 & 0
}}\newcommand{ \gme } {
\mat{ccr}{
 0 & 0 & -i \\
 0 & 0 & 0 \\
 i & 0 & 0
}}\newcommand{ \gmf } {
\mat{ccc}{
 0 & 0 & 0 \\
 0 & 0 & 1 \\
 0 & 1 & 0
}}\newcommand{ \gmg } {
\mat{ccr}{
 0 & 0 & 0 \\
 0 & 0 & -i \\
 0 & i & 0
}}\newcommand{ \gmh } {
\sthree
\mat{ccr}{
 1 & 0 & 0 \\
 0 & 1 & 0 \\
 0 & 0 & -2
}
}
%% Example matrices

\newcommand{ \Aexample }
{
    \mat{ rr }
      {
       1 & -1 \\
      -1 &  1 \\
       1 & -1
      }
}

\newcommand{ \Aplus }
{
  \frac{1}{6}
  \mat{rrr}
  {
   1 &  \phantom{-}1 &  1\\
  -1 &  1 & -1
  }
}

\newcommand{ \Atexample } 
{
    \mat{ rrr }
      {
       1 & -1 & 1 \\
      -1 & 1 & -1 \\
      }
}

\newcommand{\phivector}
{
  \mat{c}
  {
  2\\1\\0
  }
}

\newcommand{\ximatrix}
{
  \mat{c}
  {
  \xi \\ \eta
  }
}

\newcommand{ \Sigmatexamplea } {
\left[
\begin{array}{c|cc}
 \sqrt{6} & 0 & 0 \\ \hline
 0 & 0 & 0
\end{array}
\right]
}

\newcommand{ \Sigmaexampleb } {
\left[
\begin{array}{c|c}
 \sqrt{6} & 0 \\ \hline
 0 & 0 \\
 0 & 0
\end{array}
\right]
}

\newcommand{ \Sigmaexample } {
\left[
\begin{array}{cc}
 \sqrt{6} & 0 \\ 
 0 & 0 \\
 0 & 0
\end{array}
\right]
}

\newcommand{ \Sigmainverse } {
\left[
\begin{array}{c|cc}
 \ssix & 0 & 0 \\[5pt] \hline
 0 & 0 & 0
\end{array}
\right]
}

\newcommand{ \Xshadex } {
\left[
\begin{array}{ r >{\columncolor{ltgray}}r }
  \stwo & \stwo \\[5pt]
 -\stwo & \phantom{-}\stwo
\end{array}
\right]
}

\newcommand{ \Xshade } {
\stwo
\left[
\begin{array}{ r >{\columncolor{ltgray}}r }
  1 & 1 \\
 -1 & \phantom{-}1
\end{array}
\right]
}

\newcommand{ \Xtshade } {
\stwo
\left[
\begin{array}{ rr }
  1 & -1 \\[5pt]
\rowcolor{ltgray}
  \phantom{-}1 &  1
\end{array}
\right]
}

\newcommand{ \Yshade } {
\left[
\begin{array}{ r >{\columncolor{ltgray}}c >{\columncolor{ltgray}}r }
  \sthree & 0 & -\frac{2}{ \sqrt{6} } \\[5pt]
 -\sthree & \stwo &  \ssix \\[5pt]
  \sthree & \stwo & -\ssix
\end{array}
\right] 
}

\newcommand{ \Ytshade } {
\left[
\begin{array}{ rrr }
  \sthree & -\sthree & \sthree \\[5pt]
\rowcolor{ltgray}
   0 & \stwo &  \stwo \\[5pt]
\rowcolor{ltgray}
  -\frac{2}{ \sqrt{6} } & \ssix & -\ssix
\end{array}
\right] 
}

\newcommand{ \veca } {
\mat{r}{0\\1\\1}
}

\newcommand{ \vecb } {
\mat{r}{-2\\-1\\1}
}

\newcommand{ \Xexample } {
\stwo
\left[
\begin{array}{ rr }
   1 & 1 \\[5pt]
  -1 & \phantom{-}1
\end{array}
\right]
}


\newcommand{ \Xtexample } {
\stwo
\left[
\begin{array}{ rr }
  \phantom{-}1 & -1 \\[5pt]
  1 &  1
\end{array}
\right]
}

\newcommand{ \Sigmatexample } {
\left[
\begin{array}{ccc}
 \sqrt{6} & 0 & 0 \\ \hline
 0 & 0 & 0
\end{array}
\right]
}

\newcommand{ \Aexamplecol } {
\left[
\begin{array}{ r|r }
  1 & -1 \\
 -1 & 1 \\
  1 & -1
\end{array}
\right]
}


\newcommand{ \Aexamplerow } {
\left[
\begin{array}{ rr }
 1 & -1 \\ \hline
 -1 & 1 \\ \hline
 1 & -1
\end{array}
\right]
}

\newcommand{ \Aexamplepi } {
  \frac{1}{6}
  \mat{rrr}
  { 1 & -1 &  1\\
   -1 &  1 & -1
  }
}
