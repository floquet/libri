\section[Special types]{The SVD for special types of matrices}
We can boost intuition by studying the decomposition of special matrix types.

%%%
\subsection{Normal matrices}
\begin{equation}
  \brac{\A{},\A{*}} = \prdmm{*} - \prdm{*} = 0.
\end{equation}

%%%
\subsection{Idempotent matrices}
\begin{equation}
  \A{2} = 0.
\end{equation}
\begin{equation}
  \A{k} = 0.
\end{equation}

%%%
\subsection{Nilpotent matrices}
\begin{equation}
  \A{2} = \zero.
\end{equation}
\begin{equation}
  \A{k} = \zero.
\end{equation}

%%%
\subsection{Hermitian matrices}
\subsubsection{Definition}
\begin{equation}
  \A{} = \A{*} = \overline{ \A{T} } = \overline{\A{}}^{\TT}.
\end{equation}
The matrix is square, therefore 
\begin{equation}
  \sig{} = \sig{T}.
\end{equation}
\subsubsection{SVD}
\begin{equation}
  \begin{split}
     \A{} &= \A{*}, \\
     \svd{} &= \svdt{*}
  \end{split}
\end{equation}
Leads to
\begin{equation}
  \sig{} = \paren{\Y{*}\X{}} \, \sig{} \, \paren{\X{}\Y{*}}
\end{equation}


\subsubsection{Example}
\begin{equation}
  \begin{split}
    \svda{T} \\
    \mat{rrc}{
    -2 & -i & 0 \\
     i & -2 & 0 \\
     0 &  0 & 0 } &=
    \mat{cc>{\columncolor{ltgray}}c}{
    \frac{-i}{\sqrt{2}} & \frac{-i}{\sqrt{2}} & 0 \\ 
    \frac{-1}{\sqrt{2}} & \frac{-1}{\sqrt{2}} & 0 \\ 
    0                   & 0                   & 1 }
    \mat{cc|c}{
    3 & 0 & 0 \\
    0 & 1 & 0 \\\hline
    0 & 0 & 0}
    \mat{ccc}{
    \frac{-i}{\sqrt{2}} & \frac{1}{\sqrt{2}} & 0 \\ 
    \frac{i}{\sqrt{2}}  & \frac{1}{\sqrt{2}} & 0 \\ 
    \rowcolor{ltgray}
    0                   & 0                  & 1 }
  \end{split}
\end{equation}
%%
\begin{equation}
  \Y{} \, \X{*} = \X{} \, \Y{*} = \mat{rrc}{
  -1 &  0 & 0 \\
   0 & -1 & 0 \\
   0 &  0 & 1 }
\end{equation}


%%%
\subsection{RPN matrices\index{RPN matrix}\index{matrix!types!RPN}}
%%%
\subsubsection{Definition}
A matrix $\A{}$ is a range perpendicular to \ns \ (RPN) matrix when the range of $\A{}$ is perpendicular to the \ns \ of $\A{}$:
\begin{equation}
  \rnga{} \perp \nlla{}.
  \label{eq:11:rpn}
\end{equation} 
%%%
\subsubsection{Theory}
By looking at the domain matrices, 
\begin{equation}
  \begin{split}
     \X{} &= \xft{*}, \\
     \Y{} &= \yft{*},
  \end{split}
  \label{eq:11:dom}
\end{equation}
we see that these matrices must be square. (For two vectors to be orthogonal they must have the same length.)
From the \ft \ we know that 
\begin{equation}
  \begin{split}
     \rnga{}  &\perp \nlla{*}, \\
     \rnga{*} &\perp \nlla{}, \\
  \end{split}
\end{equation}
Using this and equation \eqref{eq:11:rpn} we see that
\begin{equation}
  \begin{split}
     \rnga{} &= \rnga{*},\\
     \nlla{} &= \nlla{*}.
  \end{split}
  \label{eq:11:eqls}
\end{equation}
Equations \eqref{eq:11:eqls} and \eqref{eq:11:dom} imply that the domain matrices are equal:
\begin{equation}
  \X{} = \Y{}.
\end{equation}
%%%
\subsubsection{Example}
The matrix given by
\begin{equation}
  \begin{split}
    \svda{T} \\
    \mat{rrc}{
   1 & -1 & 1 \\
  -1 &  2 & 0 \\
   1 &  0 & 2} &=
    \mat{cc>{\columncolor{ltgray}}c}{
    \sthree             & 0     & \frac{-2}{\sqrt{6}} \\ 
    \frac{-1}{\sqrt{3}} & \stwo & \frac{-1}{\sqrt{6}} \\ 
    \sthree             & \stwo & \ssix }
    \mat{cc|c}{
    3 & 0 & 0 \\
    0 & 1 & 0 \\\hline
    0 & 0 & 0}
    \mat{ccc}{
    \sthree             & \frac{-1}{\sqrt{3}} & \sthree  \\ 
    0                   & \stwo               & \stwo \\
    \rowcolor{ltgray} 
    \frac{-2}{\sqrt{6}} & \frac{-1}{\sqrt{6}} & \ssix  }
  \end{split}
\end{equation}
is an RPN matrix because
\begin{equation*}
  \X{} = \Y{}.
\end{equation*}

Also known as range-symmetric or EP matrices
For more discussion, see Meyer, \cite[p. 408]{Meyer}.

%%%
\subsection{SPD matrices}
\begin{equation}
  z^{\mathrm{T}}\A{}z \ge 0
\end{equation}

%%%
\subsection{Diagonally dominant matrices}


\endinput