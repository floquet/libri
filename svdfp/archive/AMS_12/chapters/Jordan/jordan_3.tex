\section[]{$3\times3$ Jordan forms}
\label{sec:Jordan:3}

In many cases $3\times3$ Jordan forms can be viewed as  block matrices with a $2\times2$ Jordan form and a zero matrix.

\subsection{Off-diagonal entries}
%%
Consider the forms where two off-diagonal entries are one.
\begin{equation}
  \begin{split}
    \mathcal{J}_{3,3} = 
\mat{ccc}
{
                1 & 1 & 0 \\
                0 & 0 & 0 \\
                0 & 0 & 0
}   &= \svd{T}\\
   &=
   %%  
\underbrace{\mat{c>{\columncolor{ltgray}}c>{\columncolor{ltgray}}c}
{
                1 & 0 & 0 \\
                0 & 0 & 1 \\
                0 & 1 & 0
}}_{\text{col 2 $\leftrightarrow$ col 3}}
%%
\mat{c|cc}{                
            \sqrt{2} & 0 & 0 \\\hline
            0 & 0 & 0 \\
            0 & 0 & 0
}%%  
\mat{rrr}
{                
\cos\theta & -\sin\theta & 0 \\
\rowcolor{ltgray}
0 & 0 & 1 \\
\rowcolor{ltgray}
\sin\theta & \cos\theta & 0
}
  \end{split}
\end{equation}
with
\begin{equation}
  \theta = -\frac{\pi}{4}.
\end{equation}
The $\X{}$ matrix represents a three dimensional coordinate system rotated by an angle $\theta$ about the line $x_{2}=x_{3}$. Only part of this rotation is ``visible''; the rest is ``hidden''in the null space.

The domain matrix $\X{}$ is a composition - or matrix product - of more fundamental orthogonal matrix forms. Any easy way to resolve this is to look at the action of the matrix upon an arbitrary vector:
\begin{equation}
 \mat{ccc}
{                
\cos\theta & -\sin\theta & 0 \\
0 & 0 & 1 \\
\sin\theta & \cos\theta & 0
} 
\mat{c}
{x\\y\\z}
=
\mat{c}
{x\cos\theta-y\sin\theta \\ z \\ x\sin\theta+y\cos\theta}
\end{equation}
We see these two operations:
\begin{enumerate}
\item a counterclockwise rotation of the $x-y$ axes by an angle $\theta$;
\item an interchange of the $y$ and $z$ axes.
\end{enumerate}
\begin{equation}
  \mat{ccc}
  {1&0&0\\
   0&0&1\\
   0&1&0}
  \mat{ccc}
  {\cos\theta & -\sin\theta & 0\\
   \sin\theta &  \cos\theta & 0\\
        0     &       0     & 1}
   =
\mat{ccc}
{                
\cos\theta & -\sin\theta & 0 \\
0 & 0 & 1 \\
\sin\theta & \cos\theta & 0
}
\end{equation}


%%%%%
Compare the related forms.
\begin{equation}
  \begin{split}
    \mathcal{J}_{3,5} = 
\mat{ccc}
{
                1 & 0 & 1 \\
                0 & 0 & 0 \\
                0 & 0 & 0
}   &= \svd{T}\\
   &=
   %%  
\underbrace{\mat{c>{\columncolor{ltgray}}c>{\columncolor{ltgray}}c}
{
                1 & 0 & 0 \\
                0 & 0 & 1 \\
                0 & 1 & 0
}}_{\text{col 2 $\leftrightarrow$ col 3}}
%%
\mat{c|cc}{                
            \sqrt{2} & 0 & 0 \\\hline
            0 & 0 & 0 \\
            0 & 0 & 0
}%%  
\mat{rcr}{                
                 \cos\theta & 0 & -\sin\theta \\
\rowcolor{ltgray}
                 \sin\theta & 0 &  \cos\theta \\
\rowcolor{ltgray}
                0 & 1 & 0
}
  \end{split}
\end{equation}
%%%
\begin{equation}
  \begin{split}
    \mathcal{J}_{3,6} = 
\mat{ccc}{
                0 & 1 & 1 \\
                0 & 0 & 0 \\
                0 & 0 & 0
}   &= \svd{T}\\
   &=
   %%  
\underbrace{\mat{c>{\columncolor{ltgray}}c>{\columncolor{ltgray}}c}
{
                1 & 0 & 0 \\
                0 & 0 & 1 \\
                0 & 1 & 0
}}_{\text{col 2 $\leftrightarrow$ col 3}}
%%
\mat{c|cc}{                
            \sqrt{2} & 0 & 0 \\\hline
            0 & 0 & 0 \\
            0 & 0 & 0
}%%  
\mat{rrr}{                
                0 & \cos\theta & -\sin\theta \\
\rowcolor{ltgray}
                0 & \sin\theta & \cos\theta \\
\rowcolor{ltgray}
                1 & 0 & 0
}
  \end{split}
\end{equation}

%%
\subsection{Embedding}
Embed the $\mathcal{J}_{2,7}$ block.
\begin{equation}
  \begin{split}
    \mathcal{J}_{3,11} &= 
\mat{cc|c}
{
                1 & 1 & 0 \\
                0 & 1 & 0 \\\hline
                0 & 0 & 0
}   = \svd{T}\\
   &=
   %%  
\mat{rr>{\columncolor{ltgray}}r}
{
                \cos\theta & -\sin\theta & 0 \\
                \sin\theta &  \cos\theta & 0 \\
                0 & 0 & 1
}
%%
\mat{cc|c}
{                
            \alpha_{+} & 0 & 0 \\
            0 & \alpha_{-} & 0 \\\hline
            0 & 0 & 0
}%%  
\mat{rrr}
{                
                 \cos\paren{-\theta_{c}} &  -\sin\paren{-\theta_{c}} & 0 \\
                 \sin\paren{-\theta_{c}} &   \cos\paren{-\theta_{c}} & 0 \\
\rowcolor{ltgray}
                0 & 0 & 1
}
  \end{split}
\end{equation}
where the singular values and angles have the same definitions as in equations \eqref{eq:j27:1}-\eqref{eq:j27:3}.

%%
\begin{equation}
  \begin{split}
    \mathcal{J}_{3,57} &= 
\mat{c|cc}
{
                0 & 0 & 0 \\\hline
                0 & 1 & 1 \\
                0 & 0 & 1
}   = \svd{T}\\
   &=
   %%  
\mat{rr>{\columncolor{ltgray}}r}
{
                0 & 1 & 0\\
                \cos\theta & 0 & -\sin\theta \\
                \sin\theta & 0 &  \cos\theta \\
}
%%
\mat{cc|c}
{                
            \alpha_{+} & 0 & 0 \\
            0 & \alpha_{-} & 0 \\\hline
            0 & 0 & 0
}%%  
\mat{rrr}
{                
                0 & \cos\paren{-\theta_{c}} & -\sin\paren{-\theta_{c}}\\
                1 & 0 & 0 \\
\rowcolor{ltgray}
                0 & \sin\paren{-\theta_{c}} & \cos\paren{-\theta_{c}} \\
}
  \end{split}
\end{equation}

There are three ways to embed this block in a $4\times4$ matrix.

%%
\begin{equation}
  \begin{split}
    \mathcal{J}_{4,19} &= 
\mat{cc|cc}
{
                1 & 1 & 0 & 0 \\
                0 & 1 & 0 & 0 \\\hline
                0 & 0 & 0 & 0 \\
                0 & 0 & 0 & 0
}   = \svd{T}\\
   &=
   %%  
\mat{cr>{\columncolor{ltgray}}c>{\columncolor{ltgray}}c}
{
                \cos\theta & -\sin\theta & 0 & 0 \\
                \sin\theta &  \cos\theta & 0 & 0 \\
                 0 & 0 & 0 & 1 \\
                 0 & 0 & 1 & 0 
}
%%
\mat{cc|cc}
{                
            \alpha_{+} & 0 & 0 & 0 \\
            0 & \alpha_{-} & 0 & 0 \\\hline
            0 & 0 & 0 & 0\\
            0 & 0 & 0 & 0
}%%  
\mat{ccrc}
{                
                0 & \cos\paren{-\theta_{c}} & -\sin\paren{-\theta_{c}} & 0 \\
                0 & \sin\paren{-\theta_{c}} &  \cos\paren{-\theta_{c}} & 0 \\
\rowcolor{ltgray}
                0 & 0 & 0 & 1 \\
\rowcolor{ltgray}
                1 & 0 & 0 & 0 \\
}
  \end{split}
\end{equation}
%%%
\begin{equation}
  \begin{split}
    \mathcal{J}_{4,176} &= 
\mat{c|cc|c}
{
                0 & 0 & 0 & 0 \\\hline
                0 & 1 & 1 & 0 \\
                0 & 0 & 1 & 0 \\\hline
                0 & 0 & 0 & 0 \\
}   = \svd{T}\\
   &=
   %%  
\mat{cr>{\columncolor{ltgray}}c>{\columncolor{ltgray}}c}
{
                 0 & 0 & 0 & 1 \\
                 \cos\theta & -\sin\theta & 0 & 0 \\
                 \sin\theta &  \cos\theta & 0 & 0 \\
                 0 & 0 & 1 & 0
}
%%
\mat{cc|cc}
{                
            \alpha_{+} & 0 & 0 & 0 \\
            0 & \alpha_{-} & 0 & 0 \\\hline
            0 & 0 & 0 & 0\\
            0 & 0 & 0 & 0
}%%  
\mat{ccrc}
{                
                0 & \cos\paren{-\theta_{c}} & -\sin\paren{-\theta_{c}} & 0 \\
                0 & \sin\paren{-\theta_{c}} &  \cos\paren{-\theta_{c}} & 0 \\
\rowcolor{ltgray}
                0 & 0 & 0 & 1 \\
\rowcolor{ltgray}
                1 & 0 & 0 & 0 \\
}
  \end{split}
\end{equation}
%%%
\begin{equation}
  \begin{split}
    \mathcal{J}_{4,896} &= 
\mat{cc|cc}
{
                0 & 0 & 0 & 0 \\
                0 & 0 & 0 & 0 \\\hline
                0 & 0 & 1 & 1 \\
                0 & 0 & 0 & 1
}   = \svd{T}\\
   &=
   %%  
\mat{cr>{\columncolor{ltgray}}r>{\columncolor{ltgray}}r}
{
                 0 & 0 & 0 & 1 \\
                 0 & 0 & 1 & 0 \\
                \cos\theta & -\sin\theta & 0 & 0 \\
                \sin\theta &  \cos\theta & 0 & 0
}
%%
\mat{cc|cc}
{                
            \alpha_{+} & 0 & 0 & 0 \\
            0 & \alpha_{-} & 0 & 0 \\\hline
            0 & 0 & 0 & 0\\
            0 & 0 & 0 & 0
}%%  
\mat{cccr}
{                
                0 & 0 & \cos\paren{-\theta_{c}} & -\sin\paren{-\theta_{c}} \\
                0 & 0 & \sin\paren{-\theta_{c}} &  \cos\paren{-\theta_{c}} \\
\rowcolor{ltgray}
                0 & 1 & 0 & 0 \\
\rowcolor{ltgray}
                1 & 0 & 0 & 0 \\
}
  \end{split}
\end{equation}
What happens when there are two blocks?
\begin{equation}
  \begin{split}
    \mathcal{J}_{4,915} &= 
\mat{cc|cc}
{
                1 & 1 & 0 & 0 \\
                0 & 1 & 0 & 0 \\\hline
                0 & 0 & 1 & 1 \\
                0 & 0 & 0 & 1
}   = \svd{T}\\
   \Y{} &=
   %%  
\mat{ccrr}
{
                 0 & \cos\theta & 0 & -\sin\theta \\
                 0 & \sin\theta & 0 & \cos\theta \\
                \cos\theta & 0 & -\sin\theta & 0 \\
                \sin\theta & 0 &  \cos\theta & 0
}\\
%%
\sig{} &=
\mat{cccc}
{                
            \alpha_{+} & 0 & 0 & 0 \\
            0 & \alpha_{+} & 0 & 0 \\\
            0 & 0 & \alpha_{-} & 0\\
            0 & 0 & 0 & \alpha_{-}
}\\
%%  
\X{T} &=
\mat{crcr}
{                
                0 & 0 & \cos\paren{-\theta_{c}} & -\sin\paren{-\theta_{c}} \\
                \cos\paren{-\theta_{c}} & -\sin\paren{-\theta_{c}} & 0 & 0 \\
                0 & 0 & \sin\paren{-\theta_{c}} &  \cos\paren{-\theta_{c}} \\
                \sin\paren{-\theta_{c}} &  \cos\paren{-\theta_{c}} & 0 & 0
}
  \end{split}
\end{equation}


\endinput