\numberwithin{equation}{section}
\numberwithin{figure}{section}
\numberwithin{table}{section}

%%
\providecommand{\abs}[1]{\left|#1\right|}
\providecommand{\norm}[1]{\left\lVert#1\right\rVert}
\providecommand{\normo}[1]{\left\lVert#1\right\rVert_1}
\providecommand{\normt}[1]{\left\lVert#1\right\rVert_2}
\providecommand{\norminf}[1]{\left\lVert#1\right\rVert_\infty}
\providecommand{\normf}[1]{\left\lVert#1\right\rVert_F}

\providecommand{\real}[1]{\mathbb{R}^{#1}}
\providecommand{\cmplx}[1]{\mathbb{C}^{#1}}
\providecommand{\nll}[1]{\mathcal{N}\paren{#1}}
\providecommand{\rng}[1]{\mathcal{R}\paren{#1}}

\providecommand{\A}[1] { \textbf{A}^{\mathrm{#1}} }
\providecommand{\B}[1] { \textbf{B}^{\mathrm{#1}} }
\providecommand{\C}[1] { \textbf{C}^{\mathrm{#1}} }
\providecommand{\D}[1] { \textbf{D}^{\mathrm{#1}} }
\providecommand{\G}[1] { \textbf{G}^{\mathrm{#1}} }
\providecommand{\J}[1] { \textbf{J}^{\mathrm{#1}} }
\providecommand{\M}[1] { \textbf{M}^{\mathrm{#1}} }
\providecommand{\N}[1] { \textbf{N}^{\mathrm{#1}} }
\providecommand{\P}[1] { \textbf{P}^{\mathrm{#1}} }
\providecommand{\Q}[1] { \textbf{Q}^{\mathrm{#1}} }
\providecommand{\R}[1] { \textbf{R}^{\mathrm{#1}} }
\providecommand{\T}[1] { \textbf{T}^{\mathrm{#1}} }
\providecommand{\V}[1] { \textbf{V}^{\mathrm{#1}} }
\providecommand{\U}[1] { \textbf{U}^{\mathrm{#1}} }
\providecommand{\X}[1] { \textbf{X}^{\mathrm{#1}} }
\providecommand{\Y}[1] { \textbf{Y}^{\mathrm{#1}} }
\providecommand{\Z}[1] { \textbf{Z}^{\mathrm{#1}} }
\providecommand{\pee}[1] { \textbf{P}^{\mathrm{#1}} }
\providecommand{\ess}[1] { \textbf{S}^{\mathrm{#1}} }
\providecommand{\J}[2] { \textbf{J}^{#2}_{#1} }
\providecommand{\E}[1] { \textbf{E}_{#1} }
\providecommand{\I}[1] { \textbf{I}_{\mathrm{#1}} }
\providecommand{\W}[1] { \textbf{W}_{\mathrm{#1}} }
\providecommand{\sig}[1] { \Sigma^{\mathrm{#1}} }
\providecommand{\Ain}[1] {\A{\by{m}{n}}_{#1}}

\providecommand{\prdm}[1] {\A{ \mathrm{#1} }\A{}}
\providecommand{\prdmm}[1] {\A{}\A{ \mathrm{#1} }}

\providecommand{\paren}[1] { \left( #1 \right) }
\providecommand{\inner}[1] { \langle #1 \rangle }
\providecommand{\brac}[1] { \left[ #1 \right] }
%\providecommand{\list}[1] { \lbrace #1 \rbrace }
\providecommand{\lst}[1] { \left\{ #1 \right\} }
\providecommand{\dim}[1] { \text{dim}\paren{ #1 } }

\providecommand{\mat}[2] {\paren{\begin{array}{#1}#2\end{array}}}
\providecommand{\pd}[2] {\frac{\partial #1}{\partial #2}}
\providecommand{\pdx}[2] {\frac{\partial #1}{\partial x}}
\providecommand{\pdt}[2] {\frac{\partial #1}{\partial t}}
\providecommand{\td}[2] {\frac{d #1}{d #2}}
\providecommand{\by}[2] {#1 \times #2}
\providecommand{\byt}[2] {\mathit{#1} \times \mathit{#2}}
\providecommand{\complex}[1] { \mathbb{C}^{\by{m}{n}}_{#1} }

%%
\DeclareMathOperator{\ux}{\underbar{x}}
\DeclareMathOperator{\re}{Re}
\DeclareMathOperator{\tr}{tr}
\DeclareMathOperator{\sgn}{sgn}
\DeclareMathOperator{\mx}{max}
\DeclareMathOperator{\dm}{dim}
\DeclareMathOperator{\adj}{adj}
\DeclareMathOperator{\Arg}{Arg}
\DeclareMathOperator{\spn}{sp}
\DeclareMathOperator{\rref}{rref}
\DeclareMathOperator{\rot}{\textbf{R}\paren{\theta}}
\DeclareMathOperator{\leftinv}{\A{+}\A{}}
\DeclareMathOperator{\rightinv}{\A{}\,\A{+}}

\newcommand{ \ls } {\A{} x = b}
\newcommand{ \lsa } {\A{} x & = b }
\newcommand{ \zero } {\textbf{0}}
\newcommand{ \one } {\textbf{1}}