\section{The matrix of singular values}

In this volume we have, by fiat, restricted the singular values to being nonzero. This is a convention, by no means a theoretical necessity.


\begin{table}[htdp]
\begin{center}
\begin{tabular}{lccc|cc}
  $\A{}$ & $\sig{}$ & $\sig{T}$ & $\sig{(+)}$ & $\sig{}\sig{(+)}$ & $\sig{(+)}\sig{(+)}$\\
  $\cmplx{\bys{2}}_{2}$ &
  $\mat{cc}{\sigma_{1}&0\\0&\sigma_{2}}$ & 
  $\mat{cc}{\sigma_{1}&0\\0&\sigma_{2}}$ & 
  $\mat{cc}{\frac{1}{\sigma_{1}}&0\\0&\frac{1}{\sigma_{2}}}$ & $\itwo$ & 
  $\itwo$\\ 
  %%
  $\cmplx{\by{3}{2}}_{2}$ &
  $\mat{cc|c}{\sigma_{1}&0&0\\0&\sigma_{2}&0}$ & 
  $\mat{cc}{\sigma_{1}&0\\0&\sigma_{2}\\\hline 0&0}$ & 
  $\mat{cc}{\frac{1}{\sigma_{1}}&0\\0&\frac{1}{\sigma_{2}}\\[3pt]\hline 0&0}$ &
  $\itwo$ & 
  $\mat{cc|c}{1&0&0\\0&1&0}$\\
  %%
  $\cmplx{\by{2}{3}}_{1}$ &
  $\mat{c|c}{\sigma_{1}&0\\\hline0&0\\0&0}$ & 
  $\mat{c|cc}{\sigma_{1}&0&0\\\hline0&0&0}$ & 
  $\mat{c|cc}{\frac{1}{\sigma_{1}}&0&0\\[3pt]\hline0&0&0}$ &
  $\mat{c|cc}{1&0&0\\\hline0&0&0\\0&0&0}$ & 
  $\mat{c|c}{1&0\\\hline0&0}$ \\
  \ \\
\end{tabular}
\end{center}
\caption{Different forms for the $\sig{}$ matrix for three types of matrices. The matrices $\sig{}$ and $\sig{T}$ share the same diagonal; $\sig{T}$ and $\sig{(+)}$ share the same shape.}
\label{tab:sing}
\end{table}%

Stencils
truncated identity
\begin{equation}
  \begin{split}
    \sig{}\sig{(+)} &= \mat{c|c}{\I{m}&\zero\\\hline\zero&\zero} = \J{m}{\rho}, \\
    \sig{(+)}\sig{} &= \mat{c|c}{\I{n}&\zero\\\hline\zero&\zero} = \J{n}{\rho}.
  \end{split}
\end{equation}


\endinput