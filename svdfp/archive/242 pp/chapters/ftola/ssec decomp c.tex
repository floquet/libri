\subsection{Matrix (c)}
This matrix has full rank and therefore both \ns s are trivial.

%%%%%%%%%%%
\subsubsection{Column space}
The full rank matrix is 
\begin{equation*}
  \A{} = \matrixc.
  \tag{\ref{eq:matrix c}}
\end{equation*}
This matrix takes a \vv \ and returns another \vv. Both \ns s are trivial:
\begin{equation}
  \rnlla{*} = \rnlla{} = \glzerotwo.
\end{equation}
Therefore both columns are basic and define the span of the range:
%
\begin{equation}
  \brnga{} = \spn{ \bveccf, \bveccg }.
\end{equation}
%

%%%%%%%%%%%
\subsubsection{Row space}
Similarly, both columns are basic and define the span of the range:
%
\begin{equation}
  \brnga{*} = \spn{ \bvecch, \bvecce }.
\end{equation}
%
All four fundamental spaces are resolved and we see that $\A{}\in\cmplxall{2}{2}{2}$.

\endinput