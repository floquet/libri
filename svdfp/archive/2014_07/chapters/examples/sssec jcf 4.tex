%%%%%%%%%%%%%%
%%%%%%%%%%%%%%
\subsubsection{$\bigstar$ n = 4}
This is the simplest nontrivial Jordan block.
\begin{equation}
  \J{4} = \mat{cccc}{1&1&0&0 \\ 0&1&1&0 \\ 0&0&1&1 \\ 0&0&0&1}
\end{equation}
%
The characteristic polynomial is 
%
\begin{equation}
  p\paren{\lambda} = \lambda ^4-7 \lambda ^3+15 \lambda ^2-10 \lambda + 1 .
\end{equation}
%
After some exploration, for example trying the roots $\pm1$, $\pm2$, we can factor the characteristic polynomial:
%
\begin{equation}
  p\paren{\lambda} = (\lambda - 1) \left(\lambda ^3-6 \lambda ^2+9 \lambda -1\right)
\end{equation}
%
If we define the angle
%
\begin{equation}
  \alpha = \frac{\pi}{18}
\end{equation}
%
the singular values can be written as
\begin{equation}
  \sigma = \lst{2\cos 2\alpha, 2 \cos 4\alpha, 1, \cos 2\alpha - \sqrt{3} \sin 2 \alpha}.
\end{equation}
%
To tame the domain matrices define the intermediate variables
%
\begin{equation}
  \begin{split}
    a &= 3 \sin \paren{4\alpha},  \\
    b &= 3 \cos \paren{\alpha}, \\
    c &= 2\cos^{2} \paren{\alpha} - \sin \paren{\alpha} .
  \end{split}
\end{equation}
%
and the domain matrices then have the docile forms
%
\begin{equation}
  \begin{split}
%
    \bur{} = {\bl{ \frac{2}{9} \mat{rrrr}{
		     a & -b &  \sqrt{3} & -c \\
		     b & -c & -\sqrt{3} &  a \\
		 \sqrt{3} &  \sqrt{3} &  0 & -\sqrt{3} \\
		     c &  a &  \sqrt{3} &  b } }} , \qquad
%
    \bvr{} = {\bl{ \frac{2}{9} \mat{rrrr}{
         c & -a &  \sqrt{3} & -b \\
         \sqrt{3} & -\sqrt{3} &  0 & \sqrt{3} \\
         b &  c & -\sqrt{3} & -a \\
         a &  b &  \sqrt{3} &  c } }}
%
  \end{split}
\end{equation}



\endinput