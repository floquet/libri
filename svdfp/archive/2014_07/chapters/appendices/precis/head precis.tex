\chapter{Linear algebra pr\'ecis}
Existence and uniqueness
  % = =  e q u a t i o n
  \begin{equation}
    \Axeb
  \end{equation}
  % = =
Discussed in the guise of  of identity-type matrices.

\section{0, 1, $\infty$}   %  S  S  S  S  S  S  S  S  S  S  S  S  S  S  S
Every linear system can be characterized by the number of solutions
\begin{itemize}
\item 0: no solution
\item 1: a unique solution
\item $\infty$: an infinite number of solutions
\end{itemize}

\section{Trivial: 1}   %  S  S  S  S  S  S  S  S  S  S  S  S  S  S  S
Data vector and solution vector have the same dimension. That is $b\icm$ and $x\icm$; the linear system matrix is square.
  % = =  e q u a t i o n
  \begin{equation}
    \idtwo
    \mat{c} { x_{1} \\ x_{2} } =
    \mat{c} { b_{1} \\ b_{2} }    
  \end{equation}
  % = =
There is a unique solution:
  % = =  e q u a t i o n
  \begin{equation}
    \mat{c} { x_{1} \\ x_{2} } = \bl{ \mat{c} { b_{1} \\ b_{2} } }
  \end{equation}
  % = =


\section{Tall: 0 or 1}
  % = =  e q u a t i o n
  \begin{equation}
    \mat{cc}{ 1 & 0 \\ 0 & 1 \\ 0 & 0 }
    \mat{c} { x_{1} \\ x_{2} } =
    \mat{c} { b_{1} \\ b_{2} \\ b_{3} }
  \end{equation}
  % = =
In block form we may write
  % = =  e q u a t i o n
  \begin{equation}
    \mat{c}{\I{2} \\\hline \zero }
    \mat{c} { x_{1} \\ x_{2} } =
    \mat{c} { b_{1} \\ b_{2} \\\hline b_{3} } =
    \bl{ \mat{c} { b_{1} \\ b_{2} \\ 0 } } + \rd{ \mat{c} { 0 \\ 0 \\ b_{3} } }
  \end{equation}
  % = =


When $b_{3}=0$ the solution is unique
  % = =  e q u a t i o n
  \begin{equation}
    x = \mat{c} { x_{1} \\ x_{2} } = \bl{ \mat{c} { b_{1} \\ b_{2} } }
  \end{equation}
  % = =

When $b_{3}\neq0$ there is no solution and we turn to least squares.
  % = =  e q u a t i o n
  \begin{equation}
    x_{LS} = \mat{c} { x_{1} \\ x_{2} } = \bl{ \mat{c}  { b_{1} \\ b_{2} } }
  \end{equation}
  % = =
The error is $r^{\mathrm{T}}r=\normts{\Axmb} = b_{3}^{2}$. Observe that when $b_{3}=0$ the error $r^{\mathrm{T}}r=0$; the solution is exact.

When $b_{1} = b_{2} = 0$ there is not even a least squares solution. By inspection it is clear that the linear system
  % = =  e q u a t i o n
  \begin{equation}
    \mat{cc}{ 1 & 0 \\ 0 & 1 \\ 0 & 0 }
    \mat{c} { x_{1} \\ x_{2} } =
    \rd{ \mat{c} { 0 \\ 0 \\ b_{3} } }
  \end{equation}
  % = =
has no solution either directly or by the method of least squares.
We say that the data vector is in the \ns \ of $\A{*}$: $b\in\rnlla{*}$.

\section{Wide: $\infty$}   %  S  S  S  S  S  S  S  S  S  S  S  S  S  S  S
  % = =  e q u a t i o n
  \begin{equation}
    \mat{ccc}{ 1 & 0 & 0 \\ 0 & 1 & 0 }
    \mat{c}  { x_{1} \\ x_{2} \\ x_{3} } =
    \bl{ \mat{c}  { b_{1} \\ b_{2} } }
  \end{equation}
  % = =
In block form we may write
  % = =  e q u a t i o n
  \begin{equation}
    \mat{c|c}{\I{2} & \zero }
    \mat{c} { x_{1} \\ x_{2} \\\hline x_{3} } =
    \bl{ \mat{c}  { b_{1} \\ b_{2} } }
  \end{equation}
  % = =
There are an infinite number of solutions
  % = =  e q u a t i o n
  \begin{equation}
    x = \bl{ \mat{c} { b_{1} \\ b_{2} \\ 0 } } + \alpha \rd{ \mat{c} { 0 \\ 0 \\ 1 } }, \qquad \alpha\ic .
  \end{equation}
  % = =
There are an infinite number of solutions representing the continuum of values represented by the free parameter $\alpha$.

The solution of minimum norm $x_{min}$
  % = =  e q u a t i o n
  \begin{equation}
    x_{min} = \min_{\alpha\ic} \normts{\bl{\mat{c} { b_{1} \\ b_{2} \\ 0 }} + \alpha \rd{\mat{c} { 0 \\ 0 \\ 1 }} } = \bl{ \mat{c} { b_{1} \\ b_{2} \\ 0 } }
  \end{equation}
  % = =
That is when $\alpha=0$.

\section{Transpose}
The preceding two examples represent the same matrix:
  % = =  e q u a t i o n
  \begin{equation}
    \begin{split}
      \mat{cc}  { 1 & 0 \\ 0 & 1 \\\hline 0 & 0 }^{*} &= \mat{cc|c}{ 1 & 0 & 0 \\ 0 & 1 & 0 }, \\
      \mat{cc|c}{ 1 & 0 & 0 \\ 0 & 1 & 0 }^{*}        &= \mat{cc} { 1 & 0 \\ 0 & 1 \\\hline 0 & 0 }.
    \end{split}
  \end{equation}
  % = =
The partitioning highlights this relationship.

\section{SVD}
  % = =  e q u a t i o n
  \begin{equation}
    \idtwo = \svd{*} = \bl{ \I{2} } \, \I{2} \, \bl{ \I{2} }
  \end{equation}
  % = =
  % = =  e q u a t i o n
  \begin{equation}
    \mat{cc}{ 1 & 0 \\ 0 & 1 \\ 0 & 0 } = \svd{*} 
      = \mat{ccc}{ \bone & \bzero & \rzero \\ \bzero & \bone & \rzero \\ \bzero & \bzero & \rone } \, 
        \mat{c}{ \I{2} \\\hline \zero } \, 
            \bl{ \I{2} }
  \end{equation}
  % = =
  % = =  e q u a t i o n
  \begin{equation}
    \mat{ccc}{ 1 & 0 & 0 \\ 0 & 1 & 0 } = \svd{*} 
      = \bl{ \I{2} } \, 
        \mat{c|c}{ \I{2} & \zero } \, 
        \mat{ccc}{ \bone & \bzero & \bzero \\ \bzero & \bone & \bzero \\ \rzero & \rzero & \rone }
  \end{equation}
  % = =
  % = =  e q u a t i o n
  \begin{equation}
    \mat{ccc}{ 1 & 0 & 0 \\ 0 & 1 & 0 \\ 0 & 0 & 0 } = \svd{*} 
      = \mat{ccc}{ \bone & \bzero & \rzero \\ \bzero & \bone & \rzero \\ \bzero & \bzero & \rone } \, 
        \mat{c|c}{ \I{2} & \zero \\\hline \zero & \zero } \, 
        \mat{ccc}{ \bone & \bzero & \bzero \\ \bzero & \bone & \bzero \\ \rzero & \rzero & \rone }
  \end{equation}
  % = =


\endinput