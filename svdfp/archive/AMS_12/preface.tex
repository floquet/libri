\chapter*{Preface}

For so many people, the singular value decomposition (SVD) is an objet d'art: elegant, powerful and intuitive. A foundation concept in linear algebra - the four fundamental subspaces - is brilliantly illuminated by the SVD. A fascinating topic worthy of study in its own right, the decomposition is also of central primacy in linear algebra. It is intimately connected to the generalized matrix inverse. It provides an obvious way to quantify the information content in a matrix. It is a work horse in the prolific field of least squares: the singularity of system matrices can be quantified, and it becomes a general tool in the attack of ill-posed and ill-conditioned problems.

However, this broad and far-reaching tool is used by many who feel that SVD is a Byzantine method whose machinations are murky or even opaque. This is a needless shame and the motivation for preparing this pedestrian manual.

The purpose of this work is to bring the joy and utility of the singular value decomposition to the masses.

\aufm{Daniel Topa}