\clearpage
\break

\section{Block structures}
It's important to understand the block structures of matrices and Jordan's photo offers a new perspective because we will be able to recognize the image in the various submatrices. While the concept seems trivial it is all too often a stumbling block. 

In the following table we will use white squares to represent a block on information. It may be any on of the matrices $\A{}$, $\Y{}$, $\sig{}$, or $\X{}$. The black squares represent a matrix of zeros of the proper dimension. Square brackets wrapping a matrix indicate that in has a block structure built upon the referenced data matrix. We will look at the following block forms:
\begin{table}[htdp]
\begin{center}
\begin{tabular}{ccccc}
  $\brac{\A{}}$ &=& $\brac{\Y{}}$ & $\brac{\sig{}}$ & $\brac{\X{T}}$\\
  \includegraphics[ width = 1in ]{pdf/"ch 06"/blocks/"blocks diag"} &&
  \includegraphics[ width = 1in ]{pdf/"ch 06"/blocks/"blocks diag"} &
  \includegraphics[ width = 1in ]{pdf/"ch 06"/blocks/"blocks diag"} &
  \includegraphics[ width = 1in ]{pdf/"ch 06"/blocks/"blocks diag"} \\
  %%%
  \includegraphics[ width = 1in ]{pdf/"ch 06"/blocks/"blocks all"} &&
  \includegraphics[ width = 1in ]{pdf/"ch 06"/blocks/"blocks col 1"} &
  \includegraphics[ width = 1in ]{pdf/"ch 06"/blocks/"blocks diag"} &
  \includegraphics[ width = 1in ]{pdf/"ch 06"/blocks/"blocks row 1"} \\
  %%%
  \includegraphics[ width = 1in ]{pdf/"ch 06"/blocks/"blocks row 2"} &&
  \includegraphics[ width = 1in ]{pdf/"ch 06"/blocks/"blocks lrhc"} &
  \includegraphics[ width = 1in ]{pdf/"ch 06"/blocks/"blocks lrhc"} &
  \includegraphics[ width = 1in ]{pdf/"ch 06"/blocks/"blocks row 2"} \\
  %%%
  \includegraphics[ width = 1in ]{pdf/"ch 06"/blocks/"blocks col 2"} &&
  \includegraphics[ width = 1in ]{pdf/"ch 06"/blocks/"blocks col 2"} &
  \includegraphics[ width = 1in ]{pdf/"ch 06"/blocks/"blocks lrhc"} &
  \includegraphics[ width = 1in ]{pdf/"ch 06"/blocks/"blocks lrhc"} \\
  %%%
\end{tabular}
\end{center}
\label{tab:jordan:blocks:basic}
\caption[The block forms we will examine]{The block forms we will examine.}
\end{table}%


\begin{table}[htdp]
\begin{center}
\begin{tabular}{c}
$
\mat{cc}{\A{} & \zero \\ \zero & \A{}} =
\mat{cc}{\Y{} & \zero \\ \zero & \Y{}}
\mat{cc}{\sig{} & \zero \\ \zero & \sig{}}
\mat{cc}{\X{} & \zero \\ \zero & \X{}}^{\TT}
$ \\[20pt]
\includegraphics[ width = 4in ]{pdf/"ch 06"/blocks/"blocks diagonal".png}
%%
\end{tabular}
\end{center}
\caption[Block diagonal structure]{Block diagonal structure.}
\label{tab:Jordan:diagonal}
\end{table}%

\begin{table}[htdp]
\begin{center}
\begin{tabular}{c}
$
\mat{cc}{\A{} & \A{} \\ \A{} & \A{}} =
\mat{cc}{\Y{} & \zero \\ \Y{} & \zero}
\mat{cc}{\sig{} & \zero \\ \zero & \sig{}}
\mat{cc}{\X{} & \zero \\ \X{} & \zero}^{\TT}
$ \\[20pt]
\includegraphics[ width = 4in ]{pdf/"ch 06"/blocks/"blocks full"}
%%
\end{tabular}
\end{center}
\caption[Full block structure]{Full block structure.}
\label{tab:Jordan:full}
\end{table}%

\begin{table}[htdp]
\begin{center}
\begin{tabular}{c}
$
\mat{cc}{\zero & \zero \\ \A{} & \A{}} =
\mat{cc}{\zero & \zero \\ \zero & \Y{}}
\mat{cc}{\zero & \zero \\ \zero & \sig{}}
\mat{cc}{\zero & \X{} \\ \zero & \X{}}^{\TT}
$ \\[20pt]
\includegraphics[ width = 4in ]{pdf/"ch 06"/blocks/"blocks row".png}
%%
\end{tabular}
\end{center}
\caption[Block row structure]{Block row structure.}
\label{tab:Jordan:row}
\end{table}%

\begin{table}[htdp]
\begin{center}
\begin{tabular}{c}
$
\mat{cc}{\zero & \A{} \\ \zero & \A{}} =
\mat{cc}{\zero & \Y{} \\ \zero & \Y{}}
\mat{cc}{\zero & \zero \\ \zero & \sig{}}
\mat{cc}{\zero & \zero \\ \zero & \X{}}^{\TT}
$ \\[20pt]
\includegraphics[ width = 4in ]{pdf/"ch 06"/blocks/"blocks col".png}
%%
\end{tabular}
\end{center}
\caption[Block column structure]{Block column structure.}
\label{tab:Jordan:row}
\end{table}%

\endinput