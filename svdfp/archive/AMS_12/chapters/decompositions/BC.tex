\section[$\A{}=\B{}\C{}$]{Full rank factorization}
For full rank matrices. Meyer 4.5.20, p. 221

%%
\subsection{Theory}
For $\A{}\in\cmplx{\bymn}_{\rho}$ there is a full rank factorization of the form
\begin{equation}
  \A{}=\B{}\,\C{}
\end{equation}
\begin{quote}
The foundation is the reduction to row echelon form.
\begin{equation}
  \EAR 
\end{equation}
The column vectors of the factor $\B{}$ are the basic columns of the target matrix $\A{}$. The row vectors of the factor $\C{}$ are the nonzero rows of $\EA$.

\begin{enumerate}
\item $\B{\byt{m}{\rho}} = \mat{c|c|c}{\brac{\A{}}_{*,1}&\dots&\brac{\A{}}_{*,\rho}}$: the $\rho$ basic columns of the target matrix;
\item $\C{\byt{\rho}{n}} = \mat{c}{\brac{\E{\A{}}}_{1,*}\\\hline\vdots\\\hline\brac{\E{\A{}}}_{\rho,*}}$: the $\rho$ nonzero rows of reduced form $\E{\A{}}$.
\end{enumerate}

\end{quote}

%%
\subsection{Example 1: No connection to the SVD}
Meyer 2.5.1, p. 67
The target matrix is given by this:
\begin{equation}
  \A{} = \mat{cccc}{1&2&2&3 \\ 2&4&1&3 \\ 3&6&1&4}.
\end{equation}
Attack the problem by clearing one pivot at each step. You can verify that one reduction sequence is shown here:
\begin{equation}
\begin{split}
  \E{\A{}} &= \G{2}\,\G{1}\,\A{},\\
%%
 & =
\mat{rrr}{1&0&0 \\ -2&1&0 \\ -3&1&0}
%%
\mat{rrr}{1&0&0 \\ 0&-\frac{1}{3}&0 \\ 0&-\frac{5}{3}&0}
%%
\mat{cccc}{1&2&2&3 \\ 2&4&1&3 \\ 3&6&1&4}, \\
%%%
   &= 
\mat{cccc}{1&2&2&3 \\ 0&0&1&1 \\ 0&0&0&0}.
\end{split}
\end{equation}
The reduced form reveals that the matrix $\A{}$ has rank $\rho = 2$.

The pivots in the matrix $\E{\A{}}$ are in columns one and three. Therefore the basic columns of the target matrix are one and three. This specifies the column composition of the $\B{}$ matrix:
\begin{equation}
  \B{} = \mat{c|c}{1&2 \\ 2&1 \\ 3&1}.
\end{equation}
The nonzero rows of the matrix $\E{\A{}}$ are used to construct the rows of the matrix $\C{}$:
\begin{equation}
  \C{} = \mat{cccc}{1&2&2&3 \\ \hline 0&0&1&1}.
\end{equation}
 
Verify the decomposition by checking that the matrix product produces the target matrix:
\begin{equation}
  \begin{split}
    \A{} &= \B{}\,\C{},\\
     &= \mat{cc}{1&2 \\ 2&1 \\ 3&1} \mat{cccc}{1&2&2&3 \\ 0&0&1&1},\\
     &= \mat{cccc}{1&2&2&3 \\ 2&4&1&3 \\ 3&6&1&4}.
  \end{split}
\end{equation}

The decomposition is unique as the reduced form is unique and the columns of the target matrix are unique.

%%
\subsection{SVD}
For comparison, the SVD for the matrix is shown here:
\begin{equation}
  \begin{split}
    \Y{}   & =
\left[
\begin{array}{rrr}
 \sqrt{\frac{17}{35}-\frac{31}{7 \sqrt{165}}} & -\sqrt{\frac{17}{35}+\frac{31}{7 \sqrt{165}}} & \frac{1}{\sqrt{35}} \\
 \sqrt{\frac{1}{7}+\frac{5 \sqrt{\frac{5}{33}}}{14}} & \sqrt{\frac{1}{7}-\frac{5 \sqrt{\frac{5}{33}}}{14}} & \frac{-5}{\sqrt{35}} \\
 \sqrt{\frac{13}{35}+\frac{37}{14 \sqrt{165}}} & \sqrt{\frac{13}{35}-\frac{37}{14 \sqrt{165}}} & \frac{3}{\sqrt{35}}
\end{array}
\right] \\
    \sig{} & =
\left[
\begin{array}{ccrr}
 \sqrt{55+4 \sqrt{165}} & 0 & 0 & 0 \\
 0 & \sqrt{55-4 \sqrt{165}} & 0 & 0 \\
 0 & 0 & 0 & 0
\end{array}
\right] \\
    \X{}   & =
\left[
\begin{array}{rrrr}
 \sqrt{\frac{1}{330} \left[30+\sqrt{165}\right]} & \sqrt{\frac{1}{11}-\frac{1}{2 \sqrt{165}}} & -\frac{1}{\sqrt{3}} & -\frac{4}{\sqrt{33}} \\
 \sqrt{\frac{4}{11}+\frac{2}{\sqrt{165}}} & \sqrt{\frac{4}{11}-\frac{2}{\sqrt{165}}} & 0 & \frac{3}{\sqrt{33}} \\
 \sqrt{\frac{3}{11}-\sqrt{\frac{3}{55}}} & -\sqrt{\frac{3}{11}+\sqrt{\frac{3}{55}}} & -\frac{1}{\sqrt{3}} & \frac{2}{\sqrt{33}} \\
 \sqrt{\frac{1}{330} \left[90+\sqrt{165}\right]} & -\sqrt{\frac{3}{11}-\frac{1}{2 \sqrt{165}}} & \frac{1}{\sqrt{3}} & -\frac{2}{\sqrt{33}}
\end{array}
\right] 
  \end{split}
\end{equation}

%%
\subsection{Example 2: A connection to the SVD}
Apply this same process to the familiar target matrix
\begin{equation}
  \A{} = \Aexample.
\end{equation}
The reduction to reduced row-eschelon form is immediate:
\begin{equation}
\begin{split}
  \E{\A{}} &= \G{1}\,\A{},\\
%%
   & =
\mat{rrr}{1&0&0 \\ 1&1&0 \\ -1&0&1}
%%
\Aexample, \\
%%%
   &= 
\mat{rr}{1&-1 \\ 0&0 \\ 0&0}.
\end{split}
\end{equation}
The matrix decomposition is given by these expressions:
\begin{equation}
    \B{} = \mat{r}{1 \\ -1 \\ 1}, \qquad \C{} = \mat{rr}{1&-1}.\\
\end{equation}
Compare the $\B{}$ matrix with the range vectors of the codomain matrix:
\begin{equation}
  \Y{}_{*,1} = \sthree \mat{r}{1 \\ -1 \\ 1}.
\end{equation}
They differ by the normalization factor. Now compare the $\C{}$ matrix with the range vectors of the domain matrix:
\begin{equation}
  \X{T}_{*,1} = \stwo \mat{r}{1 \\ -1}.
\end{equation}
Here too the difference the normalization factor. The singular value of 
$$
\sqrt{6} = \sqrt{3}\sqrt{2}
$$
is aportioned between the two matrices $\B{}$ and $\C{}$:
\begin{equation}
  \begin{split}
    \sqrt{3} \Y{}_{*,1} & = \B{},\\
    \sqrt{2} \X{}_{*,1} & = \C{}.
  \end{split}
\end{equation}



\endinput