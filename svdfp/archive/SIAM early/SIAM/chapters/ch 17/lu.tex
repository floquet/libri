\section[$\pee{}\A{}=\LL{}\U{}$]{Lower-upper decomposition}
For full rank matrices. Meyer p.144.

%%
\subsection{Theory}
For $\A{}\in\cmplx{\by{m}{m}}_{m}$
\begin{equation}
  \mathbf{P}\,\A{}=\LL{}\,\U{}
\end{equation}
\begin{quote}
  With permutations $\mathbf{P}$ permitted, every matrix can be written as the product of a unit lower diagonal matrix and a upper triangular matrix.
\end{quote}

%%
\subsection{Examples}
\begin{equation}
  \begin{array}{cccc}
    \A{}&=&\LL{}&\U{}\\
    \mat{cc}{1&2\\3&4} &=& \mat{cc}{1&0\\3&1}&\mat{rr}{1&2\\0&-2}
  \end{array}
\end{equation}

The decomposition is not unique. This is Meyer's example 3.10.4\cite[p. 151]{meyer2000matrix} with a different solution:

\begin{equation}
  \begin{array}{ccccc}
    \mathbf{P}&\A{}&=&\LL{}&\U{}\\
    \mat{rrr}{1&0&0 \\ 0&0&1 \\ 0&1&0}&
    \mat{rrr}{1&2&-3 \\ 4&8&12 \\ 2&3&2}&
    = &
    \mat{rrr}{
 1 & 0 & 0 \\
 2 & 1 & 0 \\
 4 & 0 & 1}&
    \mat{rrr}{
 1 & 2 & -3 \\
 0 & -1 & 8 \\
 0 & 0 & 24
 }
  \end{array}
\end{equation}

%%
\subsection{Application}
Use the $\L\,\U{}$ decomposition to solve the linear system

\begin{equation}
  \begin{array}{cccc}
    \A{}&x&=&b \\
    \mat{rrr}{1&1&1 \\ 3&0&-1 \\ -1&2&1}&
      \mat{r}{x_{1}\\x_{2}\\x_{3}} & = & 
    \mat{r}{1\\2\\-1}.
  \end{array}
\end{equation}

The decomposition is
\begin{equation}
  \begin{array}{cccc}
    \A{}&=&\LL{}&\U{}\\
    \mat{rrr}{1&1&1 \\ 3&0&-1 \\ -1&2&1} &=& \mat{ccc}{1&0&0\\2&1&0\\4&0&1}&\mat{rrr}{1&2&-3\\0&-1&8\\0&0&24}.
  \end{array}
\end{equation}

The two steps are these:
\begin{enumerate}
\item Solve for the dummy vector $y$ using \index{forward substitution}forward substitution.
\begin{equation}
  \begin{array}{cccc}
    \LL{}&y&=&b\\
    \mat{rrr}{1&0&0\\3&1&0\\-1&-1&1}&\mat{r}{y_{1}\\y_{2}\\y_{3}}&
    =&\mat{r}{2\\4\\2}
  \end{array}
\end{equation}
\item Solve for the solution vector $x$ via \index{back substitution}back substitution.
\begin{equation}
  \begin{array}{cccc}
    \U{}&x&=&y\\
    \mat{rrr}{1&1&1\\0&-3&-4\\0&0&-2}&\mat{r}{x_{1}\\x_{2}\\x_{3}}&
    =&\mat{r}{2\\-2\\2}
  \end{array}
\end{equation}
\end{enumerate}
The solution vector is then
\begin{equation}
  x = \mat{r}{x_{1}\\x_{2}\\x_{3}} = \mat{r}{1\\2\\-1}.
\end{equation}

\endinput