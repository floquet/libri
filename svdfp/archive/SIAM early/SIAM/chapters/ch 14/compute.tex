\section{Special cases of the pseudoinverse}

Here are some tempting morsels to practice with. This shows the almost symbiotic relationship between the \svdl \ and the generalized matrix inverse. We will be exploiting these three relationships:
\begin{equation}
  \begin{split}
     \A{}  &= \svd{*} = \yrn \essmatrix{} \xtrn{*},\\
     \A{*} &= \svdt{*} = \xrn \essmatrix{T} \ytrn{*},\\
     \Ap   &= \svd{*} = \xrn \essmatrix{-1} \ytrn{*},\\
  \end{split}
\end{equation}
The generic case for the target matrix is $\Amnr$.

%%
\subsection{Underdetermined case}
\label{sec:underdetermined}
Compute the pseudoinverse $\Ap$ when the inverse of the product matrix $\paren{\prdmy{*}}^{-1}$ exists.
Here the target matrix has full row rank and has more columns than row (a wide matrix):
\begin{equation}
  \Accc{m}{n}{m}, \ m\le n.
\end{equation}

Since the inverse of the product matrix $\prdmy{*}$ it must have full rank:
\begin{equation}
  \paren{\prdmy{*}}^{-1} = \paren{\A{*}}^{-1}\paren{\A{}}^{-1}.
\end{equation}
Therefore the target matrix is full rank. Therefore the pseudoinverse matches the standard inverse: 
\begin{equation}
  \Ap = \A{-1}.
\end{equation}

In terms of the SVD, the matrix product reduces to
\begin{equation}
  \begin{split}
    \prdmy{*} & = \paren{\svd{*}}\paren{\svdt{*}}\\
      &= \Y{}\, \sig{}\sig{T}  \Y{*}.
  \end{split}
\end{equation}

Is that your final answer?
\begin{equation}
  \Ap = \A{*}\paren{\prdm{*}}^{-1}
\end{equation}

As an exercise, check that
\begin{equation}
  \A{}\Ap = \Ap\A{} = \I{}.
\end{equation}

%%
\subsection{When $\prdmm{*}=\I{}$}
Because the product matrix
\begin{equation}
  \prdmm{*} = \I{}
\end{equation}
all of the singular values are unity. Therefore
\begin{equation}
  \sig{}=\sig{T}=\I{}.
\end{equation}
\begin{equation}
  \begin{split}
    \prdmm{*}&=\I{},\\
    \paren{\svdt{*}}\paren{\svd{*}}&=\I{},\\
    \X{}\X{*}&=\I{}.
  \end{split}
\end{equation}

	
\endinput