\section{Chiral inverses}

\subsection{In pictures}
\begin{table}[htdp]
\begin{center}
\begin{tabular}{cccl}
  $\X{}$ & $\Y{}$ & inverse type \\\hline
  %
  \includegraphics[]{pdf/"ch 14"/"r 2 c 2 s 0"} &
  \includegraphics[]{pdf/"ch 14"/"r 2 c 2 s 0"} &
  $\Ap = \A{-1}$ \\[20pt]
  %
  \includegraphics[]{pdf/"ch 14"/"r 2 c 2 s 0"} &
  \includegraphics[]{pdf/"ch 14"/"r 3 c 3 s 1"} &
  $\Ap = \AinvL$ \\[20pt]
  %
  \includegraphics[]{pdf/"ch 14"/"r 3 c 3 s 1"} &
  \includegraphics[]{pdf/"ch 14"/"r 2 c 2 s 0"} &
  $\Ap = \AinvR$ \\[20pt]
  %
  \includegraphics[]{pdf/"ch 14"/"r 2 c 2 s 1"} &
  \includegraphics[]{pdf/"ch 14"/"r 3 c 3 s 2"} &
  $\Ap \ne \AinvL,\ \Ap \ne \AinvR$ \\
  %
\end{tabular}
\end{center}
\label{tab:14:experiment:b}
\caption{Chiral inverses in pictures. This table shows that the presence of a null space determines the type of matrix inverse. We see representative domain matrices under $\X{}$ and codomain matrices under $\Y{}$. Following convention, column vectors in the null space are shaded.}
\end{table}%

\begin{landscape}
\subsection{Rank conditions}
\begin{table}[htdp]
\begin{center}
\begin{tabular}{llllll}
%
 chirality & $\A{}$ & equivalence & identity relation & $\rho$ & rank \\\hline
%
 ambichiral & $\cmplxmm$ & $\Ap = \A{-1}$ & $\leftinv = \rightinv = \I{m}$ & $\rho = m$ & full matrix rank \\
%
 left & $\cmplxmn$ & $\Ap = \AinvL$ & $\leftinv = \I{n}$ & $\rho = n$ & full column rank \\
%
 right & $\cmplxmn$ & $\Ap = \AinvR$ & $\rightinv = \I{m}$ & $\rho = m$ & full row rank \\
%
 ampichiral & $\cmplxmn$ &   & & $\rho \ne n$  & column rank deficiency \& \\
 &&& & $\rho \ne m$& row rank deficiency
%
\end{tabular}
\end{center}
\label{tab:14:experiment:a}
\caption{Pseudoinverse chirality and rank. A full rank condition - either in row or column - produces a chiral inverse. With rank deficiency in both row and column we obtain an ambichiral inverse, an inverse that neither right nor left handed.}
\end{table}%
\end{landscape}


\endinput