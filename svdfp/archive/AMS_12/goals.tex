\chapter*{Goals for each chapter}

Here are the goals for each chapter.

\section*{Chapter 0: What is a matrix?}
\begin{enumerate}
\item Consider a matrix as a collection of row and column vectors.
\item The row vectors describe a vector space; the column vectors describe a separate vector space.
\item Develop a geometric viewpoint of the SVD based on these two vector spaces.
\item Understand embedded domains; e.g. $\real{1}$ embedded in $\real{3}$.
\item Understand the dimensions of the matrices output by the decomposition.
\end{enumerate}

Critical concepts:
\begin{enumerate}
\item $\real{N}$ as a domain;
\item $\cmplx{N}$ as a domain;
\item vector space;
\item matrix domain;
\item matrix codomain;
\end{enumerate}

\section*{Chapter 1: SVD without eigenvalues}
\begin{enumerate}
\item Demonstrate an example of matrices that can be decomposed without solving an eigenvalue problem.
\item Understand the geometric significance of the least squares solution to an inconsistent linear system.
\item Reinforce the difference between the domain and the codomain.
\end{enumerate}

Critical concepts:
\begin{enumerate}
\item the method of least squares;
\item pseudoinverse (\textit{aka} generalized matrix inverse);
\end{enumerate}

\section*{Chapter 2: Post mortem I}
\begin{enumerate}
\item Present the most general method for performing a \svdl which requires solving an eigenvalue problem.
\end{enumerate}

Critical concepts:
\begin{enumerate}
\item solving eigenvalue problems;
\item pseudoinverse (\textit{aka} generalized matrix inverse);
\end{enumerate}


\endinput