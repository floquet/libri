\section{The product matrices and orthogonal expansions}

The product matrices $\W{x}$ and $\W{y}$ offer important insights to the SVD.

The outer products express the orthogonality of the expansion. For example
\begin{equation}
  \begin{split}
    \W{x} = \prdm{*} = \paren{\sum_{k=1}^{\rho}{\sigma_{k}\, \Y{}_{k}\, \X{*}_{k}}}^{*}\sum_{j=1}^{\rho}{\sigma_{j}\, \X{}_{j}\, \Y{*}_{j}}.
  \end{split}
\end{equation}
The orthogonality condition flows naturally from the unitary nature of the domain matrices:
\begin{equation}
  \X{*}_{k}\,\X{}_{j} = \Y{*}_{k}\,\Y{}_{j} =
  \begin{cases}
    1 & k = j\\
    0 & k \ne j
  \end{cases}.
\end{equation}

The cross terms with $k\ne j$ vanish:
\begin{equation}
  \begin{split}
    \paren{\sigma_{j}\, \Y{}_{j}\, \X{*}_{j}}^{*}\paren{\sigma_{k}\, \Y{}_{k}\, \X{*}_{k}} &= \paren{\sigma_{j}\X{}_{j}\, \Y{*}_{j}}\paren{\sigma_{k}\Y{}_{k}\, \X{*}_{k}}, \\
    &= \sigma_{k}\sigma_{j}\X{}_{k}\underbrace{\paren{\Y{*}_{k}\,\Y{}_{j}}}_{0}\X{*}_{j}\\
    & = \zero,
  \end{split}
\end{equation}
an $m\times m$ matrix of zeros.

The cross terms with $k=j$ are the survivors:
\begin{equation}
  \begin{split}
    \paren{\sigma_{k}\, \Y{}_{k}\, \X{*}_{k}}^{*}\paren{\sigma_{k}\, \Y{}_{k}\, \X{*}_{k}}&= \paren{\sigma_{k}\X{}_{k}\, \Y{*}_{k}}\paren{\sigma_{k}\Y{}_{k}\, \X{*}_{k}} \\
    &= \sigma_{k}^{2}\,\X{}_{k}\paren{\Y{*}_{k}\,\Y{}_{k}}\X{*}_{k}\\
    &= \sigma_{k}^{2}\,\X{}_{k}\,\X{*}_{k}.
  \end{split}
\end{equation}

The final result is this
\begin{equation}
  \W{x} = \prdm{*} = \sum_{k=1}^{\rho}{\sigma_{k}^{2}\,\X{}_{k}\,\X{*}_{k}}.
\end{equation}
By similar machinations we find the complementary product matrix
\begin{equation}
  \W{y} = \prdmm{*} = \sum_{k=1}^{\rho}{\sigma_{k}^{2}\,\Y{}_{k}\,\Y{*}_{k}}.
\end{equation}
Notice the similarity of this formulation to the expression using the full matrices.

In terms of the \svdl \ the product matrices are these
\begin{equation}
  \begin{array}{rccccccl}
    \W{x} &=& \prdm{*}  &=& \paren{\svd{*}}^{*}&\svd{*} &=& \wx{*},\\
    \W{y} &=& \prdmm{*} &=& \svd{*}&\paren{\svd{*}}^{*} &=& \wy{*}.
  \end{array}
\end{equation}

Compare the equivalent formulations to see the stenciling and shape arbitration effects of the $\sig{}$ matrix. The matrix formulation shows the stenciling action of the $\sig{}$ matrix. The second expression is a sum of outer products scaled by the squares of the singular values:
\begin{equation}
  \begin{array}{rcccc}
    \W{x} &=& \wx{*} &=& \sum_{k=1}^{\rho}{\sigma_{k}^{2}\,\X{}_{k}\,\X{*}_{k}},\\[10pt]
    \W{y} &=& \wy{*} &=& \sum_{k=1}^{\rho}{\sigma_{k}^{2}\,\Y{}_{k}\,\Y{*}_{k}}.
  \end{array}
\end{equation}
The Fourier-Bessel expansion represents the thin SVD\index{thin SVD}; the null space vectors are never encountered.

The following examples should strengthen these concepts.

%%
\section{Examples}
Several examples follow to reinforce the central concept of the orthogonal decomposition. Each example is basic enough to allow the reader to follow along with pencil and paper.

%%
\subsection{Rank one example}
Consider the familiar example of the matrix
\begin{equation}
  \begin{split}
    \svda{T}\\
    \Aexample &=\Yshade \Sigmaexampleb \Xtshade.
  \end{split}
\end{equation}

\begin{equation}
  \begin{array}{rcccc}
    \A{}&=&\sigma_{1} & \Y{}_{1} & \X{T}_{1}\\
      &=& \sqrt{6} & \frac{1}{\sqrt{3}}\mat{r}{1\\-1\\1} & \frac{1}{\sqrt{2}}\mat{cc}{1&-1}\\
      &=&& \Aexample
  \end{array}
\end{equation}

%%
\subsection{Rank two examples}
\begin{equation}
 \A{} = \sigma_{1}\, \Y{}_{1}\, \X{T}_{1} + \sigma_{2}\, \Y{}_{2}\, \X{T}_{2}
\end{equation}

%%
\subsection{Full rank}
\begin{equation}
  \begin{split}
    \svda{T}\\
    \mat{rr}{1&2\\-1&2}&=\frac{1}{\sqrt{2}}\mat{rr}{1&1\\1&-1}\,\sqrt{2}\mat{cc}{2&0\\0&1}\mat{cc}{0&1\\1&0}    
  \end{split}
\end{equation}
%
\begin{equation}
  \begin{array}{ccccccccc}
    \A{} &=& \sigma_{1} & \Y{}_{1} & \X{T}_{1} &+& \sigma_{2} & \Y{}_{2} & \X{T}_{2} \\[5pt]
     &=& 2\sqrt{2} & \frac{1}{\sqrt{2}}\mat{r}{1\\1} & \mat{rr}{0&1} &+& \sqrt{2} & \frac{1}{\sqrt{2}}\mat{r}{1\\-1} & \mat{rr}{1&0}\\[10pt]
     &=&& \mat{cc}{0&2\\0&2} &&+&& \mat{rr}{1&0\\-1&0}\\[5pt]
     &=&& \mat{rr}{1&2\\-1&2}
  \end{array}
\end{equation}

%%
\subsection{Rank deficient}
The Gell-Mann matrix 2:
\begin{equation}
  \begin{split}
    \svda{T}\\
    \gmb&=\left(
\begin{array}{rr>{\columncolor{ltgray}}r}
 -i & 0 & 0 \\
 0 & i & 0 \\
 0 & 0 & 1
\end{array}
\right)\left(
\begin{array}{rr|r}
 1 & 0 & 0 \\
 0 & 1 & 0 \\\hline
 0 & 0 & 0
\end{array}
\right)\left(
\begin{array}{rrr}
 0 & 1 & 0 \\
 1 & 0 & 0 \\
\rowcolor{ltgray}
 0 & 0 & 1
\end{array}
\right)    
  \end{split}
\end{equation}
%%
\begin{equation}
  \begin{array}{ccccccccc}
    \A{} &=& \sigma_{1} & \Y{}_{1} & \X{T}_{1} &+& \sigma_{2} & \Y{}_{2} & \X{T}_{2}, \\
     &=& 1 & \mat{r}{-i\\0\\0} & \mat{rrr}{0&1&0} &+& 1 & \mat{c}{0\\i\\0} & \mat{rrr}{1&0&0}.
  \end{array}
\end{equation}
%%
This leads to the matrix sum
\begin{equation}
     \A{} = \mat{rrr}{0&-i&0\\0&0&0\\0&0&0} + \mat{ccc}{0&0&0\\i&0&0\\0&0&0}= \mat{crr}{0&-i&0\\i&0&0\\0&0&0}.
\end{equation}
%%
\subsection{Full rank}
\begin{equation}
  \begin{split}
    \svda{T}\\
    \mat{ccc}{0&3&0 \\ 1&2&2} &=\frac{1}{\sqrt{2}}\mat{rr}{1&-1\\1&1}\,\mat{cc|c}{\sqrt{15}&0&0 \\ 0&\sqrt{3}&0}\, \mat{rrr}{\frac{1}{\sqrt{30}}&\frac{5}{\sqrt{30}}&\frac{2}{\sqrt{30}} \\ \frac{1}{\sqrt{6}} & \frac{-1}{\sqrt{6}} & \frac{2}{\sqrt{6}} \\ \rowcolor{ltgray}
\frac{-2}{\sqrt{5}} & 0 & \frac{1}{\sqrt{5}}}.
  \end{split}
\end{equation}

\begin{equation}
  \begin{array}{ccccccccc}
    \A{} &=& \sigma_{1} & \Y{}_{1} & \X{T}_{1} &+& \sigma_{2} & \Y{}_{2} & \X{T}_{2} \\
     &=& \sqrt{15} & \stwo \mat{r}{1\\1} & \mat{rrr}{\frac{1}{\sqrt{30}}&\frac{5}{\sqrt{30}}&\frac{2}{\sqrt{30}}} &+& \sqrt{3} & \stwo \mat{r}{-1\\1} & \mat{rrr}{\frac{1}{\sqrt{6}} & \frac{-1}{\sqrt{6}} & \frac{2}{\sqrt{6}}}
  \end{array}
\end{equation}
The two rank one component matrices combine like so:
\begin{equation}
    \A{} = \rtwo \mat{ccc}{1&5&2\\1&5&2} + \rtwo \mat{rrr}{-1&1&-2\\1&-1&2} = \mat{rrr}{0&3&0 \\ 1&2&4}
\end{equation}

%%
\subsection{Rank three examples}
The Gell-Mann matrix 8, the only matrix with a nonzero trace and full rank:
\begin{equation}
  \begin{split}
    \svda{T}\\
    \gmh&=\left(
\begin{array}{rrr}
 0 & 0 & 1 \\
 0 & 1 & 0 \\
 -1 & 0 & 0
\end{array}
\right)
\frac{1}{\sqrt{3}}\left(
\begin{array}{ccc}
 2 & 0 & 0 \\
 0 & 1 & 0 \\
 0 & 0 & 1
\end{array}
\right)\left(
\begin{array}{ccc}
 0 & 0 & 1 \\
 0 & 1 & 0 \\
 1 & 0 & 0
\end{array}
\right)
  \end{split}
\end{equation}

\begin{equation}
  \begin{array}{ccccccccccccc}
    \A{}_{1} &=& \sigma_{1} & \Y{}_{1} & \X{T}_{1} \\
    &=& \frac{2}{\sqrt{3}} & \mat{r}{0\\0\\-1}&\mat{rrr}{0&0&1}
  \end{array}
\end{equation}
\begin{equation}
  \begin{array}{ccccccccccccc}
    \A{}_{2} &=& \sigma_{2} & \Y{}_{2} & \X{T}_{2} \\
    &=& \sthree& \mat{r}{0\\1\\0}&\mat{rrr}{0&1&0}
  \end{array}
\end{equation}
\begin{equation}
  \begin{array}{ccccccccccccc}
    \A{}_{3} &=& \sigma_{3} & \Y{}_{3} & \X{T}_{3} \\
    &=& \sthree & \mat{r}{1\\0\\0}&\mat{rrr}{1&0&0}
  \end{array}
\end{equation}
The three rank one component matrices sum to the original matrix
\begin{equation}
  \begin{split}
    \A{} &= \frac{2}{\sqrt{3}}  \mat{rrr}{0&0&0 \\ 0&0&0 \\ 0&0&-1} + \frac{1}{\sqrt{3}}  \mat{rrr}{0&0&0 \\ 0&1&0 \\ 0&0&0} + \frac{1}{\sqrt{3}}  \mat{rrr}{1&0&0 \\ 0&0&0 \\ 0&0&0}\\
    &=\frac{1}{\sqrt{3}} \mat{rrr}{1&0&0 \\ 0&1&0 \\ 0&0&-2}
  \end{split}
\end{equation}
%%
\subsection{Rank one example}
Some explicit examples are included.
\begin{equation}
  \A{}=\Aexample
\end{equation}
The $\X{}$ domain product is given by:
\begin{equation}
  \W{x} = \prdm{T} = \mat{rr}{3&-3\\-3&3}.
\end{equation}
The Fourier-Bessel product is then
\begin{equation}
  \begin{split}
    \W{x} &= \sigma_{1}^{2} \X{}_{1}\X{T}_{1}\\
      &= \paren{\sqrt{6}}^{2} \stwo \mat{r}{1\\-1}\stwo \mat{rr}{1&-1}\\
      &= \mat{rr}{3&-3\\-3&3}.
  \end{split}
\end{equation}

The complementary $\Y{}$ domain product is given by:
\begin{equation}
  \W{y} = \prdmm{T} = \mat{rrr}{3&-3&3\\-3&3&-3\\3&-3&3}.
\end{equation}
The Fourier-Bessel product is then
\begin{equation}
  \begin{split}
    \W{y} &= \sigma_{1}^{2} \Y{}_{1}\Y{T}_{1}\\
      &= \paren{\sqrt{6}}^{2} \sthree \mat{r}{1\\-1\\1} \sthree \mat{rrr}{1&-1&1}\\
      &= \mat{rrr}{2&-2&2\\-2&2&-2\\2&-2&2}.
  \end{split}
\end{equation}

%%
\subsection{Rank two example}
Start with the full rank matrix
\begin{equation}
  \A{} = \mat{ccc}{0 & 3 & 2 \\ 1 & 2 & 2}
\end{equation}
which has the matrix products
\begin{equation}
  \begin{split}
    \W{x} &= \prdm{T}  = \mat{cc}{9 & 6 \\6 & 9},\\
    \W{y} &= \prdmm{T} = \mat{ccc}{1 & 2 & 2 \\ 2 & 13 & 4 \\ 2 & 4 & 4}.
  \end{split}
\end{equation}

The Fourier-Bessel products for $\W{x}$ are then
\begin{equation}
  \begin{split}
    \W{x} &= \sigma_{1}^{2}\; \X{}_{1}\X{T}_{1} + \sigma_{2}^{2}\; \X{}_{2}\X{T}_{2}\\
    &= \paren{\sqrt{15}}^{2}\stwo \mat{r}{1\\1}\stwo \mat{rr}{1&1} +  \paren{\sqrt{6}}^{2}\stwo \mat{r}{1\\-1}\stwo \mat{rr}{1&-1}\\
    &= \mat{cc}{9 & 6 \\6 & 9}.
  \end{split}
\end{equation}

The Fourier-Bessel products for the complementary product matrix $\W{y}$ are given by this
\begin{equation}
  \begin{split}
    \W{y} &= \sigma_{1}^{2}\; \Y{}_{1}\Y{T}_{1} + \sigma_{2}^{2}\; \Y{}_{2}\Y{T}_{2}\\
    &= \paren{\sqrt{15}}^{2}\frac{1}{\sqrt{30}} \mat{c}{1\\5\\2}\frac{1}{\sqrt{30}} \mat{ccc}{1&5&2} +  \paren{\sqrt{3}}^{2}\ssix \mat{r}{1\\-1\\2}\ssix \mat{crc}{1&-1&2}\\
    &= \mat{ccc}{1 & 2 & 2 \\ 2 & 13 & 4 \\ 2 & 4 & 4}.
  \end{split}
\end{equation}

Next is the Gell-Mann matrix $\lambda_{2}$
\begin{equation}
  \A{} = \gmb.
\end{equation}
The row of zeros and the column of zeros belies a rank deficiency. The independent rows reveals the matrix rank is two. The $\X{}$ domain product is given by:
\begin{equation}
  \W{x} = \prdm{T} = \mat{rrc}{-1&0&0\\0&-1&0\\0&0&0}.
\end{equation}
The Fourier-Bessel product is then
\begin{equation}
  \begin{split}
    \W{x} &= \sigma_{1}^{2} \X{}_{1}\X{T}_{1} + \sigma_{1}^{2} \X{}_{2}\X{T}_{2}\\
      &= \paren{1}^{2} \mat{c}{0\\1\\0} \mat{ccc}{0&1&0} + \paren{1}^{2} \mat{c}{1\\0\\0} \mat{ccc}{1&0&0}\\
      &= \mat{rrc}{-1&0&0\\0&-1&0\\0&0&0}.
  \end{split}
\end{equation}

The complementary $\Y{}$ domain product is given by:
\begin{equation}
  \W{y} = \prdmm{T} = \mat{rrc}{-1&0&0\\0&-1&0\\0&0&0} = \W{x}.
\end{equation}
The Fourier-Bessel product is then
\begin{equation}
  \begin{split}
    \W{y} &= \sigma_{1}^{2} \Y{}_{1}\Y{T}_{1} + \sigma_{1}^{2} \Y{}_{2}\Y{T}_{2}\\
      &= \paren{1}^{2} \mat{r}{-i\\0\\0} \mat{ccc}{-i&0&0} + \paren{1}^{2} \mat{c}{0\\i\\0} \mat{ccc}{0&i&0}\\
      &= \mat{rrc}{-1&0&0\\0&-1&0\\0&0&0},
  \end{split}
\end{equation}
the same as the previous result.

\endinput