\section{Shortcuts}
In some cases, we are able to obviate the eigenvalue problem altogether.

\begin{equation}
  \A{} = \mat{rr}{
   1 & -1 \\
  -1 &  1 \\
   1 & -1 }
\end{equation}

Let the left- and right-hand eigenvectors be
\begin{equation}
  u_{1} = \recip{\sqrt{2}} \mat{r}{1 \\ -1}, \qquad
  v_{1} = \recip{\sqrt{3}} \mat{r}{1 \\ -1 \\ 1}.
\end{equation}

What would the singular value $\sigma_{1}$ be?
%
\begin{equation}
  \begin{split}
  \A{} 
  &= \mat{rr}{
   1 & -1 \\
  -1 &  1 \\
   1 & -1 } \\
%
  &=
  \bur{}\, \ess{}\, \bvr{*} \\
%
  &=
  {\bl{ \recip{\sqrt{3}} \mat{r}{1 \\ -1 \\ 1} }} 
  \mat{c}{ \sigma_{1} }
  {\bl{ \recip{\sqrt{2}} \mat{cr}{1 & -1} }} \\
%
  &=
  \frac{\sigma_{1}} {\sqrt{6}}
  \mat{rr}{
   1 & -1 \\
  -1 &  1 \\
   1 & -1 }.
%
  \end{split}
\end{equation}
%
Certainly then $\sigma_{1} = \sqrt{6}$.
  
What of the \ns s? To complete the codomain $\cmplx{2}$, the only real choice is
\begin{equation}
  v_{2} = {\rd{ \recip{\sqrt{2}} \mat{c}{1 \\ 1} }}
\end{equation}
or the negative of this. We need two vectors to complete $\cmplx{3}$. A first attempt should include a 0.
\begin{equation}
  \begin{split}
    \bvo \cdot \rvt &= 0 , \\
%
   {\bl{ \mat{r}{1 \\ -1 \\ 1} }} \cdot
   {\rd{ \mat{r}{0 \\ \num \\ \num}}} &= 0 
%
   \quad \Rightarrow\quad  \rvt =
   {\rd{ \mat{c}{0 \\ 1 \\ 1} }}
%
  \end{split}
\end{equation}



\endinput