\subsection{Matrix (b)}
This matrix has a row rank deficiency and full column rank. The \ns \ $\rnlla{*}$ is nontrivial.

%%%%%%%%%%%
\subsubsection{Column space}
ThThe first step in the reduction is to alter the order of the rows by interchanging the first and second rows:
\begin{equation}
  \pee{} \A{} = 
  \mat{ccc}{
  0 & 1 & 0 \\
  1 & 0 & 0 \\
  0 & 0 & 1
  }
  \matrixb = 
  \recip{4}
  \mat{cc}{
  3 & 2 \\
  0 & 1 \\
  0 & 2
  }
\end{equation}
The reduction is now immediate. By inspection we see that it is equivalent to a reduced form
\begin{equation}
  \begin{split}
\arrayrulecolor{black}
    \R{} \mat{c|c}{\pee{}\A{} & \pee{}\I{3}} &= \mat{c|c}{\E{} & \R{}} \\
    \recip{4}
    \mat{rrr}{1 & 0 & 0 \\ 0 & 1 & 0 \\ \phantom{-}0 & -2 & \phantom{-}1 }
    \mat{cc|ccc}{3 & 2 & 0 & 1 & 0 \\ 0 & 1 & 1 & 0 & 0 \\ 0 & 2 & 0 & 0 & 1} &=\recip{4}
\arrayrulecolor{medgray}
    \mat{cc!{\color{black}\vline}rrr}{\boxed{3} & 2 & 1 & 0 & 0 \\  0 & \boxed{1} & \phantom{-}0 & 1 & \phantom{-}0 \\\hline 0 &  0 & \rminus\rtwo & \rzero & \rone}
  \end{split}
\end{equation}
\arrayrulecolor{black}

The codomain is resolved as
\begin{equation}
  \begin{split}
    \cmplx{3} 
      &= \qquad \qquad \ \ \brnga{} \qquad \quad \, \  \oplus \ \qquad \, \rnlla{*} \\
      &= \spn{ \bvecba, \bvecbb }\  \oplus \ \spn{ \rvecbc }.
  \end{split}
  \label{eq:ftola:b:codomain}
\end{equation}
Of course the span of the range space could also be written as
%
\begin{equation}
  \brnga{} = \spn{ {\bl{ \vecba }}, {\bl{ \vecbb }} } .
\end{equation}
%
For this chapter we deliberately include the fractions inside the spans to emphasize the origin within the target matrix.

Here too we see this set of spanning vectors in not orthogonal:
\begin{equation}
  {\bl{ \vecba }} \cdot {\bl{ \vecbb }} = 6 \ne 0.
\end{equation}
The spanning set from \asvd \ will not only be orthogonal, they will also be normalized.

%%%%%%%%%%%
\subsubsection{Row space}

Onto the row space. Again we resolve the column space of the Hermitian conjugate matrix. More specifically, the equivalent form in equation \eqref{eq:ftola:b}:
\begin{equation}
  \matrixbt \sim \recip{4}\mat{ccc}{\boxed{1} & 2 & 2 \\ 0 & \boxed{3} & 0}.
  \label{eq:ftola:b}
\end{equation}
There are two nonzero pivots and therefore the first two columns of the matrix $\A{*}$ are basic. These columns define the span of the range of $\A{*}$:
\begin{equation}
  \brnga{} = \spn{ \bvecbd, \bvecbe }.
  \label{eq:ftola:b domain}
\end{equation}
Using the rank plus nullity theorem, we see that the \ns \ must be trivial. That is
\begin{equation}
  \rnlla{*} = \lst{ \gzerotwo } = \trivial.
\end{equation}
The space $\cmplx{2}$ will be spanned by two linearly independent \vv s and the span of the $\brnga{*}$ has two linearly independent vectors. 
\begin{equation}
  \begin{split}
    \cmplx{2} 
      &= \qquad \qquad \ \ \brnga{} \qquad \quad \,  {\mg{\  \oplus \ }} \qquad \, \gnlla{*} \\
      &= \spn{ \bvecbd, \bvecbe } .
  \end{split}
  \label{eq:ftola:b:domain}
\end{equation}

%
All four fundamental spaces are resolved and we see that $\A{}\in\cmplxall{2}{3}{2}$.

\endinput