\section{Beyond the \ft}
\label{sec:beyond ft}
The \ftola \ provides a robust foundation for matrix analysis as it details the subspace decomposition of the given matrix. When we look at the matrix maps we see how the \ft \ diagnosed the dimensions of the images and predicted whether the maps where surjective or onto or one-to-one. This is invaluable information which we will exploit when we turn to the topics of least squares and the pseudoinverse.

As we leave this chapter, we that the \ft \ tells us about the subspaces induced by a matrix. We are motivated to 
\begin{enumerate}
\item resolve the subspaces with an orthonormal spanning set,
\item resolve the dilation and rotation from the preimage to the image.
\end{enumerate}


Yet a basic tool for investigation, the plot of the matrix image reveals a different type of information unavailable to the Theorem: the dilation and orientation of the maps. Such details are tied to crucial concepts such as the eigenvalue spectrum and the matrix condition. The concepts of stretching an alignment are an augury of the \asvd.


\endinput