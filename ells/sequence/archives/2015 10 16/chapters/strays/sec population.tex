\break
\clearpage
\section{\label{sec:population}Population growth}

In this section we take a nonlinear model for population growth and separate the linear and nonlinear terms.

% SUBSECTION
\subsection{Population growth}
\input{\pathtables "tab dubious data and soln"}  %  <  <  <  <  <  <  <  <  <  <  <  <

% = =  e q u a t i o n
  \begin{equation}
    %\begin{split}
      y(t) = c_{1} + c_{2} \paren{t-1900} + c_{3} e^{d\paren{t-1900}}
    %\end{split}
    %\label{eqn:}
  \end{equation}
% = =

% = =  e q u a t i o n
  \begin{equation}
    %\begin{split}
      \min_{c\in\real{3}} \normts{\A{}(d)\mat{c}{c_{1}\\c_{2}\\c_{3}}-y}
    %\end{split}
    %\label{eqn:}
  \end{equation}
% = =
\input{\pathtables "tab census"}                   %  <  <  <  <  <  <  <  <  <  <  <  <
\input{\pathfigures "fig population error plots"}  %  <  <  <  <  <  <  <  <  <  <  <  <
\input{\pathfigures "fig census data v fit"}       %  <  <  <  <  <  <  <  <  <  <  <  <
\input{\pathfigures "fig census residuals"}        %  <  <  <  <  <  <  <  <  <  <  <  <
\input{\pathfigures census/"fig merit"}            %  <  <  <  <  <  <  <  <  <  <  <  <
\input{\pathfigures census/"fig merit 3d"}         %  <  <  <  <  <  <  <  <  <  <  <  <
\input{\pathtables census/"tab results"}           %  <  <  <  <  <  <  <  <  <  <  <  <

\endinput  %  -  -  -  -  -  -  -  -  -  -  -  -  -  -  -  -  -  -  -  -