%%%%%%%%%%%%%%
%%%%%%%%%%%%%%
\subsubsection{$\bigstar$ n = 3}
This is the simplest nontrivial Jordan block for $n=3$.
\begin{equation}
  \J{3} = \mat{ccc}{1&1&0 \\ 0&1&1 \\ 0&0&1}
\end{equation}
Finding the singular values involves finding the roots of a cubic equation. The product matrix is
%
\begin{equation}
  \W{u} = \J{3}^{*}\,\J{3}^{\phantom{*}} = \mat{ccc}{1&1&0 \\ 1&2&1 \\ 0&1&2},
\end{equation}
%
and the intermediate quantities are
%
\begin{equation}
  \begin{split}
    \trace \paren{\W{u}}     &= 5, \\
    \trace \paren{\W{u}^{2}} &= 13, \\
    \det \paren{\W{u}}       &= 1.
  \end{split}
\end{equation}
%
This lead to the characteristic equation
%
\begin{equation}
  p\paren{\lambda} = -\lambda^{3} + 5 \lambda^{2} - 6 \lambda + 1
\end{equation}
%
The singular values are the square roots of the zeros of this characteristic polynomial. We solve this problem using \emph{Mathematica}. Defining the parameter
%
\begin{equation}
  \beta = \frac{\pi} {6} + \recip{3} \arctan{3 \sqrt{3}},
\end{equation}
%
the singular values are expressed as
%
\begin{equation}
  \sigma = \recip{3}
  \lst{ 1 + 2 \sqrt{7} \sin \paren{\beta} ,
       -1 + 2 \sqrt{7} \cos \paren{\beta} ,
        1 +  \sqrt{21} \cos \paren{\beta} - \sqrt{7} \sin \paren{\beta} }
\end{equation}
%
The domain matrices have a simple form revealed after suitable substitutions:
%
\begin{equation}
  \begin{split}
    a &= -1 + 2 \cos \paren{\alpha} ,  \\
    b &= -1 -   \cos \paren{\alpha} + \sqrt{3} \cos \paren{\alpha} , \\
    c &= -1 -   \cos \paren{\alpha} - \sqrt{3} \cos \paren{\alpha} ;
  \end{split}
\end{equation}
%
the domain matrices are now
%
\begin{equation}
  \begin{split}
%
    \bur{} = {\bl{ \recip{3} \mat{rrr}{
			      b & -c &  a \\
			      c &  a & -b \\
			  \ps a &  b &  c } }} , \qquad
%
    \bvr{} = {\bl{ \recip{3} \mat{rrr}{
					  a & -b &  c \\
					  c & -a & -b \\
			  \ps b &  c &  a } }}
%
  \end{split}
\end{equation}

\endinput