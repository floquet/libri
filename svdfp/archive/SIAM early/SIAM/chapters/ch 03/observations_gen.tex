\section{Observations}
We have experienced the full process of \svdl, a series of basic steps. 

The options at many steps which allow easier decomposition confuse some practitioners. These options may be mistakenly juggled with path variations and the SVD process may seem hazy as a result. Keep these things straight in your mind:
\begin{enumerate}
\item Know what you are trying to do;
\subitem resolve the domain and codomain into orthonormal coordinate systems;
\subitem resolve the singular values.
\item Know the foundation issues;
\subitem the row and vector spaces may not fully span the host space;
\subitem the row and column vectors may be far from orthogonal.
\item Remember how to perform each step;
\subitem know how to find eigenvalues and eigenvectors using augmented reduction or other techniques;
\subitem know how to construct perpendicular spaces;
\subitem know how to construct and load the matrix of singular values.
\end{enumerate}

The examples here are pedagogic and illuminate the process and the machinations. Many useful matrices will quickly submit to a \svdl. Hence these theoretical examples are both helpful and useful. However, the decomposition is intractable for many matrices.

Eigenvector problems can be arbitrarily complex. The space may be incomplete and finding mother-daughter eigenvector pairs become a permutation problem which grows factorially. This can be a major impediment.

There are many excellent texts and on-line resources to help with the machinations of finding eigenvalues and constructing orthogonal complements. A survey is listed here:

\begin{table}[htdp]
\begin{center}
\begin{tabular}{lcccc}
  topic & Anton [] & Strang1 [] & Strang2 [] & Meyer \cite{Meyer} [] \\\hline
  augmented reduction & pp. 10-20 & pp. 10-20 & pp. 10-20 & p. 118\\
  eigenvectors & pp. 10-20 & pp. 10-20 & pp. 10-20 & p. 489\\
  null spaces  & pp. 10-20 & pp. 10-20 & pp. 10-20 & p. 173
\end{tabular}
\end{center}
\label{default}
\caption{Resources to help with facets of the \svdl.}
\end{table}


\endinput