\chapter{Stitching Local Maps}

\section{What is stitching?}      %    S    S    S    S    S    S    S    S    S    S    S    S
Stitching is the process of combining local maps to create a global map.

\begin{figure}[htbp] %  figure placement: here, top, bottom, or page
   \centering
   \includegraphics[ width = 4in ]{\pathgraphics "stitch"/"piston"/"ABC"} 
   \caption{Stitching local maps together to form a global map.}
   %\label{fig:example}
\end{figure}

  \begin{enumerate}
    \item $\phi$
    \item $\nabla \phi$
    \item $\phi$ and $\nabla \phi$
  \end{enumerate}

\section{Stitch $\phi$} %    S    S    S    S    S    S    S    S    S    S    S    S
  \subsection{Genesis}  %   SS   SS   SS   SS   SS   SS   SS   SS   SS   SS   SS   SS

  \begin{equation*}   %  =   =   =   =   =
  %\begin{split}
    \phi(x) = \exp \paren{-\frac{x}{5}} \sin \paren{\pi x}  
    \label{eq:phi}
  %\end{split}
  \end{equation*}

\begin{figure}[htbp] %  figure placement: here, top, bottom, or page
   \centering
   \includegraphics[ width = 4in ]{\pathgraphics "stitch"/"piston"/"stitch input"} 
   \caption{The ideal potential function showing five measurement zones and four overlap bands.}
   \label{fig:example}
\end{figure}

\begin{figure}[htbp] %  figure placement: here, top, bottom, or page
   \centering
   \includegraphics[ width = 4in ]{\pathgraphics "stitch"/"piston"/"dots and overlaps"} 
   \caption{Waterfall diagram showing discretization within measurement zones with left and right zone overlaps.}
   \label{fig:dots and overlaps}
\end{figure}

  \begin{table}[htbp]  %  T A B L E
    \caption{The input data in continuous and discrete form.}
    \begin{center}
      \begin{tabular}{c}
        %
        \includegraphics[ width = 4in ]{\pathgraphics "stitch"/"piston"/"stitch broken labeled"} \\
        \huge{$\Downarrow$} \\
        %\includegraphics[ width = 4in ]{\pathgraphics "stitch"/"piston"/"stitch discrete in"}
        \includegraphics[ width = 4in ]{\pathgraphics "stitch"/"piston"/"stitch discrete steps"}
        %
      \end{tabular}
    \end{center}
  %\label{tab:?}
  \end{table}%

\begin{figure}[htbp] %  figure placement: here, top, bottom, or page
   \centering
   \includegraphics[ width = 4in ]{\pathgraphics "stitch"/"piston"/"stitch final"} 
   \caption{Stitching unifies the data.}
   \label{fig:stitch final}
\end{figure}

\begin{figure}[htbp] %  figure placement: here, top, bottom, or page
   \centering
   \includegraphics[ width = 4in ]{\pathgraphics "stitch"/"piston"/"pistons grid"} 
   \caption{A set of piston adjustments which restores continuity across the domain.}
   \label{fig:pistons}
\end{figure}

  \subsection{Data}  %   SS   SS   SS   SS   SS   SS   SS   SS   SS   SS   SS   SS
The central idea is simple; the mathematical expression is a tedious exercise in index gymnastics.
  \begin{equation*}   %  =   =   =   =   =
  %\begin{split}
    \zeta = 3
    %\label{eq:}
  %\end{split}
  \end{equation*}

  \begin{table}[htbp]  %  T A B L E
    \caption[Sample showing two zones with overlap.]{Sample showing an overlap of $\zeta=3$ between the first two zones.}
    \begin{center}
      \begin{tabular}{lllllllllll}
        %
        $\phi_{1,1}$ & $\phi_{2,1}$ & $\cdots$ & $\phi_{\lambda_{1}-2,1}$ & $\phi_{\lambda_{1}-1,1}$ & $\phi_{\lambda_{1},1}$ \\
        %
        &&& $\phi_{1,2}$ & $\phi_{2,2}$ & $\phi_{3,2}$ & $\cdots$ & $\phi_{\lambda_{1}-2,2}$ & $\phi_{\lambda_{2}-1,2}$ & $\phi_{\lambda_{2},2}$ \\
        %
      \end{tabular}
    \end{center}
  \label{tab:stitch raw data}
  \end{table}%

  \begin{equation*}   %  =   =   =   =   =
  %\begin{split}
    \Delta_{12} = \zeta^{-1} \paren{
      \underbrace{\paren{\phi_{\lambda_{1}-2,1} + \phi_{\lambda_{1}-1,1} + \phi_{\lambda_{1},1}}}_{\text{zone }1} - 
      \underbrace{\paren{\phi_{1,2} + \phi_{2,2} + \phi_{3,2} }}_{\text{zone }2}}
    %\label{eq:}
  %\end{split}
  \end{equation*}
	%
  \begin{equation*}   %  =   =   =   =   =
  %\begin{split}
    \Delta_{12} = \zeta^{-1} \paren{
      \underbrace{ \paren{ \phi_{\lambda_{1}-2,1} - \phi_{1,2} }}_{\text{pair 1}} + 
      \underbrace{ \paren{ \phi_{\lambda_{1}-1,1} - \phi_{2,2} }}_{\text{pair 2}} +  
      \underbrace{ \paren{ \phi_{\lambda_{1},1}   - \phi_{3,2} }}_{\text{pair 3}}}
    %\label{eq:}
  %\end{split}
  \end{equation*}
Mean value of the differences.
  \begin{equation*}   %  =   =   =   =   =
  %\begin{split}
    \Delta_{j,j+1} = \zeta^{-1}\sum_{k=1}^{\zeta} p_{j,\lambda_{j}-\zeta+k} - p_{j+1,k}
    %\label{eq:}
  %\end{split}
  \end{equation*}
First overlap region.
  \begin{equation*}   %  =   =   =   =   =
  %\begin{split}
    \Delta_{12} = \zeta^{-1} \paren{
      \underbrace{\paren{\phi_{9,1} + \phi_{10,1} + \phi_{11,1} }}_{\text{last 3 elements of zone 1}} -
      \underbrace{\paren{\phi_{1,2} + \phi_{2,2} + \phi_{3,2} }}_{\text{first 3 elements of zone 2}}}
    %\label{eq:}
  %\end{split}
  \end{equation*}
  
\subsection{Data and results}  %   SS   SS   SS   SS   SS   SS   SS   SS   SS   SS   SS   SS
%\begin{table}[h]
%	\begin{center}
%		\begin{tabular}{r r@{.}l r@{.}l r@{.}l r@{.}l r@{.}l}
%		  %
%		  & \multicolumn{2}{c}{zone 1} & \multicolumn{2}{c}{zone 2} & \multicolumn{2}{c}{zone 3} & \multicolumn{2}{c}{zone 4} & \multicolumn{2}{c}{zone 5} \\\hline
%		  %
%        1 & --0 & 2 &  0 & 500878 &  --0 & 210084 &  --0 & 0642517 &  0 & 325113 \\
%        
%        2 & 0 & 102898 &  0 & 258113 &  --0 & 0113248 &  --0 & 226982 &  0 & 458345 \\
%        
%        3 & 0 & 364738 &  0 & 0 &  0 & 2 &  --0 & 4 &  0 & 6 \\
%        
%        4 & 0 & 561904 &  --0 & 247992 &  0 & 403039 &  --0 & 566234 &  0 & 736101 \\
%        
%        5 & 0 & 677936 &  --0 & 462368 &  0 & 578555 &  --0 & 709935 &  0 & 853753 \\
%        
%        6 & 0 & 704837 &  --0 & 623794 &  0 & 710719 &  --0 & 818142 &  0 & 942345 \\
%        
%        7 & 0 & 643511 &  --0 & 718793 &  0 & 788498 &  --0 & 881821 &  0 & 994482 \\
%        
%        8 & 0 & 503326 &  --0 & 740818 &  0 & 806531 &  --0 & 896585 &  1 & 00657 \\
%        
%        9 & 0 & 300878 &  --0 & 690609 &  0 & 765423 &  --0 & 862929 &  0 & 979014 \\
%        
%        10 & 0 & 0581127 &  --0 & 575834 &  0 & 671453 &  --0 & 785993 &  0 & 916025 \\
%        
%        11 & --0 & 2 &  --0 & 410084 &  0 & 535748 &  --0 & 674887 &  0 & 825059 \\
%        
%        12 & & &  --0 & 211325 &  0 & 373018 &  --0 & 541655 &  0 & 715978 \\
%        
%        13 & & &  0 & 0 &  0 & 2 &  --0 & 4 &
%      \end{tabular}
%	\end{center}
%	\caption{Raw data and results.}
%	\label{tab:stitch data and results}
%\end{table}%

\begin{table}
\sisetup{table-number-alignment=center}
    \begin{tabular}{SS[table-format=-1.6]S[table-format=-1.6]S[table-format=-1.6]S[table-format=-1.6]S[table-format=-1.6]}
    $k$ & $y_{1}$ & $y_{2}$ & $y_{3}$ & $y_{4}$ & $y_{5}$ \\\hline
  1 & \color{medgray}-0.2     & 0.500878                 & \color{blue}-0.210084   & -0.0642517               & \color{blue}0.325113 \\
  2 & \color{medgray}0.102898 & 0.258113                 & \color{blue}-0.0113248  & -0.226982                & \color{blue}0.458345 \\
  3 & \color{medgray}0.364738 & 0.                       & \color{blue}0.2         & -0.4                     & \color{blue}0.6 \\
  4 & \color{medgray}0.561904 & \color{medgray}-0.247992 & \color{medgray}0.403039 & \color{medgray}-0.566234 & \color{medgray}0.736101 \\
  5 & \color{medgray}0.677936 & \color{medgray}-0.462368 & \color{medgray}0.578555 & \color{medgray}-0.709935 & \color{medgray}0.853753 \\
  6 & \color{medgray}0.704837 & \color{medgray}-0.623794 & \color{medgray}0.710719 & \color{medgray}-0.818142 & \color{medgray}0.942345 \\
  7 & \color{medgray}0.643511 & \color{medgray}-0.718793 & \color{medgray}0.788498 & \color{medgray}-0.881821 & \color{medgray}0.994482 \\
  8 & \color{medgray}0.503326 & \color{medgray}-0.740818 & \color{medgray}0.806531 & \color{medgray}-0.896585 & \color{medgray}1.00657 \\
  9 & 0.300878                & \color{medgray}-0.690609 & \color{medgray}0.765423 & \color{medgray}-0.862929 & \color{medgray}0.979014 \\
 10 & 0.0581127               & \color{medgray}-0.575834 & \color{medgray}0.671453 & \color{medgray}-0.785993 & \color{medgray}0.916025 \\
 11 & -0.2                    & \color{blue}-0.410084    & 0.535748                & \color{blue}-0.674887    & \color{medgray}0.825059 \\
 12 &                         & \color{blue}-0.211325    & 0.373018                & \color{blue}-0.541655    & \color{medgray}0.715978 \\
 13 &                         & \color{blue}0.           & 0.2                     & \color{blue}-0.4 
    \end{tabular}
	\caption{Measurements displaying the connection between overlap bands in figure \ref{fig:dots and overlaps}.}
	\label{tab:stitch measurements}
\end{table} 

\begin{table}[htbp]  % + + + + T A B L E
    \caption{Computation of the zone shift values.}
    \begin{center}
        \begin{tabular}{lclcl}
            %
            $\Delta_{12}$ & = & $\frac{1}{3} \paren{\paren{\phi_{9,1} + \phi_{10,1} + \phi_{11,1}} - \paren{\phi_{1,2} + \phi_{2,2} + \phi_{3,2}} }$ \\
            %
            $\Delta_{23}$ & = & $\frac{1}{3} \paren{\paren{\phi_{11,2} + \phi_{12,2} + \phi_{13,2}} - \paren{\phi_{1,3} + \phi_{2,3} + \phi_{3,3}} }$ \\
            %
            $\Delta_{34}$ & = & $\frac{1}{3} \paren{\paren{\phi_{11,3} + \phi_{12,3} + \phi_{13,3}} - \paren{\phi_{1,4} + \phi_{2,4} + \phi_{3,4}} }$ \\ 
            %
            $\Delta_{45}$ & = & $\frac{1}{3} \paren{\paren{\phi_{11,4} + \phi_{12,4} + \phi_{13,4}} - \paren{\phi_{1,5} + \phi_{2,5} + \phi_{3,5}} }$      
            %
        \end{tabular}
    \end{center}
    %\label{default}
\end{table}%

\begin{table}[htbp]  % + + + + T A B L E
    \caption{Computation of the zone shift values.}
    \begin{center}
        \begin{tabular}{lclcl}
            %
            $\Delta_{12}$ & = & $\frac{1}{3} \paren{\paren{ 0.300878 + 0.0581127 - 0.2 } - \paren{ 0.500878 + 0.258113 + 0. } }$ \\
            %
            $\Delta_{23}$ & = & $\frac{1}{3} \paren{\paren{ -0.410084 - 0.211325 + 0. } - \paren{ -0.210084 - 0.0113248 + 0.2 } }$ \\
            %
            $\Delta_{34}$ & = & $\frac{1}{3} \paren{\paren{ 0.535748 + 0.373018 + 0.2 } - \paren{ -0.0642517 - 0.226982 - 0.4 } }$ \\ 
            %
            $\Delta_{45}$ & = & $\frac{1}{3} \paren{\paren{ -0.674887 - 0.541655 - 0.4 } - \paren{ 0.325113 + 0.458345 + 0.6 } }$      
            %
        \end{tabular}
    \end{center}
    %\label{default}
\end{table}%

\begin{table}[htbp]  % + + + + T A B L E
    \caption{Input data}
    \begin{center}
        \begin{tabular}{lcr@{.}l}
            %
            & Shift & \multicolumn{2}{c}{Value} \\
            %
1 & $\Delta_{12}$ & --0 & 2 \\
            %
2 & $\Delta_{23}$ &   0 & 0 \\
            %
3 & $\Delta_{34}$ &   0 & 6 \\ 
            %
4 & $\Delta_{45}$ & --1 &  \\      
            %
        \end{tabular}
    \end{center}
    \label{tab:stitch piston in}
\end{table}%

  \begin{table}[t]  %  T A B L E
    \caption{Problem statement for linear regression.}
    \begin{center}
      \begin{tabular}{lll}
        %
        \bf{trial function} & $p_{k} - p_{k+1} = \Delta_{k, k + 1}$, $k = 1\colon n$ \\
        \bf{merit function} & $M(p) = \sum_{k=1}^{n}\paren{\Delta_{k, k + 1} - p_{k} + p_{k+1}}^{2}$ \\
        \bf{number of zones}& $m = 5$ \\
        \bf{number of overlaps}& $n = 4$ \\
        \bf{rank defect}    & $m - n  = 1$ \\
        \bf{measurements per zone}& $\lambda = \lst{11, 13, 13, 13, 12}$ \\
        \bf{measurements}   & $\phi_{k,j}$, $k=1\colon m$, $j=1\colon \lambda_{m}$ \\
        \bf{input data}     & $\Delta_{k,k+1}$, $k=1\colon n$ \\
        \bf{results}        & $p_{k}$, $k=1\colon m$ \\
        \bf{residual error} & $r = \bl{\Ap b} - \Delta$ \\
        \bf{linear system}  & $\mat{rrrrr}{
     1 & -1 &  0 &  0 &  0 \\
     0 &  1 & -1 &  0 &  0 \\
     0 &  0 &  1 & -1 &  0 \\
     0 &  0 &  0 &  1 & -1 }
     \mat{c}{ p_{1} \\ p_{2} \\ p_{3} \\ p_{4} \\ p_{5} } =
     \mat{c}{ \Delta_{12} \\ \Delta_{23} \\ \Delta_{34} \\ \Delta_{45} }$
 \\
        \bf{gauge condition} & $\sum_{k=1}^{m}p_{k} = 0$ \\
        %
      \end{tabular}
    \end{center}
  \label{tab:stitching problem statement}
  \end{table}%

\subsection{Linear System}  %   SS   SS   SS   SS   SS   SS   SS   SS   SS   SS   SS   SS
  \begin{equation*}   %  =   =   =   =   =
  %\begin{split}
    \A{} p = \Delta
    %\label{eq:}
  %\end{split}
  \end{equation*}
  \begin{equation*}   %  =   =   =   =   =
  %\begin{split}
    \mat{rrrrr}{
     1 & -1 &  0 &  0 &  0 \\
     0 &  1 & -1 &  0 &  0 \\
     0 &  0 &  1 & -1 &  0 \\
     0 &  0 &  0 &  1 & -1 }
     \mat{c}{ p_{1} \\ p_{2} \\ p_{3} \\ p_{4} \\ p_{5} } =
     \mat{c}{ \Delta_{12} \\ \Delta_{23} \\ \Delta_{34} \\ \Delta_{45} }
    \label{eq:stitch linear system}
  %\end{split}
  \end{equation*}

  \begin{equation*}   %  =   =   =   =   =
  %\begin{split}
    p_{LS} = \frac{1}{5}
    \bl{
    \mat{rrrr}{
     4 & 3 & 2 & 1 \\
     -1 & 3 & 2 & 1 \\
     -1 & -2 & 2 & 1 \\
     -1 & -2 & -3 & 1 \\
     -1 & -2 & -3 & -4 }
   \mat{c}{ \Delta_{12} \\ \Delta_{23} \\ \Delta_{34} \\ \Delta_{45} }} +
   \alpha \rd{\mat{c}{ 1 \\ 1 \\ 1 \\ 1 \\ 1 }}
   \label{eq:stitch general}
  %\end{split}
  \end{equation*}

  \begin{equation*}   %  =   =   =   =   =
  %\begin{split}
    \bl{\Ap b} = \frac{1}{25}\bl{\mat{r}{-6\\-1\\4\\-11\\14}}
    %\label{eq:}
  %\end{split}
  \end{equation*}
These are the actual plot values used in figure \ref{fig:pistons}.
  \begin{equation*}   %  =   =   =   =   =
  %\begin{split}
    \Phi_{corrected} = \Phi_{measured} - \bl{\Ap b}
    %\label{eq:}
  %\end{split}
  \end{equation*}

  \begin{table}[htbp]  %  T A B L E
    \caption{Results for stitching with piston.}
    \begin{center}
      \begin{tabular}{lll}
        %
        \bf{fit parameters}    & $p_{k}$, $k=1\colon m$ & pistons \\
        \bf{computed solution} & $p = \frac{1}{25}\bl{\mat{r}{-6\\-1\\4\\-11\\14}}$ & $\bl{\Ap b}$ \\
        \bf{data vector}       & $\frac{1}{5}\mat{r}{-1 \\ 0 \\ 1 \\ -2 \\ 3} = 
                                  \frac{1}{25}\bl{\mat{r}{-6 \\ -1 \\ 4 \\ -11 \\ 14}} -
                                  \frac{1}{5}\rd{\mat{r}{1 \\ 1 \\ 1 \\ 1 \\ 1}}$ & $\bl{p_{\atomrng}} + \rd{p_{\atomnll}}$ \\
        \bf{residual error}    & $r\cdot r$ = 0 \\\arrayrulecolor{medgray}\hline
        \bf{problem statement} & table \ref{tab:stitching problem statement} \\
        \bf{measurements}      & table \ref{tab:stitch measurements} \\
        \bf{input data}        & table \ref{tab:stitch piston in} \\
        \bf{plots}             & figure \ref{tab:stitch raw data} & raw data (bottom) \\
                               & figure \ref{fig:stitch final} & corrected data \\
        %
      \end{tabular}
    \end{center}
  %\label{tab:?}
  \end{table}%

\begin{figure}[htbp] %  figure placement: here, top, bottom, or page
   \centering
   \includegraphics[ width = 4in ]{\pathgraphics "stitch"/"piston"/"stitch merit bullseye"} 
   \caption{Looking at the merit function on the $p_{2} - p_{3}$ axis.}
   \label{fig:merit}
\end{figure}

\subsection{Least Squares Arbitration}  %   SS   SS   SS   SS   SS   SS   SS   SS   SS   SS   SS   SS
There is a fundamental ambiguity arising from gradient measurements stemming from the basic fact that
  \begin{equation*}   %  =   =   =   =   =
  %\begin{split}
    \tdx{}\phi(x) = \tdx{}\paren{\phi(x) + c}.
    %\label{eq:}
  %\end{split}
  \end{equation*}
We can recover the function shape, but not the offset. In other words, there is a translation invariance. This lone constant is the poster child for the rank one deficiency in the linear system of \eqref{eq:stitch linear system}. Realizing this, the one dimensional problem could be solved without resort to least squares. 

The system can be solved, for example, by moving from left to right and manually forcing the data to match. If the the overlap difference between zone 1 and zone 2 is $\Delta_{12}$, add $\Delta_{12}$ to every value in zone 2. Now zones 1 and 2 are stitched together. Compute $\Delta_{23}$, add this value to every point in zone 3. Zones 1, 2, and 3 are now stitched together. Continue as needed.

The least squares problem is obviated. How did this happen? The process of least squares is an exercise error arbitration which takes a peanut butter approach by trying to distribute the error evenly. In one dimension, there is no need for arbitration as there is no conflict in measurements.

In two dimensions, the problem changes. Consider the typical cell with a neighbor to the right and a neighbor above. The right--left overlap adjustment conflicts with the up--down overlap adjustment. The least squares process takes all off the overlap conflicts and provides a set of adjustments which minimizes the global error. To close, note that the least squares solution was used even though it is not necessary until dimension 2 or higher.

One last tidbit. Figure \ref{fig:pistons plus} shows the piston values that were input to distort the values. Least squares chooses a distinct set of corrections. Why was this set selected? A tantalizing clue is given by the null space vector in \eqref{eq:stitch general}. Notice this vector is perpendicular to every column vector in $\Ap$ which implies that the sum of each column vector must be 0. Therefore, the gauge condition is that the solution vector will have sum 0:
  \begin{equation*}   %  =   =   =   =   =
  %\begin{split}
    p_{1} + p_{2} + p_{3} + p_{4} + p_{5} = 0.
    %\label{eq:}
  %\end{split}
  \end{equation*}
We may now eliminate a variable; choose the last one:
  \begin{equation*}   %  =   =   =   =   =
  %\begin{split}
    p_{5} = -p_{1} - p_{2} - p_{3} - p_{4}
    %\label{eq:}
  %\end{split}
  \end{equation*}
Instead of \eqref{eq:stitch linear system}, there is now
  \begin{equation*}   %  =   =   =   =   =
  %\begin{split}
    \mat{rrrr}{
     1 & -1 & 0 & 0 \\
     0 & 1 & -1 & 0 \\
     0 & 0 & 1 & -1 \\
     1 & 1 & 1 & 2 }
     \mat{c}{ p_{1} \\ p_{2} \\ p_{3} \\ p_{4} } =
     \mat{c}{ \Delta_{12} \\ \Delta_{23} \\ \Delta_{34} \\ \Delta_{45} }.
    %\label{eq:}
  %\end{split}
  \end{equation*}
The solution is the same:
  \begin{equation*}   %  =   =   =   =   =
  %\begin{split}
    \hat{p}_{gauge} =
    \bl{ 
    \mat{rrrr}{
      4 &  3 &  2 & 1 \\
     -1 &  3 &  2 & 1 \\
     -1 & -2 &  2 & 1 \\
     -1 & -2 & -3 & 1 }
     \mat{c}{ \Delta_{12} \\ \Delta_{23} \\ \Delta_{34} \\ \Delta_{45} }}.
    %\label{eq:}
  %\end{split}
  \end{equation*}
The 0 sum, or equivalently 0 mean, condition is a gauge condition which restores the column rank of the problem.

\begin{figure}[htbp] %  figure placement: here, top, bottom, or page
   \centering
   \includegraphics[ width = 3in ]{\pathgraphics "stitch"/"piston"/"pistons grid"} \\[10pt]
   \includegraphics[ width = 3in ]{\pathgraphics "stitch"/"piston"/"pistons input grid"} 
   \caption[Pistons from the solution and pistons used to create the data]{On top, pistons output from the solution; on bottom, pistons input to create the data.}
   \label{fig:pistons plus}
\end{figure}
The piston values used to create the data set are decomposed into range and null space terms.
  \begin{equation*}   %  =   =   =   =   =
  %\begin{split}
    \frac{1}{5}\mat{r}{-1 \\ 0 \\ 1 \\ -2 \\ 3} = 
      \frac{1}{25}\bl{\mat{r}{-6 \\ -1 \\ 4 \\ -11 \\ 14}} -
      \frac{1}{5}\rd{\mat{r}{1 \\ 1 \\ 1 \\ 1 \\ 1}}
    %\label{eq:}
  %\end{split}
  \end{equation*}

\section{Stitch $\nabla \phi$}  %    S    S    S    S    S    S    S    S    S    S    S    S
The next challenge is to stitch data together using the gradient $\nabla \phi$ rather than the function value $\phi$. The outputs now will be a set of piston adjustments called tilts which restore continuity of the gradient.

The problem arises in the field of wavefront sensing. Modern devices make exquisite measurements of tilts.  
The process of wavefront reconstruction takes these tilts and reconstructs the wavefront. Measure $\nabla \phi(x)$ and compute $\nabla \phi(x)$.

\begin{figure}[htbp] %  figure placement: here, top, bottom, or page
   \centering
   \includegraphics[ width = 4in ]{\pathgraphics "stitch"/"slope"/"tilts"} 
   \caption{A set of tilt adjustments which restores continuity of the gradient across the domain.}
   \label{fig:tilts}
\end{figure}

\begin{figure}[htbp] %  figure placement: here, top, bottom, or page
   \centering
   \includegraphics[ width = 4in ]{\pathgraphics "stitch"/"slope"/"function and gradient"} 
   \caption[A function and its gradient.]{A function (black) and its gradient (blue).}
   \label{fig:function and gradient}
\end{figure}

Gradient of \eqref{eq:phi}
  \begin{equation*}   %  =   =   =   =   =
  %\begin{split}
    \nabla \phi(x) = \frac{1}{5} \exp \paren{-\frac{x}{5}} \paren{5 \pi  \cos (\pi  z)-\sin (\pi  z)}  
    \label{eq:phi}
  %\end{split}
  \end{equation*}

  \begin{equation*}   %  =   =   =   =   =
  %\begin{split}
    \tau = \frac{1}{100}\mat{r}{-90 \\ 85 \\ -40 \\ 35 \\ 10}
    %\label{eq:}
  %\end{split}
  \end{equation*}
A scaled version of these values is plotted in figure \ref{fig:tilts}.

\endinput