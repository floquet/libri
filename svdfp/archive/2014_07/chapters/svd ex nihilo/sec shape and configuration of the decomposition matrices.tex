\section{Shape and configuration of the decomposition matrices}
%%
For now, posit a decomposition of the form
%
\begin{equation}
  \aesvd{*}
  \label{eq:svd:genesis}
\end{equation}
%
where the
\begin{enumerate}
%
\item column vectors of $\U{}$ represent an orthonormal span for $\cmplxm$,
%
\item column vectors of $\V{}$ represent an orthonormal span for $\cmplxn$,
%
\item matrix $\sig{}$ contains scale factors and ensures conformability.
%
\end{enumerate}
Nomeclature alert: the word \emph{domain}\index{domain!context} appears in two different contexts.
\begin{itemize}
%
\item The matrix $\U{}$ is a unitary matrix whose column vectors span the domain, while the column vectors of $\V{}$ span the codomain.
%
\item The matrices $\U{}$ and $\V{}$ are the domain matrices.
%
\end{itemize}
Therefore, \emph{the} domain matrix is $\V{}$, \emph{a} domain matrix refers to either $\U{}$ and $\V{}$.

These basic rules define the shapes of the decomposition matrices. For not only do the column vectors span the domain and codomain, they are a \emph{minimal spanning set}.\index{minimal spanning set} As a consequence we know that there are $m$ columns in the domain matrix $\U{}$ and $n$ columns in the codomain matrix $\V{}$.

Because $\U{}\icm$ and $\V{}\icn$ then by conformability, we have the dimensions of the matrix of scale factors: $\sig{}\irmn$.

%%%%%%
\input{chapters/"svd ex nihilo"/"tab shape and configuration of the decomposition matrices"}  % table
%
\begin{equation}
  {\underset{\bymn} {\A{}} } = {\underset{\bymm} {\U{}} } \quad {\underset{\bymn} {\sig{}} } \quad {\underset{\bynn} {\V{*}} }
  \label{eq:svden:conform}
\end{equation}
%

\endinput