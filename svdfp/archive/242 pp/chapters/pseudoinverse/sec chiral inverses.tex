\section{Chiral inverses}
The pseudoinverse leads to the concept of a chiral inverse. We characterize the pseudoinverse  according to its action on the right and left-hand side of the target matrix $\A{}$. 

%%%%%%%%%%%
\subsection{Definition}
For instance, if we have
\begin{equation}
  \ApA = \I{n}
\end{equation}
the pseudoinverse would be classified as a left-inverse
\footnote{One may contend that a consistent nomenclature would call the left-handed inverse aristerochiral and the right-handed inverse dextrochiral. Here we defer to colloquialism.}
 [example matrix b, \eqref{eq:matrix b}]. If the inverse if ampichiral then
\begin{equation}
  \AAp = \ApA = \I{m}
\end{equation}
This property defines the standard inverse [example matrix c, \eqref{eq:matrix c}]. The pseudoinverse is ambichiral when
\begin{equation}
  \begin{split}
    \AAp &\ne \I{m}, \\
    \ApA &\ne \I{n}.
  \end{split}
\end{equation}

[example matrix a, \eqref{eq:matrix c}].


%%%%%%
\input{chapters/pseudoinverse/"tab chiral inverses"}  % table
\input{chapters/pseudoinverse/"tab chiral inverses and null spaces"}  % table
\input{chapters/pseudoinverse/"tab equivalence in defining the pseudoinverse chirality"}  % table
%%%%%%%%%%%
\subsection{Example matrices}
The three example matrices demonstrate the different kinds of chiral inverses, ampichiral, ambichiral and a definite chirality. To classify the chirality we compute the matrix products $\ApA$ and $\AAp$. In the next chapter we learn of the geometric significance of these forms.

%%%%%%%%%%%
\subsubsection{Example matrix (a): ampichiral inverse}
The block structure of the \asvd \ is this.
\begin{equation}
  \A{} = \matrixa = \csvdblockb{*}.
\end{equation}
The left \ns \ is nontrivial, therefore the pseudoinverse is not a left inverse.
The right \ns \ is nontrivial, therefore the pseudoinverse is not a right inverse.
Therefore the pseudoinverse is ampichiral.

\begin{equation}
  \begin{split}
    \ApA &= \prasa \\
    \AAp &= \praa
  \end{split}
\end{equation}

%%%%%%%%%%%
\subsubsection{Example matrix (b): definite chirality}
The left \ns \ is trivial, therefore the pseudoinverse is a left inverse.
The right \ns \ is nontrivial, therefore the pseudoinverse is not a right inverse.
Therefore the pseudoinverse had a definite chirality.
\begin{equation}
  \A{} = \matrixb = \csvdblockbc{*}
\end{equation}
\begin{equation}
  \begin{split}
    \ApA &= \prasb = \I{2} \\
    \AAp &= \prab
  \end{split}
\end{equation}

%%%%%%%%%%%
\subsubsection{Example matrix (c): ambichiral}
The left \ns \ is trivial, therefore the pseudoinverse is a left inverse.
The right \ns \ is trivial, therefore the pseudoinverse is a right inverse.
Therefore the pseudoinverse is ambichiral.
\begin{equation}
  \A{} = \matrixc = \svdblockbf{*}
\end{equation}
\begin{equation}
  \begin{split}
    \ApA &= \prasc = \I{2} \\
    \AAp &= \prac = \I{2}
  \end{split}
\end{equation}

%%%%%%%%%%%
\subsection{How close do we get to the identity?}

\begin{equation}
  \begin{split}
    \normt{ \ApA - \I{n} } &= 1, \\
    \normt{ \AAp - \I{m} } &= 1.
  \end{split}
\end{equation}

%%%%%%%%%%%
\subsection{Exercises}
\begin{enumerate}
%
\item Under what circumstances can a matrix $\aicmnr$ have ambichiral inverses when $m\ne n$?
%
\item Under what circumstances can a matrix $\aicmnr$ have ambichiral inverses when $m\ne n$?
%
\end{enumerate}



\endinput