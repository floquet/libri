\chapter[Least squares with exemplars]{Least squares with exemplar matrices}

\section{Linear systems}  %  S  S  S  S  S  S  S  S  S  S  S
The essential concepts of least squares and the fundamental subspaces spring to life using exemplar matrices. The canonical linear system is
  % = =  e q u a t i o n
  \begin{equation*}
    \Axeb
    %\label{eqn:}
  \end{equation*}
  % = =
The matrix $\A{}$ has $m$ rows and $n$ columns of complex numbers. The matrix rank is $\rho\le\min\paren{m,n}$. In shorthand, the three components are
\begin{enumerate}
	\item $\aicmnr$: the system matrix, an input;
	\item $b\in\cmplxm$: the data vector, an input;
	\item $x\in\cmplxn$: the solution vector, the output.
\end{enumerate}
The residual error from the best fit is 
  % = =  e q u a t i o n
  \begin{equation}
    r = \Axmb .
    %\label{eqn:}
  \end{equation}
  % = =
Ignore the trivial cases where $b=0$.

Exemplar matrices have obvious \asvd s.
$$ \idtwo,\ \exemplartall,\ \exemplarwide,\ \exemplarboth$$

$$ \mat{c}{\I{2}},\ \mat{c}{\I{2}\\\hline\zero},\ \mat{c|c}{\I{2}&\zero},\ \mat{cc}{\I{2}&\zero\\\hline\zero & 0} $$

% SECTION
\section{Full rank: $\rho = m = n$}
Start with an ideal linear system
  % = =  e q u a t i o n
  \begin{equation}
    \begin{split}
      \A{} x &= b \\
      \idtwo \xtwo &= \bvtwo.
    \label{eqn:ideal}
    \end{split}
  \end{equation}
  % = =

% SUBSECTION
\subsection{Subspace decomposition}
\input{\pathtables "tab fr"}  %  <  <  <  <  <  <  <  <  <  <  <  <  <  <  <  <  <  <
Because the matrix $\A{}$ has full column rank the null space $\rnlla{*}$ is trivial. Because the matrix $\A{*}$ has full row rank the null space $\rnlla{}$ is trivial.\\

% SUBSECTION
\subsection{Existence and uniqueness}
We have unconditional existence and uniqueness without regard to the data vector. The exact solution is 
  % = =  e q u a t i o n
  \begin{equation}
    x = \xtwo = \bvbltwo
    %\label{eqn:}
  \end{equation}
  % = =
which is also the least squares solution
  % = =  e q u a t i o n
  \begin{equation}
    \xls = x = \bvbltwo
    %\label{eqn:}
  \end{equation}
  % = =
with $\tra{r}r = 0$ residual error. More formally, the linear system has a unique solution for any value of $b_{1}, b_{2} \ic$.


% SECTION
\section{Full column rank: $\rho = n < m$}
Foreshadowing the resolution of the range and null spaces, we show a partitioning
  % = =  e q u a t i o n
  \begin{equation}
    \begin{split}
      \A{} x &= b \\
      \exemplartall \xtwo &= \bvthree.
    \label{eqn:fcr}
    \end{split}
  \end{equation}
  % = =

% SUBSECTION
\subsection{Subspace decomposition}
\ftola
\input{\pathtables "tab fcr"}  %  <  <  <  <  <  <  <  <  <  <  <  <  <  <  <  <  <  <
Thanks to the gentle behavior of the exemplar matrix, the range and null space components for the data vector are apparent:
% = =  e q u a t i o n
  \begin{equation}
    %\begin{split}
      b = \underbrace{\bl{\mat{c}{b_{1} \\ b_{2} \\ 0}}}_{\in\brnga{}} + \underbrace{\rd{\mat{c}{0\\0\\b_{3}}}}_{\in\rnlla{*}}
    %\end{split}
    %\label{eqn:}
  \end{equation}
% = =

% SUBSECTION
\subsection{Existence and uniqueness}
When the data vector component $b_{3} = 0$, 
% = =  e q u a t i o n
  \begin{equation}
    %\begin{split}
      b = \bl{\mat{c}{b_{1} \\ b_{2} \\ 0}} \in \brnga{}
    %\end{split}
    %\label{eqn:}
  \end{equation}
the linear system is consistent and we have a unique solution 
  % = =  e q u a t i o n
  \begin{equation}
    x = \bl{\xtwo} = \bl{\bvtwo}
    %\label{eqn:}
  \end{equation}
  % = =
which is also the least squares solution
  % = =  e q u a t i o n
  \begin{equation}
    \xls = x = \bvbltwo
    %\label{eqn:}
  \end{equation}
  % = =
with $\tra{r}r = 0$ residual error. Notice that the solution vector is in the complementary range space, the range space of $\A{*}$:
% = =  e q u a t i o n
  \begin{equation}
    %\begin{split}
      x \in \brnga{*}.
    %\end{split}
    %\label{eqn:}
  \end{equation}
% = =

% SUBSECTION
\subsection{No existence: }
When the data vector inhabits the null space we do not even have a least squares solution. 

% SUBSECTION
\subsection{Existence, no uniqueness: }

\input{\pathtables "tab eandu 02"}  %  <  <  <  <  <  <  <  <  <  <  <  <  <  <  <  <  <  <

% SECTION
\section{Full row rank: $\rho = m < n$}
Foreshadowing the resolution of the range and null spaces, we show a partitioning
  % = =  e q u a t i o n
  \begin{equation}
    \begin{split}
      \A{} x &= b \\
      \exemplarwide \xthree &= \bl{\bvtwo}.
    \label{eqn:frr}
    \end{split}
  \end{equation}
  % = =

% SUBSECTION
\subsection{Subspace decomposition}
\ftola
\input{\pathtables "tab frr"}  %  <  <  <  <  <  <  <  <  <  <  <  <  <  <  <  <  <  <
Thanks to the gentle behavior of the exemplar matrix, the range and null space components for the solution vector are apparent:
% = =  e q u a t i o n
  \begin{equation}
    %\begin{split}
      x = \underbrace{\bl{\mat{c}{x_{1} \\ x_{2} \\ 0}}}_{\in\brnga{*}} + \underbrace{\rd{\mat{c}{0\\0\\x_{3}}}}_{\in\rnlla{}}
    %\end{split}
    %\label{eqn:}
  \end{equation}
% = =

% SUBSECTION
\subsection{Existence and uniqueness}
When the data vector component $b_{3} = 0$, 
% = =  e q u a t i o n
  \begin{equation}
    %\begin{split}
      b = \bl{\mat{c}{b_{1} \\ b_{2} \\ 0}} \in \brnga{}
    %\end{split}
    %\label{eqn:}
  \end{equation}
the linear system is consistent and we have a unique solution 
  % = =  e q u a t i o n
  \begin{equation}
    x = \bl{\xtwo} = \bl{\bvtwo}
    %\label{eqn:}
  \end{equation}
  % = =
which is also the least squares solution
  % = =  e q u a t i o n
  \begin{equation}
    \xls = x = \bvbltwo
    %\label{eqn:}
  \end{equation}
  % = =
with $\tra{r}r = 0$ residual error. Notice that the solution vector is in the complementary range space, the range space of $\A{*}$:
% = =  e q u a t i o n
  \begin{equation}
    %\begin{split}
      x \in \brnga{*}.
    %\end{split}
    %\label{eqn:}
  \end{equation}
% = =

% SUBSECTION
\subsection{No existence}
When the data vector inhabits the null space we do not even have a least squares solution. 

% SUBSECTION
\subsection{Existence, no uniqueness}

\input{\pathtables "tab eandu 02"}  %  <  <  <  <  <  <  <  <  <  <  <  <  <  <  <  <  <  <


\endinput  %  -  -  -  -  -  -  -  -  -  -  -  -  -  -  -  -  -  -  -  -