\chapter*{Literature survey}

There are many excellent texts in linear algebra available today. Some readers may wish to augment their study of this volume by reading other texts.

Meyer's book is an encyclopedic labor of love. The author's love of the subject is evident in his writing. The style is lively and the numerous historical vignettes reveal mathematics as an intensely human endeavour. The discussions of topics are quite thorough and there is an abundance of problems, most of which come with solutions of hints. This is an excellent book for self-study.

It is difficult to write clearly and it is difficult to write concisely. Laub has managed to contribute a rare work of brevity and clarity to the study of linear algebra. The chapters are direct and augmented by a clever set of problems. These exercises run the gamut from basic material, to comprehension to very clever.

Strang's book is an excellent choice. A wide range of topics is presented in an economical fashion. Besides being a well-written text, we recognize the quality of the problems. Also the online lectures are a wondeful augment.

Noble and Daniel. 

Trefethen and Bau

Demmel

By design, this book relies on prior coursework and therefore other texts. Readers who need to work on the foundations or who wish to learn more will want to refer to other works.

%%%
\subsection*{Introductory linear algebra texts}
New and highly priced texts appear year after year. They are thicker and they are heavier and the graphics are more impressive. They may include DVDs and have elaborate web sites associated with them. Survey these by using a bookseller web site and search by bestselling order.

Here we include less audacious and more economical books
\begin{quote}
\textit{Linear Algebra}\\
Marcus and Minc\\
McGraw-Hill, 1985\\
ISBN 0-971-12345-1
\end{quote}

Here we list a few inexpensive books of considerable benefit. 

%%%%
\subsection*{Intermediate texts}
There is one text which stands above the crowd thanks to the masterful lectures which complement it.
\begin{quote}
textit{Linear Algebra}\\
Gilbert Strang\\
McGraw-Hill, 1985\\
ISBN 0-971-12345-1
\end{quote}
The lectures can be found at the MIT OCW web site:
\begin{verbatim} 
MIT/OCW
\end{verbatim}

%%%
\subsection*{Advanced texts}
The subject of the \svdl \ is covered in detail with considerable skill in
\begin{quote}
textit{Matrix Analysis}\\
Roger Horn\\
McGraw-Hill, 1985\\
ISBN 0-971-12345-1
\end{quote}

%%%
\subsection*{Journal articles}
Certainly a pedestrian text should not expect readers to be conversant with material in journals covering theory and applications. We restrict ourselves to a few noteworthy articles which would be a natural progression.

history

\endinput