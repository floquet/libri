\section{Theorem statement}

Given $\A{}\in\cmplx{\by{m}{n}}_{\rho}$, a complex matrix with $m$ rows and $n$ columns and rank $\rho$, where $\rho \le \min( m, n )$ then there exists a matrix decomposition of the format
\begin{equation}
  \svdax{*} .
\end{equation}
\begin{enumerate}
  \item The unitary matrix $\Y{}$ represents an orthonormal basis for the \textit{codomain}.
  \item The unitary matrix $\X{}$ represents an orthonormal basis for the \textit{domain}.
  \item There are $\rho$ positive \textit{singular values} which are the square roots of the nonzero eigenvalues of the product matrices $\prdm{*}$ and $\prdmm{*}$.
  \item The ordered singular values $\left( \sigma_1 \ge \sigma_2 \ge \cdots \ge \sigma_{\rho} > 0 \right)$ populate the diagonal $\textbf{S}$ matrix which is embedded in the $\sig{}$ matrix.
  \item The dimensions of associated matrices are shown here:
\begin{equation}
  \A{\byt{m}{n}}_{\rho} = 
  \Y{\byt{m}{m}}_{m}
  \mat{c|c}{
  \textbf{S}^{\by{\rho}{\rho}}_{\rho}  &  \zero \\\hline
  \zero  &  \zero
  }^{\by{m}{n}}_{\rho} 
  (\X{*})^{\by{n}{n}}_{n}.
\end{equation}
\end{enumerate}

%
\subsection{Important properties}
Thanks to the \index{Fundamental Theorem of linear algebra!dimensions for the SVD}Fundamental Theorem, the dimensions of the associated vector spaces are easily defined. Also, other important properties are evident.
\begin{enumerate}
\item The component matrices, $\sig{}$ and $\textbf{S}$ have the same rank $\rho$ as the target matrix.
\item In general the domain matrices are complex. The singular value matrix $\textbf{S}$ is always real. Therefore the sabot matrix $\sig{}$ which only pads the $\textbf{S}$ matrix with zero entries is always real.
\item The domain matrices $\Y{},\ \X{}$ along with the singular values matrix $\textbf{S}$ are square and invertible.
\item Because the domain matrices are unitary (orthogonal), the inverse matrix is the Hermitian conjugate (transpose).
\item Because the matrix of singular values is a full rank diagonal matrix, the matrix is square and the inverse matrix is has reciprocal values on the diagonal.
\item The sabot matrix $\sig{}$ has the same shape as the target matrix. This is a conformability bridge to allow the multiplication with the domain matrices.
\item The $\sig{}$ matrix is always unique.
\item The domain matrices are unique up to rotations or permutations.
\item Therefore, the \svdl \ is \textit{not} unique.
\item The nullity of the domain matrices is computed using the \index{rank plus nullity theorem:nullity of the domain matrices}rank plus nullity theorem. So while the domain matrices always have the same rank, they will have different nullities in general.
\subitem The nullity of the domain matrix is
\begin{equation}
  \eta_{x} = n - \rho.
\end{equation}
\subitem The nullity of the codomain matrix is
\begin{equation}
  \eta_{y} = m - \rho.
\end{equation}
\end{enumerate}

To reinforce these concepts we state the application of the rank plus nullity theorem to the $n-$dimensional vector space of the domain and the $m-$dimensional vector space of the codomain
\begin{equation}
\boxed{
  \begin{array}{rcllclclcl}
    \text{rank } \rng{\A{}} &+& \text{nullity } \nll{\A{}} & = & \rho + \eta_{x} & = & \rho + (n-\rho) & = & n,\\
    \text{rank } \rng{\A{*}} &+& \text{nullity } \nll{\A{*}} & = & \rho + \eta_{y} & = & \rho + (m-\rho) & = & m.
  \end{array}
}
\end{equation}
As basic as it seems it is still something users stumble over.

In terms of matrix rank, a rank deficient matrix is resolved into a product of a full rank matrix with a rank deficient matrix with another full rank matrix.

%
\subsection{Unimportant properties}
The SVD is bound by convention. One could develop an equivalent theorem based upon the formulation
\begin{equation}
  \A{} = \Y{*}\,\sig{}\,\X{}.
\end{equation}
In other words, we could move the Hermitian conjugation to the other side.

Also, as mentioned before, we could allow the singular values to be left unsorted. Keeping them arranged in descending order is a good housekeeping practice which makes applications cleaner and easier to formulate.

Some authors allow the singular values to be zero. The first $\rho$ values are nonzero, and the last $min(m,n)-\rho$ values are zeros. Again, it is a preference which could be accounted for by modest rewording of applications.

One way to state the theorem may be a bit preferable to some readers. We can write equivalently that
\begin{equation}
  \Y{*}\,\A{}\,\X{} = \sig{}.
\end{equation}
Fans of diagonalization may champion this form.

%%
\section{Four fundamental subspaces}
The \svdl \ resolves a matrix into its four fundamental subspaces:
\begin{enumerate}
\item Codomain: the range of the target matrix, $\rng{\A{}}$,
\item Codomain: the null space of the Hermitian conjugate of target matrix,  $\nll{\A{*}}$,
\item Domain: the range of the Hermitian conjugate of the target matrix, $\rng{\A{*}}$,
\item Domain: the null space of the target matrix,  $\nll{\A{}}$,
\end{enumerate}

The figure below shows how the vectors are represented in the decomposition.
\begin{enumerate}
\item First $\rho$ columns of $\Y{}$: $\rng{\A{}}$.
\item Last $m-\rho$ columns of $\Y{}$: $\nll{\A{*}}$.
\item First $\rho$ rows of $\X{*}$: $\rng{\A{*}}$.
\item Last $n-\rho$ rows of $\X{}$: $\nll{\A{}}$.
\end{enumerate}

Big four in pictures. Pedro's diagram.


%%
\section{Dimension and size}
\index{size!matrix}\index{dimension!vector space}
The size of a matrix describes the number of rows and columns. The dimension of  a vector space describes the length of the constituent vectors.

%%
\section{The spectral theorem and SVD}
We know from linear algebra that a symmetric matrix can be diagonalized by a unitary matrix. So given a matrix
\begin{equation}
  \lst{\W{} \in \cmplx{\bys{m}}_{m}\colon\mathbf{W}_{r,c}=\mathbf{W}_{c,r}, r,c=1\dots m}
\end{equation}
there exists a unitary matrix $\Q{}$ such that the matrix $\Lambda$ is diagonal.
\begin{equation}
  \W{} = \Q{}\,\Lambda\,\Q{*}.
\end{equation}

\begin{equation}
  \begin{split}
    \W{x} = \prdm{*} &= \paren{\X{}\,\sig{T}\Y{*}}\paren{\svd{*}} \\
    &= \X{}\,\sig{T}\sig{}\,\X{*}.
  \end{split}
\end{equation}
\begin{equation}
  \begin{split}
    \W{y} = \prdmm{*} &= \paren{\svd{*}}\paren{\X{}\,\sig{T}\Y{*}}\\
    & = \Y{}\,\sig{}\,\sig{T}\,\Y{*}.
  \end{split}
\end{equation}In terms of the spectral theorem, the product matrices can be decomposed into similarity transform where $\Lambda$ is a diagonal matrix of eigenvalues and the matrix $\Q{}$ is unitary.
\begin{equation}
  \begin{split}
    \W{x} &= \X{}\,\sig{T} \sig{}\,\X{*}\\
          &= \Q{}\,\Lambda\,\Q{*}.
  \end{split}
\end{equation}

\begin{equation}
  \W{x} = \mat{c}{r_{1}\\r_{2}\\\vdots\\r_{n}} \mat{c}{\sigma_{1}r_{1}^{\mathrm{T}}}
\end{equation}


\endinput