\section[Basic Jordan forms]{The most basic Jordan forms}
\label{sec:Jordan:basic}

%%
\subsection{The forms}
There are seven nonzero, upper triangular $2\times2$ matrix Jordan forms and they are listed here:
\begin{equation*}
  \begin{array}{rclrclrcl}
\mathcal{J}_{2,1} &=& \left[
                        \begin{array}{cc}
                         1 & 0 \\
                         0 & 0
                        \end{array}
                        \right], \quad
\mathcal{J}_{2,2} &=& \left[
                        \begin{array}{cc}
                         0 & 1 \\
                         0 & 0
                        \end{array}
                        \right], \quad
\mathcal{J}_{2,3} &=& \left[
                        \begin{array}{cc}
                         1 & 1 \\
                         0 & 0
                        \end{array}
                        \right], \\
\mathcal{J}_{2,4} &=& \left[
                        \begin{array}{cc}
                         0 & 0 \\
                         0 & 1
                        \end{array}
                        \right], \quad
\mathcal{J}_{2,5} &=& \left[
                        \begin{array}{cc}
                         1 & 0 \\
                         0 & 1
                        \end{array}
                        \right], \quad
\mathcal{J}_{2,6} &=& \left[
                        \begin{array}{cc}
                         0 & 1 \\
                         0 & 1
                        \end{array}
                        \right], \\
\mathcal{J}_{2,7} &=& \left[
                        \begin{array}{cc}
                         1 & 1 \\
                         0 & 1
                        \end{array}
                        \right].
  \end{array}
\end{equation*}
The \svdl s are these:
\begin{equation}
  \begin{split}
\mathcal{J}_{2,1} &= 
\mat{cc}{ 1 & 0 \\
 0 & 0} = 
\I{2}
\mat{c|c}
{
 1 & 0 \\\hline
 0 & 0}
 \I{2}\\
\mathcal{J}_{2,2} &= 
\mat{cc}
{ 0 & 1 \\
 0 & 0} = 
\I{2}
\mat{c|c}{ 1 & 0 \\\hline
 0 & 0}
 \underbrace{\mat{cc}
 { 0 & 1 \\
\rowcolor{ltgray}
 1 & 0}}_{\text{swap rows}}\\
\mathcal{J}_{2,4} &= 
\mat{cc}
{ 0 & 0 \\
 0 & 1} = 
\underbrace{\mat{c>{\columncolor{ltgray}}c}
{ 0 & 1 \\
 1 & 0}}_{\text{swap columns}}
 \mat{c|c}
 { 1 & 0 \\\hline
 0 & 0}
 \underbrace{\mat{cc}
 { 0 & 1 \\
\rowcolor{ltgray}
 1 & 0}}_{\text{swap rows}}\\
\mathcal{J}_{2,5} &= 
\itwo
 = 
\I{2}\ 
\I{2}\ 
\I{2}
  \end{split}
\end{equation}
 
These nontrivial forms are related.
%%
\begin{equation*}
  \begin{array}{rcccccccc}
    \mathcal{J}_{2,3} &=& 
&\mat{cc}{ 1 & 1 \\
 0 & 0} 
 &=& 
&\I{2}
&\mat{c|c}
{\sqrt{2} & 0 \\\hline
 0 & 0}
& \stwo
\mat{rr}
{ 1 & 1 \\
\rowcolor{ltgray}
 -1 & 1},\\
    \mathcal{J}_{2,6} &=& 
&\mat{cc}
{0 & 1 \\
 0 & 1} 
 &=& 
& \stwo
\mat{r>{\columncolor{ltgray}}r}
{ 1 &-1 \\
  1 & 1}
&\mat{c|c}{
 \sqrt{2} & 0 \\\hline
 0 & 0}
&\I{2}.
  \end{array}
\end{equation*}

%%
\subsection{Example}
The two dimensional form of interest is this one:
\begin{equation}
  \begin{split}
    \mathcal{J}_{2,3} = \mat{cc}{1&1\\0&0} &= \svd{T}\\
      &= \I{2}\, \mat{c|c}{\sqrt{2} & 0\\\hline0 & 0} \stwo\mat{rr}{1&1\\\rowcolor{ltgray}
-1&1}.
  \end{split}
\end{equation}
This decomposition can also be written as a product of rotation matrices
\begin{equation}
  \begin{split}
    \mathcal{J}_{2,3} = \mat{cc}{1&1\\0&0} &= \svd{T}\\
      &= \rat{\theta_{y}}{}\, \mat{c|c}{\sqrt{2} & 0\\\hline
0 & 0} \rat{\theta_{x}}{T}
  \end{split}
\end{equation}
where the rotation matrix has this form
\begin{equation}
  \rat{\theta}{} = \mat{rr}{\cos\theta & -\sin\theta\\\sin\theta&\cos\theta}
\end{equation}
and the angles have these values
\begin{equation}
  \theta_{y} = 0, \quad \theta_{x} = \frac{\pi}{2}.
\end{equation}

The fact that the basis matrices are not unique manifests in the periodicity of the trigonometric functions. These rotation angles will also yield the same result:
\begin{equation}
  \theta_{y} = k\pi, \quad \theta_{x} = \frac{2k+1}{2}\pi, \quad k\in\mathbb{Z}.
\end{equation}
To emphasize this point the decomposition can be cast as this:
\begin{equation}
  \begin{split}
    \mathcal{J}_{2,3} = \mat{cc}{1&1\\0&0} = \svd{T} = \rat{k\pi}{}\mat{c|c}{\sqrt{2} & 0\\\hline0 & 0}\rat{\frac{2k+1}{2}\pi}{T},\quad k\in\mathbb{Z}. 
  \end{split}
\end{equation}
Noting the isomorphism between complex numbers and matrices of the form
\begin{equation}
  z = a + ib \quad \mapsto \quad \mat{cr}{a&-b\\b&a}, \quad a,b\in\real{}
\end{equation}
the decomposition can be written in this form
\begin{equation}
  \begin{split}
    \mathcal{J}_{2,3} = e^{i \theta_{y}}\mat{c|c}{\sqrt{2} & 0\\\hline0 & 0}e^{-i \theta_{x}}.
  \end{split}
\end{equation}

The transpose of this Jordan normal form 
\begin{equation}
    \mathcal{J}_{2,6} = \mathcal{J}_{2,3}^{\mathrm{T}}= \svd{T} 
    = e^{i\theta_{x}}\, \mat{c|c}{\sqrt{2} & 0\\\hline0 & 0} e^{-i\theta_{y}}
\end{equation}

\begin{equation}
  \chi \mat{c}{x\\y} = \mat{c}{y\\x}.
\end{equation}

\endinput