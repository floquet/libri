\subsection{Functions of matrices}

Diagonalization plays critical role in evaluating functions of matrices.\cite{Higham2008}

How should functions of matrices be defined? The Cayley-Hamilton theorem\cite[p. 509]{Meyer2000}\index{Cayley-Hamilton theorem} is an important perspective. This theorem describes the behavior of polynomial functions of matrices. It says that if we define $p(\lambda)$ the \emph{characteristic polynomial}\index{characteristic polynomial} of a matrix $\Ac{m}$ as 
\begin{equation}
  p(\lambda) = \det \paren{\lambda \I{m} - \A{}}
\end{equation}
then 
\begin{equation}
  p( \A{} ) = 0.
\end{equation}
So we know how to square matrices, cube matrices etc. But what about the transcendental functions which can't be expressed in terms of a finite number of algebraic functions?

For instance, what should $\cos \A{}$ mean? We should demand that functions of matrices behave the same as functions of real variables. For example, for $\Ac{m}$,
\begin{equation}
  \cos^{2}\A{} + \sin^{2}\A{} = \I{m}.
\end{equation}

Transcendental matrix functions are defined defined by the same Taylor series used to define them in the complex plane. For example for $\Ac{m}$:
\begin{equation}
  e^{\A{}} = \I{m} + \A{} + \frac{1}{2!}\A{2} + \frac{1}{3!}\A{3} + \dots + \frac{1}{n!}\A{n} + \dots
\end{equation}
The matrix logarithm can defined as an inverse of this. That is given 

\endinput