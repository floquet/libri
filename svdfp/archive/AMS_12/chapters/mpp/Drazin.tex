\section{The Drazin inverse}
There is another common type of matrix inverse called the Drazin inverse\index{Drazin inverse}. The natural expression for this inverse is in terms of a core-nilpotent matrix decomposition.

Consider a singular matrix $\Ac{m}$ of with an index $k$ defined such that
\begin{equation}
  rank\paren{\A{k}} = \rho
\end{equation}
there exists a nonsingular matrix $\Q{}$ which decomposes the target matrix into a block matrix of the form
\begin{equation}
  \Q{}\A{}\Q{-1} = \mat{c|c}{\pee{}_{\rho\times\rho} & \zero \\[3pt]\hline \zero & \N{}}.
\end{equation}
Here the matrix $\pee{}$ represents the persistent piece and the $\paren{\bys{m-r}}$ block is nilpotent with degree $k$. That is
\begin{equation}
  \N{k} = \zero.
\end{equation}


\endinput