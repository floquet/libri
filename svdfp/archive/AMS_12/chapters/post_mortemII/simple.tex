\section{Review}
But first we will review the general method for \svdl \ this time using a much simpler example. This example will illuminate the decomposition scheme and leave us with a helpful geometric example.

\begin{landscape}
\thispagestyle{empty}
%%
\subsection{A quick review with a simpler matrix}
\begin{equation}
\boxed{
  \begin{array}{ccccccccccccc}
  \A{}&\longrightarrow&\A{*}&\longrightarrow&\W{x}&\longrightarrow&\lambda\paren{\W{x}}&\longrightarrow&\lst{\sigma_{k}}&\longrightarrow&\sig{}\\
  &&&&\downarrow\\
  &&&&\lst{x_{k}}&\longrightarrow&\lst{x_{k}}^{\perp}&\longrightarrow&\X{}\\
  &&&&&&&&\downarrow\\
  &&&&&&&&\lst{y_{k}}&\longrightarrow&\lst{y_{k}}^{\perp}&\longrightarrow&\Y{}\\
  \end{array}
 }
\end{equation}
Flow chart for a typical \svdl. This layout shows the simplicity of the underlying decomposition process. In general, it is the eigenvalue problem thats gives the SVD a reputation as being difficult. Readers more comfortable with a cookbook format may prefer table \eqref{tab:3:input}.

Essentially we the process involves finding the eigenvectors of the product matrix $\W{x}$ and completing this with a set of null space vectors to form the basis matrix $\X{}$. The singular values are the square root of the eigenvalues of $\W{x}$ and they are used to construct $\sig{}$. Given the matrices $\W{x}$ and $\sig{}$ one can construct the final matrix $\Y{}$.

A sample process with a full rank matrix is shown below.
\begin{equation*}
  \begin{array}{ccccccccccccc}
  \mat{rr}{1&2\\-1&2}&\rightarrow&\mat{rr}{1&1\\2&-2}&\rightarrow&\mat{cc}{2&0\\0&8}&\rightarrow&\lst{8,2}&\rightarrow&\lst{2\sqrt{2},\sqrt{2}}&\rightarrow&\sqrt{2}\mat{cc}{2&0\\0&1}\\
  &&&&\downarrow\\
  &&&&\lst{\mat{c}{0\\1},\mat{c}{1\\0}}&\rightarrow&\lst{\emptyset}&\rightarrow&\mat{cc}{0&1\\1&0}\\
  &&&&&&&&\downarrow\\
  &&&&&&&&\frac{1}{\sqrt{2}}\lst{\mat{c}{1\\1},\mat{r}{-1\\1}}&\rightarrow&\lst{\emptyset}&\rightarrow&\frac{1}{\sqrt{2}}\mat{cr}{1&-1\\1&1}\\
  \end{array}
  \label{eq:pmII}
\end{equation*}
\end{landscape}

The new target matrix $\A{}\in\real{\by{2}{2}}_{2}$ is decomposed as
\begin{equation}
  \begin{array}{ccccc}
    \A{} &=& \Y{} & \sig{} & \X{T}, \\
    \mat{rr}{1&2\\-1&2} &=& \stwo\mat{cr}{1&-1\\1&1}
    & \sqrt{2}\mat{cc}{2&0\\0&1}
    & \mat{cc}{0&1\\1&0}.
  \end{array}
  \label{eq:pmII:A}
\end{equation}
This example \eqref{eq:pmII} is shown against the flow chart in \eqref{eq:gen:flow}. Since this matrix has full rank there are no null spaces this is a very basic case. The matrices involved in the decomposition are all squares as depicted in figure \eqref{fig:full_rank}.

\begin{figure}[htbp] %  figure placement: here, top, bottom, or page
   \centering
   \includegraphics[width=2in]{pdf/post_mortemII/svd_02_02_02} 
   \caption{The matrices in nonsingular decompositions are all square.}
   \label{fig:full_rank}
\end{figure}

\section{SVD by inspection}
Notice another way to decompose the matrix in example.
\begin{enumerate}
\item Observe that 
\begin{equation}
  \A{}=\mat{rr}{1&2\\-1&2}
\end{equation}
has full rank. Therefore there will be no sabot matrix.
\item Compute the product matrix
\begin{equation}
  \A{T}\A{} = \W{x} = \mat{cc}{2&0\\0&8}.
\end{equation}
\item The eigenvalue spectrum is then
\begin{equation}
  \lambda\paren{\W{x}} = \lst{8,2}.
\end{equation}
\item The matrix of singular values is then
\begin{equation}
  \ess{} = \sqrt{2}\mat{cc}{2&0\\0&1}
\end{equation}
Therefore
\begin{equation}
  \sig{} = \ess{} = \sqrt{2}\mat{cc}{2&0\\0&1}.
\end{equation}
\item Because the diagonal elements are not arranged in descending order, we need a permutation matrix. Postulate the simplest possible permutation matrix for the domain:
\begin{equation}
  \X{} = \ktwo.
\end{equation}
\item Solve the two equations 
\begin{equation*}
  \A{}\X{}_{*,k} = \sigma_{k} \Y{}_{*,k}, \quad k=\lst{1,2}
\end{equation*}
to discover that
\begin{equation}
  \Y{} = \stwo \mat{rr}{1&-1\\1&1}.
\end{equation}
\end{enumerate}

\endinput