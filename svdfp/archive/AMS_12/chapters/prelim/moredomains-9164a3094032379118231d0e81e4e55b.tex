\section{More on domains}
Let's look at domains in the familiar context of the calculus. THis will allow us to connect things.

Then onto set theory. An example will keep the discussion more concrete and less abstract.

%%
\subsection{The mapping action of the function $f(x)$.}
Consider the real function
\begin{equation}
  y=\sqrt{\paren{x-1}\paren{x+2}}
\end{equation}
What is the domain and range of this function? What are the allowable values of $x$ which yield a valid $y$?

%%
\subsection{The mapping action of the matrix $\A{}$}
Start with the matrix 
\begin{equation}
  \A{}= \Aexample.
\end{equation}
This matrix maps $2-$vectors to $3-$vectors. We will represent the collection of all $2-$vectors as a circle and the collection of all $3-$vectors as a square. The target matrix $\A{}$ connects points in the circle with points in the square as shown in this figure:
$$
\text{circle to square}.
$$
Or symbolically
\begin{equation}
  \A{} \mat{c}{\star \\ \star} = \mat{c}{\bullet \\ \bullet \\ \bullet }.
\end{equation}
The problem is that this matrix cannot connect to the entire set of $3-$vectors. In fact, this matrix can only see a limited set of $3-$vectors. To understand why, look at the general problem
\begin{equation}
  \A{} x = \Aexample \mat{c}{x\\y} = \alpha\mat{r}{1\\-1\\1}
\end{equation}
where $\alpha = x-y$. For example, there is no vector in the domain which maps to the constant vectors:
\begin{equation}
  \Aexample \mat{c}{x\\y} \ne \mat{c}{1\\1\\1}.
\end{equation}
This requires that the diagram be revised. The square representing the $3-$vectors will be separated into two regions. The shaded portion represents the $3-$vectors that can't be connected to any $2-$vector under the mapping of $\A{}$.
$$
\text{circle to square}.
$$


\endinput