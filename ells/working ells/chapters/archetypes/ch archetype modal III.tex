\chapter[Modal Example III: Observations]{Modal Example III\\Observations}

A deeper look at the example.

\section{Invariances}  %    S    S    S    S    S    S    S    S    S
Separate methods, test solutions

\subsection{Translation Invariance}  %   SS   SS   SS   SS   SS   SS   SS   SS   SS   SS   SS   SS
For the example in figure \ref{fig:bar}, the temperature gradient was measured over the length of a 10 cm bar. For convenience, the end of the bar in contact with the ice bath was taken as the origin. But just as readily, that point could have been taken as $x$ = 5 cm. The end of the bar touching the boiling water bath is then 15 cm. This would not change the thermal profile shown in the figure.

  \begin{equation*}   %  =   =   =   =   =
  %\begin{split}
    \nabla T = \frac{T_{1} - T_{0}} { (x_{1} + x_{*}) - (x_{0} + x_{*}) }
             = \frac{T_{1} - T_{0}} { x_{1} - x_{0} }
    %\label{eq:}
  %\end{split}
  \end{equation*}

\subsubsection{Translation}  %  SSS  SSS  SSS  SSS  SSS  SSS  SSS  SSS  SSS  SSS  SSS  SSS
Intuitively, a translation of the coordinate system will not change the temperature measurements, which means the gradient of the temperature is invariant under translation of the coordinate system.
  \begin{equation*}   %  =   =   =   =   =
  %\begin{split}
    x_{k} \rightarrow x_{k} + x_{*}
    %\label{eq:}
  %\end{split}
  \end{equation*}

  \begin{equation*}   %  =   =   =   =   =
  %\begin{split}
    \mat{c}{x_{1} \\ x_{2} \\ \vdots \\x_{m}} \rightarrow
    \mat{c}{x_{1} + x_{*} \\ x_{2} + x_{*} \\ \vdots \\x_{m} + x_{*}} =
    \mat{c}{x_{1} \\ x_{2} \\ \vdots \\x_{m}} +
    \mat{c}{x_{*} \\ x_{*} \\ \vdots \\ x_{*}} 
    %\label{eq:}
  %\end{split}
  \end{equation*}

  \begin{equation*}   %  =   =   =   =   =
  %\begin{split}
    x \rightarrow x + \mathbf{1}x_{*}
    %\label{eq:}
  %\end{split}
  \end{equation*}

\subsubsection{Demonstrations}  %  SSS  SSS  SSS  SSS  SSS  SSS  SSS  SSS  SSS  SSS  SSS  SSS

Basic inner products
  \begin{equation*}   %  =   =   =   =   =
    \begin{split}
      \oto &\rightarrow \oto, \\
      \otx &\rightarrow \otx + x_{*} \poto, \\
      \xtx &\rightarrow \xtx + 2x_{*} \potx + x_{*}^{2} \potx, \\
      \xtt &\rightarrow \xtt + x_{*} \pott.
     \label{eq:translation:dot products}
     \end{split}
  \end{equation*}

Invariance of determinant \eqref{eq:det again}:
  \begin{equation*}   %  =   =   =   =   =
    \begin{split}
      \Delta 
        &= \poto \pxtx - \potx^{2} , \\
        &\rightarrow \poto \paren{\xtx + 2x_{*} \potx + x_{*}^{2} \potx} - \paren{\otx + x_{*} \poto}^{2} , \\
        &= \Delta + \poto \paren{2x_{*} \potx + x_{*}^{2} \potx} - \paren{2x_{*} \poto\potx} - \paren{x_{*}\poto}^{2} , \\
        &= \Delta.
    %\label{eq:}
    \end{split}
  \end{equation*}

  \begin{equation*}   %  =   =   =   =   =
  %\begin{split}
    y(0) \rightarrow y(-x_{*})
    %\label{eq:}
  %\end{split}
  \end{equation*}

Solutions in \eqref{eq:bevington:soln:vectors}.
  \begin{equation*}   %  =   =   =   =   =
    \begin{split}
      a_{1} &= \Delta^{-1} \paren{\poto \pxtt -
               \potx \pott}, \\
            &\rightarrow \Delta^{-1} \paren{\poto \paren{\xtt + x_{*} \pott} -
               \paren{\otx + x_{*} \poto} \pott}, \\
            &= a_{1} + \Delta^{-1} \paren{x_{*} \poto \pott - x_{*}\poto\pott}, \\
            &= a_{1} 
    %\label{eq:}
    \end{split}
  \end{equation*}

\subsubsection{Computations}  %  SSS  SSS  SSS  SSS  SSS  SSS  SSS  SSS  SSS  SSS  SSS  SSS

The slope is invariant, the intercept:
  \begin{equation*}   %  =   =   =   =   =
    \begin{split}
      a_{1} & \rightarrow a_{1}, \\
      a_{0} & \rightarrow a_{0} - x_{*} a_{1}.
    %\label{eq:}
    \end{split}
  \end{equation*}

Predictions for $x_{*} = 1$. 
  \begin{equation}   %  =   =   =   =   =
    \begin{split}
      a_{0} & \rightarrow -\frac{827}  {180}, \\
      a_{1} & \rightarrow  \frac{1129} {120}.
    \label{eq:bev:prediction}
    \end{split}
  \end{equation}

  % = =  e q u a t i o n
  \begin{equation*}
    \begin{split}
      \mat{c}{a_{0} \\ a_{1} } 
        &= \Delta^{-1}
           \bl{
            \mat{ll}{\ps \sum x_{k}^{2} & -\sum x_{k} \\ -\sum x_{k} & \ps \sum 1}
            \mat{l}{\sum T_{k} \\ \sum T_{k} x_{k}}} , \\
        &= 540^{-1}
           \bl{
            \mat{rr}{384 & -54 \\ -54 & 9} \frac{1}{10}
            \mat{r}{4667 \\ 33\,647}} , \\
        &= \frac{1}{360}
           \bl{
           \mat{r}{1733 \\ 3387}}.
      %\label{eqn:bevington solution product}
    \end{split}
  \end{equation*}
  % = =
In agreement with \eqref{eq:bev:prediction}.

\subsubsection{Visuals}  %  SSS  SSS  SSS  SSS  SSS  SSS  SSS  SSS  SSS  SSS  SSS  SSS

Compare to figure \ref{fig:bevington soln v data}
\begin{figure}[htbp] %  figure placement: here, top, bottom, or page
   \centering
   \begin{overpic}[ scale = \myscale ]
		{\pathgraphics "bevington III"/"translate data v soln"}
		% 
    	\put(42,65){$x_{k} \rightarrow x_{k} + x_{*}$}
    	\put(50,-3){$x,$ cm}
    	\put(-10,31.5){$T, ^{\circ}$C}
	    %
   \end{overpic}
   \caption{Solution after translation along $x$ axis: the slope is invariant.}
   \label{fig:bevington translate soln v data}
\end{figure}

Compare to \ref{fig:bevington residuals}
\begin{figure}[htbp] %  figure placement: here, top, bottom, or page
   \centering
   \begin{overpic}[ scale = \myscale ]
		{\pathgraphics "bevington III"/"translate residuals"}
		% 
    	\put(50,-3){$x,$ cm}
    	\put(-25,31.5){$T_{k}-T(x_{k}), ^{\circ}$C}
	    %
   \end{overpic}
   \caption[Scatter plot of residual errors.]{Scatter plot of residual errors after translation along $x$ axis: the residuals are invariant.}
   \label{fig:bevington translate residuals}
\end{figure}

Compare to figure \ref{fig:bevington merit}
\begin{figure}[htbp]
\centering
    \begin{overpic}[ scale = \myscale ]{\pathgraphics "bevington III"/"translate merit"}
        \put(40,102){$x_{k} \rightarrow x_{k} + x_{*}$}
    	\put(50,-3){$a_{0}$}
    	\put(-5,52){$a_{1}$}
    \end{overpic}
   \label{fig:bevington translate merit}
   \caption[The merit function after translation.]{Contour plot of merit function after translation along $x$ axis: $a_{1}$ is invariant.}
\end{figure}


\subsection{Reflection Invariance}  %   SS   SS   SS   SS   SS   SS   SS   SS   SS   SS   SS   SS
A demonstration of the reflection method of \S \ref{sec:reflection invariance}.
  \begin{equation*}   %  =   =   =   =   =
  %\begin{split}
    \sum_{k=1}^{m} r^{2} = \sum_{k=1}^{m} \paren{-r}^{2}
    %\label{eq:}
  %\end{split}
  \end{equation*}

table \ref{tab:bevington solution} 

\begin{figure}[htbp] %  figure placement: here, top, bottom, or page
   \centering
   \begin{overpic}[ scale = \myscale ]
	   {\pathgraphics "bevington plus"/"reflect bars"}
        %
      	\put(50,-3) {$a_{0}$}
      	\put(-4,33) {$a_{1}$}
	      %
   \end{overpic}
   \caption{The merit function for the learning curve showing the minimum and the value.}
   \label{fig:learn:merit}
\end{figure}

\section{Fitting To Higher Orders}  %    S    S    S    S    S    S    S    S    S

\endinput

%\section{Section Title}  %    S    S    S    S    S    S    S    S    S
%\subsection{Subsection Title}  %   SS   SS   SS   SS   SS   SS   SS   SS   SS   SS   SS   SS
%\subsubsection{Subsubsection Title}  %  SSS  SSS  SSS  SSS  SSS  SSS  SSS  SSS  SSS  SSS  SSS  SSS