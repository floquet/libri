\chapter[The Gram-Schmidt process]{The \\Gram-Schmidt process}
\label{sec:gs}

The Gram-Schmidt process takes a set of vectors and provides an orthonormal set of basis vectors. This process has two primary uses in the \asvd:
\begin{enumerate}
\item To build an orthonormal basis for a null space given a set of vectors in the image;
\item To orthonormalize the set of null space vectors output from a matrix reduction.
\end{enumerate}

Because this process is essential for computing the \asvd \ an appendix is warranted.

%%
\section{Process schematic}
Start with a set of $k$ $\mv$s
\begin{equation}
  U = \lst{u_{1},u_{2},\dots,u_{m}}, \quad k\le m.
\end{equation}
If $j$ vectors are linearly independent, the Gram-Schmidt process will produce a set of $j$ orthonormal vectors
\begin{equation}
  V = \lst{\hat{v}_{1},\hat{v}_{2},\dots,\hat{v}_{j}}, \quad j\le k.
\end{equation}

A sample set input and output for two linearly independent vectors is shown in table \eqref{tab:gs:io}. The input set $U$ is shown on the left and the set $V$ is shown on the right. Part of the unit circle is included to show the effect of dilations upon the vectors. The Gram-Schmidt orthogonalization process is then displayed pictorially in the following table \eqref{tab:gs:guts}.

%%%%%%
\input{chapters/appendices/"gram schmidt"/"tab the gram-schmidt orthogonalization process"}  % table
\input{chapters/appendices/"gram schmidt"/"tab the gram-schmidt orthogonalization process for two vectors"}  % table

%%
\section{Projections}
The language of projections is a natural choice for a discussion of the Gram-Schmidt process as one may surmise from the previous table. Recall that the projection of a vector $y$ onto the vector $x$ is defined this way
\begin{equation}
  \pee{}_{x}(y) = \frac{y\cdot x}{x\cdot x}y.
\end{equation}
This notation allows for a compact and intuitive representation of the orthogonalization.

In two dimensions orthogonalization is simple. The generalization to higher dimensions looks follows here. Start with a list with a total of $m$ vectors $U$ in arbitrary ordering. The output will be a set of $n$ orthogonal vectors $V$ which depends upon the ordering of the vectors in the input set. The first few steps look like this:
\begin{equation}
  \begin{array}{cccccccccc}
    v_{1} &=& u_{1}\\
    v_{2} &=& u_{2} &-& \pee{}_{v_{1}}(u_{2})\\
    v_{3} &=& u_{3} &-& \pee{}_{v_{1}}(u_{3}) &-& \pee{}_{v_{2}}(u_{3})\\
    v_{4} &=& u_{4} &-& \pee{}_{v_{1}}(u_{4}) &-& \pee{}_{v_{2}}(u_{4}) &-& \pee{}_{v_{3}}(u_{4})\\
     & \vdots
  \end{array}
\end{equation}
The general formula takes the compact form
\begin{equation}
  \check{v}_{k} = u_{k} - \sum_{j}^{m-1}{\pee{}_{v_{j}}(u_{k})}
\end{equation}
where $\check{v}$ denotes an unnormalized vector.

If the vectors in the collection $U$ are linearly independent, then the collection $V$ will have the same number of vectors. That is, $m=n$. If the input vectors have linear dependencies then there will be fewer vectors in the output list and $m>n$. For this discussion only nonzero vectors are relevant.
 
At this juncture the vectors in the set $V$ are orthogonal, but not yet orthonormal. To normalize them use the prescription
\begin{equation}
  v_{k} = \frac{\check{v}_{k}}{\normt{\check{v}_{k}}}, \quad k=1,n.
\end{equation}

%%
\section{Application}
Let's return to a familiar example matrix to use the Gram-Schmidt orthogonalization process to compute a \asvd.

Consider the rank-deficient rectangular matrix
\begin{equation}
  \A{} = \matrixat.
\end{equation}

%%
\subsection{Column space: codomain}
As noted early and often, the column space contains one independent column vector
\begin{equation}
  c_{1} = \mat{r}{1\\-1\\1}.
\end{equation}
We need to stir in two other vectors. A good choice is to use two unit vectors. They are simple and it is easy to check that they are not linearly dependent upon the target vector $c_{1}$. The input vectors are these:
\begin{equation}
  U = \lst{u_{1},u_{2},u_{3}} = \lst{
  \mat{r}{1\\-1\\1},
  \mat{r}{1\\0\\0},
  \mat{r}{0\\1\\0}
  }.
\end{equation}

%%
\subsubsection{First vector}
The first vector is the easiest. It requires a scaling, or dilation, to normalize the length:
\begin{equation}
  v_{1} = \frac{u_{1}}{\normt{u_{1}}} = \rsthree
  \mat{r}{1\\-1\\1}.
\end{equation}

This action corresponds to step 1 in table \eqref{tab:gs:guts} above.

%%
\subsubsection{Second vector}
The first part of this step is to compute the projection of the input vector $u_{2}$ onto the previous output vector $v_{1}$:
\begin{equation}
  \pee{}_{v_{1}}(u_{2}) = \rsthree\mat{r}{1\\-1\\1}.
\end{equation}
This projection corresponds to the red vector in step 2 in table \eqref{tab:gs:guts}.

The unnormalized output vector becomes this
\begin{equation}
  \check{v}_{2}= u_{2} - \pee{}_{v_{1}}(u_{2}) = \rthree \mat{r}{2\\1\\-1}.
\end{equation}
This corresponds step 3 in table \eqref{tab:gs:guts}.

The normalized form is then
\begin{equation}
  v_{2} = \rssix \mat{r}{2\\1\\-1}.
\end{equation}
This is the final step, the fourth step in table \eqref{tab:gs:guts}.

%%
\subsubsection{Third and final vector}
We need to project the final input vector $u_{3}$ onto the subordinate output vectors. The projections are these
\begin{equation}
  \begin{split}
    \pee{}_{v_{1}}(u_{3}) = \rthree \mat{r}{-1\\1\\-1}, \quad \pee{}_{v_{2}}(u_{3}) &= \rssix   \mat{r}{1\\1\\-1}.
  \end{split}
\end{equation}

The unnormalized output vector becomes this
\begin{equation}
  \check{v}_{2}= u_{3} - \pee{}_{v_{1}}(u_{3}) - \pee{}_{v_{2}}(u_{3}) = \rsthree \mat{r}{0\\1\\1}.
\end{equation}

The final normalized form is given by this
\begin{equation}
  v_{2} = \rstwo \mat{r}{0\\1\\1}.
\end{equation}

%%
\subsubsection{Assemble the codomain matrix}
The result from the Gram-Schmidt procedure are these three orthonormal vectors:
\begin{equation}
  V = \lst{v_{1},v_{2},v_{3}} = \lst{
  \rsthree \mat{r}{1\\-1\\1},
  \rssix   \mat{r}{2\\1\\-1},
  \rstwo   \mat{r}{0\\1\\1}
  }.
\end{equation}
These vectors are the column vectors of the codomain matrix:
\begin{equation}
  \U{} = \mat{c|c|c}{v_{1} & v_{2} & v_{3}} = \matrixaY
\end{equation}

%%
\subsubsection{Check the final answer}
A quick, easy and helpful check involves verifying the orthogonality of the vectors $V$:
\begin{equation}
  \begin{split}
    v_{1}\cdot v_{2} &= 0,\\
    v_{1}\cdot v_{3} &= 0,\\
    v_{2}\cdot v_{3} &= 0.\\
  \end{split}
\end{equation}

%%
\subsection{Row space: domain}
\label{sec:gs:row}
The row space here too contains one independent vector
\begin{equation}
  r_{1} = \mat{r}{1\\-1}.
\end{equation}
This time we need only one candidate \vv \ to complete the space. The two input vectors are these:
\begin{equation}
  U = \lst{u_{1},u_{2}} = \lst{
  \mat{r}{1\\-1},
  \mat{r}{1\\0}
  }.
\end{equation}

%%
\subsubsection{First vector}
The first vector only requires normalization:
\begin{equation}
  v_{1} = 
    \frac{u_{1}}{\normt{u_{1}}} = 
    \rstwo \mat{r}{1\\-1}.
\end{equation}

Notice that $v_{1}\cdot v_{2}=0$.

%%
\subsubsection{Second and final vector}
The projection of the second input vector $u_{2}$ onto the first output vector $v_{1}$:
\begin{equation}
  \pee{}_{v_{1}}\paren{u_{2}}  = \rstwo \mat{r}{1\\-1};
\end{equation}
this leads to the unnormalized output vector:
\begin{equation}
  \check{v}_{2}= u_{2} - \pee{}_{v_{1}}(u_{2}) = \rstwo\mat{r}{1\\1};
\end{equation}
with the normalized form given here
\begin{equation}
  v_{2} = \rstwo \mat{r}{1\\1}.
  \label{eq:gs:a}
\end{equation}

%%
\subsubsection{Assemble the domain matrix}
The column vectors of the domain matrix are these output vectors
\begin{equation}
  \V{} = \mat{c|c}{v_{1} & v_{2}} = \matrixaX.
\end{equation}

%%
\subsection{Assemble the SVD}
Given the domain matrices $\V{}$ and $\U{}$, the only remaining piece is the singular values. Since the target matrix had only one independent column\footnote{One could also base the rank argument on the number of independent rows.} the rank is $\rho = 1$. The $\sig{}$ matrix always has the same shape as the target matrix. Therefore we are solving for this matrix:
\begin{equation}
  \sig{} = \mat{c|c}{\sigma_{1} & 0 \\\hline 0 & 0 \\ 0 & 0}.
\end{equation}
We can glean the lone singular value $\sigma_{1}$ from this equation:
\begin{equation}
  \begin{split}
    \A{}\V{}_{*,1} &= \sigma_{1}\U{}_{*,1},\\
    \matrixa \rstwo \mat{r}{1\\-1} &= \sigma_{1} \rsthree \mat{r}{1\\-1\\1}.
  \end{split}
\end{equation}
This leads to the reassuring conclusion that
\begin{equation}
  \sigma_{1} = 6^{-1/2}.
\end{equation}

Compare the top SVD from the Gram-Schmidt process to the bottom form computed in \eqref{eq:simple:svd}
\begin{equation}
  \begin{array}{rcccc}
    \A{} 
      &=& \U{} & \sig{} & \V{*} \\
      &=& \matrixaY  
        & \sigmaat 
        & \matrixaX   
  \end{array}
\end{equation}

Two different processes yielded different yet equivalent \asvd s. The $\sig{}$ matrices must be the same. The domain matrices may have some sign difference in the range vectors. In this case the range vectors, the unshaded vectors in the $\V{}$ and $\U{}$ matrices match for both decomposition methods. The null space (shaded) vectors in the codomain matrix $\U{}$ are ordered differently. Also, there are sign differences between the shaded column vectors.

%%
\subsection{Superfluous vectors}
Let's explore the case where the vector collection $U$ has two many vectors. For example, suppose $U$ has three \vv s. The output collection $V$ will only have two orthogonal unit \vv s. How does a vector dissappear?

Go back to \S\eqref{sec:gs:row} and append a vector to the input collection:
\begin{equation}
  U = \lst{
  \mat{r}{1\\-1},
  \mat{r}{1\\0}
  }
  \quad \rightarrow \quad
  \lst{
  \mat{r}{1\\-1},
  \mat{r}{1\\0},
  \mat{r}{0\\1}
  }.
\end{equation}

We can pick up the calculation for $\check{v}_{3}$ after equation \eqref{eq:gs:a}:
\begin{equation}
  \begin{split}
    \check{v}_{3} &= u_{3} - \pee{}_{v_{1}}(u_{3}) - \pee{}_{v_{2}}(u_{3})\\
    & = \mat{c}{0\\1} - \rstwo\mat{r}{-1\\1} - \rstwo\mat{c}{1\\1}\\
    & = \mat{r}{0\\0}.
  \end{split}
\end{equation}
The first two output vectors, $v_{1}$ and $v_{2}$, completely span the space $\real{2}$. No other vectors are needed and the Gram-Schmidt process will produce zero vectors after this iteration.

Further readings: \cite[ch. 5.5, p. 307]{Meyer2000}, \cite[ch. 5.5, p. 307]{Laub2005},  \cite[]{Strang2005}.


\endinput