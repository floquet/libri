\clearpage
\thispagestyle{empty}
\begin{landscape}
%%
\begin{table}[htdp]
\caption[The two different \asvd s for the Jordan form $\J{2}$]{The two different \asvd s for the Jordan form $\J{2}$. The first column represents the codomain matrix, the second the domain matrix, and the final column the $\sig{}$ matrix. The top row represents even values of the parameter $k$ in equation \eqref{eq:examples:jcf2}; the bottom row odd. The column vectors of the domain matrices are arrows plotted below. The first column vector is the black arrow, the second is blue. These coordinate systems are right-handed because we advance from the first vector to the second in a \emph{clockwise} manner. The $\sig{}$ matrix is represented as an ellipse with radii $\lst{\sigma_{1}, \sigma_{2}}$. All three figures are plotted against a unit circle.}
\begin{center}
\begin{tabular}{ccc}
%
  $\U{}$ & \quad $\V{}$ & \quad $\sig{}$ \\[10pt]\hline
%
\includegraphics[ width = 1.5in ]{images/examples/Jordan/"J2 U k = 1"} & \quad
\includegraphics[ width = 1.5in ]{images/examples/Jordan/"J2 V k = 1"} & \quad
\includegraphics[ width = 3.93in ]{images/examples/Jordan/"J2 s"} \\[10pt]
%
\includegraphics[ width = 1.5in ]{images/examples/Jordan/"J2 U k = 0"} & \quad
\includegraphics[ width = 1.5in ]{images/examples/Jordan/"J2 V k = 0"} & \quad
\includegraphics[ width = 3.93in ]{images/examples/Jordan/"J2 s"} \\
%
\end{tabular}
\end{center}
\label{tab:J2:k}
\end{table}
%%
\end{landscape}


\endinput