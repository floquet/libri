\chapter{The pseudoinverse}
\label{chap:pseudoinverse}
In the last chapter, in equation \eqref{eq:13:bingo}, we derived an expression for the pseudoinverse solution for least squares problem. While very satisfying, the fact that this object generalizes the matrix inverse is far from obvious. Let's now discuss the pseudoinverse in a more general setting in linear algebra, not tied specifically to least squares problems.

The pseudoinverse, or generalized matrix inverse, is a powerful tool in the study of linear systems. We have seen already how this generalization is a potent tool in the arena of least squares problems. We will go on to see that the pseudoinverse and the \svdl \ work together like hand and glove.

%%
\section{Definition of the matrix pseudoinverse}
Can the matrix inverse be generalized? Can we find a consistent framework to talk about the inverse of a rectangular matrix? A singular matrix? Why would we search for such a beast?

We have seen in chapter \eqref{chap:simple} that general linear systems do not have an inverse, but they still offer a least squares solution. A reasonable expectation would be to connect the least squares solution with a generalized matrix inverse. In fact we saw that a generalized inverse would allow a direct solution to the least squares problem. 

The generalized matrix inverse also offers generalized nomenclature. It is also known as the pseudoinverse or the Moore-Penrose inverse.

Connect to the Drazin inverse.

So the concept of a generalized inverse is not far fetched. However, important theoretical work needed to be done to solidify the ideas. We begin with important guidelines which serve 
An elegant collection of necessary and conditions define the pseudoinverse. These conditions are credited to Sir Arthur Penrose [??].

\newtheorem*{thm}{Theorem}
\begin{thm}
The matrix $\G{}$ is a pseudoinverse of the matrix $\A{}$ if and only these four properties are satisfied:
\begin{enumerate}
\item $\A{}\G{}\A{} = \A{}$
\item $\G{}\A{}\G{} = \G{}$
\item $\paren{\A{}\G{}}^{*} = \A{}\G{}$
\item $\paren{\G{}\A{}}^{*} = \G{}\A{}$
\end{enumerate}
\end{thm}

Notice the conventional inverse also satisfies these four properties.

%
\subsection{The sabot matrix}
Perhaps the most confusing details involve the sabot matrix $\sig{}$. This container of zeros has the same shape as the target matrix. It contains a diagonal matrix $\textbf{S}$ of non-zero singular values. As an example a matrix $\A{}\in\cmplx{\by{4}{3}}_{2}$ the sabot matrix looks like
\begin{equation}
  \sig{} = \mat{c|c}{\textbf{S} & \zero \\\hline \zero & \zero} = 
  \mat{cc|c}{\sigma_{1}&0&0\\0&\sigma_{2}&0\\\hline0&0&0\\0&0&0}.
\end{equation}
The transpose behaves as expected:
\begin{equation}
  \sig{T} = \mat{c|c}{\textbf{S}^{\mathrm{T}} & \zero \\\hline \zero & \zero} = 
  \mat{cc|cc}{\sigma_{1}&0&0&0\\0&\sigma_{2}&0&0\\\hline0&0&0&0}.
\end{equation}
The confusion comes when we try to invert the sabot matrix. The singular value matrix $\textbf{S}$ inverts easily: just take the reciprocal of the diagonal entries. But what about the sabot entries? By looking at the conformability we see that we must also take the transpose of the sabot matrix. That is, if we want to multiply the size $\by{m}{n}$ sabot by its inverse on either the right or the left the inverse must have size $\by{n}{m}$. To form the inverse of the sabot matrix $\sig{}$ form the transpose and invert the matrix $\textbf{S}$. To remind us that this is special process use the following symbol
\begin{equation}
  \sig{(+)} = \mat{c|c}{\textbf{S}^{-1} & \zero \\\hline \zero & \zero} = 
  \mat{cc|cc}{\frac{1}{\sigma_{1}}&0&0&0\\0&\frac{1}{\sigma_{2}}&0&0\\[3pt]\hline0&0&0&0}.
\end{equation}

%%
\section{Manipulations}
\begin{equation}
  \begin{split}
    \svda{*}\\
    \A{*}&=\paren{\svd{*}}^{*}=\X{}\,\sig{T}\Y{*}.
  \end{split}
\end{equation}

\begin{equation}
  \begin{split}
  \sig{T}\sig{} &= 
  \mat{c|c}{\textbf{S}^{\mathrm{T}} & \zero \\\hline \zero & \zero}^{\by{n}{m}}_{\rho} 
  \mat{c|c}{\textbf{S} & \zero \\\hline \zero & \zero}^{\by{m}{n}}_{\rho} = 
  \mat{c|c}{\textbf{S}^{2} & \zero \\\hline \zero & \zero}^{\by{n}{n}}_{\rho} \\
  \sig{}\,\sig{T} &= 
  \mat{c|c}{\textbf{S} & \zero \\\hline \zero & \zero}^{\by{m}{n}}_{\rho} 
  \mat{c|c}{\textbf{S}^{\mathrm{T}} & \zero \\\hline \zero & \zero}^{\by{n}{m}}_{\rho} = 
  \mat{c|c}{\textbf{S}^{2} & \zero \\\hline \zero & \zero}^{\by{m}{m}}_{\rho}
  \end{split}
\end{equation}
These special cases where there is no sabot follow naturally
\begin{enumerate}
\item if $\rho=m$ then $\sig{}\sig{T} = \ess{2}$;
\item if $\rho=n$ then $\sig{T}\sig{} = \ess{2}$;
\item if $\rho=m=n$ then $\sig{T}\sig{} = \sig{}\sig{T} = \ess{2}$.
\end{enumerate}

\begin{equation}
  \begin{split}
  \sig{T}\sig{} &= \mat{cc|cc}{\sigma_{1}^{2}&0&0&0\\0&\sigma_{2}^{2}&0&0\\\hline0&0&0&0\\0&0&0&0} \\
  \sig{}\,\sig{T} &= \mat{cc|c}{\sigma_{1}^{2}&0&0\\0&\sigma_{2}^{2}&0\\\hline0&0&0}
  \end{split}
\end{equation}


\begin{equation}
  \begin{split}
  \sig{(+)}\sig{} &= 
  \mat{c|c}{\textbf{S}^{-1} & \zero \\\hline \zero & \zero}^{\by{n}{m}}_{\rho} 
  \mat{c|c}{\textbf{S} & \zero \\\hline \zero & \zero}^{\by{m}{n}}_{\rho} = 
  \mat{c|c}{\I{\rho} & \zero \\\hline \zero & \zero}^{\by{n}{n}}_{\rho}=\J{n}{\rho} \\
  \sig{}\,\sig{(+)} &= 
  \mat{c|c}{\textbf{S} & \zero \\\hline \zero & \zero}^{\by{m}{n}}_{\rho} 
  \mat{c|c}{\textbf{S}^{-1} & \zero \\\hline \zero & \zero}^{\by{n}{m}}_{\rho} = 
  \mat{c|c}{\I{\rho} & \zero \\\hline \zero & \zero}^{\by{m}{m}}_{\rho}=\J{m}{\rho}
  \end{split}
\end{equation}

For $\A{}\in\cmplx{\by{3}{4}}_{2}$

\begin{equation}
  \begin{split}
  \sig{(+)}\sig{} &= 
  \mat{cc|cc}{1&0&0&0\\0&1&0&0\\\hline0&0&0&0\\0&0&0&0} = \J{4}{2} \\
  \sig{}\,\sig{(+)} &= 
  \mat{cc|c}{1&0&0\\0&1&0\\\hline0&0&0} = \J{3}{2}
  \end{split}
\end{equation}

\begin{figure}[htbp] %  figure placement: here, top, bottom, or page
   \centering
   \includegraphics[ ]{pdf/svd/mask_06_04_04} 
   \caption{The stencil action of the truncated identity matix.}
   \label{fig:svd:stencil}
\end{figure}

%%
\subsection[Construction: general case]{Constructing the pseudoinverse when $\A{}$ is singular}
Let's back up and look at the process more carefully.
Given a \svdl \ decomposition, how do we construct the pseudoinverse? The simplest step would be to apply the inverse operation to the decomposition. The domain matrices invert easily - they are orthogonal, so forming the Hermitian conjugate forms the inverse. The sticky issue comes from the matrix of singular values, $\sig{}$. How should we ``invert'' a matrix which is not square and which may have zero elements on the lower diagonals?

If two matrices $\U{}$ and $\V{}$ are both nonsingular, then we can relate the inverse of the product to the product of the inverses:
\begin{equation}
  \paren{\U{}\V{}}^{-1} = \V{-1}\U{-1}.
\end{equation}

Consider the case of a nonsingular matrix $\A{}$.
\begin{equation}
  \paren{\A{}}^{-1} \Rightarrow \paren{\svd{*}}^{-1} \Rightarrow \paren{\X{*}}^{-1}\paren{\sig{}}^{-1}\paren{\Y{}}^{-1}=\X{}\,\sig{-1}\,\Y{*}.
\end{equation}
Because the domain matrices are unitary, their inverses are trivial to compute: form the Hermitian conjugate. When the target matrix is nonsingular the $\sig{}$ matrix is diagonal and can also be inverted easily.

Retreat to the safe case: when $\sig{}$ is square and diagonal with a full diagonal with no zero elements. There we would invert the diagonal elements like so
\begin{equation}
  \begin{split}
   \sig{}       & \quad \Rightarrow \quad \sig{-1},\\    
   \sig{}_{k,k} & \quad \Rightarrow \quad \frac{1}{\sig{}_{k,k}}, \quad k=1,2,\dots,\rho.    
  \end{split}
\end{equation}This motivates us to move to the pathological cases and simply invert all of the singular values. By definition, in this work singular values are non-zero. In order to be conformable we also need to form the transpose.

To invert a singular \index{singular values matrix!inversion}singular values matrix $\sig{}$, perform these two steps:
\begin{enumerate}
\item form the transpose matrix $\sig{T}$,
\item invert the singular values.
\end{enumerate}
In terms of the $\ess{}$ matrix the operations look like this:
\begin{equation}
  \begin{split}
    \sig{} &= \mat{c|c}
    {
    \ess{} & \zero \\\hline
    \zero & \zero
    }^{\paren{m\times n}} \\
    \sig{(+)} &= \mat{c|c}
    {
    \ess{-1} & \zero \\\hline
    \zero & \zero
    }^{\paren{n\times m}}
  \end{split}
\end{equation}
The trouble with this formulation is that it obscures the transpose of the sabot matrix.  Hopefully the examples will clarify this transposition of the sabot.

For our well-travelled example matrix this process looks like
\begin{equation}
\begin{array}{ccc}
\sig{} &\Rightarrow& \sig{(+)} \\
 \mat{c|c}{\sigma_{1} & 0\\\hline0 & 0 \\0 & 0} & \Rightarrow & \mat{c|cc}{\frac{1}{\sigma_{1}} & 0 & 0\\[3pt]\hline0 & 0 & 0} \\
\Sigmaexampleb  & \Rightarrow & \mat{c|cc}{\ssix & 0 & 0\\[4pt]\hline0 & 0 & 0}.
\end{array}
\end{equation}
Because the process is unique to the $\sig{}$ matrix, it uses a dedicated superscript ``(+)''. In conversational mathematics, we would say
\begin{quote}
  To form the inverse of the $\sig{}$ matrix of singular values invert all non-zero entries and form the transpose matrix.
\end{quote}

These simplistic, intuitive steps work. 
\begin{equation}
    \mpgiax{*}
\end{equation}
The relationships between \svdl s for the target matrix, the Hermitian conjugate and the psuedoinverse are shown here:
\begin{equation}
  \begin{array}{lcccc}
    \A{} &=& \Y{} & \sig{} & \X{*} \\
    \A{*} &=& \X{} & \sig{T} & \Y{*} \\
    \Ap &=& \X{} & \sig{(+)} & \Y{*} \\
  \end{array}
\end{equation}

%%
\subsection[Construction: special case]{Constructing the pseudoinverse when $\A{}$ is nonsingular}
For the case when the target matrix $\A{}$ is nonsingular there is no sabot matrix: $\sig{}=\ess{}$ and the inversion is process is
\begin{equation}
  \begin{split}
    \sig{} &= \ess{} \\
    \sig{(+)} &= \ess{-1}
  \end{split}
\end{equation}
Restated another way, a matrix is nonsingular if and only if the matrix of singular values fils the sabot matrix.
\begin{equation}
  \A{} \text{ is nonsingular}\qquad \iff \qquad \sig{}=\ess{}
\end{equation}
In these cases the $\sig{}$ matrix will be square and diagonal with no zero entries on the diagonal.

%%
\subsection[Verification: singular case]{Verification of the pseudoinverse: nonsingular case}

\endinput
\input{chapters/"ch 14"/"chiral inverses"}
\input{chapters/"ch 14"/"fundamental projectors"}
\input{chapters/"ch 14"/"special cases"}
\section{Back to vectors}
What is the pseudoinverse of a vector? Laub \cite[p. ??]{Laub2005} presents the formula for the pseudoinverse of a vector $v\in\cmplxm$
\begin{equation}
  v^{\psymbol} = \paren{v^{*}v}^{\psymbol}v^{*} = \frac{ v^{*} }{ v^{*}v }.
\end{equation}
The derivation follows.

%%
\subsection{Derivation of the pseudoinverse}
We will construct the pseudoinverse from the thin SVD. That is, we will only compute the range space contributions to the SVD. As is so often the case, we don't need to the full rigor of the general decomposition. Instead, we can
\begin{enumerate}
\item Form the product matrix $\W{x} = v^{*}v$: dimension $\byy{1}$.
\item Form the domain basis matrix $\X{} = \mat{c}{1}$: dimension $\byy{1}$.
\item Find the lone eigenvalue: $\lambda\paren{\W{x}}$.
\item Compute the lone singular value $\sigma_{1} = \sqrt{\lambda_{1}}$.
\item Form the matrix $\ess{} = \mat{c}{\sigma_{1}}$: dimension $\byy{1}$.
\item Find the lone range space column vector $y_{1}$ for the codomain matrix $\Y{}$: dimension $\by{n}{1}$.
\item The SVD is the $v = y_{1}\sigma_{1}x_{1}$.
\item The pseudoinverse is $v^{*} = x_{1}\sigma_{1}^{-1}y_{1}$.
\end{enumerate}

(1) Form the product matrix:
\begin{equation*}
  \W{x} = v^{*}v = \mat{c}{\normt{v}^{2}}.
\end{equation*}
(2) The domain basis has dimension $\byy{1}$:
\begin{equation*}
  \X{} = \mat{c}{1}.
\end{equation*}
(3) The matrix $\W{x} = \mat{c}{\normt{v}^{2}}$ is diagonal and the eigenvalue is therefore given by the diagonal entry:
\begin{equation}
  \lambda\paren{\W{x}} = \lst{\normt{v}^{2}}.
\end{equation}
(4) The singular value is $\sigma_{1} = \sqrt{\lambda_{1}} = \normt{v}$.\\
(5) The matrix of singular values $\ess{} = \mat{c}{\normt{v}}$.\\
(6) The range space column vector is the solution to this:
\begin{equation*}
  \begin{split}
     \A{}x_{1} &= \sigma_{1} y_{1},\\
     v \mat{c}{1} &= \normt{v} y_{1},\\
     \frac{v}{\normt{v}} &= y_{1}.
  \end{split}
\end{equation*}
(7) The SVD is given by this:
\begin{equation*}
  v = \svd{*} = v \mat{c}{\normt{v}}\mat{c}{1}.
\end{equation*}
(7) The pseudoinverse is given by this:
\begin{equation*}
  v = \mpgi{*} = \mat{c}{1} \mat{c}{\normt{v}^{-1}}v^{*} = \frac{v*}{\normt{v}}.
\end{equation*}

We have yet another example of an important SVD which required little. The next step is to check the Moore-Penrose conditions.
In this case the target matrix $\Arr{n}{1}$ is a vector which maps $1-$vectors (scalars) into $n-$vectors and so the vector space of the domain is a point. Since there is no null space component, if a solution exists it will be unique. The subspace decomposition takes the form:
\begin{equation}
  \A{} = \Y{}\,\Sigma\,\X{T} = 
  \mat{cc}{\Y{}_{\rng{\A{}}} & \Y{}_{\nll{\A{T}}}} 
  \mat{c}{\ess{} \\\hline \zero} 
  \mat{c}{1}
\end{equation}


%%%%%%%%%%%%%%%%%%%%%%%%%%%%%%%%%%%%%%%
\subsubsection{Construction of the SVD}
In the general SVD construction one starts by resolving the eigensystem of the product matrix 
\begin{equation}
  \W{x} = \A{*}\A{}
\end{equation}
and then uses these right singular vectors $x_{k}$ to compute the left-singular vectors $y_{k}$ using the relationship
\begin{equation}
  \A{}x_{k} = \sigma_{k} y_{k}, \quad k=1\colon\rho.
\end{equation}
In this case the algebra is trivial because the product matrix is a $\byy{1}$ matrix:
\begin{equation}
  \W{x} = \A{T}\A{} = \mat{c}{n}.
\end{equation}
The lone eigenvector is obviously
\begin{equation}
  x_{1} = \mat{c}{1}.
\end{equation}
and the single eigenvalue is then
\begin{equation}
  \lambda_{1} = n.
\end{equation}
The singular value is the square root of this eigenvalue
\begin{equation}
  \sigma_{1} = \sqrt{\lambda_{1}} = \sqrt{n}.
\end{equation}

We may now complete the domain matrix $\X{}$ and the $\Sigma$ matrix:
\begin{equation}
  \Sigma = \mat{c}{\sigma_{1} \\ 0 \\ \vdots \\ 0},
\end{equation}
and 
\begin{equation}
  \X{} = \mat{c}{1}.
\end{equation}
There is one range space vector in the domain matrix and therefore there will be one range space vector in codomain matrix. This vector, $y_{1}$, is the solution to this equation:
\begin{equation}
  \A{}x_{1} = \sigma_{1}y_{1} \qquad \Longrightarrow \qquad \mat{c}{1\\1\\\vdots\\1}\mat{c}{1} = \sqrt{n} y_{1}
\end{equation}
which has the solution
\begin{equation}
  y_{1} = \recip{\sqrt{n}}\mat{c}{1\\1\\\vdots\\1}
\end{equation}
This is the column vector describing the range space of the codomain, $\rng{\A{}}$.

To complete the decomposition of the codomain, we need an orthonormal basis which spans the null space of the domain, $\nll{\A{T}}$. By inspection we see one set of such vectors can be these:
\begin{equation}
  \mat{r}{-1\\1\\0\\0\\\vdots\\0}, \mat{r}{-1\\0\\1\\0\\\vdots\\0}, \mat{r}{-1\\0\\0\\1\\\vdots\\0}, \dots, \mat{r}{-1\\0\\0\\0\\\vdots\\1}.
\end{equation}

The decomposition is complete and can be written as this:
\begin{equation}
  \begin{split}
     \A{} &= \Y{}\,\Sigma{}\,\X{T},\\
     \mat{c}{1\\1\\\vdots\\1} & =
     \mat{c|>{\columncolor{ltgray}}c>{\columncolor{ltgray}}c>{\columncolor{ltgray}}c}{
     \mat{c}{\recip{\sqrt{n}}\\\recip{\sqrt{n}}\\\vdots\\\recip{\sqrt{n}}} &
     \mat{c}{-\recip{\sqrt{2}}\\\recip{\sqrt{2}}\\0\\0\\\vdots\\0} &
     \dots &
     \mat{c}{-\recip{\sqrt{2}}\\0\\0\\0\\\vdots\\\recip{\sqrt{2}}} 
     }
     \mat{c}{\sqrt{n}\\0\\\vdots\\0}
     \mat{c}{1}
  \end{split}
\end{equation}
The null space vectors are shaded to remind us that they are silent.

%%%%%%%%%%%%%%%%%%%%%%%%%%%%%%%%%%%%%%%
\subsubsection{Constructing the pseudoinverse}
The psuedoinverse in \eqref{eq:1:ap} can be constructed from either the full SVD or the thin SVD which only uses the range space components and the matrix of singular values $\ess{}$. Our pseudoinverse is given by this:
\begin{equation}
  \begin{split}
     \Ap &= \X{}_{\rng{\A{*}}} \, \ess{-1} \, \Y{*}_{\rng{\A{}}},\\
         &= \mat{c}{1} \mat{c}{\recip{\sqrt{n}}} \recip{\sqrt{n}}\mat{cccc}{1&1&\dots&1}, \\
         &= \recip{n}\mat{cccc}{1&1&\dots&1}.
  \end{split}
\end{equation} 

%%
\subsection{Verification of the pseudoinverse}
Does the pseudoinverse for a vector satisfy the four Moore-Penrose conditions? We can simplify the process if we do the last two identities first.
%%%
\begin{enumerate}
\item Does $v v^{\psymbol} v =v$?
Yes.
\begin{equation}
  \begin{split}
     v \paren{v^{\psymbol}} v &= v \paren{\frac{v^{*}v}{v^{*}v}} v,\\
     &= v \frac{v^{*}v}{v^{*}v},\\
     &= v
  \end{split}
\end{equation}
\item Does $v^{\psymbol} v v^{\psymbol} = v^{\psymbol} $?
Yes.
\begin{equation}
  \begin{split}
     v^{\psymbol} \paren{v v^{\psymbol}} &= v^{\psymbol}\paren{\frac{v^{*}v}{v^{*}v}},\\
     &= v^{\psymbol}
  \end{split}
\end{equation}
\item Does $\paren{v v^{\psymbol}}^{*} = v v^{\psymbol}$?
Yes, because $vv^{\psymbol}$ is a scalar and unchanged by the adjoint operation.
\item Does $\paren{v^{\psymbol}v}^{*} = v^{\psymbol}v $?
Yes, because $v^{\psymbol}v$ is a scalar.
\begin{equation}
  \begin{split}
     v^{\psymbol}v &= \mat{c}{v_{1}v\\\hline v_{2}v\\\hline \vdots\\\hline v_{n}v} = \mat{cccc}{v_{1}v_{1} & v_{1}v_{2} & \dots & v_{1}v_{n} \\\hline v_{2}v_{1} & v_{2}v_{2} & \dots & v_{2}v_{n} \\\hline  \vdots & \dots && \vdots \\ v_{n}v_{1} & v_{n}v_{2} & \dots & v_{n}v_{n} }
  \end{split}
\end{equation}
\end{enumerate}


\endinput
\section{Generalized Fourier expansion}
In chapter 5 we looked at a generalized Fourier expansion for the target matrix using a sum of rank one matrices. We mentioned the trouble we would have with the pseudoinverse. While in the matrix decomposition we can happily discard the smallest singular values, we see that these are the values which contribute to the sum the most.

Rank one decomposition for the pseudoinverse:
\begin{equation}
  \Ap = \sum_{k=1}^{rho} {\sigma_{k}^{-1}x_{k}y_{k}^{\TT}}  = \sum_{k=1}^{rho} {\sigma_{k}^{-1}x_{k}\otimes y_{k}}
\end{equation}

\endinput
\section{Approaching the pseudoinverse}
%%%
\begin{equation}
  \Ap = \lim_{\delta\to 0} \paren{\prdmx{*} + \delta^{2}\I{m}}\A{*}
  \label{eq:14:column}
\end{equation}
%%%
\begin{equation}
  \Ap = \lim_{\delta\to 0} \A{*}\paren{\prdmy{*} + \delta^{2}\I{n}}
  \label{eq:14:row}
\end{equation}

We recognize these cases as the full column rank and full row rank solutions!

\endinput
\section{The Drazin inverse}
There is another common type of matrix inverse called the Drazin inverse\index{Drazin inverse}. The natural expression for this inverse is in terms of a core-nilpotent matrix decomposition.

Consider a singular matrix $\Ac{m}$ of with an index $k$ defined such that
\begin{equation}
  rank\paren{\A{k}} = \rho
\end{equation}
there exists a nonsingular matrix $\Q{}$ which decomposes the target matrix into a block matrix of the form
\begin{equation}
  \Q{}\A{}\Q{-1} = \mat{c|c}{\pee{}_{\rho\times\rho} & \zero \\[3pt]\hline \zero & \N{}}.
\end{equation}
Here the matrix $\pee{}$ represents the persistent piece and the $\paren{\bys{m-r}}$ block is nilpotent with degree $k$. That is
\begin{equation}
  \N{k} = \zero.
\end{equation}


\endinput
\section{Chiral inverses}

\subsection{In pictures}
\begin{table}[htdp]
\begin{center}
\begin{tabular}{cccl}
  $\X{}$ & $\Y{}$ & inverse type \\\hline
  %
  \includegraphics[]{pdf/"ch 14"/"r 2 c 2 s 0"} &
  \includegraphics[]{pdf/"ch 14"/"r 2 c 2 s 0"} &
  $\Ap = \A{-1}$ \\[20pt]
  %
  \includegraphics[]{pdf/"ch 14"/"r 2 c 2 s 0"} &
  \includegraphics[]{pdf/"ch 14"/"r 3 c 3 s 1"} &
  $\Ap = \AinvL$ \\[20pt]
  %
  \includegraphics[]{pdf/"ch 14"/"r 3 c 3 s 1"} &
  \includegraphics[]{pdf/"ch 14"/"r 2 c 2 s 0"} &
  $\Ap = \AinvR$ \\[20pt]
  %
  \includegraphics[]{pdf/"ch 14"/"r 2 c 2 s 1"} &
  \includegraphics[]{pdf/"ch 14"/"r 3 c 3 s 2"} &
  $\Ap \ne \AinvL,\ \Ap \ne \AinvR$ \\
  %
\end{tabular}
\end{center}
\label{tab:14:experiment:b}
\caption{Chiral inverses in pictures. This table shows that the presence of a null space determines the type of matrix inverse. We see representative domain matrices under $\X{}$ and codomain matrices under $\Y{}$. Following convention, column vectors in the null space are shaded.}
\end{table}%

\begin{landscape}
\subsection{Rank conditions}
\begin{table}[htdp]
\begin{center}
\begin{tabular}{llllll}
%
 chirality & $\A{}$ & equivalence & identity relation & $\rho$ & rank \\\hline
%
 ambichiral & $\cmplxmm$ & $\Ap = \A{-1}$ & $\leftinv = \rightinv = \I{m}$ & $\rho = m$ & full matrix rank \\
%
 left & $\cmplxmn$ & $\Ap = \AinvL$ & $\leftinv = \I{n}$ & $\rho = n$ & full column rank \\
%
 right & $\cmplxmn$ & $\Ap = \AinvR$ & $\rightinv = \I{m}$ & $\rho = m$ & full row rank \\
%
 ampichiral & $\cmplxmn$ &   & & $\rho \ne n$  & column rank deficiency \& \\
 &&& & $\rho \ne m$& row rank deficiency
%
\end{tabular}
\end{center}
\label{tab:14:experiment:a}
\caption{Pseudoinverse chirality and rank. A full rank condition - either in row or column - produces a chiral inverse. With rank deficiency in both row and column we obtain an ambichiral inverse, an inverse that neither right nor left handed.}
\end{table}%
\end{landscape}


\endinput

\endinput