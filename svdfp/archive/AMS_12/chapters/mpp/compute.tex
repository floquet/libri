\section{Compute the pseudoinverse}

Here are some tempting morsels to practice with. This shows the almost symbiotic relationship between the \svdl \ and the generalized matrix inverse.

%%
\subsection{When $\paren{\A{}\A{*}}^{-1}$ exists}
Compute the pseudoinverse $\A{+}$ when the inverse of the product matrix $\paren{\A{}\A{*}}^{-1}$ exists.

Since the inverse of the product matrix exists, the inverse of each component must exist:
\begin{equation}
  \paren{\A{}\A{*}}^{-1} = \paren{\A{*}}^{-1}\paren{\A{}}^{-1}.
\end{equation}
Therefore the target matrix is full rank. Therefore the pseudoinverse matches the standard inverse: 
\begin{equation}
  \A{+} = \A{-1}.
\end{equation}

In terms of the SVD, the matrix product reduces to
\begin{equation}
  \begin{split}
    \A{}\A{*} & = \paren{\svd{*}}\paren{\svdt{*}}\\
      &= \Y{}\, \sig{}\sig{T}  \Y{*}.
  \end{split}
\end{equation}

Is that your final answer?
\begin{equation}
  \A{+} = \A{*}\paren{\prdm{*}}^{-1}
\end{equation}

As an exercise, check that
\begin{equation}
  \A{}\A{+} = \A{+}\A{} = \I{}.
\end{equation}

%%
\subsection{When $\prdmm{*}=\I{}$}
Because the product matrix
\begin{equation}
  \prdmm{*} = \I{}
\end{equation}
all of the singular values are unity. Therefore
\begin{equation}
  \sig{}=\sig{T}=\I{}.
\end{equation}
\begin{equation}
  \begin{split}
    \prdmm{*}&=\I{},\\
    \paren{\svdt{*}}\paren{\svd{*}}&=\I{},\\
    \X{}\X{*}&=\I{}.
  \end{split}
\end{equation}

	
\endinput