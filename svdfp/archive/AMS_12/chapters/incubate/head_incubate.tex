\chapter[]{Incubator}
Here are some random thoughts which might require further development.

\section{SVD by inspection}
\begin{equation}
  \A{} = \mat{cc}{1&0\\0&1\\1&0}
\end{equation}
Full column rank. Therefore $\Ap = \AinvL$. Try an identity matrix for the domain matrix and assemble a null space vector for the codomain matrix: 
\begin{equation}
  \X{} = \stwo\itwo, \qquad \Y{} = \mat{rr>{\columncolor{ltgray}}r}{\stwo & 0 & \stwo \\ 0 & 1 & 0 \\ \stwo & 0 & \frac{-1}{\sqrt{2}}}.
\end{equation}
If these two matrices are adequate we can find the two singular values using the following relationships
\begin{equation}
  \begin{split}
    \A{}\X{}_{*,1} & = \sigma_{1}\Y{}_{*,1},\\
    \A{}\X{}_{*,2} & = \sigma_{2}\Y{}_{*,2}.\\
  \end{split}
\end{equation}
When solved, we see that the matrix of singular values is this:
\begin{equation}
  \ess{} = \sig{} = \mat{cc}{\sqrt{2} & 0 \\ 0 & 1}.
\end{equation}
The final \svdl \ is this:
\begin{equation}
  \begin{split}
    \svda{T},\\
     \mat{cc}{1&0\\0&1\\1&0} & = 
     \mat{ccc}{\stwo & 0 & \stwo \\ 0 & 1 & 0 \\ \stwo & 0 & \frac{-1}{\sqrt{2}}}
     \mat{cc} {\sqrt{2} & 0 \\ 0 & 1}
     \stwo \itwo.
  \end{split}
\end{equation}

\section{Almost linear}
Consider a matrix which allows us to experiment with the linear independence of the column vectors. To wit
\begin{equation}
  \A{} = \mat{cc}{1&1\\\delta&0\\0&\delta}
\end{equation}
where $\delta$ is a real and positive number.

This matrix too can be resolved informally. Since the target matrix has rank 2, we can try an identity matrix again for the domain matrix:
\begin{equation}
  \X{} = \itwo.
\end{equation}
This won't work, so we try the next guess, a rotation by $\pi/4$:
\begin{equation}
  \X{} = \stwo \mat{rr}{1&-1\\1&1}.
\end{equation}
This works and we find that
\begin{equation}
  \begin{split}
    \svda{T},\\
     \mat{cc}{1&1\\\delta&0\\0&\delta} & = 
\left[
\begin{array}{rr>{\columncolor{ltgray}}r}
 0 & \frac{\sqrt{2}}{\sqrt{2+\delta ^2}} & \frac{\delta ^2}{\sqrt{\delta ^2 \left[2+\delta ^2\right)}} \\
 \frac{1}{\sqrt{2}} & \frac{\delta }{\sqrt{2} \sqrt{2+\delta ^2}} & -\frac{\delta }{\sqrt{\delta ^2 \left[2+\delta ^2\right)}} \\
 -\frac{1}{\sqrt{2}} & \frac{\delta }{\sqrt{2} \sqrt{2+\delta ^2}} & -\frac{\delta }{\sqrt{\delta ^2 \left[2+\delta ^2\right)}}
\end{array}
\right]    
     \mat{cc} {\delta & 0 \\ 0 & \sqrt{2+\delta^{2}} \\\hline 0 & 0}
     \stwo \mat{rr}{1&1\\-1&1}.
  \end{split}
\end{equation}

Hmmm.
\begin{equation}
  \sqrt{2+\delta ^2} > \delta
\end{equation}

\section{Slow decay}
Consider a matrix which allows us to experiment with the linear independence of the
\begin{equation}
  \A{} = \mat{rr}{2\alpha & \alpha \\ -\alpha & 0 }
\end{equation}
\begin{equation}
  \begin{split}
    \Y{} & = 
\left(
\begin{array}{cc}
 \frac{\sqrt{2-\sqrt{2}}}{2} & \frac{\sqrt{2+\sqrt{2}}}{2} \\
 \frac{\sqrt{2+\sqrt{2}}}{2} & -\frac{1}{2} \sqrt{2-\sqrt{2}}
\end{array}
\right), \\
   \sig{} & =
\left(
\begin{array}{cc}
 \sqrt{3-2 \sqrt{2}} \alpha  & 0 \\
 0 & \sqrt{3 \alpha ^2+2 \sqrt{2} \alpha ^2}
\end{array}
\right), \\
   \X{} &=
\left(
\begin{array}{cc}
 -\frac{1}{2} \sqrt{2-\sqrt{2}} & \frac{\sqrt{2+\sqrt{2}}}{2} \\
 \frac{\sqrt{2+\sqrt{2}}}{2} & \frac{1}{\sqrt{2 \left(2+\sqrt{2}\right)}}
\end{array}
\right)
  \end{split}
\end{equation}

 


\endinput