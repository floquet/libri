\section{Complex numbers}
The field of complex numbers 
\begin{equation}
  z = a+ib
\end{equation}
with $a, b \in \real{}$ is \index{complex numbers!isomorphism} isomorphic to matrices of the form
\begin{equation}
  \Z{} = \mat{cr}{a&-b\\b&a}.
  \label{eq:fun:Z}
\end{equation}
You can check the isomorphism easily. For example
\begin{equation}
  z^{2} = a^{2} - b^{2} + i2ab.
\end{equation}
In the matrix notation this has the form
\begin{equation}
  z^{2}=\mat{cc}{a^{2} - b^{2} & -2ab\\2ab & a^{2} - b^{2}}.
\end{equation}
This matches the square of the matrix operator
\begin{equation}
  \Z{2}= \mat{cr}{a&-b\\b&a}\mat{cr}{a&-b\\b&a} =\mat{cc}{a^{2} - b^{2} & -2ab\\2ab & a^{2} - b^{2}}.
\end{equation}

%%
\subsection{Finding the decomposition}
With complex numbers in matrix form we are able to talk about the \svdl \ of a complex number. There are a few ways to approach this problem. One way is to guess $\X{}=\I{2}$ and reason that the fundamental length scale is the length of the vector. This leads to
\begin{equation}
  \sig{} = r \itwo.
\end{equation}
The codomain matrix then assembles quickly. However, we detail an approach which relies less on intuition and more on deduction.

The problems [XX] for the second chapter explored the interpretation of a complex number as rotation operator. To rotate a vector in $\real{2}$ by an angle $\theta$ multiply by the orthogonal matrix
\begin{equation}
  \R{}(\theta) = \mat{cr}{\cos \theta & -\sin \theta \\ \sin \theta & \cos \theta}.
\end{equation}
For this matrix the elements are constrained by
\begin{equation}
  \cos^{2} \theta + \sin^{2} \theta = 1.
\end{equation}
For the $\Z{}$ matrix in equation \eqref{eq:fun:Z} the constraint is
\begin{equation}
  a^{2}+b^{2} = r^{2}.
\end{equation}
The first constraint equation describes the unit circle; the second equation a circle of radius $r$.

As demonstrated in the exercises, the rotation matrix has an \svdl \ of the form
\begin{equation}
  \begin{split}
    \R{}(\theta) &= \svd{T} \\
    \R{}(\theta) &= \R{}(\theta) \, \I{2} \, \I{2}\\
    \mat{cr}{\cos \theta & -\sin \theta \\ \sin \theta & \cos \theta} &= \mat{cr}{\cos \theta & -\sin \theta \\ \sin \theta & \cos \theta}\, \itwo \, \itwo.
  \end{split}
\end{equation}
By straightforward extension
\begin{equation}
  \begin{split}
    \Z{} &= \svd{T} \\
    \Z{} &= \R{}(\theta)\,\paren{r\I{2}}\,\I{2}\\
    \mat{cr}{a & -b \\ b & a} &= \mat{cr}{\cos \theta & -\sin \theta \\ \sin \theta & \cos \theta}\, \mat{cc}{r&0\\0&r} \, \itwo
  \end{split}
\end{equation}
where the angle has the usual definition
\begin{equation}
  \theta = \arctan \frac{b}{a}.
\end{equation}
In terms of the original parameters and $r$ the expansion is
\begin{equation}
  \Z{} = \svd{T} = \paren{\frac{1}{r}\mat{cr}{a&-b\\b&a}}\,\paren{r\I{2}}\,\I{2}.
\end{equation}
This completes an analytic decomposition that is most unforgiving if treated directly.

%%
\subsection{Equivalent solutions}
The singular value matrix $\sig{}$ is unique, but the domain matrices are not. One immediate variant is
\begin{equation}
  \Z{} = \svd{T} = \paren{\frac{1}{r}\mat{rr}{-b&a\\a&b}}\,\paren{r\I{2}}\,\mat{cc}{0&1\\1&0}.
\end{equation}

%%
\subsection{Working with the decomposition}

%%
\subsubsection{Complex conjugates}
From complex variable theory we know that
\begin{equation}
  z\overline{z} = \abs{z} = a^{2}+b^{2}
\end{equation}
is a scalar.
\begin{equation}
  \begin{split}
    Z \overline{Z} &= \mat{rr}{a&-b\\b&a}\mat{rr}{a&b\\-b&a}\\
      &= \paren{\svd{T}}\paren{\X{}\,\sig{T}\,\Y{T}}\\
      &= \brac{\paren{\frac{1}{r}\mat{rr}{a&-b\\b&a}}\,\paren{r\I{2}}\,\I{2}} \brac{\paren{\frac{1}{r}\mat{rr}{a&b\\-b&a}}\,\paren{r\I{2}}\,\I{2}} \\
      &= \mat{rr}{a^{2}+b^{2}&0\\0&a^{2}+b^{2}}.
  \end{split}
\end{equation}

%%
\subsubsection{Matrix exponential}

%%
\subsubsection{Linear fractional transformations}


\endinput