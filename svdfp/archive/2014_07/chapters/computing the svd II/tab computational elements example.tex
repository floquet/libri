\clearpage
\thispagestyle{empty}
\begin{landscape}

%%
%\arrayrulecolor{blue}
\begin{table}[htdp]
\caption[An example of the computational elements of the \asvd]{An example of the computational elements of the \asvd. This is table \eqref{tab:computational elements} for the example matrix (c). In the instance both \ns s are trivial and the truncated SVD is the same as the full SVD.}
\begin{center}
\begin{tabular}{ccccccrcl}
%
  $\matrixc$   & $\xrightarrow[]{ \phantom{1} 1 \phantom{1} }$ 
& $\wxc$ & $\xrightarrow[]{ \phantom{1} 2 \phantom{1} }$ 
& $\sigma = \sqrt{ \lst{8, 2} } $ & $\xrightarrow[]{ \phantom{1} 3 \phantom{1} }$
& $\ess{}$ &=& $\sigmac$ \\[20pt]
%
&&&& {\scriptsize{4}}\ $\downarrow$\\[15pt]
%
&&&& $\wxc\, \bvk = \sigma_{k} \bvk$ & $\xrightarrow[]{ \phantom{1} 5 \phantom{1} }$ 
& $\bvr{}$ &=& $\matrixcX $ \\[20pt]
%
&&&& {\scriptsize{6}}\ $\downarrow$ \\[10pt]
%
&&&& $\buk = \sigma_{k}^{-1}\matrixc\, \bvk$ & $\xrightarrow[]{ \phantom{1} 7 \phantom{1} }$ 
& $\bur{}$ & = & $\matrixcY $ \\[15pt]
%
&&&&&&& {\small{8}}\ $\downarrow$\\[5pt]
%
&&&&&& $\A{}$  &=& $\svdecompc$ 
%
\end{tabular}
\end{center}
\label{tab:computational elements example}
\end{table}
%\arrayrulecolor{black}

\end{landscape}


\endinput