\section{The breadth and depth of the SVD}
This shit is not only useful, it is cool. If we have an SVD we can do a lot of other things.

%%
\subsection{Polar decomposition}
The \index{polar factorization}\textit{polar factorization} is another useful matrix decomposition. It is analogous to finding the polar form of a complex number $z$ into the form
\begin{equation}
  z = r e^{i \theta}.
\end{equation}
This form resolves the complex number into a length $r$ and a direction $e^{i \theta}$. 

For the matrix form of the polar decomposition a target matrix $\A{}$ is resolved into the product of a unitary matrix $\Q{}$ and an Hermitian matrix $\mathbf{P}$ that is also positive semi-definite. Every matrix
$$
\A{}\in\cmplx{\by{n}{n}}_{\rho}
$$
has a polar decomposition, even when $\rho<n$.
The decomposition takes the form
\begin{equation}
  \A{} = \Q{}\mathbf{P}.
\end{equation}

The task here is to create the polar decomposition from the \svdl. The common trick is to insert a form of the identity matrix. In this case we will use
\begin{equation}
  \I{n} = \X{*}\X{}.
\end{equation}
The chain of reasoning is 
\begin{equation}
  \begin{split}
    \svdax{*} &= \Y{}\I{n}\sig{}\X{*} \\
      &= \Y{}\paren{\X{*}\X{}}\sig{}\X{*} \\
      &= \underbrace{\paren{\Y{}\X{*}}}_{\Q{}}\underbrace{\paren{\X{}\sig{}\X{*}}}_{\mathbf{P}}.
  \end{split}
\end{equation}
The polar decomposition can be expressed in terms of the \svdl with the assignemnts
\begin{equation}
  \begin{split}
    \Q{} &= \Y{}\X{*},\\
    \mathbf{P} &= \X{}\sig{}\X{*}.
  \end{split}
\end{equation}

Is the $\Q{}$ matrix unitary? We must demonstrate that $\Q{*}=\Q{-1}$.
\begin{equation}
  \Q{*} = \paren{\Y{}\X{*}}^{*} = \X{}\Y{*}.
\end{equation}
\begin{equation}
  \Q{-1} = \paren{\Y{}\X{*}}^{\mathrm{-1}} = \paren{\X{*}}^{\mathrm{-1}}\paren{\Y{}}^{\mathrm{-1}} = \X{}\Y{*}.
\end{equation}


Is the $\mathbf{P}$ matrix unitary? Does $\mathbf{P}^{*}=\mathbf{P}^{-1}$. The left-hand side is the most direct as it relies only upon the transpose of a product:
\begin{equation}
  \mathbf{P}^{*} = \paren{\X{}\sig{}\X{*}}^{*} = \X{*}\sig{T}\X{}.
\end{equation}
The right-hand-side uses both the transpose and inverse of a product:
\begin{equation}
  \Q{-1} = \paren{\Y{}\X{*}}^{\mathrm{-1}} = \paren{\X{*}}^{\mathrm{-1}}\paren{\Y{}}^{\mathrm{-1}} = \X{}\Y{*}.
\end{equation}



\endinput