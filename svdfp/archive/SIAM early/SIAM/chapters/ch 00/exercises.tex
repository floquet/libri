\section{Exercises}
\begin{enumerate}
\item Show that the dot product is commutative. That is, for conformable vectors $u$ an $v$ show that
\begin{equation}
  u\cdot v=v \cdot u.
\end{equation}
\item Reinforce the view of a matrix as a collection of vectors and verify the transpose formula \eqref{eq:mattran}. Start with component equation \eqref{eq:dot} and write the form for the components of the transpose:
\begin{equation}
  \begin{array}{rcccc}
    \C{}_{r,c} &=& \A{}_{r,*}&\cdot&\B{}_{*,c} \\
    \therefore\ \C{}_{c,r} &=& \A{}_{c,*}&\cdot&\B{}_{*,r} \\
  \end{array}
\end{equation}
Show that
\begin{equation}
  \A{}_{c,*}\cdot\B{}_{*,r} = \B{T}_{r,*}\cdot\A{T}_{*,c}
\end{equation}
and explain in a short sentence how this validates \eqref{eq:mattran}.
\item Rank one matrices are constructed most naturally from the unit vectors $\paren{e_{k}}_{j}$ which is the $k$th column of the identity matrix $\I{j}$. We can express the outer product in matrix form
\begin{equation}
  e_{r}\otimes e_{c} = e_{r}e_{c}^{\mathrm{T}} = \M{}
\end{equation}
where the $\by{m}{n}$ matrix has entries given by
\begin{equation}
  \M{}_{\mu,\nu} = \delta_{r}^{\mu}\delta_{c}^{\nu}.
\end{equation}
\subitem Given the expression
\begin{equation}
  \mat{ccrr}
  {
  i & 1 & 3 & -1\\
  1-i & 1 + i & -1 & -1
  }
  -
  \mat{cccc}
  {
  0 & 0 & 1 & 0\\
  0 & 0 & 0 & 0
  }
\end{equation}
write down the values $\lst{r,c}$ so that the above may be cast as
\begin{equation}
  \A{} - e_{r}\otimes e_{c}.
\end{equation}
\subitem Write out the vectors $e_{r}$ and $e_{c}$.
\subitem Answer: The first vector must have length 2; the second vector must have length 4. The index values are these $\lst{r,c} = \lst{1,3}$
\begin{equation*}
  \paren{e_{1}}_{2} =\mat{c}{1\\0}, \quad \paren{e_{3}}_{4}= \mat{c}{0\\0\\1\\0}
\end{equation*}
\item An important and useful way to verify your command of 	matrix products is to verify the formula
\begin{equation}
 \paren{\I{n}+\alpha \beta^{T}}^{-1} = \I{n} - \frac{\alpha \beta^{T}}{1+\beta^{T}\alpha}.
 \label{eq:1:rou}
\end{equation}
Here $\I{n}$ is the $\by{n}{n}$ identity matrix and the column vectors
\begin{equation}
  \alpha, \beta \in \cmplx{\by{n}{1}}
\end{equation}
have the restriction $\beta^{T}\alpha\ne-1$.
Two different ways to verify equation \eqref{eq:1:rou} involve direct multiplication. Verify either
\begin{equation}
  \begin{split}
    \paren{\I{n}+\alpha \beta^{T}}\paren{\I{n} - \frac{\alpha \beta^{T}}{1+\beta^{T}\alpha}} &= \I{n},\\
    \paren{\I{n} - \frac{\alpha \beta^{T}}{1+\beta^{T}\alpha}}\paren{\I{n}+\alpha \beta^{T}} &= \I{n}.
  \end{split}
\end{equation}
The significance of this equation is profound. It hints that one can express the inverse of a perturbed matrix in terms of the original inverse and another perturbation. This means that small changes to a large system will not require the expensive computation of a new inverse. Instead, we just need to compute a rank one update matrix.

Suppose you have a circuit with a thousand components. The system matrix that must be inverted has $1,000^{2}=1,000,000$ elements. If a single resistor is changed there is no need to invert a new megamatrix; just compute an update.
\item A more complicated problem is to check the \textit{Sherman-Morrison} formula for updating regular matrices. The formula is
\begin{equation}
  \paren{\A{}+\alpha \beta^{T}}^{-1} = \A{-1}\paren{1+\frac{\alpha \beta^{T}\A{-1}}{1+\beta^{T}\A{-1}\alpha}}.
\end{equation}
Verify this inverse. You may need to use the fact that scalars commute with matrices:
\begin{equation}
  \alpha \beta^{T}\alpha \beta^{T}=\alpha \paren{\beta^{T}\alpha} \beta^{T}=\paren{\beta^{T}\alpha}\alpha \beta^{T}.
\end{equation}
In order to assure that the outer product attains rank one, what restrictions must be placed on $\alpha$ and $\beta$?
%%
\item A matrix has domain $\real{2}$ and has full column rank yet no row rank deficiency. Specify the matrix in terms of $\A{\byt{m}{n}}_{\rho}$.\\
$\A{}\in\real{\by{2}{2}}_{2}$
%%
\item A matrix has domain $\real{4}$ and has a row rank deficiency of 2. Specify the matrix in terms of $\A{\byt{m}{n}}_{\rho}$.\\
$\A{}\in\real{m\times 4}_{2}$
%%
\item A matrix has codomain $\real{3}$ and has a row rank deficiency of 2. Specify the matrix in terms of $\A{\byt{m}{n}}_{\rho}$.\\
$\A{}\in\real{3\times n}_{n-2}$
%%
\item A matrix has rank 2, a column rank deficiency of 3 and no row rank deficiency. Specify the matrix in terms of $\A{\byt{m}{n}}_{\rho}$.\\
$\A{}\in\real{2\times 5}_{2}$
%%
\item A matrix has has twice as many rows as columns and the matrix rank is twice the column rank deficiency. Specify the matrix in terms of $\A{\byt{m}{n}}_{\rho}$.\\
$\A{}\in\real{6k\times 3k}_{2k}, \quad k\in\mathbb{N}$
%%
\item A matrix has has twice as many rows as columns and the row rank deficiency is half the column rank deficiency. Specify the matrix in terms of $\A{\byt{m}{n}}_{\rho}$.\\
This implies $\rho=0$; no such matrix exists.

\end{enumerate}


\endinput