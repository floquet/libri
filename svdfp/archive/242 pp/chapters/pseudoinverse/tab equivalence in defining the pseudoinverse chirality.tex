\begin{table}[htdp]
\caption[Equivalence in defining pseudoinverse chirality]{Equivalence in defining pseudoinverse chirality.}
\begin{center}
\begin{tabular}{cccccc}
%
  \raisebox{-0.4\height}{ \includegraphics[ width = 0.45in ]{images/pseudoinverse/"transparent right"} } &
  $\ApA = \I{n}$ & $\mathbf{\iff}$ & $\rvn{} = \trivial$ & $\iff$ & $\rho = n$ \\
%
  \raisebox{-0.4\height}{ \includegraphics[ width = 0.45in ]{images/pseudoinverse/"transparent left"} } &
  $\AAp = \I{m}$ & $\mathbf{\iff}$ & $\run{} = \trivial$ & $\iff$ & $\rho = m$
%
\end{tabular}
\end{center}
\label{tab:pseudoinverse:equivalent}
\end{table}%


\endinput