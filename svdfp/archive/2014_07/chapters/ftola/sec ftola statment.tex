\section{Theorem Statement}

Consider a matrix of scalars $\A{}$ with $m$ rows and $n$ columns. A scalar is a number, either real or complex. This matrix $\aicmn$ and it induces two vector spaces: $\cmplxm$ and $\cmplxn$. Furthermore, these spaces have an orthogonal decomposition into range spaces and \ns s. This is the crux of the \ft.

In spaces with finite dimension one mathematical prescription for the \ft \ is this:
%%%%%%
\input{chapters/ftola/"tab ftola statement"}  % table

Here $\bl{\rng{\cdot}}$ specifies the span of the range and $\rd{\nll{\cdot}}$ the span of the \ns. Throughout this book we will follow this coloring scheme. Vectors in the range are blue, those in the \ns \ are red. The asterisk denotes the Hermitian conjugate of a matrix. That is,
%
\begin{equation}
  \A{*} = \overline{\A{T}} = \tra{\overline{\A{}}}.
\end{equation}
%
Note that the asterisk only appears once in each decomposition.
For example, the codomain $\cmplxn$ is resolved into a range space $\brnga{*}$ and its orthogonal complement $\rnlla{}$ and the symbol $\oplus$ signifies a direct sum.

In colloquial mathematics one may say that 
\begin{itemize}
%
\item the \emph{column} space or \emph{codomain} of $\A{}$ in the direct sum of the range space $\brnga{}$ and the \ns \ $\rnlla{*}$.
%
\item the \emph{row} space or \emph{domain} of $\A{}$ in the direct sum of the range space $\brnga{*}$ and the \ns \ $\rnlla{}$.
%
\end{itemize}
%
The range and null spaces are complimentary:
%
\begin{equation}
  \begin{split}
    \brnga{*} \, &\cap \, \rnlla{}  = \zero, \\
    \brnga{}  \, &\cap \, \rnlla{*} = \zero .
  \end{split}
\end{equation}
%
The range of $\A{}$ is also spanned by the columns of $\A{}$ 
%
\begin{equation}
  \brnga{} = \spn{{\bl{ a_{1} }},\dots,{\bl{ a_{n} }}}
\end{equation}
%
%

The range space is defined \cite[p. 140]{Meyer2000} 

%
\begin{equation}
  \brnga{} \quad = \quad \lst{\Ax \colon x \in \cmplxn } \quad \subseteq \quad \cmplxm
\end{equation}
%
\begin{equation}
  \brnga{*} \quad = \quad \lst{\A{*}y \colon y \in \cmplxm } \quad \subseteq \quad \cmplxn
\end{equation}
%
The null space can be defined \cite[p. 174]{Meyer2000} 
%
\begin{equation}
  \rnlla{*} \quad = \quad \lst{x \in \cmplxn \colon \Axez } \quad \subseteq \quad \cmplxm
\end{equation}
%
\begin{equation}
  \rnlla{} \quad = \quad \lst{y \in \cmplxm \colon \paren{ y^{*}\A{} }^{*} = \zero } \quad \subseteq \quad \cmplxn
\end{equation}


\endinput