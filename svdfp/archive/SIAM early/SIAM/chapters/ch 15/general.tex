\section{The general solution}

Let's resume the discussion with the issue posed in \S\eqref{sec:pi}. The issue was the general system to the linear system given by 
$$
\ls.
$$
By generalizing the matrix inverse we were able to solve the general problem. THe target matrix did not have to be square; it was no longer restricted to being nonsingular. This general solution was expressed in equation \eqref{eq:solution:general}:
$$
    \Ap\paren{\A{}x} = \A{+}b.
$$
We were troubled by the fact that the matrix product on the left-hand side is in general not an identity matrix:
\begin{equation}
  \Ap\A{} \ne \I{n}.
\end{equation}
(Later on in \S\eqref{sec:chiral} we saw that the pseudoinverse is a left inverse
\begin{equation}
  \Ap\A{} = \I{n}
\end{equation}
when the target matrix has full column rank.) We also went on to discover in \S\eqref{sec:solution:complete} that while $\A{+}b$ is the point solution, it is not the complete solution as seen, for example, in equation \eqref{eq:simple:fullsoln}.

Now armed with the knowledge of the four fundamental projectors from \S\eqref{sec:orthproj} we can bring this issue into full theoretical appreciation. In equation \eqref{eq:fourprojpi} we saw that the matrix product of interest is the operator which projects vectors onto the perpendicular complement of the of the null space of the target matrix:
\begin{equation}
  \projnap = \leftinv.
\end{equation}


\endinput