\section{Example: matrix (b)}
\label{sec:svd I:b}
The target matrix $\dimsb$. It has full column rank but is row rank deficient:
\begin{equation}
  \A{} = \matrixb.
\end{equation}
The \ns \ for the domain is trivial:
\begin{equation}
  \rnlla{} = \trivial.
\end{equation}
The \asvd \ takes the block form
\begin{equation}
  \aesvd{*} = \csvdblockbc{*}
\end{equation}


%%%%%%%%%%%%%%
%%%%%%%%%%%%%%
\subsubsection{Step 1: singular values}
The Hermitian conjugate and the product matrix are
\begin{equation}
  \A{*} = \matrixbt, \quad \wv = \wx{*} = \wxb.
\end{equation}
To compute the characteristic polynomial, we will use
\begin{equation}
  \begin{split}
    p\paren{\lambda} 
      &= \lambda^{2} - \lambda \text{tr} \paren{\wv} + \det \paren{\wv} \\
      &= \lambda^{2} - \frac{9}{8}\lambda + \frac{45}{256}
  \end{split}
\end{equation}
The roots of the characteristic equation are the eigenvalues:
\begin{equation}
  \lambda \paren{\wv} = \lst{ \frac{15}{16}, \frac{3}{16} }. 
\end{equation}
The singular values are
\begin{equation}
  \sigma = \lst{ \frac{\sqrt{15}}{4}, \frac{\sqrt{3}}{4} }.
\end{equation}
The matrix of singular values is
\begin{equation}
  \ess{} = \text{diagonal} \lst{\sigma} = \essb
\end{equation}
and the $\sig{}$ matrix becomes
\begin{equation}
  \sig{} = \sbc{} = \sigmab.
\end{equation}


%%%%%%%%%%%%%%
%%%%%%%%%%%%%%
\subsubsection{Step 2: $\brnga{*}$}
Resolve the domain by computing the eigenvectors in the same order as the eigenvalues. Since we use the normalized form of the eigenvectors, we can ignore scaling factor of $1/16$.
\begin{equation}
  \begin{split}
    \paren{\wv - \lambda_{1}\I{2}}\bvo = \mat{cc}{9-15 & 6 \\ 6 & 9-15}\bvo = \mat{rr}{-6 & 6 \\ 6 & -6}\bvo = \zerotwo
  \end{split}
\end{equation}
\begin{equation}
  \bvo = \vecbxa.
\end{equation}
%
\begin{equation}
  \begin{split}
    \paren{\wv - \lambda_{2}\I{2}}\bvo = \mat{cc}{9-3 & 6 \\ 6 & 9-3}\bvt = \mat{cc}{6 & 6 \\ 6 & 6}\bvt = \zerotwo
  \end{split}
\end{equation}
\begin{equation}
  \bvt = \vecbxb.
\end{equation}
Of course the negative version of the vectors would work as well.

\begin{equation}
  \V{} = \bvr{} = \mat{cc}{ \vecbxa & \vecbxb }.
  \label{eq:b:v}
\end{equation}


%%%%%%%%%%%%%%
%%%%%%%%%%%%%%
\subsubsection{Step 3: $\brnga{}$}
We resolve the range space for the codomain by direct contraction. The \mv s are scaled images of \nv s:
\begin{equation}
  \begin{split}
    \A{}\,\bvo       &= \sigma_{1} \buo \\
    \matrixb \vecbxa &= \frac{\sqrt{15}}{4} \paren{\buo}  
  \end{split}
\end{equation}
from which we conclude
\begin{equation}
  \buo = \obvecbm.
\end{equation}
%
We resolve the range space for the codomain by direct contraction. The \mv s are scaled images of \nv s:
\begin{equation}
  \begin{split}
    \A{}\,\bvt       &= \sigma_{2} \buo \\
    \matrixb \vecbxb &= \frac{\sqrt{3}}{4} \paren{\buo}  
  \end{split}
\end{equation}
from which we conclude
\begin{equation}
  \but = \obvecbn.
\end{equation}
We can complete the range space components:
\begin{equation}
  \ur{} = \mat{c|c}{ \obvecbm & \obvecbn }.
\end{equation}


%%%%%%%%%%%%%%
%%%%%%%%%%%%%%
\subsubsection{Step 4: $\rnlla{*}$, $\rnlla{}$}
Because this matrix has full column rank, the \ns \ $\rnlla{}$ is trivial. This leaves the \ns \ $\rnlla{*}$. Using the rank plus nullity theorem, we conclude that there is only one vector in this space. One way to construct this vector is to use the cross product
\begin{equation}
  a \times b = c
\end{equation}
\begin{equation}
  \mat{c}{ a_{1} \\ a_{2} \\ a_{3} } \times 
  \mat{c}{ b_{1} \\ b_{2} \\ b_{3} } =
  \mat{c}{ a_{2}b_{3} - b_{3}a_{2} \\ 
           a_{3}b_{1} - b_{1}a_{3} \\ 
           a_{1}b_{2} - b_{2}a_{1} }
\end{equation}
\begin{equation}
  \buo \times \but = {\rd{ u_{3} }}
\end{equation}
\begin{equation}
  \begin{split}
    \buo \times \but 
      &= {\rd{ \tilde{u}_{3} }}, \\
    \obvecbm \times \obvecbn
      &= \recip{2\sqrt{45}} 
         \mat{r}{ 10 + 2 \\ 2 - 2 \\ -1 - 5 } \\
      &= \recip{2\sqrt{45}} 
         \mat{r}{ 12 \\ 0 \\ -6 }
  \end{split}
\end{equation}
The normalized vector is
\begin{equation}
  {\rd{ u_{3} }} = \orvecbo
\end{equation}
The codomain is resolved into a complete orthonormal basis
\begin{equation}
  \U{} = \ublockf{} = \mat{c|c|c}{ \obvecbm & \obvecbn & \orvecbo }.
\end{equation}

\begin{equation}
  \aesvdecompb
\end{equation}

%%%%%%%%%%%%%%
%%%%%%%%%%%%%%
\subsubsection{Observations}

\endinput