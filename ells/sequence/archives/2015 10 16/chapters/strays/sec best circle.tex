\section{\label{sec:best circle}Finding the best circle}
This is an example of an unnumbered first-level heading.

%\input{\pathchapter XXX/"ssec YYY"}  %  <  <  <  <  <  <  <  <  <  <  <  <  <  <  <

\subsection{Nonlinear formulation}  %  SS  SS  SS  SS  SS
This is an example of a numbered first-level heading.

A circle is characterized by two parameters: an origin and a radius. The origin is a vector quantity, the radius a scalar.
  % = =  e q u a t i o n
  \begin{equation}
    \org = \mat{c}{x_{0} \\ y_{0}}
    %\label{eqn:}
  \end{equation}
  % = =
Given a set of measurements $p_{j}$, $j=1\colon m$.
  % = =  e q u a t i o n
  \begin{equation}
    \paren{ x - x_{0} }^{2} - \paren{ y - y_{0} }^{2} = \rho^{2}
    \label{eqn:nonlinear trial}
  \end{equation}
  % = =

This implies a trial function
  % = =  e q u a t i o n
  \begin{equation}
    \chi^{2}\paren{\org,\rho} = \sum_{j=1}^{m} \paren{\rho^{2} - \paren{ x_{j} - x_{0} }^{2} + \paren{ y_{j} - y_{0} }^{2}}^{2}
    %\label{eqn:}
  \end{equation}
  % = =

In equation \eqref{eqn:nonlinear trial} the fit parameters for the origin appear in a nonlinear fashion, making this a nonlinear problem. There are many ways to solve such a problem. However, our focus is on linear problems.

\subsection{Linear formulation}  %  SS  SS  SS  SS  SS
We start with the simple vector equation
  % = =  e q u a t i o n
  \begin{equation}
    p_{j} = r_{k} + \org 
    %\label{eqn:}
  \end{equation}
  % = =
from which we conclude
   % = =  e q u a t i o n
  \begin{equation}
    p_{j}^{2} = r_{j}^{2} + \org^{2} + 2 r_{j} \cdot \org 
    %\label{eqn:}
  \end{equation}
  % = =
The trick is make one parameter disappear. To do so examine differences between the measurements
  % = =  e q u a t i o n
  \begin{equation}
    \Delta_{jk} = p_{j} - p_{k} = r_{j} - r_{k} 
    \label{eqn:difference}
  \end{equation}
  % = =
The data is no longer a list of $m$ measurements of $p$; instead it is a list of $\tau$ differences where 
  % = =  e q u a t i o n
  \begin{equation}
    \tau = \half m(m-1)
    %\label{eqn:}
  \end{equation}
  % = =
For example, when $m=4$
\begin{table}[htdp]
\caption{The new data set compared to the old. The measured values $p$ are converted to a set of differences $\Delta_{jk}$.}
\begin{center}
  \begin{tabular}{rcc}
    %
    & measurements & inputs\\\hline
    %
    1 & $p_{1}$ & $\Delta_{12} = p_{1} - p_{2}$ \\
    %
    2 & $p_{2}$ & $\Delta_{13} = p_{1} - p_{3}$ \\
    %
    3 & $p_{3}$ & $\Delta_{14} = p_{1} - p_{4}$ \\
    %
    4 & $p_{4}$ & $\Delta_{23} = p_{2} - p_{3}$ \\
    %
    5 &         & $\Delta_{24} = p_{2} - p_{4}$ \\
    %
    6 &         & $\Delta_{34} = p_{3} - p_{4}$
  \end{tabular}
\end{center}
\label{tab:p's}
\end{table}%

  % = =  e q u a t i o n
  \begin{equation}
    p_{j}^{2} - p_{k}^{2} = r_{j}^{2} - r_{k}^{2} + 2 \paren{r_{j} - r_{k}} \cdot \org
    %\label{eqn:}
  \end{equation}
  % = =
  % = =  e q u a t i o n
  \begin{equation}
    r_{j}^{2} = \rho^{2} \qquad j = 1\colon m
    %\label{eqn:}
  \end{equation}
  % = =
  % = =  e q u a t i o n
  \begin{equation}
    r_{j}^{2} - r_{k}^{2} = 0 \qquad j,k = 1\colon m
    %\label{eqn:}
  \end{equation}
  % = =
The final trial function is this using equation \eqref{eqn:difference}
  % = =  e q u a t i o n
  \begin{equation}
    p_{j}^{2} - p_{k}^{2} = 2 \paren{p_{j} - p_{k}} \cdot \org
    %\label{eqn:}
  \end{equation}
  % = =
The trial function is then
  % = =  e q u a t i o n
  \begin{equation}
    p_{j}^{2} - p_{k}^{2} = 2 \paren{p_{j} - p_{k}} \cdot \org
    %\label{eqn:}
  \end{equation}
  % = =
and the merit function
  % = =  e q u a t i o n
  \begin{equation}
    \merit{\org} = \sum_{j=1}^{m-1} \sum_{k=1}^{m} \sq{p_{j}^{2} - p_{k}^{2} - 2 \paren{p_{j} - p_{k}} \cdot \org}
    %\label{eqn:}
  \end{equation}
  % = =
Label the pairs
  % = =  e q u a t i o n
  \begin{equation}
    \xi = \lst{ \mat{c}{1\\2}, \mat{c}{1\\3}, \dots, \mat{c}{m-1\\m} }
    %\label{eqn:}
  \end{equation}
  % = =
  % = =  e q u a t i o n
  \begin{equation}
    \merit{\org} = \sum_{\mu=1}^{\tau} \sq{2\Delta_{\xi} \cdot \org -p_{\xi_{1}}^{2} + p_{\xi_{2}}^{2}}
    %\label{eqn:}
  \end{equation}
  % = =
Linear system
\begin{equation}
  \begin{split}
    p_{1}^{2} - p_{2}^{2} &= 2\paren{p_{1}-p_{2}}\cdot \org \\ 
    p_{1}^{2} - p_{3}^{2} &= 2\paren{p_{1}-p_{3}}\cdot \org \\ 
    &  \ \, \vdots \\
    p_{m-1}^{2} - p_{m}^{2} &= 2\paren{p_{m-1}-p_{m}}\cdot \org
  \end{split}
\end{equation}
solve for the origin $\org$.
The problem statement
  % = =  e q u a t i o n
  \begin{equation}
    \Delta \org = b
    %\label{eqn:}
  \end{equation}
  % = =
In $d$ dimensions the matrix dimensions are
$$\Delta \in\real{\tau \times d}_{d}, \quad \org   \in\real{d \times 1},  \quad b      \in\real{\tau \times 1}$$
and the matrices are defined as
  % = =  e q u a t i o n
  \begin{equation}
    \Delta = 2\mat{c}{p_{1} - p_{2} \\ \vdots \\p_{m-1} - p_{m}}, \quad
    \org   = \mat{c}{x_{1} \\ \vdots \\x_{d}}, \quad
    b      = \mat{c}{p_{1}^{2} - p_{2}^{2} \\ \vdots \\p_{m-1}^{2} - p_{m}^{2}}
    %\label{eqn:}
  \end{equation}
  % = =


\endinput  %  -  -  -  -  -  -  -  -  -  -  -  -  -  -  -  -  -  -  -  -