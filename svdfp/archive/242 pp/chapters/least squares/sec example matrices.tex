\section{Example matrices}

%%%%%%%%%%%
\subsection{Example matrix (a)}
%
\begin{equation}
  \begin{split}
    \A{} x &= b \\
    \matrixa \xthree &= \dataa
  \end{split}
\end{equation}
%
The full least squares solution\index{least squares!solutions!example (a)} is
%
\begin{equation}
  x_{LS} = \xthree = \xlsa + \alpha_{1} \nllao  + \alpha_{2} \nllat, \quad \alpha\ic.
\end{equation}
%
The pseudoinverse or point solution is
%
\begin{equation}
  \xmp = \xlsa
\end{equation}
%
\begin{equation}
  \A{}x_{LS} - b = \dataar
\end{equation}
%
Orthogonal resolution of the data vector
%
\begin{equation}
  \begin{split}
     b &= \datarange + \datanull \\
     \dataa &= {\bl{ \half \mat{r}{ 1 \\ -1 } }} + {\rd{ \frac{3}{2}\mat{c}{1\\1} }}
  \end{split}
\end{equation}
%
Consistent linear system
\begin{equation}
  \begin{split}
    \A{}\, \xbl &= \datarange, \\
    \matrixa \xlsa &= {\bl{ \half \mat{r}{ 1 \\ -1 } }} .
  \end{split}
\end{equation}



%%%%%%%%%%%
\subsection{Example matrix (b)}
\begin{equation}
  \begin{split}
    \A{} x &= b \\
    \matrixb \xtwo &= \datab
  \end{split}
\end{equation}
%
The least squares solution is
%
\begin{equation}
  x_{LS} = \xtwo = \xlsb + \alpha \nllb, \quad \alpha\ic
\end{equation}
%
The pseudoinverse or point solution is
%
\begin{equation}
  \xmp = \xlsb
\end{equation}
%
\begin{equation}
  \A{}x_{LS} - b = \databr
\end{equation}
%
Orthogonal resolution of the data vector
%
\begin{equation}
  \begin{split}
     b &= \datarange + \datanull \\
     \datab &= {\bl{ \recip{5} \mat{c}{ 2 \\ 5 \\ 4 } }} + {\rd{ \recip{5} \mat{r}{ 8 \\ 0 \\ -4 } }}
  \end{split}
\end{equation}
%
\begin{equation}
  \U{} = \matrixbY
\end{equation}

%
\begin{equation}
  \databt \cdot \obbuo = \frac{7}{\sqrt{30}}, \quad \databt \cdot \obbut = \recip{\sqrt{6}}
\end{equation}
%
\begin{equation}
  \frac{7}{\sqrt{30}} \obbuo + \recip{\sqrt{6}} \obbut = {\bl{ \recip{5} \mat{c}{ 2 \\ 5 \\ 4 } }}
\end{equation}


%
Consistent linear system
\begin{equation}
  \begin{split}
    \A{}\, \xbl &= \datarange, \\
    \matrixb \xlsb &= {\bl{ \recip{5} \mat{c}{ 2 \\ 5 \\ 4 } }} .
  \end{split}
\end{equation}

%%%%%%%%%%%
\subsection{Example matrix (c)}
\begin{equation}
  \begin{split}
    \A{} x &= b \\
    \matrixc \xtwo &= \datac
  \end{split}
\end{equation}
	
\begin{equation}
  x_{LS} = \xtwo = \xlsc
\end{equation}

\begin{equation}
  \A{}x_{LS} - b = \datacr
\end{equation}


\endinput