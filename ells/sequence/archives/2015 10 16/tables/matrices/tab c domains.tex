%    Figure insertion; default placement is top; if the figure occupies
%    more than 75% of a page, the [p] option should be specified.
\begin{table}[t]
	\begin{center}
		\begin{tabular}{ccc}
		  %
		  \textbf{Solution space} & \textbf{Measurement space} \\
		  result: $\xls = \mat{c}{1\\6}$ & data: $b = \mat{c}{2\\1}$\\[15pt]
		  %
		  \includegraphics[ width = 2.25in ]{\pathgraphics matrices/"c domain"} & \qquad
		  %
		  \includegraphics[ width = 2.25in ]{\pathgraphics matrices/"c codomain"} \\
		  %
		  Domain & Codomain \\[5pt]
		  %
		  $\V{} \in \cmplx{n}$ & $\U{} \in \cmplx{m}$ \\[10pt]
		  %
		\end{tabular}
	\end{center}
	%\label{tab:}
	\caption{The domain and the codomain for example matrix c. Both null spaces, $\rnlla{}$ and $\rnlla{*}$ are trivial. The unit circles show the SVD decomposition: the column vectors of $\V{}$ on the left and $\U{}$ on the right. The green shading indicates a right-handed coordinate system, the red a left-handed system. The matrix $\A{}$ maps the solution to the data: $\A{}\xls=b$.}
\end{table}%

\endinput  %  -  -  -  -  -  -  -  -  -  -  -  -  -  -  -  -  -  -  -  -

%\input{\pathtables "tab XXX"}  %  <  <  <  <  <  <  <  <  <  <  <  <