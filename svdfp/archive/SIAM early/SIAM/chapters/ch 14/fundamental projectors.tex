\section[The fundamental projections]{The four fundamental orthogonal projections}
\label{sec:orthproj}

One of the bonuses we reap from the pseudoinverse is the four fundamental orthogonal projections. Each of these matrices projects onto a specific vector space. The spaces are these:
\begin{enumerate}
\item the range $\rng{\A{}}$,
\item the orthogonal complement to the range, $\rng{\A{}}^{\perp}$,
\item the null space $\nll{\A{}}$,
\item the orthogonal complement to the null space, $\nll{\A{}}^{\perp}$.
\end{enumerate}

We can make a simple yet powerful observation about the projectors.
Given the $\pee{}_{\mathcal{V}}$, the projector onto an $m-$dimensional subspace $\mathcal{V}$, the projector onto the orthogonal complement is
\begin{equation}
  \pee{}_{\mathcal{V}^{perp}} = \I{m}-\pee{}_{\mathcal{V}}.
\end{equation}

Below are these projectors for a matrix $\Accmn_{\rho}$ with pseudoinverse $\Ap$:
\begin{equation}
\boxed{
  \begin{array}{lcllcl}
    \projra& = &\rightinv & \quad \projrap& = &\I{m}-\rightinv\\
    \projnap& =& \leftinv & \quad \projna& = &\I{n}-\leftinv\\    
  \end{array}
  }
  \label{eq:fourprojpi}
\end{equation}

%%
\subsection{Projection onto the range of $\A{}$}
Let's explore the matrix product
\begin{equation}
  \begin{split}
    \projra &= \rightinv,\\
    &= \Y{}\,\sig{}\,\sig{(+)}\Y{T},\\
    & = \Y{}\,\J{m}{\rho}\,\Y{*}, \\
    & = \pra{*}.
  \end{split}
\end{equation}
The truncated identity is acting like screen which only allows the first $\rho$ column vectors of $\Y{}$ into play. These are the vectors which span the image of the target matrix.

%%
\subsection{Projection onto the orthogonal complement\\ of the range of $\A{}$}
Now subtract this projector from the identity:
\begin{equation}
  \begin{split}
    \projrap& = \I{m}-\projra,\\
      &= \I{m} - \rightinv,\\
      &= \I{m} - \pra{*}.
  \end{split}
\end{equation}

%%
\subsection{Projection onto the null space of $\A{}$}
\begin{equation}
  \begin{split}
    \projna &= \leftinv,\\
    &= \X{}\,\sig{+}\,\sig{}\X{*},\\
    & = \X{}\,\J{n}{\rho}\,\X{*},\\
    & = \pnap{*}.
  \end{split}
\end{equation}The truncated identity is acting like screen which only allows 
%%
\subsection{Projection onto the orthogonal complement\\ of the null space of $\A{}$}
Let's explore the matrix product
\begin{equation}
  \begin{split}
    \projnap& = \I{n}-\projna,\\
      &= \I{n}-\leftinv,\\
      &= \I{n} - \pnap{*}.
  \end{split}
\end{equation}

Below are the same projectors shown in \eqref{eq:fourprojpi} for the same matrix $\Accmn_{\rho}$ with \svdl\ $\svd{*}$:
\begin{equation}
\boxed{
  \begin{array}{lcllcl}
    \projra&  = &\pra{*}  & \qquad  \projrap & = & \I{m}-\pra{*}\\[3pt]
    \projnap& = &\pnap{*} & \qquad  \projna  & = & \I{n}-\pnap{*}\\    
  \end{array}
  }
  \label{eq:fourprojsvd}
\end{equation}

%%%
%%%
\subsection{Primitives}
Consider the following simple matrix:
\begin{equation}
  \A{}= \mat{cc}{1&0\\0&1\\0&0}.
\end{equation}
This matrix is a map from $\cmplx{2}$ to $\cmplx{3}$. This means that the projection onto $\rng{\A{}}$ and the projection onto the orthogonal complement $\nll{\A{*}}$ will be matrices of dimension $\byy{3}$. The projection onto $\rng{\A{*}}$ and the projection onto the orthogonal complement $\nll{\A{}}$ will be matrices of dimension $\byy{2}$.

We can discern the primitives by inspection. Consider the action of this matrix on a \vvv:
\begin{equation}
  \A{}x = \mat{cc}{1&0\\0&1\\0&0}
          \mat{c}{x \\ y}
        = \mat{c}{x \\ y \\ 0}.
\end{equation}
No matter vector $x$ we present to $\A{}$, the result will be in the $x-y$ plane where $z=0$.
The range of $\A{}$ is the span of the unit vectors $e_{1}^{(3)}$ and $e_{2}^{(3)}$:
\begin{equation}
  \rnga{T} = \spn \lst{\xhattt,\yhattt}
\end{equation}
The orthogonal complement to the $x-y$ plane in $\cmplx{3}$ is the $z-$axis, or the span
\begin{equation*}
  \spn \lst{\mat{c}{0 \\ 0 \\ 1}}.
\end{equation*}
Therefore, we know the first two projectors immediately:
\begin{equation}
  \projra = \mat{ccc}{1&0&0\\0&1&0\\0&0&0}, \quad \projnat = \mat{ccc}{0&0&0\\0&0&0\\0&0&1}.
\end{equation}

For the next two projectors we need to look at the transpose matrix:
\begin{equation}
  \A{T}y = \mat{ccc}{1&0&0\\0&1&0}.
\end{equation}
We see that the range of this matrix is given by this
\begin{equation}
  \rnga{T} = \spn \lst{\xhatt,\yhatt} = \cmplx{2}
\end{equation}
Since the range is the entire space $\cmplx{2}$, there is no orthogonal complement.
The last two projectors are then these:
\begin{equation}
  \projrat = \mat{ccc}{1&0\\0&1}, \quad \projna = \mat{cc}{0&0\\0&0}.
\end{equation}
The moral is that many matrices can be interpreted quickly without recourse to formal computation.

To complete the example, let's show the comptation. Using SVD by inspection we write
\begin{equation}
\begin{split}
  \svda{T},\\
  &= \I{3}\, \A{}\, \I{2},\\
  &= \mat{cc>{\columncolor{ltgray}}c}{1&0&0\\0&1&0\\0&0&1} \mat{cc}{1&0\\0&1\\\hline0&0} \itwo.
\end{split}
\end{equation}
The pseudoinverse is then
\begin{equation}
  \begin{split}
     \mpgia{T}, \\
     & = \itwo \mat{ccc}{1&0&0\\0&1&0} \mat{ccc}{1&0&0\\0&1&0\\\rowcolor{ltgray}0&0&1},\\
     & = \mat{ccc}{1&0&0\\0&1&0}.
  \end{split}
\end{equation}

The projection operators are these:
\begin{equation}
  \begin{array}{lcllcl}
    \projra& = &\rightinv = \mat{ccc}{1&0&0\\0&1&0\\0&0&0} & \quad \projnap& = &\I{3}-\projra = \mat{ccc}{0&0&0\\0&0&0\\0&0&1}\\[10pt]
    \projrap& =& \leftinv = \mat{ccc}{1&0\\0&1} & \quad \projna& = &\I{2}-\projnap = \mat{cc}{0&0\\0&0}\\    
  \end{array}
  \label{eq:primitives}
\end{equation}

%%
\subsection{Examples}
To solidify the theory work through these examples. They will show both formulations of the four fundamental projectors: using the pseudoinverse and using the \svdl.

%%
\subsubsection{$\Ap=\A{-1}$}
This class of matrices has full row and full column rank: $\rho=n=m$.

%%
\subsubsection{$\Ap=\AinvL$}
This class of matrices has full column rank: $\rho=n$, $m\ge n$.
%%
\subsubsection{$\Ap=\AinvR$}
This class of matrices has full row rank: $\rho=m$, $m\le n$.
%%
\subsubsection{$\Ap\ne\A{-1}_{L,R}$}
This class of matrices is deficient in both row and column rank: $\rho<m$, $\rho<n$.
%%
\textbf{Range projections and their complements}
For this example we need a matrix with both row and column rank deficiency. 
\begin{equation}
  \A{} = \Aexample, \quad \Ap = \Aexamplepi.
\end{equation}

The projection represents these two 
\begin{equation}
  \begin{split}
    \projra &= \rightinv = \rthree
    \mat{rrr}
    { 
    1 & -1 &  1\\
   -1 &  1 & -1\\
    1 & -1 &  1\\
    }\\[5pt]
    & = \pra{T} =
    \Yshade\,
    \mat{ccc}{
    1 & 0 & 0\\
    0 & 0 & 0\\
    0 & 0 & 0}
    \Ytshade\\
    & =\sthree
    \mat{r}
    {1\\-1\\1}
    \sthree
    \mat{ccc}
    {1 & -1 & 1}
    = \rthree\mat{rrr}
    { 1 & -1 &  1\\
     -1 &  1 & -1\\
      1 & -1 &  1}.
  \end{split}
\end{equation}

Tee projection onto the space perpendicular to the range is this
\begin{equation}
  \projrap = \I{3} - \rightinv = \rthree
    \mat{rrr}
    { 2 & -1 &  1\\
     -1 &  2 & -1\\
      1 & -1 &  2}.
\end{equation}

Given 
\begin{equation}
  \A{} = 
  \mat{ccc}
  {0 & 3 & 0\\
   1 & 2 & 2}
   , \quad \Ap = \rfive
   \mat{rr}
   {-2 & 3 \\ 5 & 0 \\ -4 & 6}.
\end{equation}

The projection represents these two 
\begin{equation}
  \begin{split}
    \projra &= \rightinv = \rthree
    \mat{rrr}
    { 
    1 & -1 &  1\\
   -1 &  1 & -1\\
    1 & -1 &  1\\
    }\\[5pt]
    & = \pra{T} =
    \Yshade\,
    \mat{ccc}{
    1 & 0 & 0\\
    0 & 0 & 0\\
    0 & 0 & 0}
    \Ytshade\\
    & =\sthree
    \mat{r}
    {1\\-1\\1}
    \mat{ccc}
    {1 & -1 & 1}
    = \rthree\mat{rrr}
    { 1 & -1 &  1\\
     -1 &  1 & -1\\
      1 & -1 &  1}.
  \end{split}
\end{equation}

Tee projection onto the space perpendicular to the range is this
\begin{equation}
  \projrap = \I{3} - \rightinv = \rthree
    \mat{rrr}
    { 2 & -1 &  1\\
     -1 &  2 & -1\\
      1 & -1 &  2}.
\end{equation}


\section{A survey of the four fundamental projectors}
In the following pages we will look at three different kinds of mappings for a matrix $\Acc{m}{n}$: no frustration, frustration in one direction, frustration in both directions.

There are two different ways to view frustrated\index{frustration} mappings:
\begin{enumerate}
\item geometric deficiency\index{geometric deficiency} - mapping into a lower dimensional object;
\item algebraic deficiency\index{algebraic deficiency} - rank deficiency in row or column.
\end{enumerate}

In the examples that follow we will see ellipses mapped into other ellipses. These mappings are not frustrated. But once we map into a lower dimensional object, say from the unit sphere onto a line, the map is frustrated. This also means that we can't reverse the map. We can't make a finite linear map from a line with one parameter onto a sphere with three parameters.

These tables specify critical properties of the target matrix.

\textbf{Plots: }All plots start with the unit circle which is either
\begin{equation}
  \begin{array}{rcll}
     S(\theta) &=& \mat{c}{\cos \theta\\\sin \theta},\ \theta\in[0,2\pi) \qquad &n=2,\\
     S(\theta,\phi) &=& \mat{c}{\cos \theta\sin \phi\\\sin \theta \sin \phi\\\cos \phi},\ \theta\in[0,2\pi),\ \phi\in[0,\pi), \qquad &n=3.
  \end{array}
\end{equation}
Then look at the mapping action of the matrix. The result is either
\begin{equation}
  \A{}S(\theta) 
\end{equation}
when the target matrix has two columns or
\begin{equation}
  \A{}S(\theta,\phi) 
\end{equation}
when the target matrix has three columns.

The circles and ellipses have the color determined the the angular variable to provide an clearer idea of how the unit circle is distorted. So for the color red starts at $\theta=0$ and progresses through the spectrum until $\theta=2pi$ where the color is violet.\\

\textbf{Matrix images:} This block summarizes the plot above. For example, it may say that the plot represents a unit sphere being mapped to a line.\\

%%
\textbf{Vector space mappings:} These mappings are based on the dimensions of the spaces for the row and column vectors, $m$ and $n$. They disregard the issue of rank and and concerned purely with the mappings $\real{m}\mapsto\real{n}$ and  $\real{n}\mapsto\real{m}$. This map addresses the geometric deficiency of the mappings. For example are we going from a plane to a plane or a plane to a line. If the map is into a higher dimensional space we will have a frustrated map.\\

\textbf{Matrix ranks:} Are there rank deficiencies in the row or column space? If there is a rank deficiency we will see a frustrated map. 


\clearpage

%%
%% 2 x 2
%%
\begin{table}[htdp]
\begin{center}
\begin{tabular}{cc}
  $\A{}x=y$ & $\A{T}y=x$\\
$\mat{rr}{1&2\\-1&2}\mat{c}{x_{1}\\x_{2}} = \mat{c}{y_{1}\\y_{2}}$ &
$\mat{rr}{1&-1\\2&2}\mat{c}{x_{1}\\x_{2}} = \mat{c}{y_{1}\\y_{2}}$ \\
\ \\
\includegraphics[ width = 2.15in ]{pdf/post_mortemII/2_2_2} &
\includegraphics[ width = 2.15in ]{pdf/post_mortemII/2_2_2_t} \\
%%
\ \\
 $\Ap = \frac{1}{4}\mat{rr}{2&-1\\2&1}$ & $\paren{\A{T}}^{+} = \frac{1}{4}\mat{rr}{2&2\\-1&1}$ \\
\ \\
 $\projra = \itwo$ & $\projrat = \itwo$ \\
\ \\
 $\projrap = \mat{cc}{0&0\\0&0}$ & $\projratp = \mat{cc}{0&0\\0&0}$ \\
\ \\
 $\projna = \itwo$ & $\projnat = \itwo$ \\
\ \\
 $\projnap = \mat{cc}{0&0\\0&0}$ & $\projnatp = \mat{cc}{0&0\\0&0}$ \\
\end{tabular}
\end{center}
\label{tab:proj:a}
\caption{The four fundamental projectors for a full rank matrix and its transpose. There is no null space for either $\A{}$ or $\A{T}$.}
\end{table}%

\clearpage
%%
%% 2 x 3
%%
\begin{table}[htdp]
\begin{center}
\begin{tabular}{cc}
  $\A{}x=y$ & $\A{T}y=x$\\
$\mat{ccc}{0&3&0\\1&1&2}\mat{c}{x_{1}\\x_{2}\\x_{3}} = \mat{c}{y_{1}\\y_{2}}$ &
$\mat{cc}{0&1\\3&1\\0&2}\mat{c}{y_{1}\\y_{2}} = \mat{c}{x_{1}\\x_{2}\\x_{3}}$ \\
\includegraphics[ width = 2.5in ]{pdf/post_mortemII/3_2_2.png} &
\includegraphics[ width = 2.5in ]{pdf/post_mortemII/3_2_2_t} \\
%%
 $\Ap = \frac{1}{15}\mat{rr}{-2&3\\5&0\\-4&2}$ & $\paren{\A{T}}^{+} = \frac{1}{5}\mat{rrr}{-2&5&-4\\3&0&2}$ \\
\ \\
 $\projra = \itwo$ & $\projrat = \frac{1}{5}\mat{ccc}{1&0&2\\0&1&0\\2&0&4}$ \\
\ \\
 $\projrap = \mat{cc}{0&0\\0&0}$ & $\projratp = \frac{1}{5}\mat{rrr}{4&0&-2\\0&\phantom{-}0&0\\-2&0&1}$ \\
\ \\
 $\projna = \frac{1}{5}\mat{ccc}{1&0&2\\0&1&0\\2&0&4}$ & $\projnat = \itwo$ \\
\ \\
 $\projnap = \frac{1}{5}\mat{rrr}{4&0&-2\\0&\phantom{-}0&0\\-2&0&1}$ & $\projnatp = \mat{cc}{0&0\\0&0}$ \\[15pt]
\end{tabular}
\end{center}
\label{tab:proj:b}
\caption{The four fundamental projectors for a matrix with full row rank. Because the target matrix has full row rank the range spans the domain space and  there is no perpendicular complement.}
\end{table}

\clearpage
%%
%% 3 x 2
%%
\begin{table}[htdp]
\begin{center}
\begin{tabular}{cc}
  $\A{}x=y$ & $\A{T}y=x$\\
$\Aexample \mat{c}{x_{1}\\x_{2}} = \mat{c}{y_{1}\\y_{2}\\y_{3}}$ &
$\Atexample\mat{c}{x_{1}\\x_{2}\\x_{3}} = \mat{c}{y_{1}\\y_{2}}$ \\
\includegraphics[ width = 2.5in ]{pdf/post_mortemII/3_2_1_a} &
\includegraphics[ width = 2.5in ]{pdf/post_mortemII/3_2_1_t_a} \\
%%
 $\Ap = \Aplus$ & $\paren{\A{T}}^{+} = \frac{1}{6}\Aexample$ \\
\ \\
 $\projra = \frac{1}{3}\Aexample$ & $\projrat = \frac{1}{2}\mat{rr}{1&-1\\-1&1}$ \\
\ \\
 $\projrap = \frac{1}{3}\mat{rrr}{2&1&-1\\1&\phantom{-}2&1\\-1&1&2}$ & $\projratp = \frac{1}{2}\mat{rr}{1&1\\1&1}$ \\
\ \\
 $\projna = \frac{1}{2}\mat{rr}{1&-1\\-1&1}$ & $\projnat = \frac{1}{3}\Aexample$ \\
\ \\
 $\projnap = \frac{1}{2}\mat{rr}{1&1\\1&1}$ & $\projnatp = \frac{1}{3}\mat{rrr}{2&1&-1\\1&\phantom{-}2&1\\-1&1&2}$ \\[10pt]
\end{tabular}
\end{center}
\label{tab:proj:c}
\caption{The four fundamental projectors for a matrix with both row and column rank deficiences. Because the target matrix has full row rank the range spans the domain space and  there is no perpendicular complement.}
\end{table}
\clearpage
\endinput


	
\endinput