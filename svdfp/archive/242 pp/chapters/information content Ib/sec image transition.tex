\section{A transition between two images}
In the previous section, the image contrast was reduced until the image was completely destroyed. We see that the SVD is very good at finding a weak signal in the image. Now the task is to blend between two images and see how the singular values change.
%
\begin{equation}
  \C{} = (1-j)\A{} + j\B{}, \quad j \in \brac{0,1}
  \label{eq:transition}
\end{equation}
%
%%%%%%
\begin{table}[htdp]
\caption{A transition between two images. After suitable cropping, we have two images of Camille Jordan, the elderly and the younger. The two images are blended according to the prescription in eqref{eq:transition}. The change in the singular value spectra is shown in table \eqref{fig:contrast:blended singular values}.}
\begin{center}
\begin{tabular}{ccccc}
%
 $\A{}$ && $\C{}$ && $\B{}$ \\
%
   \raisebox{-0.5\height}{\includegraphics[ width = 1.25in ]{images/"information content I"/transition/"camille old"}} & $\Rightarrow$  &
   \raisebox{-0.5\height}{\includegraphics[ width = 1.25in ]{images/"information content I"/transition/"camille blend"}} & $\Rightarrow$  &
   \raisebox{-0.5\height}{\includegraphics[ width = 1.25in ]{images/"information content I"/transition/"camille young"}} \\[5pt]
%
   ${\rd{ j = 0 }}$ && $j=\frac{1}{2}$ && ${\bl{ j = 1 }}$
%
\end{tabular}
\end{center}
\label{tab:contrast:youth rediscovered}
\end{table}%

\begin{figure}[htbp] %  figure placement: here, top, bottom, or page
   \centering
   \includegraphics[ width = 4in ]{images/"information content I"/transition/"dorian sigma"} 
   \caption[The singular value spectra for blended images]{The singular value spectra for blended images. The red line represents the spectrum for the elderly Camille $(j=0)$, the blue line for the younger Camille $(j=1)$. Spectra for the 14 intermediate states are shown in gray.}
   \label{fig:contrast:blended singular values}
\end{figure}

\endinput