\chapter*{Literature}

\section*{Linear algebra texts}
There are excellent texts available on the topic of linear algebra. The selection of books here is restricted not on the basis of a systematic survey of the literature, but instead on the basis of personal, and perhaps myopic experience. These are works which have a personal appeal for reasons listed below. 

It is difficult to write well and it is difficult to write with brevity. Laub has managed to do both. His succinct works are a joy to read and he does a fine job of summarizing the vital topics of concern. 
Laub1
Laub2

Meyer's work is much lengthier, and yet it still seems economical. The text, including solutions and hints for all problems, is available on CD on the authors website. It is an invaluable aid for self-study. Applied mathematicians who prefer computation to proof will find that Meyer makes his proofs come alive with relevance and elegance.
Meyer

Strang's works in applied mathematics are de facto gold standards for the community. His linear algebra book is well written and rich with insightful explanation and problems. Also, his classroom lectures are available on the MIT OCW website and they are a superb adjunct to the written words.
Strang 


\section*{Why this book?}
Classroom and laboratory experience suggests a need for this topic. A number of issues arise
\begin{itemize}
\item Contact with material is distant.
\item Contact with material is distant.
\end{itemize}

The target reader is someone who has already had a course in linear algebra and wishes to revisit the foundation concepts. Perhaps the class was a long time ago. Perhaps the SVD, an alluring topic at the start of the term, had been sacrificed on the altar of necessity in the interest of foundation topics. Perhaps the SVD was discussed in a rushed manner at semester's end. Perhaps the SVD was presented in a way that appealed to mathematicians yet did not inspire the reader.\footnote{For instance, constructive proofs of the SVD have different valuations depending upon proximity to the math department.}

Who should find this book helpful? Working scientists and engineers who are working in linear algebra and are unable to recall the \ftola \ or the \asvd \ theorem. This book offers a perspective on linear algebra from the viewpoint of the SVD.





\endinput