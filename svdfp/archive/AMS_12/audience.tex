\chapter*{Intended audience}

This book is intended for working scientists and engineers who have a need to use the \svdp \ and reflects years of working with such a group and so addresses the questions and issues that reoccur. Many people have had linear algebra, but it may not have addressed the SVD, or it may have been covered in a fleeting or perfunctory way. Also, of the restricted subset who did study the SVD, some were left feeling a bit mystified by the constructive proof that was their introduction. The effects of being out of the classroom for several years do not help.

And so the goal here is to rely upon basic knowledge of linear algebra (Gaussian elimination, homogeneous and inhomogeneous solutions, eigenvalues) and to refresh essential concepts like rank and Fundamental Theorem of linear algebra. Hopefully readers will connect rapidly to the material here and be doing decompositions within minutes. The emphasis is on the simplicity of the concept. As such, theoretical details occur later.

Experience suggests that difficulty with the SVD is due in part to a scattering and sublimation of key concepts in linear algebra. Certainly this is natural after several years in the work force and concentration on the specialities of the career field. The approach here is to bring these concepts to the fore while performing decompositions; a form of learning through example.

The repetition here honors the fact that this book is used outside of a classroom and away from an instructor who may clarify concepts and emphasis specific issues. It also acknowledges that books are not always read linearly from front cover to back cover. The design encourages the reader to follow along with pencil and paper. Omission of intermediate steps is kept to a minimum. Sometimes books seem to shout ``the author can do complicated mathematics.'' The purpose here is to demonstrate that the reader can perform these operations.

The book begins immediately with the SVD (assuming that the review in chapter 0 can be skipped). We start with the simplest form and progress to more involved techniques. We pause to reflect back on the significance of the solutions. Along the way the command of linear algebra will blossom. At the end we hope the reader not only understands the SVD but also appreciates its place in linear algebra. 

We would be remiss to end the book so abruptly; readers may wish to explore further. To allow this more theoretical discussions and analysis round out the remainder of the text. Certainly some readers need only a few chapters to be able to understand and apply the SVD in their work. But other readers will see this as only a beginning. There is a great deal to be learned from the SVD and we hope to show numerous examples of its utility.

Should you start with the review or the \svdl \ in chapter 1? Look at the goals for the review in chapter 0. If you are comfortable with those concepts then proceed to chapter 1.

\endinput