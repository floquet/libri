\chapter{Linearization}

\section{Linear Transformation}  %    S    S    S    S    S    S    S    S    S

A linear transformation $T$ satisfies the requirement
  \begin{equation*}   %  =   =   =   =   =
  %\begin{split}
    T(x + \alpha y) = T(x) + \alpha T(y).
    \label{eq:defn lt}
  %\end{split}
  \end{equation*}
An immediate consequence is
  \begin{equation*}   %  =   =   =   =   =
  %\begin{split}
    T(\underbrace{x - x}_{0}) = \underbrace{T(x) - T(x)}_{0} = 0,
    %\label{eq:}
  %\end{split}
  \end{equation*}
therefore, because $x-x=0$, we must have $T(0) = 0$.

Is the transformation $T(x) = a_{0} + a_{1} x$ a linear transformation? No, because $T(0) = b$. This is, however, an example of an affine transformation.

\section{Mythology}  %    S    S    S    S    S    S    S    S    S

An enduring misadventure in least squares is to hope that an exponential function like
  \begin{equation*}   %  =   =   =   =   =
  %\begin{split}
    y(x) = a_{0} e^{a_{1}x}
    %\label{eq:}
  %\end{split}
  \end{equation*}
can be linearized with a logarithmic transformation:
  \begin{equation*}   %  =   =   =   =   =
  %\begin{split}
    \tilde{y}(x) = \ln \paren{y(x)} = \ln a_{0} + a_{1} x.
    %\label{eq:}
  %\end{split}
  \end{equation*}
Is the logarithm a linear transformation? Of course not:
  \begin{equation*}   %  =   =   =   =   =
  %\begin{split}
    \ln \paren{x + \alpha y} \ne \ln x + \alpha \ln y.
    %\label{eq:}
  %\end{split}
  \end{equation*}

\endinput