\section[Full Jordan forms]{The full $2\times2$ Jordan form}
\label{sec:Jordan:full}

The most vexations \svdl s involve the full Jordan forms. There are forms which are an entire upper triangular matrices of unit entries. If the target matrix is similar to a full Jordan form the \svdl \ will be arduous.

Consider the full form for the $2\times2$ Jordan forms. The depleted forms have trivial decompositions; the full form is much more complex.

\subsection{$2\times2$ form}
The full Jordan normal form for size $m=2$ and its decomposition is given by this
\begin{equation}
  \begin{split}
    \mathcal{J}_{2,7} = 
\mat{cc}{
 1 & 1 \\
 0 & 1
}
 &= \svd{T}\\
\Y{}&=\mat{rr}{
 \sqrt{\frac{1}{10} \left(5+\sqrt{5}\right)} & -\sqrt{\frac{1}{10} \left(5-\sqrt{5}\right)} \\
 \sqrt{\frac{1}{10} \left(5-\sqrt{5}\right)} & \sqrt{\frac{1}{10} \left(5+\sqrt{5}\right)}
}\\
\sig{}&=\mat{cc}{
 \frac{1}{2} \left(1+\sqrt{5}\right) & 0 \\
 0 & \frac{1}{2} \left(-1+\sqrt{5}\right)
}
\\
\X{}&=\mat{rr}{
 \sqrt{\frac{1}{10} \left(5-\sqrt{5}\right)} & -\sqrt{\frac{1}{10} \left(5+\sqrt{5}\right)} \\
 \sqrt{\frac{1}{10} \left(5+\sqrt{5}\right)} & \sqrt{\frac{1}{10} \left(5-\sqrt{5}\right)}
}.
  \end{split}
\end{equation}
%%%

This looks quite daunting until we see that we are dealing with rotator matrices using complementary angles:
%%%
\begin{equation}
  \begin{array}{ccccccc}
    \mathcal{J}_{2,7} &=& 
\mat{cc}
{
 1 & 1 \\
 0 & 1
}
 &=& \svd{T}\\
 &&&=& \rat{\theta}{}
&\mat{cc}{
 \alpha_{+} & 0 \\
 0 & \alpha_{-}
}
&\rat{\theta_{c}}{T}\\
 &&&=& e^{i\theta_{y}}
&\mat{cc}{
 \alpha_{+} & 0 \\
 0 & \alpha_{-}
}
&e^{-i\theta_{x}}
  \end{array}
\end{equation}
where the singular values are
\begin{equation}
  \alpha_{\pm} = \frac{1}{2} \left(\pm1+\sqrt{5}\right);
  \label{eq:j27:1}
\end{equation}
and the rotation angle is given by this
\begin{equation}
  \theta = \arcsin\paren{\sqrt{\frac{1}{10} \left(5-\sqrt{5}\right)}},
  \label{eq:j27:2}
\end{equation}
with $\theta_{c}$ being the complementary angle:
\begin{equation}
  \theta + \theta_{c} = \frac{\pi}{2}.
  \label{eq:j27:3}
\end{equation}


\endinput