\section{Find the pseudoinverse}
Build a few samples of the pseudoinverse. The target matrix in all cases is $\aicmn$

%%%%%%%%%%%
\subsection{When $\paren{\wy{*}}^{-1}$ exists}
\begin{equation}
  \begin{split}
    \wy{*} 
      &= \paren{\svd{*}} \paren{\svdtt{*}{T}} \\
      &= \U{} \sig{} \sig{T} \U{*} \\
      &= \I{m} \\
  \end{split}
\end{equation}
\begin{equation}
  \sig{} \sig{T} = \I{m}
\end{equation}
Therefore the matrix $\A{}$ has full column rank: $\rho = n$.
\begin{equation}
  \sig{} \sig{T} = \sbr{}\sbctt{} = \ess{2}
\end{equation}
\begin{equation}
  \A{} = \csvdblockbr{*}
\end{equation}



%%%%%%%%%%%
\subsection{When $\wx{*}=\I{n}$}
We cannot restrict the matrix $\A{}$ to being unitary.

%%%%%%%%%%%
\subsection{When $\A{*}=\A{}$ and $\A{2}=\A{}$}
The target matrix is Hermitian and idempotent. Because the target matrix is Hermitian, the matrix is square so $m = n$. Intuitively, we expect that $\ess{}=\I{\rho}$ when we think of the matrix image on the unit circle. Because
\begin{equation}
  \A{}\circletwo = \A{}\paren{\A{}\circletwo}  = \A{}\paren{\A{2}\circletwo} = \cdots
\end{equation}
we see that repeated applications of $\A{}$ to the image leave the image unchanged. 
\begin{equation}
  \sig{} = \stencil
\end{equation}
\begin{equation}
  \A{} = \cublockf \stencil \cvblockfs{ * }
\end{equation}


We conclude that pseudoinverse matrix is identical to the Hermitian conjugate:
\begin{equation}
  \Ap = \A{*}
\end{equation}



\endinput