\chapter{Information content II\\Basic matrix images}

This chapter explores the information content of the singular values. There are two perspectives: the canonical image as matrix and then matrix as image. This will compel the following chapter to examine more basic shapes of matrix images. Study most basic features: points, edges, curves.

%%%%%%
\input{chapters/"information content II"/"tab the singular value decomposition of Camille's photo"}  % table
\input{chapters/"information content II"/"sec summaries"}
%
%
\clearpage
\section{Low rank images}
\input{chapters/"information content II"/"tab cross"}
\input{chapters/"information content II"/"tab checkerboard"}
\input{chapters/"information content II"/"tab stripes, diagonal"}
%\input{chapters/"information content II"/"tab horizontal stripes"}
\input{chapters/"information content II"/"tab stripes, vertical"}
%
%%
%\clearpage
%\section{Moderate rank images}
%\input{chapters/"information content II"/"tab disk png"}
%\input{chapters/"information content II"/"tab soft disk"}
%\input{chapters/"information content II"/"tab cross"}
%\input{chapters/"information content II"/"tab ring"}
%\input{chapters/"information content II"/"tab diamond"}
%\input{chapters/"information content II"/"tab gradient"}
%\input{chapters/"information content II"/"tab text"}
%\input{chapters/"information content II"/"tab potential"}
%
%%
%\clearpage
%\section{Full rank images}
%\input{chapters/"information content II"/"tab circulant"}
%\input{chapters/"information content II"/"tab random"}
%\input{chapters/"information content II"/"tab tones"}
%
%%
%\clearpage
%\section{Koch snowflake}
%Start with a equilateral triangle. Trisect each segment. On the middle third, construct an equilateral triangle.
%
%\begin{figure}[htbp] %  figure placement: here, top, bottom, or page
%   \centering
%   \raisebox{-0.5\height}{\includegraphics[ width = 1.2in ]{images/"information content II"/Koch/"fractal segment"}} 
%   \qquad $\Rightarrow$ \qquad
%   \raisebox{-0.5\height}{\includegraphics[ width = 1.2in ]{images/"information content II"/Koch/"fractal triangle"}} 
%   \caption{Generating the Koch snowflake. Each line segment is trisected and then a equilateral triangle is created in the middle section.}
%   \label{fig:info II:koch:rules}
%\end{figure}
%
%\input{chapters/"information content II"/"tab snowflakes"}
%\input{chapters/"information content II"/"tab snowflake spectra"}


\endinput