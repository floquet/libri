\clearpage

\begin{enumerate}
\item After considering the range and codomain of the two formulations, write a single sentence to explain how the equation
\begin{equation}
  \A{}x=b
\end{equation}
may not have a solution but the normal equations
\begin{equation}
  \A{T}\A{}x=\A{T}b
\end{equation}
will have a solution. 
\begin{quotation}
The vector $b$ may not be in the image or range of the matrix $\A{}$, however the vector $\A{T}b$ is in the range of $\A{T}$ by of construction.
\end{quotation}
%%
\item In the example we found that the product matrix of the matrix $\A{\by{3}{2}}_{2}$
\begin{equation}
   \W{y}=\A{}\,\A{T} =
\mat{cc}
 {9 & 6 \\
  6 & 9} 
\end{equation}
in equation \eqref{eq:problemgen:Wy} has the eigenvalue spectrum
\begin{equation}
  \lambda\paren{\W{y}} = \lst{15,3}.
\end{equation}
Using reasoning alone without computation, what is the eigenvalue spectrum of $\W{x}$?
\begin{quotation}
  The eigenvalue spectrum of either product matrix contains all of the nonzero eigenvalues. Therefore the nonzero eigenvalues of $\W{x}$ are also $\lst{15,3}$. By inspecting the size of the target matrix, we see that the $\W{x}\in\cmplx{\bys{3}}$ and therefore will have three eigenvalues. Therefore the third eigenvalue is zero. Therefore
\begin{equation}
  \lambda\paren{\W{x}}=\lst{15,3,0}.
\end{equation}
\end{quotation}
\end{enumerate}
%%

\endinput