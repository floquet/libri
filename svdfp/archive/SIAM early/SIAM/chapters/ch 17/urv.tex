\section[$\A{}=\U{}\R{}\V{*}$]{$\U{}\R{}\V{*}$ decomposition}

The complete orthogonal transformation. Given $\Amnr$
\begin{equation}
  \A{}=\U{}\R{}\V{*} = \U{}\ceematrix{}\V{*}
\end{equation}
Just like the SVD, the column vectors of the unitary matrices $\U{}$ and $\V{}$ provide an orthonormal basis for the codomain and domain. The shape arbitrator matrix $\R{}\Accmn$ contains the full rank matrix $\C{}\in\Accc{\rho}{\rho}{\rho}$.

This decomposition is very much like the SVD. It provides an orthonormal decomposition of the four fundamental subspaces:
\begin{equation}
  \A{} = \U{}\R{}\V{*} = \yft{} \ceematrix{} \xtft{*}.
\end{equation}
The columns are organized as follows.
\begin{enumerate}
\item First $\rho$ columns of $\U{}$ span $\rnga{}$.
\item First $\rho$ columns of $\V{}$ span $\rnga{*}$.
\item Last  $m - \rho$ columns of $\U{}$ span $\nlla{*}$.
\item Last  $n - \rho$ columns of $\V{}$ span $\nlla{}$.
\end{enumerate}
One could argue that this matrix is a natural development on the way to the SVD. In fact, in Meyer's book\cite{p. 407}[Meyer] this factorization leads directly to his discussion of the SVD.

However, there are some profound differences.

%%%
\subsection{Theory}
\begin{quote}
\end{quote}

%%%
\subsection{Examples}
Full row rank and full column rank:
\begin{equation}
  \begin{split}
    \A{} & = \Q{}\,\R{},\\
    \Ab & =
    \stwo   \mat{rr}{1&1\\-1&1}
    \sqrt{2}\mat{cc}{2 & 0 \\ 0 & 1}
  \end{split}
\end{equation}

Full row rank with column rank deficiency:
\begin{equation}
  \begin{split}
    \A{} & = \Q{}\,\R{},\\
    \Aa & =
    \ktwo
    \mat{ccc}{1 & 2 & 2 \\ 0 & 3 & 0}
  \end{split}
\end{equation}
The $\Q{}$ matrix is the exact same matrix $\Y{}$ in the example.


Row and column rank deficiency:
\begin{equation}
  \begin{split}
    \A{} & = \Q{}\,\R{},\\
    \Aexample & =
    \Yshade
    \mat{rr}{\sqrt{3}&-\sqrt{3}\\0&0\\0&0}
  \end{split}
\end{equation}
The $\Q{}$ matrix is the exact same matrix $\Y{}$ in the example.


%%%
\subsection{Application}

%%%
\subsection{Comparison to the SVD}
This matrix factorization most closely resembles the SVD. In fact, SVD is a special case where the the upper-triangular matrix $\T{}$ is instead the diagonal matrix $\sig{}$.

Show how to convert the URV to SVD.

\endinput