\section{Chiral inverses}
\label{sec:chiral}
Chiral matrix inverses are inverses which operate on the left- or right-hand side of the target matrix. In analogy with the concept of chirality in chemistry, the chiral inverses have different shapes; in fact they are transposes.

%%
\subsection{Definitions}
For a target matrix $\A{}\in\cmplx{\by{m}{n}}_{\rho}$ the pseudoinverse $\Ap\in\cmplx{\by{n}{m}}_{\rho}$. That is the pseudoinverse has the shape of the transpose of the target matrix and retains the same rank. Figure \eqref{fig:mpp:chiral} shows when the pseudoinverse will be a chiral inverse. When the target matrix has full row rank ($\rho=m$), the pseudoinverse $\Ap$ is a right inverse. When the matrix has full column rank $\Ap$ is a left inverse. When the target matrix has full row and column rank it is square and the pseudoinverse is the standard inverse: $\Ap=\A{-1}$.

\begin{table}[htdp]
\begin{center}
\begin{tabular}{l|clc}
chirality, $\chi$ & requirement & equivalence & expression\\\hline
right & $\rho = m$ & $\Ap=\AinvR$ & $\rightinv=\I{m}$\\
left  & $\rho = n$ & $\Ap=\AinvL$ & $\leftinv=\I{n}$\\
ambichiral & $\rho=m=n$ & $\Ap=\A{-1}$ & $\rightinv=\leftinv=\I{m}$\\
ampichiral & $\rho< m,\rho< n$ & \quad none & $\rightinv\ne\I{m},\ \leftinv\ne\I{n}$\\
\end{tabular}
\end{center}
\label{fig:mpp:chiral}
\caption[Conditions for when the pseudoinverse is a chiral inverse]{Conditions for when the pseudoinverse is a chiral inverse. The conditions are when the target matrix has either full row rank or full column rank.}
\end{table}%

Examine cases illustrating the four different classifications for the actions of the pseudoinverse. The rules are basic:
\begin{enumerate}
\item If the matrix has full \textit{row} rank, the pseudoinverse is a \textit{right} inverse.
\begin{equation}
  \rho = m: \quad \Ap = \AinvL
\end{equation}
\item If the matrix has full \textit{column} rank, the pseudoinverse is a \textit{left} inverse.
\begin{equation}
  \rho = n: \quad \Ap = \AinvR
\end{equation}
\end{enumerate}

From this we conclude that if the matrix has full row rank and full column rank the pseudoinverse both a left and right inverse. That is, the pseudoinverse in the classic inverse.

We also conclude that if the matrix is rank deficient in both row and column then the pseudoinverse in neither a right not a left inverse.

Examples of all four cases follow.

%%%
\subsection{Ambichiral or classic inverse: $\Ap = \A{-1}$}
What are the two rank conditions?
\begin{enumerate}
\item The matrix has full row rank: $\rho = m$.
\item The matrix has full column rank: $\rho = n$.
\end{enumerate}
Therefore the matrix has $\rho$ row and $\rho$ columns. Therefore it is square and of full rank. We don't know what the rank is or what the size is, but we can state
\begin{equation}
  \rho = m = n.
\end{equation}

Choose a target matrix $\Arrr{2}{2}{2}$.
\begin{equation*}
  \begin{array}{ccccc}
    \A{} &=& \Y{} & \sig{} & \X{T} \\
    \matrixalpha &=
    & \matrixalphaY 
    & \matrixalphasigma 
    & \matrixalphaXt.
  \end{array}
\end{equation*}
The pseudoinverse is given by
\begin{equation*}
  \begin{array}{ccccc}
    \Ap &=& \X{} & \sig{(+)} & \Y{T} \\
    \matrixalphapi &=
    & \matrixalphaX 
    & \matrixalphasigmapi
    & \matrixalphaYt.
  \end{array}
\end{equation*}
The matrix products produce the same result:
\begin{equation*}
  \begin{array}{cccc}
  \A{}&\Ap&=&\I{2},\\
  \matrixalpha &
  \matrixalphapi & = &
  \itwo;\\[25pt]
  \Ap&\A{}&=&\I{2},\\
  \matrixalphapi &
  \matrixalpha & = &
  \itwo.
  \end{array}
\end{equation*}

\subsection{Right inverse only: $\Ap=\AinvR$}
What are the two rank conditions?
\begin{enumerate}
\item The matrix has full row rank: $\rho = m$.
\item The matrix has partial column rank: $\rho < n$.
\end{enumerate}
Therefore the matrix has $\rho$ rows and $\rho$ columns. Therefore it is square and of full rank. We don't know what the rank is or what the size is, but we can state
\begin{equation}
  \begin{split}
     \rho = m,\\
     \rho < n.
  \end{split}
\end{equation}

Consider a target matrix $\Arrr{2}{3}{2}$.
\begin{equation*}
  \begin{array}{ccccc}
  \A{} &=& \Y{} & \sig{} & \X{T}\\
  \matrixbravo &=
  & \matrixbravoY  
  & \matrixbravosigma
  & \matrixbravoXt.
  \end{array}
\end{equation*}
The pseudoinverse is constructed in this fashion:
\begin{equation*}
  \begin{array}{ccccc}
    \Ap &=& \X{} & \sig{(+)} & \Y{T} \\
    \matrixbravopi &=
    & \matrixbravoX 
    & \matrixbravosigmapi
    & \matrixbravoYt.
  \end{array}
\end{equation*}
Postmultiplication by the pseudoinverse produces an identity matrix:
\begin{equation*}
  \begin{array}{cccc}
    \A{}&\Ap&=&\I{2},\\
   \matrixbravo &
   \matrixbravopi &=&
   \itwo.
  \end{array}
\end{equation*}
The complementary result is not as nice. Premultiplication by the pseudoinverse produces this result:
\begin{equation*}
  \begin{array}{cccc}
    \Ap&\A{}&\in&\real{\by{3}{3}}_{2},\\
   \matrixbravopi &
   \matrixbravo &=&
   \frac{1}{5}
   \mat{ccc}
   {
   1&0&2\\
   0&5&0\\
   2&0&4
   }.
  \end{array}
\end{equation*}

%%%
\subsection{Left inverse only: $\Ap = \AinvL$}
The two rank conditions are these:
\begin{enumerate}
\item The matrix has partial row rank: $\rho < m$.
\item The matrix has full column rank: $\rho = n$.
\end{enumerate}
These two statements are equivalent to these mathematical relationships:
\begin{equation}
  \begin{split}
     \rho &= n, \\
     \rho &< m
  \end{split}
\end{equation}

The target matrix for this example is $\A{}\in\cmplx{\by{3}{2}}_{2}$
\begin{equation*}
  \begin{array}{ccccc}
  \A{} & = & \Y{} & \sig{} & \X{T},\\
  \mat{rr}{1&2\\0&2i\\1&-2} &=&
  \left[
\begin{array}{rc>{\columncolor{ltgray}}r}
  \frac{1}{\sqrt{3}} & \frac{1}{\sqrt{2}} & -\frac{1}{\sqrt{6}} \\
  \frac{i}{\sqrt{3}} & 0 & \frac{2i}{\sqrt{6}} \\
 -\frac{1}{\sqrt{3}} & \frac{1}{\sqrt{2}} & \frac{1}{\sqrt{6}}
\end{array}
\right] &
\left[
\begin{array}{cc}
 2 \sqrt{3} & 0 \\
 0 & \sqrt{2} \\\hline
 0 & 0
\end{array}
\right] &
\left[
\begin{array}{cc}
 0 & 1 \\
 1 & 0
\end{array}
\right]
  \end{array}.
\end{equation*}
The pseudoinverse is constructed as so:
\begin{equation*}
  \begin{array}{ccccc}
  \Ap & = & \X{} & \index{Sigma matrix!inverse}\sig{(+)} & \Y{*},\\
  \frac{1}{6}\left[
\begin{array}{crr}
 3 & 0 & 3 \\
 1 & -i & -1
\end{array}
\right] &=&
  \left[
\begin{array}{cc}
 0 & 1 \\
 1 & 0
\end{array}
\right]&
\left[
\begin{array}{cc|c}
 \frac{1}{2 \sqrt{3}} & 0 & 0 \\
 0 & \frac{1}{\sqrt{2}} & 0
\end{array}
\right] &
\left[
\begin{array}{rcr}
  \frac{1}{\sqrt{3}} & \frac{i}{\sqrt{3}} & -\frac{1}{\sqrt{3}} \\
  \frac{1}{\sqrt{2}} & 0                  &  \frac{1}{\sqrt{2}} \\
  \rowcolor{ltgray}
 -\frac{1}{\sqrt{6}} & \frac{2i}{\sqrt{6}} & \frac{1}{\sqrt{6}}
\end{array}
\right].
\end{array}
\end{equation*}
Premultiplcation by the pseudoinverse produces an identity matrix:
\begin{equation*}
  \begin{array}{cccc}
  \Ap&\A{}&=&\I{2},\\
  \frac{1}{6}\left[
\begin{array}{crr}
 3 &  0 & 3 \\
 1 & -i & -1
\end{array}
\right] &
  \mat{rr}{1&2\\0&2\\1&-2} & = &
  \itwo.
  \end{array}
\end{equation*}
Postmultiplcation by the pseudoinverse does not produce an identity matrix:
\begin{equation*}
  \begin{array}{cccc}
  \A{}&\Ap&\in&\real{\by{3}{3}}_{2},\\
  \mat{rr}{1&2\\0&2i\\1&-2} &
  \frac{1}{6}\left[
\begin{array}{crr}
 3 &  0 & 3 \\
 1 & -i & -1
\end{array}
\right]
   & = &
  \frac{1}{6}
\left[
\begin{array}{crr}
 5 & -2 i & 1 \\
 2 i & 2 & -2 i \\
 1 & 2 i & 5
\end{array}
\right]
  \end{array}
\end{equation*}

%%%
\subsection{Ampichiral or generalized inverse: $\Ap\ne\AinvB$}
The two rank conditions are these:
\begin{enumerate}
\item The matrix has partial row rank: $\rho < m$.
\item The matrix has partial column rank: $\rho < n$.
\end{enumerate}

The target matrix $\A{}\in\real{\by{3}{2}}_{1}$ is studied in the first chapter and the \svdl \ is presented in equation \eqref{eq:simple:svd}; the pseudoinverse is given in equation \eqref{eq:simple:mppdecomp}. The multiplicative properties follow.

%%%
Multiplication on the left by the pseudoinverse does not produce an identity matrix:
\begin{equation*}
  \begin{array}{cccc}
  \Ap&\A{}&\in&\real{\by{2}{2}}_{1},\\
  \Aexamplepi &
  \Aexample 
   & = &
  \frac{1}{2}
  \mat{rrr}{1&-1&\\-1&1}
  \end{array}.
\end{equation*}

%
Multiplication on the right by the pseudoinverse does not produce an identity matrix:
\begin{equation*}
  \begin{array}{cccc}
  \A{}&\Ap&\in&\real{\by{3}{3}}_{1},\\
  \Aexample &
  \Aexamplepi
   & = &
  \frac{1}{3}
  \mat{rrr}{1&-1&1\\-1&1&-1\\1&-1&1}
  \end{array}.
\end{equation*}

%%
\section{The $\sig{}$ matrix}
The condition for the chirality of the pseudoinverse is encoded in the $\sig{}$ matrices. That is, we can classify the chirality of the inverse (left, right, both, neither) simply by inspecting the $\sig{}$ matrix. 

These conditions are summarized in table \eqref{tab:mpp:chiralsigma}.
%%%%
\begin{table}[htdp]
\begin{center}
%
\begin{tabular}{l|rclcrcl}
    $\chi$  & product  && & iff & condition \\\hline
    left  & $\leftinv$  &$=$& $\I{n}$ & $\iff$ & $\sig{(+)}\sig{}$ &=& $\I{n}$\\
    right & $\rightinv$ &$=$& $\I{m}$ & $\iff$ & $\sig{}\sig{(+)}$ &=& $\I{m}$\\
    both  & $\rightinv = \leftinv$ &$=$& $\I{m}=\I{n}$ & $\iff$ & $\sig{}\,\sig{(+)}=\sig{(+)}\sig{}$ &=& $\I{m}$\\
    neither  & $\rightinv $ &$\ne$& $\I{m}$ & $\iff$ & $\sig{}\,\sig{(+)}$ &$\ne$& $\I{m}$\\
             & $\leftinv $  &$\ne$& $\I{n}$ &        & $\sig{(+)}\,\sig{}$ &$\ne$& $\I{n}$\\[5pt]
\end{tabular}
%
\end{center}
\label{tab:mpp:chiralsigma}
\caption[Necessary and sufficient conditions to classify the chirality of the pseudoinverse]{The necessary and sufficient conditions to classify the chirality of the pseudoinverse can be expressed in terms of the $\sig{}$ matrix.}
\end{table}%

In this case with full row rank the pseudoinverse is also a \index{right inverse}right inverse. In this case with full column rank the pseudoinverse is also a \index{left inverse}left inverse. When the matrix has full rank (the column rank equals the row rank) the pseudoinverse reverts to being the standard inverse. Table \eqref{tab:pmii:rank} summarizes these points.
\begin{table}[bottom]
\begin{center}
\begin{tabular}{lll}
rank condition   & \ parameters \ & \ inverse condition\\\hline\hline
full row rank    & \ $\rho = m $  & \ $\Ap = \AinvR$ \\[3pt]
full column rank & \ $\rho = n $  & \ $\Ap = \AinvL$ \\[3pt]\hline
full row and column rank \ & \ $\rho = m = n $ \ & \ $\Ap = \AinvL = \AinvR$\\  row and column rank deficit \ & \ $\rho < m, \rho < n $ \ & \ $\Ap \ne \AinvB$ \\[6pt]
\end{tabular}
\end{center}
\label{tab:pmii:rank}
\caption[Classifications of the pseudoinverse]{Classifications of the pseudoinverse depend upon the rank condition of the target matrix. We see a concise definition of when the pseudoinverse is also the standard matrix inverse.}
\end{table}%
%%%

The formulaically minded may prefer this more formal mathematical presentation:
\begin{equation}
  \begin{array}{rclrcr}
    \leftinv &=& \I{n}, & \quad \A{}&\in&\cmplx{\by{m}{n}}_{n},\\
    \rightinv &=& \I{m}, & \quad \A{}&\in&\cmplx{\by{m}{n}}_{m},\\
    \leftinv = \rightinv &=& \I{m}, & \quad \A{}&\in&\cmplx{\by{m}{m}}_{m}.\\
  \end{array}
\end{equation}

%%
\subsection{Demonstrations of the four types of chiral inverses}

%%
\subsubsection{$\Ap=\A{-1}$}
What are the two rank conditions?
\begin{enumerate}
\item The matrix has full row rank.
\item The matrix has full column rank.
\end{enumerate}
Therefore the matrix has $\rho$ row and $\rho$ columns. Therefore it is square and of full rank. We don't know what the rank is or what the size is, but we can state
\begin{equation}
  \rho = m = n.
\end{equation}

%%
\subsubsection{$\Ap=\AinvL$}
\label{sec:lchiral}
The two rank conditions are these:
\begin{enumerate}
\item The matrix has partial row rank.
\item The matrix has full column rank.
\end{enumerate}
These two statements are equivalent to these mathematical relationships:
\begin{equation}
  \begin{split}
     \rho &= n, \\
     \rho &< m
  \end{split}
\end{equation}

%%
\subsubsection{$\Ap=\AinvR$}
This example flips the tank conditions.
\begin{enumerate}
\item The matrix has full row rank.
\item The matrix has partial column rank.
\end{enumerate}
These two statements are equivalent to these mathematical relationships:
\begin{equation}
  \begin{split}
     \rho &= m, \\
     \rho &< n
  \end{split}
\end{equation}


%%
\subsubsection{$\Ap\ne\AinvB$}
The final case is the least restrictive.
\begin{enumerate}
\item The matrix has partial row rank.
\item The matrix has partial column rank.
\end{enumerate}
These two statements are equivalent to these mathematical relationships:
\begin{equation}
  \begin{split}
     \rho &< m, \\
     \rho &< n
  \end{split}
\end{equation}
We can make no statement about the relationship between the number of rows, $m$ and the number of columns, $n$.

%%
\subsection{Proximity}
A quick note before closing. What about the first result in equation \eqref{eq:gen:lr}? How ``close'' is this matrix to the identity matrix?
\begin{equation}
\normt{\leftinv-\I{m}} = 
\normt{\frac{1}{5}
  \mat{ccc}
  {
 1 & 0 & 2 \\
 0 & 5 & 0 \\
 2 & 0 & 4
  }
  -
  \ithree} = 1.
\end{equation}
We will see this last result a few more times.

\endinput